\documentclass{elsarticle}

\usepackage[english]{babel}
\usepackage[utf8]{inputenc}
\usepackage[T1]{fontenc}

\usepackage{amsmath}  % Maths
\usepackage{amsfonts} % Maths
\usepackage{amssymb}  % Maths
\usepackage{stmaryrd} % Maths (crochets doubles)

\usepackage{url}     % Mise en forme + liens pour URLs
\usepackage{hyperref}
\usepackage{array}   % Tableaux évolués

\usepackage{comment}

% Theorems and definitions
\newdefinition{definition}{Definition}
\newdefinition{condition}{Condition}
\newtheorem{theorem}{Theorem}
\newtheorem{lemma}{Lemma}
\newproof{example}{Example}
\newproof{rawproof}{Proof}

\newenvironment{proof}{\begin{rawproof}}{\hfill$\Box$\end{rawproof}}

% Pretty references
\usepackage{prettyref}
\newrefformat{def}{Def.~\ref{#1}}
\newrefformat{fig}{Fig.~\ref{#1}}
\newrefformat{lem}{Lemma~\ref{#1}}
\newrefformat{th}{Theorem~\ref{#1}}
\newrefformat{co}{Corollary~\ref{#1}}
\newrefformat{sec}{Sect.~\ref{#1}}
\newrefformat{ssec}{Subsect.~\ref{#1}}
\newrefformat{suppl}{Appendix~\ref{#1}}
\newrefformat{cr}{Condition~\ref{#1}}
\newrefformat{eq}{Eq.~\eqref{#1}}
\def\pref{\prettyref}

\usepackage{tikz}
\newdimen\pgfex
\newdimen\pgfem
\usetikzlibrary{arrows,shapes,shadows,scopes}
\usetikzlibrary{positioning}
\usetikzlibrary{matrix}
\usetikzlibrary{decorations.text}
\usetikzlibrary{decorations.pathmorphing}

% Macros relatives à la traduction de PH avec arcs neutralisants vers PH à k-priorités fixes

% Macros générales
%\newcommand{\ie}{\textit{i.e.} }

% Notations générales pour PH
\newcommand{\PH}{\mathcal{PH}}
%\newcommand{\PHs}{\mathcal{S}}
\newcommand{\PHs}{\Sigma}
%\newcommand{\PHp}{\mathcal{P}}
\newcommand{\PHp}{\textcolor{red}{\mathcal{P}}}
%\newcommand{\PHproc}{\mathcal{P}}
\newcommand{\PHproc}{\mathbf{Proc}}
\newcommand{\Proc}{\PHproc}
\newcommand{\PHh}{\mathcal{H}}
\newcommand{\PHa}{\PHh}
%\newcommand{\PHa}{\mathcal{A}}
\newcommand{\PHl}{\mathcal{L}}
\newcommand{\PHn}{\mathcal{N}}

\newcommand{\PHhitter}{\mathsf{hitter}}
\newcommand{\PHtarget}{\mathsf{target}}
\newcommand{\PHbounce}{\mathsf{bounce}}
%\newcommand{\PHsort}{\Sigma}
\newcommand{\PHsort}{\PHs}

\newcommand{\hitter}[1]{\PHhitter(#1)}
\newcommand{\target}[1]{\PHtarget(#1)}
\newcommand{\bounce}[1]{\PHbounce(#1)}

%\newcommand{\PHfrappeur}{\mathsf{frappeur}}
%\newcommand{\PHcible}{\mathsf{cible}}
%\newcommand{\PHbond}{\mathsf{bond}}
%\newcommand{\PHsorte}{\mathsf{sorte}}
%\newcommand{\PHbloquant}{\mathsf{bloquante}}
%\newcommand{\PHbloque}{\mathsf{bloquee}}

%\newcommand{\PHfrappeR}{\textcolor{red}{\rightarrow}}
%\newcommand{\PHmonte}{\textcolor{red}{\Rsh}}

\newcommand{\PHhitA}{\rightarrow}
\newcommand{\PHhitB}{\Rsh}
%\newcommand{\PHfrappe}[3]{\mbox{$#1\PHhitA#2\PHhitB#3$}}
%\newcommand{\PHfrappebond}[2]{\mbox{$#1\PHhitB#2$}}
\newcommand{\PHhit}[3]{#1\PHhitA#2\PHhitB#3}
\newcommand{\PHfrappe}{\PHhit}
\newcommand{\PHhbounce}[2]{#1\PHhitB#2}
\newcommand{\PHobj}[2]{\mbox{$#1\PHhitB^*\!#2$}}
\newcommand{\PHconcat}{::}
%\newcommand{\PHneutralise}{\rtimes}

\def\PHget#1#2{{#1[#2]}}
%\newcommand{\PHchange}[2]{#1\langle #2 \rangle}
%\newcommand{\PHchange}[2]{(#1 \Lleftarrow #2)}
%\newcommand{\PHarcn}[2]{\mbox{$#1\PHneutralise#2$}}
\newcommand{\PHplay}{\cdot}

\newcommand{\PHstate}[1]{\mbox{$\langle #1 \rangle$}}

\def\supp{\mathsf{support}}
\def\first{\mathsf{first}}
\def\last{\mathsf{last}}

\def\DNtrans{\rightarrow_{ADN}}
\def\DNdef{(\mathbb F, \langle f^1, \dots, f^n\rangle)}
\def\DNdep{\f{dep}}
%\def\PHPtrans{\rightarrow_{PH}}
\newcommand{\PHPtrans}[1][\PH]{\rightarrow_{#1}}
\def\get#1#2{#1[{#2}]}
\def\encodeF#1{\mathbf{#1}}
\def\toPH{\encodeF{PH}}
\def\card#1{|#1|}
\def\decode#1{\llbracket#1\rrbracket}
\def\encode#1{\llparenthesis#1\rrparenthesis}
\def\Hits{\PHa}
\def\hit{\PHhit}
\def\play{\cdot}

\newcommand{\angles}[1]{{\langle #1 \rangle}}
\newcommand{\trans}[1]{\rightarrow_{#1}}
\newcommand{\PHdep}[2]{\f{dep}^{#1}(#2)}
\newcommand{\PHflat}{\f{flat}}

\usepackage{ifthen}
\usepackage{tikz}
\usetikzlibrary{arrows,shapes}

\definecolor{lightgray}{rgb}{0.8,0.8,0.8}
\definecolor{lightgrey}{rgb}{0.8,0.8,0.8}

\tikzstyle{boxed ph}=[]
\tikzstyle{sort}=[fill=lightgray,rounded corners]
\tikzstyle{process}=[circle,draw,minimum size=15pt,fill=white,
font=\footnotesize,inner sep=1pt]
\tikzstyle{black process}=[process, fill=black,text=white, font=\bfseries]
\tikzstyle{gray process}=[process, draw=black, fill=lightgray]
\tikzstyle{current process}=[process, draw=black, fill=lightgray]
\tikzstyle{process box}=[white,draw=black,rounded corners]
\tikzstyle{tick label}=[font=\footnotesize]
\tikzstyle{tick}=[black,-]%,densely dotted]
\tikzstyle{hit}=[->,>=stealth']
\tikzstyle{selfhit}=[min distance=30pt,curve to]
\tikzstyle{bounce}=[densely dotted,>=stealth',->]
\tikzstyle{hl}=[font=\bfseries,very thick]
\tikzstyle{hl2}=[hl]
\tikzstyle{nohl}=[font=\normalfont,thin]

\tikzstyle{prio}=[draw,very thick,-stealth]
\tikzstyle{coopupdate}=[dashed]
\tikzstyle{superprio}=[draw,very thick,double,-stealth]

\tikzstyle{apdot}=[circle, fill=black, inner sep=1.2pt, draw=transparent]
\tikzstyle{apdotsimple}=[]
\tikzstyle{aan}=[apdotsimple/.style={apdot}]

\newcommand{\currentScope}{}
\newcommand{\currentSort}{}
\newcommand{\currentSortLabel}{}
\newcommand{\currentAlign}{}
\newcommand{\currentSize}{}

\newcounter{la}
\newcommand{\TSetSortLabel}[2]{
  \expandafter\repcommand\expandafter{\csname TUserSort@#1\endcsname}{#2}
}
\newcommand{\TSort}[4]{
  \renewcommand{\currentScope}{#1}
  \renewcommand{\currentSort}{#2}
  \renewcommand{\currentSize}{#3}
  \renewcommand{\currentAlign}{#4}
  \ifcsname TUserSort@\currentSort\endcsname
    \renewcommand{\currentSortLabel}{\csname TUserSort@\currentSort\endcsname}
  \else
    \renewcommand{\currentSortLabel}{\currentSort}
  \fi
  \begin{scope}[shift={\currentScope}]
  \ifthenelse{\equal{\currentAlign}{l}}{
    \filldraw[process box] (-0.5,-0.5) rectangle (0.5,\currentSize-0.5);
    \node[sort] at (-0.2,\currentSize-0.4) {\currentSortLabel};
   }{\ifthenelse{\equal{\currentAlign}{r}}{
     \filldraw[process box] (-0.5,-0.5) rectangle (0.5,\currentSize-0.5);
     \node[sort] at (0.2,\currentSize-0.4) {\currentSortLabel};
   }{
    \filldraw[process box] (-0.5,-0.5) rectangle (\currentSize-0.5,0.5);
    \ifthenelse{\equal{\currentAlign}{t}}{
      \node[sort,anchor=east] at (-0.3,0.2) {\currentSortLabel};
    }{
      \node[sort] at (-0.6,-0.2) {\currentSortLabel};
    }
   }}
  \setcounter{la}{\currentSize}
  \addtocounter{la}{-1}
  \foreach \i in {0,...,\value{la}} {
    \TProc{\i}
  }
  \end{scope}
}

\newcommand{\TTickProc}[2]{ % pos, label
  \ifthenelse{\equal{\currentAlign}{l}}{
    \draw[tick] (-0.6,#1) -- (-0.4,#1);
    \node[tick label, anchor=east] at (-0.55,#1) {#2};
   }{\ifthenelse{\equal{\currentAlign}{r}}{
    \draw[tick] (0.6,#1) -- (0.4,#1);
    \node[tick label, anchor=west] at (0.55,#1) {#2};
   }{
    \ifthenelse{\equal{\currentAlign}{t}}{
      \draw[tick] (#1,0.6) -- (#1,0.4);
      \node[tick label, anchor=south] at (#1,0.55) {#2};
    }{
      \draw[tick] (#1,-0.6) -- (#1,-0.4);
      \node[tick label, anchor=north] at (#1,-0.55) {#2};
    }
   }}
}
\newcommand{\TSetTick}[3]{
  \expandafter\repcommand\expandafter{\csname TUserTick@#1_#2\endcsname}{#3}
}

\newcommand{\myProc}[3]{
  \ifcsname TUserTick@\currentSort_#1\endcsname
    \TTickProc{#1}{\csname TUserTick@\currentSort_#1\endcsname}
  \else
    \TTickProc{#1}{#1}
  \fi
  \ifthenelse{\equal{\currentAlign}{l}\or\equal{\currentAlign}{r}}{
    \node[#2] (\currentSort_#1) at (0,#1) {#3};
  }{
    \node[#2] (\currentSort_#1) at (#1,0) {#3};
  }
}
\newcommand{\TSetProcStyle}[2]{
  \expandafter\repcommand\expandafter{\csname TUserProcStyle@#1\endcsname}{#2}
}
\newcommand{\TProc}[1]{
  \ifcsname TUserProcStyle@\currentSort_#1\endcsname
    \myProc{#1}{\csname TUserProcStyle@\currentSort_#1\endcsname}{}
  \else
    \myProc{#1}{process}{}
  \fi
}

\newcommand{\repcommand}[2]{
  \providecommand{#1}{#2}
  \renewcommand{#1}{#2}
}
\newcommand{\THit}[5]{
  \path[hit] (#1) edge[#2] node[pos=.5,apdotsimple] {} (#3#4);
  \expandafter\repcommand\expandafter{\csname TBounce@#3@#5\endcsname}{#4}
}
\newcommand{\TBounce}[4]{
  (#1\csname TBounce@#1@#3\endcsname) edge[#2] (#3#4)
}

\newcommand{\TState}[1]{
  \foreach \proc in {#1} {
    \node[current process] (\proc) at (\proc.center) {};
  }
}



% ex : \TAction{c_1}{a_1.west}{a_0.north west}{}{right}
% #1 = frappeur
% #2 = cible
% #3 = bond
% #4 = style frappe
% #5 = style bond
\newcommand{\TAction}[5]{
  \THit{#1}{#4}{#2}{}{#3}
  \path[bounce, bend #5=50] \TBounce{#2}{}{#3}{};
}

\newcommand{\TCoopHit}[6]{
  \node[#2, apdot] at (#3) {};
  \foreach \proc in {#1} {
    \draw[#2,-] (#3) edge (\proc);
  }
  \path[hit] (#3) edge[#2] (#4#5);
  \expandafter\repcommand\expandafter{\csname TBounce@#4@#6\endcsname}{#5}
}



% ex : \TActionPlur{f_1, c_0}{a_0.west}{a_1.south west}{}{3.5,2.5}{left}
% #1 = frappeur
% #2 = cible
% #3 = bond
% #4 = style frappe
% #5 = coordonnées point central
% #6 = direction bond
\newcommand{\TActionPlur}[6]{
  \TCoopHit{#1}{#4}{#5}{#2}{}{#3}
  \path[bounce, bend #6=50] \TBounce{#2}{}{#3}{};
}

% procedure, abstractions and dependencies
\newcommand{\abstr}[1]{#1^\wedge}%\text{\textasciicircum}}
\def\BS{\mathbf{BS}}
\def\aBS{\abstr{\BS}}
\def\abeta{\abstr{\beta}}
\def\aZ{\abstr{\zeta}}
\def\aY{\abstr{\xi}}

\def\beforeproc{\vartriangleleft}

\def\powerset{\wp}

\def\Sce{\mathbf{Sce}}
\def\OS{\mathbf{OS}}
\def\Obj{\mathbf{Obj}}
%\def\Proc{\mathbf{Proc}}
%\def\Sol{\mathbf{Sol}}
\newcommand{\Sol}{\mathbf{Sol}}

\usepackage{galois}
\newcommand{\theOSabstr}{toOS}
\newcommand{\OSabstr}[1]{\theOSabstr(#1)}
\newcommand{\theOSconcr}{toSce}
\newcommand{\OSconcr}[1]{\theOSconcr(#1)}

% \def\gO{\mathbb{O}}
% \def\gS{\mathbb{S}}
\def\aS{\mathcal{A}}
\def\Req{\mathrm{Req}}
%\def\Sol{\mathrm{Sol}}
\def\Cont{\mathrm{Cont}}
\def\cBS{\BS_\ctx}
\def\caBS{\aBS_\ctx}
\def\caS{\aS_\ctx}
\def\cSol{\Sol_\ctx}
\def\cReq{\Req_\ctx}
\def\cCont{\Cont_\ctx}

\def\any{\star}

% \def\gProc{\mathrm{maxPROC}}
\def\mCtx{\mathrm{maxCtx}}

\def\procs{\f{procs}}
\def\objs{\f{objs}}
\def\sat#1{\lceil #1\rceil}

\def\gCont{\f{maxCont}}
\def\lCont{\f{minCont}}
\def\lProc{\f{minProc}}
\def\gProc{\f{maxProc}}

\def\join{\oplus}
\def\concat{\!::\!}
\def\emptyseq{\varepsilon}
\def\ltw{\preccurlyeq_{\OS}}
\def\indexes#1{\mathbb{I}^{#1}}
%\def\indexes#1{\{1..|#1|\}}
\def\supp{\f{support}}
\def\w{\omega}
\def\W{\Omega}
\def\ctx{\varsigma}
\def\Ctx{\mathbf{Ctx}}
\def\mconcr{\gamma}
\def\concr{\mconcr_\ctx}
\def\obj#1#2{{#1\!\Rsh^*\!\!#2}}
\def\objp#1#2#3{\obj{{#1}_{#2}}{{#1}_{#3}}}
\def\A{\mathcal{A}}
\def\cwA{\A_\ctx^\w}
\def\cwReq{\Req_\ctx^\w}
\def\cwSol{\Sol_\ctx^\w}
\def\cwCont{\Cont_\ctx^\w}
\def\gCtx{\f{maxCtx}}
\def\endCtx{\f{endCtx}}
\def\ceil{\f{end}}

%\def\lfp{\mathrm{lfp}\;}
%\def\mlfp#1{\mathrm{lfp}\{#1\}\;}
\newcommand{\lfp}[3]{\mathbf{lfp}\{#1\}\left(#2\mapsto#3\right)}
\def\maxobjs{{\f{maxobjs}}}
\def\maxprocs{{\f{maxprocs}_\ctx}}
\def\objends{{\f{ends}}}

\def\ra{\rho}
\def\rb{\rho^\wedge}
\def\rc{\widetilde{\rho}}
\def\interleave{\f{interleave}}

\def\join{\concat}

\tikzstyle{aS}=[every edge/.style={draw,->,>=stealth}]
\tikzstyle{Asol}=[draw,circle,minimum size=5pt,inner sep=0,node distance=1cm]
\tikzstyle{Aproc}=[draw,node distance=1.2cm]
\tikzstyle{Aobj}=[node distance=1.5cm]
\tikzstyle{Anos}=[font=\Large]
%\tikzstyle{AprocPrio}=[Aproc,double]
\tikzstyle{AsolPrio}=[Asol,double]



\def\procs{\mathsf{procs}}
%\def\allprocs{\mathsf{allProcs}}
\def\allprocs{\procs}
%\def\pfp{\mathsf{pfp}}
\def\pfp{\mathsf{lst}}
\def\pfpprocs{\mathsf{pfpProcs}}
\def\bounceprocs{\mathsf{bounceProcs}}
\def\newprocs{\mathsf{newProcs}}

\def\aB{\mathcal{B}}
\def\sat#1{\lceil #1\rceil}
\def\cwB{\sat{\aB_\ctx^\w}}
\def\mycwB#1#2{\sat{\aB_{#1}^{#2}}}
\def\Bsol{\sat{\Sol^\w_\ctx}}
\def\Breq{\sat{\Req^\w_\ctx}}
\def\Bcont{\sat{\Cont^\w_\ctx}}

\def\myB{\aB^\w_\ctx}
\def\mysol{\overline{\Sol^\w_\ctx}}
\def\myreq{\overline{\Req^\w_\ctx}}
\def\mycont{\overline{\Cont^\w_\ctx}}

\begin{comment}
\def\PrioCont{\textcolor{red}{\mathrm{PrioCont}}}
\def\mypriocont{\overline{\PrioCont^\w_\ctx}}
\def\cwPrioCont{\PrioCont_\ctx^\w}
\def\Bpriocont{\sat{\PrioCont^\w_\ctx}}
\def\Sat{\PrioCont}
\def\mysat{\overline{\Sat^\w_\ctx}}
\def\cwSat{\Sat_\ctx^\w}
\def\Bsat{\sat{\Sat^\w_\ctx}}

\def\ReqSolPrio{\textcolor{blue}{\mathrm{ReqSolPrio}}}
\def\RSP{\ReqSolPrio}
\def\myrsp{\overline{\RSP^\w_\ctx}}
\def\cwRSP{\RSP_\ctx^\w}
\def\Brsp{\sat{\RSP^\w_\ctx}}
\end{comment}

\newcommand{\csState}{\mathsf{procState}}

\newcommand{\V}{V}
\newcommand{\E}{E}
\newcommand{\cwV}{\V_\ctx^\w}
\newcommand{\cwE}{\E_\ctx^\w}
\newcommand{\VProc}{\V_\PHproc}
\newcommand{\VObj}{\V_\Obj}
%\newcommand{\VSol}{\V_{Sol}}
\newcommand{\VSol}{\V_{\Sol}}

\def\Bv{\sat{\cwV}}
\def\Be{\sat{\cwE}}
\def\BvProc{\textcolor{red}{\sat{\cwV}^\PHproc}}
\def\BvObj{\textcolor{red}{\sat{\cwV}^\Obj}}
%\def\BvSol{\sat{\cwV}^{Sol}}
\def\BvSol{\textcolor{red}{\sat{\cwV}^{\Sol}}}

\newcommand{\Bee}[2]{\Be^{#1}_{#2}}

%\def\mlfp#1{\f{pppf}\{#1\}}

\def\PHobjp#1#2#3{\PHobj{{#1}_{#2}}{{#1}_{#3}}}
\def\Obj{\mathbf{Obj}}
\def\powerset{\wp}
\def\gCont{\f{maxCont}}

\def\muconcr{\ell}
\def\uconcr{\muconcr_\ctx}

\begin{comment}
%\newcommand{\abstr}[1]{#1^\wedge}%\text{\textasciicircum}}
%\def\priomax{\mathsf{prio}_{max}}
\def\procs{\mathsf{procs}}
\def\allprocs{\mathsf{allProcs}}
\def\pfp{\mathsf{pfp}}
\def\pfpprocs{\mathsf{pfpProcs}}
%
\def\ctx{\varsigma}
\def\w{\omega}
%\def\aBS{\abstr{\BS}}
%
\def\Req{\mathrm{Req}}
\def\Sol{\mathrm{Sol}}
\def\Cont{\mathrm{Cont}}
\def\A{\mathcal{A}}
\def\cwA{\A_\ctx^\w}
\def\cwReq{\Req_\ctx^\w}
\def\cwSol{\Sol_\ctx^\w}
\def\cwCont{\Cont_\ctx^\w}
%
%
%
\end{comment}

% Macros
\newcommand{\sN}{\mathbb{N}}
\newcommand{\sNN}{\mathbb{N}^\bullet}
\def\DEF{\stackrel{\Delta}=}
\def\EQDEF{\stackrel{\Delta}\Leftrightarrow}
\newcommand{\segm}[2]{\llbracket #1; #2 \rrbracket}
\def\indexes#1{\mathbb{I}^{#1}}
\def\f#1{\mathsf{#1}}
\def\prio{\mathsf{prio}}
\newcommand{\bottom}{\perp}
\newcommand{\stable}{\mathsf{stable}}
\newcommand{\update}{\mathsf{update}}
\newcommand{\components}{\Gamma}
%\newcommand{\cs}{CS}
\newcommand{\cs}{\Delta}

% Restriction notation
\def\restriction#1#2{\mathchoice
              {\setbox1\hbox{${\displaystyle #1}_{\scriptstyle #2}$}
              \restrictionaux{#1}{#2}}
              {\setbox1\hbox{${\textstyle #1}_{\scriptstyle #2}$}
              \restrictionaux{#1}{#2}}
              {\setbox1\hbox{${\scriptstyle #1}_{\scriptscriptstyle #2}$}
              \restrictionaux{#1}{#2}}
              {\setbox1\hbox{${\scriptscriptstyle #1}_{\scriptscriptstyle #2}$}
              \restrictionaux{#1}{#2}}}
\def\restrictionaux#1#2{{#1\,\smash{\vrule height .8\ht1 depth .85\dp1}}_{\,#2}} 

%\renewcommand{\restriction}[2]{#1_{#2}}

% Commandes À FAIRE
\usepackage{color} % Couleurs du texte
%\definecolor{darkgreen}{rgb}{0,0.5,0}
%\newcommand{\towrite}[1]{\textcolor{darkgreen}{[#1]}}
\newcommand{\todo}[1]{\textcolor{red}{\textbf{[#1]}}}

% Ajouts et progression
\def\modMF#1{\textcolor{teal}{#1}}
\def\modLP#1{\textcolor{magenta}{#1}}
\def\modMM#1{\textcolor{blue}{#1}}
\def\modOR#1{\textcolor{olive}{#1}}
%\def\modMF#1{#1} \def\modLP#1{#1} \def\modMM#1{#1} \def\modOR#1{#1} 




% Special notations
\newcommand{\ie}{i.e.,\ }
\newcommand{\Ie}{I.e.,\ }
\newcommand{\eg}{e.g.,\ }
\newcommand{\Eg}{E.g.,\ }
\newcommand{\resp}{resp.\ }
\newcommand{\Resp}{Resp.\ }

% Césures
\hyphenation{pa-ra-me-tri-za-tion}
\hyphenation{pa-ra-me-tri-za-tions}



\def\lastname{Folschette \emph{et al.}}

\begin{document}
\begin{frontmatter}
\title{Sufficient Conditions for Reachability in Automata Networks with Priorities}

\author[irccyn]{Maxime Folschette}
\ead{Maxime.Folschette@irccyn.ec-nantes.fr}
\author[lri,amib]{Loïc Paulevé}
\author[irccyn]{Morgan Magnin}
\author[irccyn]{Olivier Roux}

\address[irccyn]{LUNAM Universit\'e, \'Ecole Centrale de Nantes, IRCCyN UMR CNRS 6597\\
(Institut de Recherche en Communications et Cybern\'etique de Nantes)\\
1 rue de la No\"e - B.P. 92101 - 44321 Nantes Cedex 3, France.}

\address[lri]{CNRS, Laboratoire de Recherche en Informatique (LRI)\\
		Université Paris-Sud - CNRS UMR 8623, France}
\address[amib]{AMIB group, Inria Saclay, France}{}



\begin{abstract}
The Process Hitting is a recently introduced framework designed for the modelling of concurrent systems.
Its originality lies in a compact representation of both components of the model and its corresponding actions:
each action can modify the status of a component, and is conditioned by the status of at most one other component.
This allowed to define very efficient static analysis based on local causality to compute reachability properties.
However, in the case of cooperations between components (for example, when two components are supposed to interact with a third one only when they are in a given configuration), the approach leads to an over-approximated interleaving between actions, because of the pure asynchronous semantics of the model.

To address this issue, we propose an extended definition of the framework, including priority classes for actions.
In this paper, we focus on a restriction of the Process Hitting with two classes of priorities and a specific behaviour of the components, that is sufficient to tackle the aforementioned problem of cooperations.
We develop a new refinement for the under-approximation of the static analysis to give accurate results for this class of Process Hitting models.
Then we show that this class of models is sufficient to represent any
Process Hitting model with an arbitrary number of classes of priorities,
and even any Asynchronous Discrete Networks, either Boolean or multivalued.
Our method thus allows to efficiently under-approximate reachability properties
in Asynchronous Discrete Networks,
as it is illustrated on the model-checking of a signalling network of 94
components, which is unprecedented.
\end{abstract}
\begin{keyword}
discrete networks \sep
abstract interpretation \sep
reachability \sep
qualitative models \sep
systems biology
\end{keyword}
\end{frontmatter}



\section{Introduction}
\label{sec:intro}

Discrete modeling frameworks for biological networks is an active research field where formal methods have proved that they were very powerful.
Such a work started in the seventies.
It was later enriched in many directions and widely used to elucidate many biological questions.
Among these questions, a major one is to understand precisely how biological systems evolve and behave; why and how they change their usual behaviors...
This leads to questions about the accessibility (possible or inevitable) of certain states.
The ultimate goal is to discover how it could be possible to prevent from reaching some pathological states.

Of course, such formal models on which analyses are performed are abstract representations of the actual studied systems.
They are associated with parameters that have to be synthesized so as to be as much as possible fitting with the real systems having some observed behaviors.
As a matter of fact, the abstractions we get are more or less rough or accurate.
Usual formal frameworks for such modeling activities are state-transition systems or Petri nets or process algebras.
We developed a quite similar framework named the Process Hitting \cite{PMR10-TCSB}, with the aim of avoiding to build the whole state space so as to be able to tackle very large systems (which would have led to a huge number of states, hopelessly far to big to be analyzed).

Besides, one further objective of our work is now to be more accurate in the description of the dynamics of the studied systems.
The idea for this is to introduce timing features of a system into its model.
Indeed, we are interested in taking into account some knowledge about the relative length of some phenomena as it is a way to refute some kinds of models (parameters) non convenient with the observed dynamic behaviors.
In this paper, we are introducing these timing properties through priorities.
It is based on the simple founding idea that the highest priority actions have to be processed before the other ones.

Until now, such a priority scheduling of the actions was not studied very much in the different formal modeling frameworks of systems biology.
Nevertheless, such an attempt has been carried out for Petri nets by F. Bause \cite{Bause97} and the concept of priority relations among the transitions of a network has also more recently been introduced by A. K. Wagler \textit{et al.} in \cite{waw,WaglerW12} in order to allow the modelisation of deterministic systems for biological applications.

Our paper is organized as follows.
The Process Hitting framework is defined in section 2;
we introduce static analysis of the PH in section 3;
section 4 illustrates the approach on an example
before the discussion and conclusion in section 5.

\paragraph*{Notations}
We denote: $\segm{a}{b} = \{ a, a+1, \dots, b-1, b \}$.
If $A$ is a finite set,
$|A|$ is the cardinality of $A$
and $\powerset(A)$ is the power set of $A$.
If $x = (x_i)_{i \in \segm{a}{b}}$ is a sequence of elements indexed by $i \in \segm{a}{b}$,
we denote $|x| = (b-a)+1$ the size of this sequence
and $\indexes{x} = \segm{1}{|x|}$ its set of indexes.
We also denote by $\emptyseq$ the empty sequence.
Finally, if $A$ and $B$ are sets,
$f : A \rightarrow B$ denotes an application $f$ that maps elements of $A$ to elements of $B$.


\section{The Process Hitting Framework}

We give in this section the definition and the semantics of the Process Hitting (PH) with priorities, which is an extension of the basic semantics given in~\cite{PMR10-TCSB}.
Then we describe the modelling of cooperation between components and discuss how the new aforementioned semantics makes this modelling more accurate.
Finally, in order to perform a static analysis adapted to this new semantics, we give several criteria to restrict the class of models that we can study,
and give several lemmas that follow.
This class of models is equivalent to Asynchronous Discrete Networks.

\subsection{Definition of the Process Hitting with $k$ classes of priorities}
\label{ssec:PH}
A PH with $k$ classes of priorities (\pref{def:ph}), also simply called “PH” in the following when it is not ambiguous, gathers a finite number of concurrent \emph{processes} divided into a finite set of \emph{sorts}.
A process belongs to a unique sort and is noted $a_i$ where $a$ is the sort and $i$ the identifier of the process within the sort $a$.
Each process stands for a kind of “activity level” of its sort; a state of the PH thus corresponds to a set of processes containing exactly one process of each sort.

The concurrent interactions between processes are defined by a set of \emph{actions} divided into classes of priorities.
Actions describe the replacement of a process by another of the same sort conditioned by the presence of at most one other process and by the fact that no other action of higher priority can be played in the considered state of the PH.
An action is denoted by $\PHhit{a_i}{b_j}{b_k}$ where $a_i,b_j,b_k$ are processes of sorts $a$ and $b$.
It is required that $b_j \neq b_k$ and that $a=b\Rightarrow a_i=b_j$.
An action $h=\PHfrappe{a_i}{b_j}{b_k}$ is read as ``$a_i$ \emph{hits} $b_j$ to make it bounce to $b_k$'', and $a_i,b_j,b_k$ are called respectively \emph{hitter}, \emph{target} and \emph{bounce} of the action, and can be referred to as $\PHhitter(h), \PHtarget(h), \PHbounce(h)$, respectively.

\begin{definition}[Process Hitting with $k$ classes of priorities]
\label{def:ph}
  If $k \in \sNN$, a \emph{Process Hitting with $k$ classes of priorities} is a triplet $\PH = (\PHs; \PHl; \PHa^{\langle k \rangle})$,
  where $\PHa^{\langle k \rangle} = (\PHa^{(1)}; \dots; \PHa^{(k)})$ is a $k$-tuple and:
  \begin{itemize}
    \item $\PHs \DEF \{a, b, \dots\}$ is the finite set of \emph{sorts},
    \item $\PHl \DEF \underset{a \in \PHs}{\times} \PHl_a$ is the finite set of \emph{states}, where $\PHl_a = \{a_0, \ldots, a_{l_a}\}$ is the finite set of \emph{processes} of sort $a \in \PHs$ and $l_a \in \sN^*$. Each process belongs to a unique sort: $\forall (a_i; b_j) \in \PHl_a \times \PHl_b, a \neq b \Rightarrow a_i \neq b_j$,
    \item $\forall n \in \llbracket 1; k \rrbracket, \PHa^{(n)} \DEF \{\PHfrappe{a_i}{b_j}{b_l} \mid (a; b) \in \PHs^2 \wedge (a_i; b_j; b_l) \in \PHl_a \times \PHl_b \times \PHl_b \wedge b_j \neq b_l \wedge a = b \Rightarrow a_i = b_j \}$ is the finite set of \emph{actions of priority $n$}.
  \end{itemize}
  We call $\PHproc \DEF \bigcup_{a \in \PHs} \PHl_a$ the set of all processes, and $\PHh \DEF \bigcup_{n \in \segm{1}{k}} \PHh^{(n)}$ the set of all actions.
\end{definition}
\noindent
The sort of a process $a_i$ is referred to as $\PHsort(a_i) = a$.
Given a state $s\in \PHl$, the process of sort $a \in \PHs$ present in $s$ is denoted by $\PHget{s}{a}$, that is the $a$-coordinate of the state $s$.
If $a_i \in \PHl_a$, we define the notation $a_i \in s \EQDEF \PHget{s}{a} = a_i$.
The override of a state $s$ by a process $a_i$ is defined in \pref{def:statecap} as the same state in which the process of sort $a$ has been replace by $a_i$,
which then allows to define the dynamics of a PH in \pref{def:play}.
\begin{definition}[$\Cap : \PHl \times \PHproc \rightarrow \PHl$]
\label{def:statecap}
  Given a state $s \in \PHl$ and a process $a_i \in \PHproc$, $(s \Cap a_i)$ is the state defined by:
  $\PHget{(s \Cap a_i)}{a} = a_i \wedge \forall b \neq a, \PHget{(s \Cap a_i)}{b} = \PHget{s}{b}$.
  We also extend this definition to a set of processes $ps$ given that all processes are from different sorts by the override of each process:
  $\forall as \subseteq \PHs, \forall ps \in \underset{a \in as}{\times} \PHl_a, s \Cap ps = s \underset{a_i \in ps}{\Cap} a_i$.
\end{definition}
\begin{definition}[Dynamics of a PH ($\PHPtrans$)]
\label{def:play}
  An action $h = \PHhit{a_i}{b_j}{b_k} \in \PHa^{(n)}$ of priority $n$ is \emph{playable} in $s \in \PHl$
  if and only if $\PHget{s}{a} = a_i$, $\PHget{s}{b} = b_j$ and $\forall m < n, \forall g \in \PHa^{(m)}, \PHhitter(g) \notin s \vee \PHtarget(g) \notin s$.
  In such a case, $(s \PHplay h)$ stands for the state resulting from the play of the action $h$ in $s$ and is defined by: $(s \PHplay h) = s \Cap \PHbounce(h)$.
  Moreover, we denote: $s \PHPtrans (s \PHplay h)$.

  If $s \in \PHl$,
  a \emph{scenario} $\delta$ from $s$ is a sequence of actions of $\PHh$ that can be played successively in $s$.
  The set of all scenarios from $s$ is noted $\Sce(s)$.
\end{definition}

\pref{def:substate} defines the notion of sub-state on a set of sorts,
that is used to consider the interesting part of a complete state.
\begin{definition}[Sub-states ($\PHsublize{\PHl}$)]
\label{def:substate}
  If $S \subset \PHs$ is a set of sorts, a sub-set on $S$ is an element of:
  $\PHsubl[\PHl]_S \DEF \bigtimes{a \in S} \PHl_a$.
  The set of all sub-sets is:
  $\PHsubl[\PHl] \DEF \bigcup_{S \in\powerset(\PHs)} \PHsubl[\PHl]_S$.
  
  \noindent
  Furthermore, if $\mysigma \in \PHsubl[\PHl]$ and $s \in \PHl$, we note:
    \[\mysigma \subseteq s \EQDEF \forall a_i \in \Proc, a_i \in \mysigma \Rightarrow a_i \in s \enspace.\]
\end{definition}
We note that a state is \textit{a fortiori} a sub-state: $\PHl \subset \PHsubl[\PHl]$.

In \pref{def:restriction}, we define the $n$-reduction of a given PH as the PH with $n$ classes of priorities in which only actions of priority lower or equal to $n$ are considered.
\begin{definition}[PH $n$-reduction]
\label{def:restriction}
  If $\PH = (\PHs; \PHl; \PHa^{\langle k \rangle})$ is a Process Hitting with $k$ classes of priorities and $n \in \segm{1}{k}$, we denote $\restriction{\PH}{n}$
  the $n$-reduction of $\PH$.
  $\restriction{\PH}{n} = (\PHs; \PHl; \PHa'^{\langle n \rangle})$ is a PH with $n$ classes of priorities with:
  $$\PHa'^{\langle n \rangle} = (\PHa^{(1)}; \dots; \PHa^{(n)})$$
  Furthermore, we denote: $\restriction{\Sce}{n}(s)$ the set of scenarios from $s$ in $\restriction{\PH}{n}$.
\end{definition}



\begin{example}
  \pref{fig:ph-livelock} gives an example of PH with $2$ classes of priorities where:
  \begin{align*}
    \PHs &= \{ a, b, c, ab \} \enspace, \\
    \PHl_a &= \{ a_0, a_1 \} \enspace, & \PHl_b &= \{ b_0, b_1 \} \enspace, \\
    \PHl_c &= \{ c_0, c_1 \} \enspace, & \PHl_{ab} &= \{ ab_{00}, ab_{01}, ab_{10}, ab_{11} \} \enspace.
  \end{align*}
  There also is especially: $\{ \PHhit{ab_{11}}{c_0}{c_1}, \PHhit{a_1}{a_1}{a_0}, \PHhit{a_0}{b_0}{b_1} \} \subseteq \PHh^{(2)}$.

\begin{figure}[tb]
  \centering
  \scalebox{1.2}{
  \begin{tikzpicture}
    \TSort{(0,0)}{a}{2}{l}
    \TSort{(0,4)}{b}{2}{l}
    \TSort{(7,2.5)}{c}{2}{r}

    \TSetTick{ab}{0}{00}
    \TSetTick{ab}{1}{01}
    \TSetTick{ab}{2}{10}
    \TSetTick{ab}{3}{11}
    \TSort{(2,2.5)}{ab}{4}{t}

    \THit{a_0}{prio}{ab_3}{.south}{ab_1}
    \THit{a_0}{prio}{ab_2}{.south}{ab_0}
    \THit{a_1}{prio}{ab_1}{.south}{ab_3}
    \THit{a_1}{prio}{ab_0}{.south}{ab_2}

    \THit{b_0}{prio}{ab_3}{.north}{ab_2}
    \THit{b_0}{prio}{ab_1}{.north}{ab_0}
    \THit{b_1}{prio}{ab_2}{.north}{ab_3}
    \THit{b_1}{prio}{ab_0}{.north}{ab_1}
    
    \THit{a_1}{selfhit}{a_1}{.west}{a_0}
    \THit{b_1}{selfhit}{b_1}{.west}{b_0}
    \THit{a_0.north}{bend left}{b_0}{.west}{b_1}
    \THit{b_0.south}{bend right=60}{a_0}{.west}{a_1}

    \THit{ab_3}{}{c_0}{.west}{c_1}

	\path[bounce, bend right=55]
      \TBounce{ab_0}{}{ab_2}{.west}
      \TBounce{ab_1}{}{ab_3}{.west}
	;
	\path[bounce, bend left=20]
      \TBounce{ab_3}{}{ab_1}{.south east}
      \TBounce{ab_2}{}{ab_0}{.south east}
	;
    \path[bounce, bend right=20]
      \TBounce{ab_3}{}{ab_2}{.north east}
      \TBounce{ab_1}{}{ab_0}{.north east}
    ;
    \path[bounce, bend left=30]
      \TBounce{ab_2}{}{ab_3}{.west}
      \TBounce{ab_0}{}{ab_1}{.west}
    ;
    \path[bounce, bend right]
      \TBounce{a_1}{}{a_0}{.north west}
      \TBounce{b_1}{}{b_0}{.north west}
    ;
    \path[bounce, bend left]
      \TBounce{a_0}{}{a_1}{.south west}
      \TBounce{b_0}{}{b_1}{.south west}
    ;
    \path[bounce, bend left]
      \TBounce{c_0}{}{c_1}{.south west}
    ;
    \TState{a_1, b_0, ab_2, c_0}
  \end{tikzpicture}
  }
  \caption{
  \label{fig:ph-livelock}
    An example of PH with $2$ classes of priorities.
    Sorts are represented as labelled boxes and processes as circles with their identifier on the side.
    Actions of $\PHh^{(1)}$ are represented by thick arrows and actions of $\PHh^{(2)}$ are represented by single arrows;
    the hit part of each action in drawn in plain line and the bounce part is in dotted line.
    Greyed processes stand for a possible state $s = \PHstate{a_1, b_0, c_0, ab_{10}}$.
  }
\end{figure}

\end{example}



\subsection{Modelling cooperation}
\label{ssec:cooperation}

Cooperation between processes to make another process bounce can be expressed in PH by building a \emph{cooperative sort}, as described in \cite{PMR10-TCSB}.
\pref{fig:ph-livelock} shows an example of cooperation between processes $a_1$ and $b_1$ to make $c_0$ bounce to $c_1$:
a cooperative sort $ab$ is defined with 4 processes (one for each sub-state of the presence of processes $a_1$ and $b_1$).
For the sake of clarity, the processes of $ab$ are indexed using the sub-state they represent.
Hence, $ab_{10}$ represents the sub-state $\PHstate{a_1,b_0}$, and so on.
Each process of sort $a$ and $b$ hit $ab$ to make it bounce to the process reflecting the status of the sorts $a$ and $b$
(\eg $\PHfrappe{a_1}{ab_{00}}{ab_{10}}$ and $\PHfrappe{a_1}{ab_{01}}{ab_{11}}$).
Then, to represent the cooperation between $a_1$ and $b_1$, the process $ab_{11}$ hits $c_0$ to make it bounce to $c_1$ instead of independent hits from $a_1$ and $b_1$.

We note that cooperative sorts are standard PH sorts and do not involve any
special treatment regarding the semantics of related actions.
Furthermore, it is possible to “factorise” cooperative sorts in order to decrease the number of processes created within each cooperative sort.
For example, if three processes $x_1$, $y_1$ and $z_1$ cooperate,
it is preferable to create a cooperative sort $xy$ with 4 processes to state the presence of $x_1$ and $y_1$
and a second cooperative sort $xyz$ with 4 processes to state the presence of $xy_{11}$ and $z_1$,
rather than a unique cooperative sort with 8 processes stating the presence of $x_1$, $y_1$ and $z_1$.
This “factorisation” allows to prevent the combinatorial explosion of the number of processes in cooperative sorts,
especially for cooperations between more than three processes.
It may have computational consequences as the static analysis method developed in~\pref{sec:sa} does not suffer from the number of sorts but from the number of processes in each sort.

The construction of cooperation in PH allows to encode any Boolean function between cooperating processes \cite{PMR10-TCSB}.
Due to the introduction of priorities into the PH framework,
it is possible to build cooperations with no temporal shift by defining actions updating the cooperative sorts with the highest class of priority.
This allows to gain the same expressivity in PH than in Boolean networks, as stated in \pref{ssec:hypothesis}.
The aim of this paper is to allow the static analysis of the dynamics to be handled on PH models comprising such higher priority actions updating cooperative sorts.



\subsection{Restrictions}
\label{ssec:hypothesis}

In the scope of this paper, we focus on a specific class of PH models
in which actions of $\PHh^{(1)}$ are called \emph{prioritized actions} and allow to model non-biological actions.
%and show that they are equivalent to discrete networks.
We consider in this section a PH model with $k$ classes of priorities: $\PH = (\PHs; \PHl; \PHa^{\langle k \rangle})$, with $k \in \sNN$
and we define here the restrictions that lead to this class of models.

%\pref{cr:2prio} allows to distinguish two kinds of actions:
%\emph{unprioritised actions} modelling the non-determinacy of biological evolutions
%and \emph{prioritised actions} used to model non-biological behaviours in the model, namely the update of cooperative sorts.
\pref{cr:bounded} states that the dynamics of the studied model $\PH$ contains no infinite sequence of prioritised actions.
As these actions can be considered as non-biological and therefore instantaneous, we thus prevent the existence of any Zeno-like behaviour
which would allow the play of an infinite sequence of prioritised actions in “zero time”.
As a consequence, the set $\restriction{\Sce}{1}$ is finite.
%\begin{condition}[2 classes of priorities]
%\label{cr:2prio}
%  In this paper, we only consider Process Hitting with $2$ classes of priorities:
%  $\PH = (\PHs; \PHl; \PHa^{\langle 2 \rangle})$.
%\end{condition}
%
\begin{condition}[Bounded termination]
\label{cr:bounded}
  The dynamics of $\restriction{\PH}{1}$ contains no cycles:
  $\exists N \in \sN, \forall s \in \PHl, \forall \delta \in \restriction{\Sce}{1}(s), |\delta| \leq N$.
\end{condition}

Given a sort $a\in\Sigma$ and a state $s\in L$, 
we denote $\pfp_s(a)$ (\pref{def:pfp}) the processes of sort $a$ that can be present after
playing all actions of priority $1$,
which is always defined because of \pref{cr:bounded}.

\begin{definition}[$\pfp : \PHl \times \PHs \rightarrow \powerset(\PHproc)$]
\label{def:pfp}
  For all $s \in \PHl$ and $a \in \PHs$,
  \begin{align*}
    \pfp_s(a) = \{ \get{(s\play\delta)}{a} \in \PHl_a &\mid \delta \in \restriction{\Sce}{1}(s)
					\wedge\nexists h\in\PHh^{(1)}, (\delta; h) \in\restriction{\Sce}{1}(s) \}
  \end{align*}
\end{definition}

Our second condition (\pref{cr:compcs}) imposes that any sort in the Process
Hitting is either a well-formed component (\pref{def:component}) or a well-formed cooperative sort
(\pref{def:cs}).
The former is a sort that is hit only by actions with priority greater than $2$.
The latter is a sort that is hit only by actions with priority $1$ and which
always converge to the same process with respect to the state of the components
that have an influence on it (\pref{def:conn}).
We note $\components\subset\Sigma$ the set of well-formed components,
and $\cs\subset\Sigma$ the set of well-formed cooperative sorts.

\begin{definition}[Well-formed component ($\components$)]
\label{def:component}
A sort $a \in \PHs$ is a \emph{well-formed component} if and only if:
    $\forall h \in \PHh, \PHsort(\PHtarget(h)) = a \Rightarrow \prio(h) \geq 2$ \enspace.
\end{definition}

\begin{definition}[Components influence $(\compin: \PHs\to\powerset(\components))$]
\label{def:conn}
Given a sort $a$, $\compin(a)\DEF \conn(a)\cap\components$ where
$\conn(a)$ is the smallest set of sorts satisfying the following properties:
\begin{align*}
a\in\conn(a) &
\\
\forall h\in  \PHh^{(1)},
	\Sigma(\target{h})\in\conn(a) & \Longrightarrow \Sigma(\hitter{h})\in\conn(a)
\end{align*}
\end{definition}

\begin{definition}[Well-formed cooperative sorts ($\cs$)]
\label{def:cs}
A sort $a \in \PHs$ is a \emph{well-formed cooperative} sort if and only if:
\begin{enumerate}[(i)]
\item $\forall h\in\PHh, \Sigma(\PHtarget(h))=a \Longrightarrow \prio(h) = 1$;
\item\label{csai} $\forall s\in\PHl, \card{\pfp_s(a)} = 1$;
\item\label{css} $\forall a_i \in \PHl_a, \exists s \in \PHl, a_i\in\pfp_s(a)$;
\item $\forall \sigma\in\PHsubl[\PHl]_{\compin(a)},
			\forall s,s'\in\PHl,
				\sigma\subseteq s\wedge \sigma\subseteq s'\Rightarrow 
							\pfp_s(a) = \pfp_{s'}(a)$\enspace.
\end{enumerate}
\end{definition}

\begin{condition}[Components \& cooperative sorts partition]
\label{cr:compcs}
  $$\PHs = \components \cup \cs \wedge \components \cap \cs = \emptyset$$
\end{condition}

Because of \pref{def:cs}(\ref{csai}), we denote in the following: $\pfp_s(a) = a_i$.
Furthermore, because of \pref{def:cs}(\ref{css}), we denote $\csState(a_i)$ the set of sub-states represented by the process $a_i$ of any cooperative sort $a$ (\pref{def:csState}).
\begin{definition}[$\csState : \PHproc \rightarrow \powerset(\PHproc)$]
\label{def:csState}
  If $a \in \cs$ and $a_i \in \PHl_a$, 
  $$\csState(a_i) \DEF \{ \toset{ps} \mid ps\in\PHsubl[\PHl]_{\compin(a)} \wedge
  							\exists s\in\PHl, (ps\subset s\wedge \pfp_s(a)=a_i)
							\}$$
\end{definition}

In the following we simply write “component” (\resp “cooperative sort”) instead of “well-formed component” (\resp “well-formed cooperative sort”).

\begin{example}
  The PH in \pref{fig:ph-livelock} contains three components $a$, $b$ and $c$ and a cooperative sort $ab$ that models cooperation between sorts $a$ and $b$.
\end{example}


\subsection{Consequences of the restrictions}

In this subsection, we give several general lemmas that can be derived from the restrictions of \pref{ssec:hypothesis}, %(\pref{cr:bounded}, \ref{cr:cyclefreeness} \& \ref{cr:compcs})
in the special case of a PH model with $2$ classes of priorities $\PH = (\PHs; \PHl; \PHa^{\langle 2 \rangle})$.
These results will help building the static analysis of \pref{sec:sa}.

We first denote by $\update(s)$ the state equivalent to $s$ but in which all cooperative sorts are updated (\pref{def:update}).
This state is unique due to the properties of $\pfp$ given in the previous subsection.
Then, \pref{lem:update} states that from any state, there exists a scenario updating the cooperative sorts of this state.
%
\begin{definition}[$\update : \PHl \rightarrow \PHl$]
\label{def:update}
  For all $s \in \PHl$, we define:
  \begin{align*}
    \update(s) = s \Cap \{ \pfp_{s}(a) \mid a \in \cs \} \enspace.
  \end{align*}
\end{definition}
%
\begin{lemma}
\label{lem:update}
  $\forall s \in \PHl, \exists \delta \in \restriction{\Sce}{1}(s), s \PHplay \delta = \update(s)$
\end{lemma}
%
\begin{proof}
  Let $a$ be a cooperative sort so that $\PHget{s}{a} \neq \pfp_s(a)$.
  Given the definition of $\pfp_s(a)$, there exists a scenario $\delta$ updating $a$ in $s$ so that
  $\forall \delta' \in \restriction{\Sce}{1}(s \PHplay \delta)$, $\PHget{(s \PHplay \delta \PHplay \delta')}{a} = \pfp_s(a)$.
\end{proof}

\pref{lem:hcompcomp} states that for a given state $s$, and for any action $h = \PHhit{a_i}{b_j}{b_k}$ where $a$ and $b$ are components,
if $\PHget{s}{a} = a_i$ and $\PHget{s}{b} = b_j$, then
$h$ can always be played after a series of prioritised hits (and these hits do not prevent it to be fired).
\pref{lem:hcscomp} states the same if $a$ is a cooperative sort, under the condition that $a$ is updated in $s$.
\begin{lemma}
\label{lem:hcompcomp}
  $\forall s \in \PHl, \forall a,b \in \components, \forall h = \PHhit{a_i}{b_j}{b_k} \in \PHh,$\\
  $(\PHget{s}{a} = a_i \wedge \PHget{s}{b} = b_j) \Rightarrow (\exists \delta \in \restriction{\Sce}{1}(s),
  (s \PHplay \delta) \PHPtrans (s \PHplay \delta \PHplay h))$
\end{lemma}
%
\begin{proof}
  From \pref{lem:update}, there exists a scenario $\delta$ with: $(s \PHplay \delta) = \update(s)$.
  As $a,b \in \components$, $a_i \in (s \PHplay \delta)$ and $b_j \in (s \PHplay \delta)$.
  Finally, by definition of $\update(s)$, no prioritised action can be played in $(s \PHplay \delta)$, thus $h$ can be played in $(s \PHplay \delta)$.
\end{proof}
%
\begin{lemma}
\label{lem:hcscomp}
  $\forall s \in \PHl, \forall h = \PHhit{a_i}{b_j}{b_k} \in \PHh, a \in \cs, b \in \components,$\\
  $(\PHget{s}{a} = a_i \wedge \PHget{s}{b} = b_j \wedge \pfp_s(a) = a_i) \Rightarrow (\exists \delta \in \restriction{\Sce}{1}(s),
  (s \PHplay \delta) \PHPtrans (s \PHplay \delta \PHplay h))$
\end{lemma}
\begin{proof}
  Similar to the proof of \pref{lem:hcompcomp};
  as $a_i \in \pfp_s(a)$, $a_i \in (s \PHplay \delta)$.
\end{proof}


% vi:spell spelllang=en:
\section{Static Analysis}\label{sec:sa}

\todo{Replacer notre méthode par rapport aux méthodes d'interprétation abstraite en général}

The aim of this section is to define the problem of reachability in an AAN,
and propose an under-approximation allowing to efficiently solve it.
The static analysis presented here is inspired from~\cite{PMR12-MSCS}.
We consider in this section an AAN $\PH = (\PHs; \PHl; \PHa)$.

\medskip

\modMF{
The main idea behind the static analysis presented in this section
is to abstract the dynamics of an AAN model
by a dynamics that is more general and easier to analyse.
For this, we focus on the notion of objective:
an objective is a couple of local states (\eg $\PHobj{a_i}{a_j}$) whose reachability
is required to ensure the global reachability property considered;
in other words, it is required that a set of actions hitting $a$ exists so that,
starting from a state containing $a_i$,
it is possible to play these actions (possibly intertwined with other actions)
and reach a state containing $a_j$.
Given the particular form of the actions in AANs,
the solving of such an objective raises new objectives in the general case.
%because each action can be triggered by (at most) one local state from a different automaton.
Indeed, each of these local states also have to be reached before the required actions are played.
However, the order of the considered actions is abstracted,
and so is the order of the related objectives,
thus resulting in an over-approximation of the requirements.
}



\subsection{Preliminary Definitions}
\label{ssec:sa-def}

The reachability of a local state $a_j$ of a given automaton $a$,
starting from another local state $a_i$,
is called an objective and is denoted $\PHobjp{a}{i}{j}$ (\pref{def:obj}).
\begin{definition}[Objective ($\Obj$) \& Objective Sequence ($\OS$)]
\label{def:obj}
  If $a \in \components$, the reachability of a local state $a_j$ from a local state $a_i$ is called an \emph{objective}, noted $\PHobj{a_i}{a_j}$.
  The set of all objectives is called $\Obj \DEF \{ \PHobj{a_i}{a_j} \mid a \in \components \wedge (a_i, a_j) \in \PHl_a \times \PHl_a \}$.
  For an objective $P = \PHobj{a_i}{a_j} \in \Obj$, we define: $\PHsort(P) \DEF
  a$, $\PHtarget(P)\DEF a_i$, $\PHbounce(P)\DEF a_j$,
  and $P$ is said \emph{trivial} if $a_i = a_j$.

  We define an \emph{objective sequence} as a sequence of objectives in which each objective target must be equal to the previous objective bounce of the same automaton, if it exists.
  The set of all objective sequences is denoted by $\OS$.
\end{definition}

A context (\pref{def:context}) extends the notion of state to a set of possible initial states:
\modMF{
to each automaton in the model, a context maps a set of local states in this automaton.
}
We also extend the override operator to contexts (\pref{def:ctxcap}).
\begin{definition}[Context ($\Ctx$)]
\label{def:context}
  A \emph{context} $\ctx$ associates to each automaton in $\PHs$ a non-empty subset of its local states:
  $\forall a \in \PHs, \PHget{\ctx}{a} \subseteq \PHl_a \wedge \PHget{\ctx}{a} \neq \emptyset$.
  $\Ctx$ is the set of all contexts.
\end{definition}
%
\begin{definition}[$\Cap: \Ctx \times \powerset(\PHproc) \rightarrow \Ctx$]
\label{def:ctxcap}
  For any $\ctx\in\Ctx$ and set of local states $ps \in \powerset(\PHproc)$,
  the override of $\ctx$ by $ps$ is noted $\ctx \Cap ps$ and is defined by:
  \[ \forall a \in \PHs, \PHget{(\ctx \Cap ps)}{a} \DEF
  \begin{cases}
    \{ p \in ps \mid \PHsort(p)=a \} & \text{if } \exists p \in ps, \PHsort(p)=a,\\
    \PHget{\ctx}{a} & \text{else.}
  \end{cases}
  \]
\end{definition}
\noindent
For a given context $\ctx$, we note $a_i \in \ctx$ if and only if $a_i \in \PHget{\ctx}{a}$,
and for all $ps \in \powerset(\PHproc)$ or $ps \in \PHl$, $ps \subseteq \ctx \EQDEF \forall a_i \in ps, a_i \in \ctx$.
A sequence of actions $\delta$ is \emph{playable} in a context $\ctx$ if and only if 
$\exists s \subseteq \ctx, \delta \in \Sce(s)$.
We denote then: $\delta \in \Sce(\ctx)$,
and the play of $\delta$ in $\ctx$ is $\ctx \PHplay \delta = \ctx \Cap \ceil(\delta)$,
where $\ceil(\delta)$ is the set containing the last local state in the sequence $\delta$ (hitter or bounce) of every automaton mentioned in $\delta$.

Finally, a bounce sequence on a automaton $a$ (\pref{def:bs}) is a sequence of actions hitting $a$
in which the bounce of each action equals the hitter of the following action.
Bounce sequences are used to find local solutions to a given objective;
\modMF{
indeed, the bounce sequences solving an objective $\PHobj{a_i}{a_j}$
only depend on the bounce part of the actions hitting the automaton $a$
and not on the hitters.
}%
Then, a bounce sequence on $a$ can be abstracted into
its combinations of hitters outside automaton $a$ (\pref{def:aBS});
% the minimal sets of its sets of hitters that are not in automaton $a$ (\pref{def:aBS}).
This abstraction is later used to propagate an objective on a given automaton
by creating new objectives on other automata,
by considering the sets of hitters required by the actions involved.

\begin{definition}[Bounce sequence ($\BS$)]
\label{def:bs}
  A \emph{bounce sequence} $\zeta$ is a sequence of actions so that $\forall n \in \indexes{\zeta}, n < |\zeta|, \PHbounce(\zeta_{n}) = \PHtarget(\zeta_{n+1})$.
  $\BS$ denotes the set of all bounce sequences, and
  $\BS(P)$ denotes the set of bounce sequences \emph{solving} an objective $P$:
  \[
    \BS(\PHobj{a_i}{a_j}) \DEF \{ \zeta \in \BS \mid \PHtarget(\zeta_1)= a_i \wedge \PHbounce(\zeta_{|\zeta|}) = a_j \} \enspace.
  \]
  Furthermore, $\BS(\obj{a_i}{a_j}) = \emptyset$ if there is no way to reach $a_j$ from $a_i$
  and $\emptyseq \in \BS(\obj{a_i}{a_i})$
  \modMF{
  for any trivial objective.
  }
\end{definition}

In the following, we denote: $\sSol = \powerset(\PHproc)$
\moda{%
and $\Sol = \powerset(\sSol)$.
}%
\modMF{%
The elements of $\sSol$ are sets of local states,
and will later represent sets of hitters of a given action
that are required to be active simultaneously in order to play this related action.
The elements in $\Sol$ are sets of sets of hitters,
and represent the requirements to play a whole bounce sequence.
}%

\begin{definition}[$\aBS:\Obj \rightarrow \powerset(\Sol)$]
\label{def:aBS}
  The \emph{abstractions of bounce sequences} of an objective $P$, denoted by the set $\aBS(P)$, are the minimal sets of hitters of the bounce sequences solving $P$:
  \[
    \aBS(P) \DEF \{ \abstr{\zeta} \in \Sol \mid
      \zeta \in \BS(P) \wedge
      \nexists \zeta' \in \BS(P), \abstr{\zeta'} \subsetneq \abstr{\zeta} \} \enspace,
  \]
  \modMF{
  where $\forall P \in \Obj, \forall \zeta \in \BS(P)$:
  $\abstr{\zeta} \DEF \{ \PHhitter(\zeta_n) \mid n \in \indexes{\zeta} \} \setminus \emptyset$.
  }%
  
  \noindent
  \modMF{
  As a consequence,
  $\aBS(\obj{a_i}{a_j}) = \emptyset$ if $\BS(\obj{a_i}{a_j}) = \emptyset$,
  and $\BS(\obj{a_i}{a_i}) = \{ \emptyset \}$.
  }%
\end{definition}



\subsection{Under-approximation}
\label{ssec:ua}

We denote by $\concr(\w)$ (\pref{def:concr}) the set of scenarios concretising an objective sequence $\w$ in the context $\ctx$.
In \pref{def:uconcr}, we define $\uconcr(\w)$ as equal to $\concr(\w)$ if and only if $\concr(\w)$ contains scenarios starting from all states $s \subseteq \ctx$.
\pref{lem:uconcr-ctx} is used to over-approximate the initial context $\ctx$.

\begin{definition}[$\concr: \OS \to \powerset(\Sce)$]\label{def:concr}
Given $\w\in\OS$, $\concr(\w)$ is the set of minimal scenarios concretising $\w$ in the
context $\ctx$. It is defined as the largest set satisfying the following conditions:
\begin{enumerate}[(i)]
\item $\forall\delta\in\concr(\w), \modMF{\delta\in\Sce(\ctx)}$
\item $\forall\delta\in\concr(\w),
  \exists \phi:\indexes{\w}\to\indexes{\delta},
    (\forall n,m\in\indexes{\w}, n<m \Leftrightarrow \phi(n)\leq\phi(m)),$
\\\hspace*{2cm}$\forall n\in\indexes{\w}, \PHbounce(\w_n) \in \ctx\play\delta_{1..\phi(n)}$
\item $\forall\delta,\delta'\in\concr(\w)
				\card{\delta}\leq\card{\delta'} \Rightarrow
					\delta\neq\delta'_{1..\card{\delta}}$.
\end{enumerate}
\end{definition}

\noindent
The notation $\delta_{a..b}$ in the previous definition
denotes the subsequence of $\delta$ between indexes $a$ and $b$,
as defined on page \pageref{notations}.

\begin{definition}[$\uconcr: \OS \rightarrow \powerset(\Sce)$]
\label{def:uconcr}
  \[
  \modMF{\forall \w \in \OS,}
  \uconcr(\w) \DEF
  \begin{cases}
    \concr(\w) & \text{if } \forall s \in \PHl, s \subseteq \ctx, \exists \delta \in \concr(\w), \delta \in \Sce(s) \\
    \emptyset & \text{else.}
  \end{cases}
  \]
\end{definition}
% 
\begin{lemma}
\label{lem:uconcr-ctx}
  $\ctx \subseteq \ctx' \wedge \muconcr_{\ctx'}(\w) \neq \emptyset \Longrightarrow \muconcr_{\ctx}(\w) \neq \emptyset$.
\end{lemma}

For any objective $P$ and context $\ctx$, \pref{def:maxCont} gives the set of local states of automaton $\PHsort(P)$ that are required to solve $P$ in $\ctx$, denoted by $\gCont_\ctx(\PHsort(P), P)$.
\begin{definition}[$\gCont_\ctx : \Sigma \times \Obj \rightarrow \powerset(\PHproc)$]
  \label{def:maxCont}
  \begin{align*}
    \gCont_\ctx(a,P) \DEF
    \{ p \in \PHproc &\mid \exists ps \in \aBS(P), \exists b_i \in ps, b = a \wedge p = b_i \\
      & \vee b \neq a \wedge p \in \gCont_\ctx(a, \PHobj{b_j}{b_i}) \wedge b_j \in \PHget{\ctx}{b} \}
    \enspace.
  \end{align*}
\end{definition}

\modMF{
A graph of local causality $\myB = (\cwV, \cwE)$
is a graph where $\cwV \subseteq \Proc \cup \Obj \cup \sSol \cup \Sol$ is the set of vertices
and $\cwE \subseteq \cwV \times \cwV$ is the set of oriented edges.
Such a graph aims a describing the requirements of a given reachability
problem, that is, the reachability of $\w$ from the initial context $\ctx$.
Thus, a node in $\PHproc$ represents a local state required to play an action,
a node in $\Obj$ is an objective to reach a given local state,
a node in $\sSol$ is a set of hitters that have to be active simultaneously
to solve one step of a given objective,
and a node in $\Sol$ is a set of sets of hitters,
thus encompassing all requirement for the solving of a whole objective.
}

\modMF{
The edges of such a graph allow to draw links between these requirements.
Thus, an objective $P \in \Obj$ is solvable if
at least one related bounce sequence in $\BS(P)$ can be played;
therefore, we can consider all minimal sets of sets of hitters of these bounce sequences
(that is, the sets of $\aBS(P) \subseteq \Sol$, see \pref{def:aBS})
as requirements to solve this objective (\pref{eq:ESol1}).
Such a set can then naturally be split into as many sets of hitters (\pref{eq:ESol3})
which all have to be simultaneously active.
These sets of hitters can in turn be split into several local state nodes (\pref{eq:ESol2}).
This creates new requirements, as the reachability of a required local state $a_i$
is approximated by the ability to solve all objectives of the form
$\PHobjp{a}{j}{i} \in \Obj$ for all $a_j$ in the initial context (\pref{eq:EReq}).
Furthermore, we note that the solving of an objective $P$ may require
a local state of $\PHsort(P)$,
\ie $\gCont(\PHsort(P), P) \neq \emptyset$ (see \pref{def:maxCont});
in this case, $P$ is re-targeted (\pref{eq:ECont}).
Finally, \pref{eq:Vw} forces the graph to contain all the initial objectives,
and \pref{eq:VE} ensures the consistency of $\cwV$.
}

\modMF{
However, this graph $\myB$ alone is not sufficient to check the reachability of
an objective sequence $\w$ from an initial context $\ctx$;
indeed, the reachability of each intermediate required local state
is only checked starting from the context $\ctx$,
although this context may be modified by other objectives.
Therefore, such a reachability requires the building of the
\emph{saturated graph of local causality} (\pref{def:glc})
which is obtained by iteratively saturating the initial context with all
local states referenced in the graph,
until reaching a fixed point.
Such a fixed point always exists because the size of the context
is limited by the size of the initial PH model.
The set of local states contained in a graph of local causality
is given by:
\begin{align*}
  \allprocs(\cwV, \cwE) = (\cwV \cap \Proc) &\cup \{ \PHtarget(P) \mid P \in \cwV \cap \Obj \}\\
  & \cup \{ P \neq \w \Rightarrow \PHbounce(P) \mid P \in \cwV \cap \Obj \}
\end{align*}
where, if the objective sequence $\w$ consists of only one element,
we can omit the hitter of this objective
provided that it does not appear elsewhere in the graph.
}

\begin{definition}
\label{def:glc}
  The \emph{\modMF{saturated} graph of local causality} $\cwB \DEF (\Bv, \Be)$ is defined as:
  $\cwB \DEF \lfp{\aB^\w_\ctx}{\myB}{\aB^\w_{\ctx \Cap \allprocs(\myB)}}$,
  where $\myB \DEF (\cwV, \cwE)$ is the smallest graph with
  $\cwV \subseteq \Proc \cup \Obj \cup \sSol \cup \Sol$ and $\cwE \subseteq \cwV \times \cwV$
  so that:
  \begin{align}
    \modMF{\toset{\w}} &\subseteq \cwV \label{eq:Vw} \\
%    P \in \cwVObj &\Rightarrow \PHbounce(P) \in \cwV \label{eq:Vproc} \\
    (x, y) \in \cwE &\Rightarrow y \in \cwV \label{eq:VE} \\
    P \in \cwVObj \wedge pps \in \modMF{\aBS}(P) &\Rightarrow (P, pps) \in \cwE \label{eq:ESol1} \\
    \moda{pps \in \cwVSol \wedge A \in pps} &\moda{\Rightarrow (pps, A) \in \cwE} \label{eq:ESol3} \\
    A \in \cwVsSol \wedge a_i \in A &\Rightarrow (A, a_i) \in \cwE \label{eq:ESol2} \\
    a_i \in \cwVProc \wedge a_j \in \ctx &\Rightarrow (a_i, \PHobjp{a}{j}{i}) \in \cwE \label{eq:EReq} \\
%    a \in \cs \wedge a_i \in \cwVProc \wedge ps \in \csState(a_i) &\Rightarrow (a_i, ps) \in \cwE \label{eq:EPrio} \\
    P \in \cwVObj \wedge q \in \gCont_\ctx(\PHsort(P), P) &\Rightarrow (P, \PHobj{q}{\PHbounce(P)}) \in \cwE \label{eq:ECont} \!
  \end{align}
\end{definition}

\modMF{
In the saturated graph of local causality, a set of hitters $A \in \sSol$
is said to be \emph{coherent} (\pref{def:coherent})
if none of the local states in $A$ conflict with
a successor of $A$ in $\Bv \cap \Proc$,
that is, if there exists no local state node $a_j$ in the successors of $A$
and no element $a_i \in A$ so that $a_i \neq a_j$.
}
Then, \pref{th:approxinf} gives a sufficient condition for the concretisation
of an objective sequence in a given context,
which is derived immediately from the saturated graph of local causality,
and the condition that all sets of hitters are coherent.
A proof of this theorem is given in \pref{suppl:demoapproxinf}.
\begin{definition}[Coherent node]
\label{def:coherent}
  \modMF{
  In a given saturated graph of local causality $\cwB = (\Bv, \Be)$,
  a node $x \in \Bv$ is said to be \emph{coherent} if and only if:
  $x \in \Bv \cap \sSol \Rightarrow x$ has no successor $a_j \in \Bv \cap \Proc$
  so that $\exists a_i \in x$, $a_i \neq a_j$.
  }
\end{definition}

\begin{theorem}[Under-Approximation]
\label{th:approxinf}
  Given an AAN $(\PHs; \PHl; \PHa)$,
  a context $\ctx$ and an objective sequence $\w$,
  if the graph $\cwB$ contains no cycle,
  all objectives have at least one solution
  and all nodes are coherent,
  then $\uconcr(\w) \neq \emptyset$.
\end{theorem}

\modMF{
Regarding the complexity of the method,
computing the saturated graph of local causality is polynomial in the number of automata in $\PH$ and exponential in the number of local states in one automaton.
Checking the properties allowing to apply \pref{th:approxinf} is polynomial in the size of the graph.
Therefore, the building and checking process can be considered as polynomial in the size
of the AAN, provided that each automaton only contains a few local states.
We note that this is particularly true for biological models, where
each component usually contains a limited number of expression levels.
}

\modMF{
We note furthermore that in the case where
the method developed in this section is not conclusive,
it is possible to compute only a subset of $\Bv \cap \Sol$,
by removing some solutions from the initial graph of local causality $\myB$,
or by removing them from the saturated graph of local causality $\cwB$ and by trimming it.
In other words, this consists in ignoring some of the bounce sequences
for some of the objectives,
which intuitively cannot create false positives.
Indeed, \pref{th:approxinf} is then still valid on the partial graph obtained,
and this removal can lead to more conclusiveness by trimming parts of the graph
that were not necessary for the reachability
(especially cycles or unnecessary but inconclusive branches).
If one wants to try each possible subset of solutions,
then the overall method turns out to be exponential
in the number of solutions to each objective.
Our method can thus still be considered as efficient
compared to regular model-checking which is usually PSPACE-complete~\cite{Harel02}.
}



\begin{example}
  \moda{%
  If we confider the AAN of \pref{fig:ph-livelock},
  from the initial state $\PHstate{a_1, b_0, c_0}$ depicted,
  the under-approximation given in \pref{th:approxinf}
  does not conclude regarding the reachability of $c_1$.
  This is due to the fact that the node $\{ a_1, b_1 \} \in \Bv \cap \sSol$
  is not coherent because of its successor $a_0$ (and $b_0$).
  (However, from the inconclusiveness of \pref{th:approxinf},
  one cannot conclude about the unreachability of $c_1$.
  Such analysis should be driven for instance
  with over-approximation methods developed in~\cite{PMR12-MSCS}.)
  }%
  
  \moda{%
  This result is new compared to the method proposed in~\cite{PMR12-MSCS}.
  Indeed, the representation based on the Process Hitting that was proposed
  in this paper only allowed to represent “over-approximated” Boolean gates
  with the use of cooperative sorts,
  as explained in \pref{ssec:cooperation}.
  This especially did not allow to model the fact that $a_1$ and $b_1$ could not
  be activated in the same state, but only in successive states.
  Thus, $c_1$ was indeed reachable,
  contrary to the behaviour expected from an accurate Boolean gate.
  }%
  
  \moda{%
  Finally, we note however that,
  if $\PHhits{a_0}{b_0}{b_1}$ and $\PHhits{b_0}{a_0}{a_1}$ are replaced by the actions
  $\PHhitm{a_0}{a_1}$ and $\PHhitm{b_0}{b_1}$,
  then the resulting saturated graph of local causality changes, and
  \pref{th:approxinf} concludes that $c_1$ is reachable from $\ctx$.
  }%

\begin{figure}[tp]
  \centering
  \begin{tikzpicture}[aS]
%    \node[Aproc] (c1) {$c_1$};
%    \node[Aobj,below of=c1] (c01) {$\PHobj{c_0}{c_1}$};
    \node[Aobj] (c01) {$\PHobj{c_0}{c_1}$};
    \node[Asol,below of=c01] (c01s) {};

    \node[Assol,below of=c01s] (a1a1ss) {$\{ a_1, b_1 \}$};
    \node[Aproc,below left of=a1a1ss] (a1) {$a_1$};
    \node[Aobj,below of=a1] (a11) {$\PHobj{a_1}{a_1}$};
    \node[Asol,below of=a11] (a11s) {};
    \node[Assol,below of=a11s] (na11s) {$\emptyset$};
    \node[Aobj,below left of=a1] (a01) {$\PHobj{a_0}{a_1}$};
    \node[Asol,below of=a01] (a01s) {};
    \node[Aproc,below of=a01s] (b0) {$b_0$};
    \node[Aobj,below of=b0] (b00) {$\PHobj{b_0}{b_0}$};
    \node[Asol,below of=b00] (b00s) {};
    \node[Assol,below of=b00s] (nb00s) {$\emptyset$};
    \node[Aobj,below left of=b0] (b10) {$\PHobj{b_1}{b_0}$};
    \node[Asol,below of=b10] (b10s) {};
    \node[Assol,below of=b10s] (nb10s) {$\emptyset$};

    \node[Aproc,below right of=a1a1ss] (b1) {$b_1$};
    \node[Aobj,below of=b1] (b11) {$\PHobj{b_1}{b_1}$};
    \node[Asol,below of=b11] (b11s) {};
    \node[Assol,below of=b11s] (nb11s) {$\emptyset$};
    \node[Aobj,below right of=b1] (b01) {$\PHobj{b_0}{b_1}$};
    \node[Asol,below of=b01] (b01s) {};
    \node[Aproc,below of=b01s] (a0) {$a_0$};
    \node[Aobj,below of=a0] (a00) {$\PHobj{a_0}{a_0}$};
    \node[Asol,below of=a00] (a00s) {};
    \node[Assol,below of=a00s] (na00s) {$\emptyset$};
    \node[Aobj,below right of=a0] (a10) {$\PHobj{a_1}{a_0}$};
    \node[Asol,below of=a10] (a10s) {};
    \node[Assol,below of=a10s] (na10s) {$\emptyset$};

    \path
%    (c1) edge (c01)
    (c01) edge (c01s)
    (c01s) edge (a1a1ss)
    (a1a1ss) edge (a1) edge (b1)

    (a1) edge (a01) edge (a11)
    (a01) edge (a01s)
    (a01s) edge (b0)
    (a11) edge (a11s)
    (a11s) edge (na11s)
    (a0) edge (a10) edge (a00)
    (a10) edge (a10s)
    (a10s) edge (na10s)
    (a00) edge (a00s)
    (a00s) edge (na00s)

    (b0) edge (b10) edge (b00)
    (b10) edge (b10s)
    (b10s) edge (nb10s)
    (b00) edge (b00s)
    (b00s) edge (nb00s)
    (b1) edge (b01) edge (b11)
    (b01) edge (b01s)
    (b01s) edge (a0)
    (b11) edge (b11s)
    (b11s) edge (nb11s)
    ;
    \end{tikzpicture}
  \caption{
  \label{fig:sa-livelock}
    The saturated graph of local causality of the AAN in \pref{fig:ph-livelock}
    for the objective $\w = \PHobj{c_0}{c_1}$
    and the initial context $\ctx = \PHstate{a_1, b_0, c_0}$.
    Elements in $\Proc$ are represented by rectangular nodes,
    elements in $\Sol$ are represented by circle nodes,
    and elements in $\sSol$ and $\Obj$ are the remaining borderless nodes.
    \pref{th:approxinf} is inconclusive on this example as node $\{ a_1, b_1 \}$
    is not coherent (see \pref{def:coherent}).
    Indeed, $a_0 \in \Proc$ is a successor of $\{ a_1, b_1 \}$, but $a_0 \neq a_1$
    (and the same also stands for $b_0$).
  }
\end{figure}
\end{example}



\subsection{Reachability of a State}
\label{ssec:simult-ua}

\newcommand{\total}{\tau}
\newcommand{\reach}{\sigma}

The reachability property studied so far concerns a single local state at a time.
However, we remark that the reachability of a global state or a sub-state can be
addressed with the very same analysis by introducing a dedicated automata.
Such analysis was not possible with the Process Hitting framework,
because of the lack of the notion of simultaneity for more than two components.

In order to check the reachability of a global state $s \in \PHl$,
let $\PH = (\PHs, \PHl, \PHh)$ be an AAN.
We define a new AAN $\PH' = (\PHs', \PHl', \PHh')$ with:
$\PHs' = \PHs \cup \{ \reach \}$, $\PHl' = \PHl \times \PHl_\reach$,
where $\PHl_\reach = \{ \reach_0, \reach_1 \}$,
and $\PHh' = \PHh \cup \{ \PHhit{\toset{s}}{\reach_0}{\reach_1} \}$.
Given an initial context $\ctx$, the reachability of $s$ in $\PH$
is equivalent to the concretisation of $\PHobjp{\reach}{0}{1}$ in $\PH'$
from the initial context $\ctx \cup \{ \reach_0 \}$,
which can be efficiently under-approximated using \pref{th:approxinf}.

It is of course also possible to compute the reachability
of a sub-state $s \in \PHsubl[\PHl]_S$ of a set of components $S \subseteq \PHs$
with the same method.
One can also check the reachability of a set of states $\Lambda \subseteq \PHl$
by creating several actions
$\PHhit{\toset{s}}{\reach_0}{\reach_1}$ in $\PHh$ for each state $s \in \Lambda$.



\subsection{Sequential Under-approximation}
\label{ssec:ordered-ua}

In this section, we briefly explain an alternative sufficient condition that
progressively takes into account the successive objectives, instead of
considering all of them at a time, as it is done in \pref{ssec:ua}.
Because objectives are taken into account individually, such an approach
considers only a subset of scenarios.
However, because each iteration focuses on a smaller part of the network, this
sequential under-approximation may be more conclusive.

Let us define a sequence of objectives $\w=\obj{a_i}{a_j}\concat\w'$ with
$a_i\neq a_j$ and a state $s\in \PHl$ with $\get{s}{a}=a_i$.
One can remark that any scenario reaching $a_j$ necessarily includes one of the
bounce sequences in $\BS(\obj{a_i}{a_j})$, and, in particular,
any minimal scenario reaching $a_j$ ends in a state where $a_j$ is present but
also the hitters of the last bounce of one bounce sequence in $\BS(\obj{a_i}{a_j})$.
\pref{def:lastprocs} defines $\lastprocs(\obj{a_i}{a_j})$ as the set of set of
local states that may be present just after reaching $a_j$.

\begin{definition}[$\lastprocs : \Obj\to\Sol$]
\label{def:lastprocs}
  Given an objective $\obj{a_i}{a_j} \in \Obj$, $\lastprocs(\obj{a_i}{a_j})$ is
  defined as the largest set such that, $\forall ps\in\lastprocs(\obj{a_i}{a_j})$, 
  $ps\in\sSol$,
  \begin{enumerate}
    \item $a_j \in ps$;
    \item $\exists \zeta \in \BS(\obj{a_i}{a_j}),
      \hitter{\zeta_{\card{\zeta}}} \subseteq ps$;
    \item $\nexists ps' \in \lastprocs(\obj{a_i}{a_j}),
      ps' \subset ps \wedge ps' \neq ps$.
  \end{enumerate}
\end{definition}

From \pref{th:approxinf}, we can deduce that
for any scenario $\delta$ in $\uconcr(P)$,
there exists a set of local states $ps\in\lastprocs(P)$
such that $ps \subseteq (s\play\delta)$.
Hence, if $\muconcr_{\ctx'\Cap ps}(\w')\neq\emptyset$,
with $\ctx'=\ctx\Cap\procs(\mycwB{\ctx}{P})$,
there exists a scenario $\delta'$ concretising $\w'$ from the
state $(s\play\delta)$.
Therefore, the scenario $\delta\concat\delta'$ concretises
$\w$.

\begin{theorem}[Sequential under-approximation]
\label{thm:ordered-ua}
Given an AAN $(\PHs; \PHl; \PHa)$,
a context $\ctx$ and an objective sequence $\w =
P\concat\w'\in\OS$,
$\uconcr(P)\neq\emptyset \wedge
	\forall ps \in\lastprocs(P),
	\muconcr_{\ctx'\Cap ps}(\w')\neq\emptyset
	\Longrightarrow \uconcr(\w)\neq\emptyset$,
where $\ctx' = \ctx\Cap\procs(\mycwB{\ctx}{P})$.
\end{theorem}
\begin{proof}
If $\uconcr(P)\neq\emptyset$,
for all $s\in \PHl, s\subseteq\ctx$,
there exists a scenario $\delta\in\uconcr(P)\cap\Sce(s)$;
from \pref{def:lastprocs} and proof of \pref{th:approxinf},
$\exists ps\in\lastprocs(P)$ such that
$(s\play\delta)\subseteq\ctx'\Cap ps$.
Hence, if $\muconcr_{\ctx'\Cap ps}(\w')\neq\emptyset$,
there exists a scenario $\delta'\in\muconcr_{\ctx'\Cap ps}(\w')$ such that
$\delta'\in\Sce(s\play\delta)$.
Hence, $\delta\concat\delta'$ is a scenario playable in $s$.
Therefore, for all $s\in \PHl, s\subseteq\ctx$, there exists a scenario
concretising $\w$.
Hence, $\uconcr(\w)\neq\emptyset$.
\end{proof}



\modMF{
\subsection{Extraction of a Scenario}
}

\modMF{
This section gives a recursive method to find a scenario that concretizes
a given objective sequence $\w \in \OS$.
All the definitions above are well-defined provided that
$\uconcr(\w)\neq\emptyset$,
which can be demonstrated by \pref{th:approxinf} or \pref{thm:ordered-ua}.
The justification of these definitions can be found by analysing the demonstrations
of these theorems.
}

\newcommand{\res}{\mathbf{res}}
\newcommand{\osres}{\mathbf{OSres}}
\newcommand{\autoBS}{\mathbf{autoBS}}
\newcommand{\rec}{\mathbf{rec}}
% \newcommand{\thisobj}[1]{P_{#1}}
% \newcommand{\thisseq}[1]{\Theta_{#1}}
\newcommand{\thisobj}{P}
\newcommand{\thisseq}{\Theta}
\newcommand{\R}{R}

\modMF{%
First, $\osres$ is used to split the objective sequence $\w$ to solve,
by considering successive objectives, in the fashion of \pref{thm:ordered-ua}.
This function calls $\res$ which allows to solve an objective $P$
by either re-target it (if $\gCont(\PHsort(P), P) \neq \emptyset$)
of by searching for a solution to this objective in $\BS(P)$.
This search may require to try several objective sequences $\zeta \in \BS(P)$.
For each trial, the chosen objective sequence $\zeta$ is solved with $\rec$,
which recursively created a scenario to play the whole sequence.
This scenario is itself recursively created by combining
smaller scenarios allowing to reach each required hitter of each action
with the function $\thisseq$.
Finally, $\thisseq$ calls $\res$ anew to reach the hitters
that are necessarily contained in other automata.
In the end, a scenario can be obtained by analysing all values of
$\osres_s(\w)$ (\pref{def:concret})
for all possible initial state $s \subseteq \ctx$ in the considered context,
and all possible bounce sequences $\zeta \in \BS(P)$ that are chosen during the resolution.
}%

\begin{definition}[$\osres : \PHl \times \OS \to \Sce$]
\label{def:concret}
\modMF{%
For all $s \in \PHl$ and $\w \in \OS$, we define:
\[
  \osres_s(\w) \DEF
    \begin{cases}
      \emptyseq
        & \text{ if $\w = \emptyseq$} \\
      \res_s(\w_1) \concat
      \osres_{s \play \res_s(\w_1)}(\w_2 \concat \dots \concat \w_{\card{\w}})
        & \text{ otherwise} \\
    \end{cases}
\]
and, for all $s \in \PHl$ and $P \in \Obj$:
\[\res_s(P) \DEF
  \begin{cases}
    \varepsilon
      & \text{if } \target{P} = \bounce{P} \\
    \osres_s(P^Q \concat Q)
      & \text{if } \exists Q \in \Obj, (P, Q) \in \Bee{\Obj}{\Obj} \\
    \rec_s(\zeta)
      & \text{otherwise, with } \zeta \in \BS(P)
  \end{cases}
\]
with $P^Q = \PHobj{\PHtarget(P)}{\PHtarget(Q)}$.
Furthermore:
\begin{align*}
  \forall \zeta \in \BS,
  \rec_s(\zeta) &\DEF
    \begin{cases}
      \emptyseq
        & \text{ if $\zeta = \emptyseq$} \\
      \thisseq_s(\zeta_1) \concat \zeta_1 \concat
      \rec_{s \play \thisseq_s(\zeta_1) \concat \zeta_1}
      (\zeta_2 \concat \dots \concat \zeta_{\card{\zeta}})
        & \text{ otherwise}
    \end{cases} \\
  \forall h \in \PHh,
  \thisseq_s(h) &\DEF
    \begin{cases}
      \emptyseq
        & \text{ if $\hitter{h} = \emptyset$} \\
      \R_s(h_1) \concat \thisseq_{s \play \R_s(h_1)}(h_2 \concat \dots \concat h_{\card{h}})
        & \text{ otherwise}
    \end{cases} \\
  \forall a_i \in \Proc,
  \R_s(a_i) &\DEF \res_s(\PHobj{\PHget{s}{\PHsort(a_i)}}{a_i})
\end{align*}
and for all action $h \in \PHh$, $(h_i)_{i \in \segm{1}{\card{\hitter{h}}}}$
is a sequence containing all hitters of $h$ in an arbitrary order.
}
\end{definition}


\section{Encodings in Asynchronous Automata Networks}
\label{sec:encodings}

% Equivalence with ADNs
\subsection{Weak Bisimulation of Asynchronous Discrete Networks}
\label{sec:dn}

We exhibit in this section an encoding of Asynchronous Discrete Networks (ADN),
also called Logical Networks or equivalently René Thomas Models \cite{Thomas95,deJong02},
into AANs and prove a weak bisimulation relation.
This translation is important as it allows to efficiently study the dynamics of ADNs
by using the static analysis developed in \pref{sec:sa};
indeed, the bisimulation relation ensures that the dynamics of the
resulting AAN model are accurately the same than the original.

A \emph{Discrete Network} gathers a finite number of components $i\in\segm{1}{n}$ having a discrete finite domain
$\mathbb F^i$ that we note $\mathbb{F}^i = \segm{1}{l_i}$.
For each component $i\in\segm{1}{n}$, a map $\mathbb F \rightarrow \mathbb F^i$ is defined, where
$\mathbb F = \mathbb F^1 \times \cdots \times \mathbb F^n$, giving the next value of the component
with respect to the global state of the network.
Typically $f^i$ depends on a subset of components that we denote $\DNdep(f^i)$.
In the case of \emph{Asynchronous Discrete Networks} (ADN), a transition relation $\DNtrans\subseteq \mathbb
F\times \mathbb F$ is defined such that $x\DNtrans x'$ if and only if there exists a unique
$i\in\segm{1}{n}$ such that $\get{x'}{i}=f^i(x)$ and $\forall j\in\segm{1}{n}, j\neq i, \get{x'}{j}
=\get{x}{j}$, \ie one and only one component has been updated.
This is formalised in \pref{def:DN}.

\begin{definition}[Asynchronous Discrete Network]
\label{def:DN}
  An \emph{Asynchronous Discrete Networks} (ADN) is defined by a couple $(\mathbb F, \langle f^1, \dots, f^n \rangle)$
  where $\mathbb{F} = \mathbb{F}^1\times\dots\times\mathbb{F}^n$,
  and $\forall i\in\segm{1}{n}$,
  $f^i: \mathbb{F} \rightarrow \mathbb{F}^i$ with
  $\mathbb{F}^i = \segm{1}{l_i}$ and $l_i \in \sNN$.
  Given two states $x,x'\in\mathbb F$, the transition relation $\DNtrans$ is given by
  \[
  x\DNtrans x' \Longleftrightarrow
    \exists i\in\segm{1}{n}, f^i(x)=\get{x'}{i}
    \wedge \forall j\in\segm{1}{n}, j\neq i, \get{x}{j}=\get{x'}{j}
  \enspace,
  \]
  where $\get{x}{i}$ is the $i$\textsuperscript{th} component of $x$.
  %
  We note $\DNdep(f^i)\subseteq \segm{1}{n}$ the set of components on which the value of $f^i$
  depends: $\forall x,x'\in \mathbb F$ such that $\forall
  j\in\DNdep(f^i), \get{x}{j}=\get{x'}{j}$, necessarily $f^i(x)=f^i(x')$.
\end{definition}

Let us denote the encoding of a given ADN $\DNdef$ into an AAN by $\toPH\DNdef$ (\pref{def:DN2PH}).
For each component $i\in\segm{1}{n}$ of the ADN,
one automaton $a^i$ is built, acting for the component value.
Obviously, $a^i$ has one local state $a^i_k$ per element $k \in \mathbb F^i$.
It is then sufficient to build actions towards $a^i$
depending on any possible state of the components in $\DNdep(f^i)$
and the related value of the evolution function $f^i$.



$f^i_\varsigma$ hits the local states of automaton $a^i$ to make them bounce to the local state
$a^i_{k'}$ if and only if $k'=f^i(\decode \varsigma)$;
$\decode \varsigma$ being the ADN state corresponding to the PH (partial) state $\varsigma$ (note that
$f^i(\decode \varsigma)$ is fully defined because $\decode \varsigma$ specifies the state for all
the components in $\DNdep(f^i)$).


% 
% between the components $\DNdep(f^i)$.
% 
% Two classes of actions are then defined:
% $\Hits^{(1)}$ is the set of actions updating the cooperative automata according to the current state of the
% components:
% if $j\in\DNdep(f^i)$, $a^j_k$ hits each local state $f^i_\varsigma$ where $\get{\varsigma}{a^j}\neq
% a^j_k$ to make it bounce to the local state $f^i_{\varsigma \Cap a^j_k}$.
% $\Hits^{(2)}$ is the set of actions encoding the transitions in the ADN:
% $f^i_\varsigma$ hits the local states of automaton $a^i$ to make them bounce to the local state
% $a^i_{k'}$ if and only if $k'=f^i(\decode \varsigma)$;
% $\decode \varsigma$ being the ADN state corresponding to the PH (partial) state $\varsigma$ (note that
% $f^i(\decode \varsigma)$ is fully defined because $\decode \varsigma$ specifies the state for all
% the components in $\DNdep(f^i)$).

\begin{definition}
\label{def:DN2PH}
  $\toPH\DNdef=(\Sigma,\PHl,\Hits)$ is the AAN encoding the ADN $\DNdef$, with:
  \begin{itemize}
    \item $\Sigma = \{ a^1, \dots, a^n \}$;
%      the automata representing components;
    
    \item $\PHl=\underset{i\in\segm{1}{n}}{\times} \PHl_{a^i}$, where
      $\PHl_{a^i}=\{a^i_0, \dots, a^i_{l_i}\}$;
    
    \item $\PHh = \{ \PHfrappe{A}{a^i_j}{a^i_k} \mid
      \exists \varsigma \PHsubl[\PHl]_{\DNdep(f^i)}, \exists x \in \mathbb{F},
      \varsigma \subseteq \encode{x} \wedge
      f^i(x) = k \wedge \PHget{x}{i} = j \wedge j \neq k \wedge
      A = \toset{\varsigma} \}$ where $\encode{x}$ is defined below.
    
%     \item $\Hits^{(1)}= \{ \hit{a^j_k}{f^i_\varsigma}{f^i_{\varsigma'}}
%       \mid i\in\segm{1}{n} \wedge
%       j\in\DNdep(f^i) \wedge a^j_k\in \PHl_{a^j} \wedge f^i_\varsigma\in \PHl_{f^i}
%       \wedge
%       \get{\varsigma}{a^j}\neq a^j_k 
%       \wedge \get{\varsigma'}{a^j}=a^j_k
%       \wedge (\get{\varsigma'}{a^{l}}=\get{\varsigma}{a^l},
%       \forall l\in\segm{1}{n},l\neq j)
%       \}$, 
%       the set of actions with priority $1$ that update cooperative automata;
% 
%     \item $\Hits^{(2)}=\{ \hit{f^i_\varsigma}{a^i_k}{a^i_{k'}} \mid i\in\segm{1}{n} \wedge
%           f^i_\varsigma\in \PHl_{f^i} \wedge
%           a^i_k\in \PHl_{a^i}\wedge k\neq k' \wedge f^i(\decode \varsigma) = k'
%         \}$,
%       the set of actions with priority $2$ for updating the components using their respective discrete
%       maps.
%       $\decode \varsigma$ is defined below.
  \end{itemize}
  Given a state $s \in \PHl$ of the AAN,
  $\decode s=x$ is the corresponding state in the ADN:
  $\forall i\in\segm{1}{n}, \get{s}{a^i}=a^i_k \Rightarrow \get{x}{i}=k$.
  Conversely, given a state $x\in \mathbb F$ of the ADN, 
  $\encode x=s$ is the corresponding state in the AAN:
  $\forall i\in\segm{1}{n}, \get{x}{i}=k \Rightarrow \get{s}{a^i}=a^i_k$.
\end{definition}

Finally,
\pref{th:bisimDN} states the weak bisimulation relation between an ADN and its encoding in AAN.
Intuitively, the actions follow strictly the possible transitions of the ADN.
A detailed proof is given in \pref{suppl:proofbisimADN}.

\begin{theorem}[$\DNdef \approx \toPH\DNdef$]~
\label{th:bisimDN}
  \begin{enumerate}
    \item \label{adn2ph} $\forall x,x'\in\mathbb F$,
      $x\DNtrans x' \Longrightarrow \encode x \PHPtrans^* \encode{x'}$,
      where $\PHPtrans^*$ is a finite sequence of $\PHPtrans$ transitions.

    \item \label{ph2adn} $\forall s,s'\in \PHl$,
      $s\PHPtrans s' \Longrightarrow \decode s = \decode {s'} \vee \decode s \DNtrans
      \decode{s'}\enspace.$
  \end{enumerate}
\end{theorem}


% Flattening of a PH with k priorities
\section{Asynchronous Automata Networks with classes of priorities}
\label{sec:flattening}

In this section, we define the notion of AANs with classes of priorities,
and give a translation from any of these AANs into
an AAN without priorities, as defined in \pref{sec:ph}.

The idea behind AANs with classes of priorities (\pref{def:php})
is to split the set of actions into several subsets assigned to priorities,
and adapting the behaviour of the model to make any action unplayable
until no other action of higher priority is playable (\pref{def:playp}).
Such model allows to model preemptions between sets of actions,
which can be helpful to abstract time or duration properties under certain conditions.

\begin{definition}[AAN with $k$ classes of priorities]
\label{def:php}
  If $k \in \sN^*$,
  an \emph{Asynchronous Automata Network with $k$ classes of priorities} (AAN$k$)
  is a triplet $\PH = (\PHs; \PHl; \PHa^{\angles{k}})$,
  where $\PH^{\angles{k}} = (\PHh^{(1)} \dots; \PHh^{(k)})$,
  and:
  \begin{itemize}
    \item $\PHs \DEF \{a, b, \dots, z\}$ is the finite set of \emph{automata};
    \item $\PHl \DEF \underset{a \in \PHs}{\times} \PHl_a$ is the finite set of
      (global) \emph{states},
      where $\PHl_a = \{a_0, \ldots, a_{l_a}\}$ is the finite set of \emph{local states}
      of automaton $a \in \PHs$, with $l_a \in \sN^*$,
      and so that:
      $\forall (a_i; b_j) \in \PHl_a \times \PHl_b, a \neq b \Rightarrow a_i \neq b_j$;
    \item $\forall n \in \llbracket 1; k \rrbracket,
      \PHa^{(n)} \DEF \{\PHfrappe{A}{b_j}{b_k} \mid
      b \in \PHs \wedge (b_j; b_k) \in \PHl_b \times \PHl_b \wedge
      b_j \neq b_k \wedge
      \forall a \in \PHs, \card{A \cap \PHl_a} \leq 1 \wedge
      A \cap \PHl_b = \emptyset \}$ is the finite set of \emph{actions of priority $n$}.
  \end{itemize}
  We also use the same notations as those defined in \pref{sec:ph}, when applicable.
  Furthermore,
  we denote by $\PHh \DEF \bigcup_{n \in \segm{1}{k}} \PHh^{(n)}$ the set of all actions
  and, for all $h \in \PHh$,
  by $\prio(h) \DEF \min\{ n \in \segm{1}{k} \mid h \in \PHh^{(n)} \}$
  the priority of action $h$.
\end{definition}

\begin{definition}[Semantics of an AAN$k$ ($\PHPtrans$)]
\label{def:playp}
  An action $h = \PHhit{A}{b_j}{b_k} \in \PHa^{(n)}$ of priority $n$
  is \emph{playable} in $s \in \PHl$
  if and only if $A \subseteq s$, $\PHget{s}{b} = b_j$ and
  $\forall m < n, \forall g \in \PHa^{(m)},
    \neg (\PHhitter(g) \subseteq s \wedge \PHtarget(g) \in s)$.
  In such a case, $(s \PHplay h)$ stands for the state resulting from the play
  of the action $h$ in $s$, which is defined by: 
    $(s \PHplay h)_b = b_k$
	and
	$\forall a\in\PHs, a\neq b, (s\PHplay h)_a = s_a$.
  Moreover, we denote: $s \PHPtrans (s \PHplay h)$.
\end{definition}

We consider in the following an AAN$k$ with $k \in \sNN$:
$\ov{\PH} = (\ov{\PHs}; \ov{\PHl}; \ov{\PHa}^{\angles{k}})$.
The aim of the rest this section is to propose a translation of $\oPH$
into an AAN with $1$ class of priority $\PH = (\PHs; \PHl; \PHa^{\angles{1}})$
called \emph{flattening}, which is bisimilar.
This AAN with 1 class of priority is equivalent to a regular AAN without priorities;
Therefore, such a translation is particularly useful to be able to study the dynamics of
any kind of AAN with priorities by using the static analysis developed in \pref{sec:sa}.

This translation is based on the notion of \emph{playability property} defined in \pref{def:pp}
which is a Boolean formula where the atoms are local states of $\ov{\PH}$.

\begin{definition}[Playability property language ($\F$)]
  \label{def:pp}
  A \emph{playability property} is an element of the language $\F$ inductively defined by:
  \begin{itemize}
    \item $\top$ and $\bot$ belong to $\F$;
    \item if $a \in \ov{\PHs}$ and $a_i \in \ov{\PHl}_a$, then $a_i \in \F$ and we call it an \emph{atom};
    \item if $P \in \F$ and $Q \in \F$, then $\neg P \in \F$, $P \wedge Q \in \F$ and $P \vee Q \in \F$.
  \end{itemize}
  If $P \in \F$ is a playability property and $\mysigma \in \PHsubl$ is a sub-state of $\oPH$,
  we note $\Fsem{P}{\mysigma}$ the \emph{evaluation} of $P$ in $\mysigma$:
  \begin{itemize}
    \item if $P = a_i \in \ov{\PHl}_a$ is an atom, with $a \in \ov{\PHs}$, then $\Fsem{a_i}{\mysigma}$ is true iff $a_i \in \mysigma$;
    \item if $P$ is not an atom, then $\Fsem{P}{\mysigma}$ is true iff all its subformulas are inductively true in $\mysigma$
      with the classical semantics for the logic operators $\top$, $\bot$, $\neg$, $\wedge$ and $\vee$.
  \end{itemize}
\end{definition}

Because we only use classical logic operators, the formulas of Boolean logic on 
distributivity, associativity and commutativity can be used, together with De Morgan's laws on negation.
We also have the following property for the negation of an atom:
\[\forall a \in \ov{\PHs}, \forall a_i \in \ov{\PHl}_a, \forall \mysigma \in \PHsubl,
  \Fsem{\neg a_i}{\mysigma} \Leftrightarrow \Fsem{\bigvee_{\substack{a_j \in \ov{\PHl}_a\\a_j \neq a_i}} a_j}{\mysigma}\]
Indeed, if a local state is not active in a state, this means that another local state of the same automaton is active.
Moreover, we note that:
\[\forall a \in \ov{\PHs}, \forall a_i, a_j \in \ov{\PHl}_a,
  a_i \neq a_j \Longrightarrow a_i \wedge a_j \equiv \bot \]
because two different processes can never be active simultaneously.

In \pref{def:fop}, we define the the operator $\Fopsymbol$ which characterises the playability of an action
given the semantics of AAN$k$s (see \pref{def:playp}).
This operator simply states that the hitters of an action have to be active,
and no other action of higher priority have to be playable.
% The target of the considered action is not taken into account
% as it is not needed for the rest of the flattening.

\begin{definition}[Playability property operator ($\Fopsymbol : \PHh \rightarrow \F$)]\label{def:fop}
  For all $h = \PHfrappe{A}{b_j}{b_m} \in \ov{\PHh}$, we define:
  \[
    \Fop{h} \equiv
    b_j \wedge
    \left(\bigwedge_{a_i \in A} a_i\right)
    %\hitter{h}
    \wedge
      \left( \bigwedge_{\substack{g \in \ov{\PHh}^{(n)}\\1 < n < \prio(h)}}
      \neg \left( \target{g} \wedge \left(\bigwedge_{c_l \in \hitter{g}} c_l\right)\right) \right)
  \]
\end{definition}
%
By construction of this operator and given the dynamics of a AAN$k$,
an action $h$ is playable in a state $s \in \ov{\PHl}$ if and only if: $\Fsem{\Fop{h}}{s}$.

Because we only use classical logic operators, we can compute the Disjunctive Normal Form (DNF) of any playability property.
For any action $h \in \ov{\PHh}$, this DNF takes the form:
\[\Fop{h} \equiv \bigvee_{i \in \segm{1}{\n}} \left( \bigwedge_{j \in \segm{1}{\m}} p_{i,j} \right)\]
where $\n \in \sN$ and $\forall i \in \segm{1}{\n}, \m \in \sN^*$.
If $\n = 0$, then $\Fop{h} \equiv \bot$; this means that $h$ can never be played
due to preemptions by other actions with higher priorities.
If $\Fop{h} \not\equiv \bot$, on the other hand, then in this case $\Fop{h}$
can be seen as a disjunction of $\n$ smaller playability properties consisting only of conjunctions of atoms.
These $\n$ conjunctions can be translated to as many actions,
thus creating a new AAN$1$.
In this case, we denote, for any $i \in \segm{1}{\n}$:
$\PHdep{i}{h} = \{ \PHsort(p_{i,j}) \mid j \in \segm{1}{\m} \}$.

With \pref{lem:ppplaysubset}, we can then characterise the playability of an action in a state only with a sub-state.
This sub-state corresponds to one of the conjunctions of its playability property's DNF.
Finally, \pref{def:flattening} gives the construction of the flattening of $\oPH$:
for each action $h \in \ov{\PHh}$, several actions $f^{h,i}$ are built to reflect each of the conjunctions in $\Fop{h}$,
\ie for $i \in \segm{1}{\n}$.
This construction allows to obtain the same dynamics as $\oPH$, as stated by \pref{th:bisimPHP}.

\begin{lemma}
\label{lem:ppplaysubset}
  Let $h \in \ov{\PHh}$ and $s \in \ov{\PHl}$;
  $h$ is playable in $s$ if and only if:
  \[\target{h} \in s \wedge \exists \mysigma \subseteq s, \Fsem{\Fop{h}}{\mysigma} \enspace.\]
\end{lemma}
%
\begin{proof}
  ($\Rightarrow$)
    If $h$ is playable in $s$, then $\target{h} \in s$ and $\Fsem{\Fop{h}}{s}$.
    Thus, $\Fop{h} \not\equiv \bot$ and, by property of a DNF,
    at least one of the $\n$ conjunctions of $\Fop{h}$ is true in $s$.
    Suppose the $i$\textsuperscript{th} conjunction is true in $s$, with $i \in \segm{1}{\n}$;
    then we have: $\forall j \in \segm{1}{\m}, p_{i,j} \in s$.
    Let $\mysigma \in \PHsubl_{\PHdep{i}{h}}$
    with $\forall b \in \PHdep{i}{h}, \PHget{\mysigma}{b} = \PHget{s}{b}$.
    We immediately have: $\mysigma \subseteq s$,
    and, by construction of $\PHdep{i}{h}$, $\Fsem{\Fop{h}}{\mysigma}$.
  
  ($\Leftarrow$)
    $\Fsem{\Fop{h}}{\mysigma}$ therefore $\Fsem{\Fop{h}}{s}$; as $\target{h} \in s$,
    $h$ is playable in $s$.
\end{proof}

\begin{definition}[Flattening ($\PHflat$)]
  \label{def:flattening}
  If $k \in \sNN$ and $\oPH = (\ov{\PHs}; \ov{\PHl}; \ov{\PHa}^{\angles{k}})$ is an AAN$k$,
  we denote by
  $\PHflat(\ov{\PHs}; \ov{\PHl}; \ov{\PHa}^{\angles{k}}) = (\PHs; \PHl; \PHa)$
  the \emph{flattening} of $\oPH$, where:
  \begin{itemize}
    \item $\PHs = \ov{\PHs}$;
    
    \item $\PHl = \ov{\PHl}$;
    
    \item $\PHh = \{
      \PHfrappe{(\toset{\mysigma} \setminus \{ \target{h} \})}{\target{h}}{\bounce{h}} \mid
      h \in \ov{\PHh} \wedge \n \geq 1 \wedge i \in \segm{1}{\n} \wedge
      \mysigma \in \PHsubl_{\PHdep{i}{h}} \wedge
      \Fsem{\Fop{h}}{\mysigma} \}$.
  \end{itemize}
%   Given a state $s \in \PHl$, $\unflats{s} = \os$ is the corresponding state in $\ov{\PHl}$:
%   $\forall a \in \ov{\PHs}, \PHget{\os}{a} = \PHget{s}{a}$.
%   Conversely,
%   given a state $\os \in \ov{\PHl}$, $\flats{\os} = s$ is the corresponding state in $\PHl$:
%   $\forall a \in \ov{\PHs}, \PHget{s}{a} = \PHget{\os}{a}$.
\end{definition}

We note that the set of global states of an AAN$k$
and the set of global states of its flattening are the same.

\begin{theorem}[$(\ov{\PHs}; \ov{\PHl}; \ov{\PHa}^{\angles{k}}) \approx \PHflat(\ov{\PHs}; \ov{\PHl}; \ov{\PHa}^{\angles{k}})$]
\label{th:bisimPHP}
  If $\ov{\PH} = (\ov{\PHs}; \ov{\PHl}; \ov{\PHa}^{\angles{k}})$ is an AAN$k$
  and $\PH = \PHflat(\ov{\PHs}; \ov{\PHl}; \ov{\PHa}^{\angles{k}}) =
    (\PHs; \PHl; \PHa)$ is its flattening, then:
  \[ \forall s, s' \in \PHl, s \PHPtrans[\oPH] s' \Longleftrightarrow s \PHPtrans[\PH] s' \]
\end{theorem}

\begin{proof}
  By definition of $\PHflat$.
\end{proof}



We showed in this subsection that it is possible to model any AAN$k$
as an AAN (or, equivalently, as an AAN$1$).
This translation thus extends the applicability of the static analysis developed in
\pref{sec:sa} to any AAN$k$, with $k \in \sN^*$.
Moreover, it allows to represent
any Process Hitting model with classes of priorities~\cite{FPMR13-CS2Bio}
under the form of an AAN
(or, equivalently, of a Process Hitting model with $2$ classes of priorities).
The translation given in this section
is exponential in the number of actions of higher priority for each action.



% vim:set spell spelllang=en:

\section{Large-scale Biological Example}\label{sec:example}

In order to support the scalability and applicability of our under-approximation of reachability, we
apply our new approach for the analysis of large-scale model of the T-cell receptor (TCR)
signalling pathway \cite{tcrsig94}.
This model gathers 94 interacting components and is specified as a Boolean network.
The under-approximation presented in this paper has been implemented in the existing Pint
software\footnote{Pint is freely available at \url{http://process.hitting.free.fr}.}.

The Boolean model has been automatically encoded into a Process Hitting with 2 classes of priority%
\footnote{Model and scripts are available at
\url{http://www.irccyn.ec-nantes.fr/~folschet/underapprox-tcrsig94.tbz2}.}.
Then, we verified the reachability for the independent activation of 4 outputs of the signalling
cascade (SRE, AP1, NFkB, NFAT) from all possible input combinations (CD45, CD28, TCRlib) using our
new reachability under-approximation (answering either \emph{yes} or \emph{inconclusive}) and a 
previously defined reachability over-approximation \cite{PMR12-MSCS} (answering either \emph{no} or
\emph{inconclusive}).
All result in conclusive decisions, and the under-approximation has been satisfied in 12 cases (over
32) proving the satisfiability of the concerned reachability property in the encoded Boolean network
(and non-satisfiability in the other cases).

Computations times are in the order of a few hundredths of a second on a 2.4GHz processor with 2GB
of RAM.
To give a comparison, we did the same experiments with a standard symbolic model-checker, libDDD
\cite{libddd}, known for its good performances, the input model being the Boolean network expressed
as a Petri net.
However, due to the large scale of the model, the program runs out of memory for all the experiments.

While ensuring a low complexity for the analysis of reachability in Boolean and discrete networks, our
under-approximation method reveals to be conclusive in numerous cases when applied to real
large-scale biological models, which were not tractable with exact model-checking.




\section{Discussion \& Conclusion}\label{sec:ccl}

We introduced a new semantics to include priorities into the Process Hitting framework, which prove especially useful to model cooperations.
Then, we developed a method to efficiently perform a reachability analysis of a sequence of objectives in a restricted class of Process Hitting models,
but it is also useful to establish the reachability of a partial state.
This analysis is based on an under-approximation of the true reachability solutions.
%; however, the most usual cases can be handled.

We showed that the class of Process Hitting models that can be handled by the aforementioned method are equivalent to Asynchronous Discrete Networks, and therefore to Asynchronous Boolean Networks.
This allows to efficiently compute reachability results on large biological models provided that they are equivalent to Asynchronous Discrete Networks and that a translation from the original framework into a Process Hitting model is possible.
Such a translation for interaction graphs of Thomas modelling was proposed in~\cite{PMR10-TCSB} \towrite{[À garder ?] and is made possible by the Pint software}.

Further work can be derived from what have been presented in this paper.
The over-approximation on Process Hitting models without priorities proposed in~\cite{PMR12-MSCS}
is still accurate in the framework with priorities (by “flattening” all actions),
but may be refined given the restrictions proposed in this paper,
and a specific search of key processes or cut sets may be derived.
%but turns out to be too wide even in some obvious cases that are consequently not conclusive.
%This approximation may be refined in order to better fit the introduction of priorities, and mak the overall approximation approach 
%and mode precisely the class of models studied in this paper.
Furthermore, a more general under-approximation could be developed in order to handle a larger class of Process Hitting models, that is,
models with more than two classes of priorities, that do not only contain components of cooperative sorts, or whose behaviour may contain cycles or cyclic attractors.
Finally, in order to take into account quantitative data in transition delays, the overall approximation method could be extended to handle evolution that are chronometric instead of only chronologic.


\section*{References}
\bibliographystyle{elsarticle-num}
\bibliography{biblio}

\appendix

% Fix appendix references
\renewcommand*{\thesection}{\Alph{section}}

\section{Proof of Under-approximation (\pref{th:approxinf})}
\label{suppl:demoapproxinf}

In the following, we denote:
%$\BvProc = \Bv \cap \PHproc$, $\BvObj = \Bv \cap \Obj$ and $\BvSol = \Bv \cap \Sol$.
$\Bee{X}{Y} = \Be \cap (X \times Y)$, with $X, Y$ amongst $\PHproc$, $\Obj$ and $\Sol$.

\begin{proof}
We note $max\ctx = \ctx \Cap \allprocs(\cwB)$ the context supported by $\cwB$.

Let $(a_i, ps) \in \Bee{\Proc}{\Sol}$ be an edge linking the required process of a cooperative sort to a solution set and suppose all children of $ps$ are concretisable.
We label all processes of $ps$ by an integer: $ps = \{ p_n \}_{n \in \indexes{ps}}$.
Let us prove by induction that for all $n \in \indexes{ps}$, there exists a scenario $\delta_n$ so that:
$\forall i \in \segm{1}{n}, \PHget{(s \PHplay \delta_n)}{\PHsort(p_i)} = p_i$.
\begin{itemize}
  \item It is straightforward for $\delta_0 = \varepsilon$.
  \item Suppose such $\delta_n$ exists and let $q = \PHget{(s \PHplay \delta_n)}{\PHsort(p_{n+1})}$.
    By hypothesis, $(a_i, ps)$ is coherent (\pref{def:coherent}) and all processes of $ps$ are processes of components;
    this means that none of the processes needed to solve $p_{n+1}$ is another process of the same sort than another process of $ps$.
    Therefore, there exists $\delta' \in \muconcr_{s \PHplay \delta_n}(\PHobj{q}{p_{n+1}})$,
    so that $\forall i \in \segm{1}{n+1}$, $\PHget{(s \PHplay \delta_n \PHplay \delta')}{\PHsort(p_{i})} = p_{i}$.
    Finally, by \pref{th:update}, there exists a scenario $\delta'' \in \restriction{\Sce}{1}(s \PHplay \delta_n \PHplay \delta')$
    so that, if we denote $\delta_{n+1} = \delta_n \PHplay \delta' \PHplay \delta''$,
    we have: $\update(s \PHplay \delta_n \PHplay \delta') = s \PHplay \delta_{n+1}$ and the same property about processes (by \pref{th:hcscomp}).
\end{itemize}
Therefore, $\delta = \delta_{|ps|}$ exists, and given its properties, we have: $\PHget{(s \PHplay \delta)}{a} = a_i$
and $\update(s \PHplay \delta) = s \PHplay \delta$.

As there is no cycle in $\cwB$, we show by induction that $\forall s\in L, s\subseteq max\ctx$, 
for all objective $P$ in $\Bv \cap \Obj$ so that $\PHtarget(P) \in s$,
$\exists \delta \in \muconcr_s(P)$.% and $\ceil(\delta) \subseteq max\ctx$.
\begin{itemize}
  \item If $(P, \emptyset) \in \Bee{\Obj}{\Sol}$, either $\PHtarget(P) = \PHbounce(P)$ and $\delta = \emptyseq$;
    or $\forall \zeta \in \BS(P), \zeta \in \Sce(s) \wedge \PHsort(\zeta) = \{ \PHsort(P) \}$
    and $\delta = \delta_1 \PHplay \zeta_1 \PHplay \dots \PHplay \delta_{|\zeta|} \PHplay \zeta_{|\zeta|}$ is a valid sequence given by \pref{th:hcompcomp}.

  \item Suppose all children objectives of $P$ are concretizable.
%  \begin{itemize}
    %\item 
    If $\exists (P, Q) \in \Bee{\Obj}{\Obj}$, then by hypothesis,
      $\muconcr_{s}(\obj{\PHtarget(P)}{\PHtarget(Q)} \concat Q) \neq \emptyset$, thus
      $\muconcr_{s}(P) \neq \emptyset$.
    %\item 
    Else, by \pref{def:maxCont}, the concretizations of the children of $P$ require no process of sort $\PHsort(P)$.
      Furthermore, there exists $\zeta \in \BS(P)$ so that $(P, \aZ) \in \Bee{\Obj}{\Sol}$.
      We show by induction that for all $n \in \indexes{\zeta}$, there is a scenario $\delta_n$ so that $\PHget{(s \PHplay \delta_n)}{\PHsort(P)} = \PHbounce(\zeta_n)$.
      \begin{itemize}
        \item[$\circ$] Suppose that $\delta_n$ exists and let $\zeta_n = \PHhit{b_i}{a_j}{a_k}$.
        By hypothesis there exists $\delta' \in \muconcr_{s \PHplay \delta_n}(\PHobj{\any}{b_i})$ with $\PHsort(P) \notin \PHsort(\delta')$ (by \pref{def:maxCont}).
        By \pref{th:update} there exists $\delta'' \in \restriction{\Sce}{1}(s \PHplay \delta')$ so that $\update(s \PHplay \delta') = s \PHplay \delta' \PHplay \delta''$.
        Furthermore, $\PHget{(s \PHplay \delta' \PHplay \delta'')}{b} = b_j$ (by \pref{th:hcompcomp} if $b \in \components$ or \pref{th:hcscomp} if $b \in \cs$).
        Therefore, $\delta_{n+1} = \delta_n \PHplay \delta' \PHplay \delta'' \PHplay \zeta_n$.
      \end{itemize}
      Thus, $\delta_{|\zeta|} \in \muconcr_s(P)$. % and $\ceil(\delta) \subseteq max\ctx$.
%  \end{itemize}
\end{itemize}
Finally, as $\muconcr_{max\ctx}(\w) \neq \emptyset$, $\uconcr(\w) \neq \emptyset$ (\pref{th:uconcr-ctx}).
\end{proof}



% vim:set spell spelllang=en:

\section{Weak Bisimulation of Asynchronous Discrete Networks}

\def\DNtrans{\rightarrow_{ADN}}
\def\DNdef{(\mathbb F, \langle f^1, \dots, f^n\rangle)}
\def\DNdep{\f{dep}}
\def\PHPtrans{\rightarrow_{PHP}}
\def\get#1#2{#1[{#2}]}
\def\encodeF#1{\mathbf{#1}}
\def\toPH{\encodeF{PH}}
\def\card#1{|#1|}
\def\decode#1{\llbracket#1\rrbracket}
\def\encode#1{\llparenthesis#1\rrparenthesis}
\def\Hits{\PHa}
\def\hit{\PHhit}
\def\play{\cdot}

\todo{synchronise notations and terminology with the main text}
\todo{define $\PHPtrans$? or defined in main text?}

We exhibit an encoding of Asynchronous Discrete Networks (ADN) with the Process
Hitting using two priority classes, and prove a weak bisimulation relation.

A Discrete Network gathers a finite number of components $i\in[1;n]$ having a discrete finite domain
$\mathbb F^i$ that we note $\mathbb{F}^i = [0;l_i]$.
For each component $i\in[1;n]$, a map $\mathbb F \mapsto F^i$ is defined, where
$\mathbb F = \mathbb F^1 \times \cdots \times \mathbb F^n$, giving the next value of the component
with respect to the global state of the network.
Typically $f^i$ depends on a subset of components that we denote $\DNdep(f^i)$.
In the case of Asynchronous Discrete Networks (ADN), a transition relation $\DNtrans\subset \mathbb
F\times \mathbb F$ is defined such that $x\DNtrans x'$ if and only if there exists a unique
$i\in[1;n]$ such that $\get{x'}{i}=f^i(x)$ and $\forall j\in[1;n], j\neq i, \get{x'}{j}
=\get{x}{j}$, i.e. one and only component has been updated.
This is formalised in \pref{def:DN}.

\begin{definition}[Asynchronous Discrete Network (ADN)]
\label{def:DN}
An ADN is defined by a couple $(\mathbb F, \langle f^1, \dots, f^n \rangle)$
where $\mathbb{F} = \mathbb{F}^1\times\dots\times\mathbb{F}^n$,
and $\forall i\in[1;n]$,
$f^i: \mathbb{F} \mapsto \mathbb{F}^i$ with
$\mathbb{F}^i = [0;l_i]$.
Given two states $x,x'\in\mathbb F$, the transition relation $\DNtrans$ is given by
\[
x\DNtrans x' \Longleftrightarrow
  \exists i\in[1;n], f^i(x)=\get{x'}{i}
  \wedge \forall j\in[1;n], j\neq i, \get{x}{j}=\get{x'}{j}
\enspace,
\]
where $\get{x}{i}$ is the $i$-th component of $x$.
%
We note $\DNdep(f^i)\subseteq \{1,\dots,n\}$ the set of components on which the value of $f^1$
depends: $\forall x,x'\in \mathbb F$ such that $\forall
j\in\DNdep(f^i), \get{x}{j}=\get{x'}{j}$, necessarily $f^i(x)=f^i(x')$.
\end{definition}

Let us denote by $\toPH\DNdef$ the encoding of the ADN $\DNdef$ in Process Hitting with $2$ priority
classes (\pref{def:DN2PH}).
For each component $i\in[1;n]$ of the ADN, two sorts are built: $a^i$ acting for the component
value, and $f^i$ acting for a cooperative sort between the components $\DNdep(f^i)$.
Sorts $a^i$ have one process $a^i_k$ per element in $k\in\mathbb F^i$.
Sorts $f^i$ have one process $f^i_\varsigma$ per state $\varsigma \in \prod_{j\in\DNdep(f^i)}
L_{a^j}$.
Two classes of actions are then defined:
$\Hits^1$ is the set of actions updating the cooperative sorts according to the actual state of the
components:
if $j\in\DNdep(f^i)$, $a^j_k$ hits each process $f^i_\varsigma$ where $\get{\varsigma}{a^j}\neq
a^j_k$ to make it bounce to the process $f^i_{\varsigma'}$ where $\get{\varsigma'}{a^j}=a^j_k$.
$\Hits^2$ is the set of actions encoding the transitions in the ADN:
$f^i_\varsigma$ hits the processes of sort $a^i$ to make them bounce to the process
$a^i_{k'}$ if and only if $k'=f^i(\decode \varsigma)$;
$\decode \varsigma$ being the ADN state correspond to the PH (partial) state $\varsigma$ (note that
$f^i(\decode \varsigma)$ is fully defined because $\decode \varsigma$ specifies the state for all
the components in $\DNdep(f^i)$).

\begin{definition}
\label{def:DN2PH}
$\toPH\DNdef=(\Sigma,L,\Hits^1,\Hits^2)$ is the Process Hitting with 2 priority classes encoding the
ADN $\DNdef$, with:
\begin{itemize}
\item $\Sigma = \{ a^1, \dots, a^n \} \cup \{ f^1, \dots, f^n \}$,
the sorts for components ($a^i$) and cooperative sorts ($f^i$).

\item $L=\prod_{i\in[1;n]} L_{a^i} \times \prod_{i\in[1;n]} L_{f^i}$, where
$L_{a^i}=\{a^i_0, \dots, a^i_{l_i}\}$, and
$L_{f^i}=\{f^i_\varsigma \mid \varsigma\in\prod_{j\in\DNdep(f^i)} L_{a^i} \}$
if $\DNdep(f^i)\neq\emptyset$, otherwise
$L_{f^i}=\{f^i_\emptyset\}$;
  
\item $\Hits^1= \{ \hit{a^j_k}{f^i_\varsigma}{f^i_{\varsigma'}}
 \mid i\in[1;n] \wedge
j\in\DNdep(f^i) \wedge a^j_k\in L_{a^j} \wedge f^i_\varsigma\in L_{f^i}
\wedge
\get{\varsigma}{a^j}\neq a^j_k 
\wedge \get{\varsigma'}{a^j}=a^j_k
\wedge (\get{\varsigma'}{a^{l}}=\get{\varsigma}{a^l},
\forall l\in[1;n],l\neq j)
\}$, 
the set of actions with priority $1$ for updating cooperative sorts.

\item $\Hits^2=\{ \hit{f^i_\varsigma}{a^i_k}{a^i_{k'}} \mid i\in[1;n] \wedge
      f^i_\varsigma\in L_{f^i} \wedge
      a^i_k\in L_{a^i}\wedge k\neq k' \wedge f^i(\decode \varsigma) = k'
    \}$,
the set of actions with priority $2$ for updating the components using their respective discrete
maps.
$\decode \varsigma$ is defined below.
\end{itemize}
Given a state $s\in L$ of the Process Hitting, 
$\decode s=x$ is the corresponding state in the ADN:
$\forall i\in[1;n], \get{s}{a^i}=a^i_k \Rightarrow \get{x}{i}=k$.

\noindent
Given a state $x\in \mathbb F$ of the ADN, 
$\encode x=s$ is the corresponding state in the Process Hitting:
$\forall i\in[1;n], \get{x}{i}=k \Rightarrow \get{s}{a^i}=a^i_k$
and
$\forall i\in[1;n], \get{s}{f^i}=f^i_\varsigma$ with $f^i_\varsigma\in L_{f^i}$
and $\forall j\in\DNdep(f^i), \get{\varsigma}{j}=\get{s}{a^j}$.
\end{definition}

\pref{thm:bisimDN} states the weak bisimulation relation between an ADN and its encoding in
Process Hitting with $2$ priority classes.
Intuitively, actions updating cooperative sorts being of higher priority, actions updating component
sorts follow strictly the possible transitions of the ADN.

\begin{theorem}[$\DNdef \approx \toPH\DNdef$]~
\label{thm:bisimDN}
\begin{enumerate}
\item $\forall x,x'\in\mathbb F$,
$x\DNtrans x' \Longrightarrow \encode x \PHPtrans^* \encode{x'}$,
where $\PHPtrans^*$ is a finite sequence of $\PHPtrans$ transitions.

\item $\forall s,s'\in L$,
$s\PHPtrans s' \Longrightarrow \decode s = \decode {s'} \vee \decode s \DNtrans
\decode{s'}\enspace.$
\end{enumerate}
\end{theorem}
\begin{proof}
(1) From \pref{def:DN} $x\DNtrans x'\Rightarrow \exists i\in[1;n],
f^i(x)=\get{x'}{i} \wedge \forall j\in[1;n],i\neq j, \get{x}{j}=\get{x'}{j}$.
Let us assume (without loss of generality) that $f^i(x)=k'$, $\get{x}{i}=k$ and
$\varsigma\in\prod_{j\in\DNdep(f^i)} L_{a^j}$ such that
$\forall j\in\DNdep(f^i), \get{\varsigma}{j}=a^j_{\get{x}{j}}$.
From \pref{def:DN2PH}, $h=\hit{f^i_\varsigma}{a^i_k}{a^i_{k'}}\in\Hits^2$.
From $\encode x$ definition,
$a^i_k\in \encode x$ and $f^i_\varsigma\in \encode x$;
moreover, as there is no action in $\Hits^1$ applicable in $\encode x$,
$h$ is applicable in $\encode x$:
$\encode x\PHPtrans \encode x\play h$.
In $\encode x\play h$, the only applicable actions of priority $1$ are those having
$a^i_{k'}$ as hitter and hitting cooperative sorts, giving a finite number of transitions towards
$\encode{x'}$.

(2) $s\PHPtrans s'$ only if there exists an action $h$ applicable in $s$ such that
$s\play h=s'$.
If $\prio(h)=1$, then, by $\Hits^1$ definition, 
$\decode s=\decode {s'}$.
If $\prio(h)=2$, then $\forall i\in[1;n]$,
if $\get{s}{f^i} = f^i_\varsigma$, then, $\forall j\in\DNdep(f^i),
\get{\varsigma}{a^j}=\get{s}{a^j}$.
Let us define $i\in[1;n]$ such that $\get{s}{a^i}\neq\get{s'}{a^i}$ ($i$ is unique for this
transition).
By \pref{def:DN2PH}, if $\get{s'}{a^i}=a^i_{k'}$, necessarily $f^i(\decode s)=k'$, hence
$\decode s\DNtrans \decode{s'}$.
\end{proof}






\end{document}
