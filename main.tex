\documentclass{elsarticle}

\usepackage[english]{babel}
\usepackage[utf8]{inputenc}
\usepackage[T1]{fontenc}

\usepackage{amsmath}  % Maths
\usepackage{amsfonts} % Maths
\usepackage{amssymb}  % Maths
\usepackage{stmaryrd} % Maths (crochets doubles)

\usepackage{url}     % Mise en forme + liens pour URLs
\usepackage{hyperref}
\usepackage{array}   % Tableaux évolués

\usepackage{comment}

% Theorems and definitions
\newdefinition{definition}{Definition}
\newdefinition{condition}{Condition}
\newtheorem{theorem}{Theorem}
\newtheorem{lemma}{Lemma}
\newproof{example}{Example}
\newproof{rawproof}{Proof}

\newenvironment{proof}{\begin{rawproof}}{\hfill$\Box$\end{rawproof}}

% Pretty references
\usepackage{prettyref}
\newrefformat{def}{Def.~\ref{#1}}
\newrefformat{fig}{Fig.~\ref{#1}}
\newrefformat{tab}{Table~\ref{#1}}
\newrefformat{lem}{Lemma~\ref{#1}}
\newrefformat{th}{Theorem~\ref{#1}}
\newrefformat{co}{Corollary~\ref{#1}}
\newrefformat{sec}{Sect.~\ref{#1}}
\newrefformat{ssec}{Subsect.~\ref{#1}}
\newrefformat{suppl}{Appendix~\ref{#1}}
\newrefformat{cr}{Condition~\ref{#1}}
\newrefformat{eq}{Eq.~\eqref{#1}}
\def\pref{\prettyref}

\usepackage{tikz}
\newdimen\pgfex
\newdimen\pgfem
\usetikzlibrary{arrows,shapes,shadows,scopes}
\usetikzlibrary{positioning}
\usetikzlibrary{matrix}
\usetikzlibrary{decorations.text}
\usetikzlibrary{decorations.pathmorphing}

% Macros relatives à la traduction de PH avec arcs neutralisants vers PH à k-priorités fixes

% Macros générales
%\newcommand{\ie}{\textit{i.e.} }

% Notations générales pour PH
\newcommand{\PH}{\mathcal{PH}}
\newcommand{\PHs}{\mathcal{S}}
%\newcommand{\PHp}{\mathcal{P}}
\newcommand{\PHp}{\textcolor{red}{\mathcal{P}}}
\newcommand{\PHproc}{\mathcal{P}}
\newcommand{\PHh}{\mathcal{H}}
\newcommand{\PHa}{\PHh}
%\newcommand{\PHa}{\mathcal{A}}
\newcommand{\PHl}{\mathcal{L}}
\newcommand{\PHn}{\mathcal{N}}

\newcommand{\PHhitter}{\mathsf{hitter}}
\newcommand{\PHtarget}{\mathsf{target}}
\newcommand{\PHbounce}{\mathsf{bounce}}
%\newcommand{\PHsort}{\Sigma}
\newcommand{\PHsort}{\PHs}

%\newcommand{\PHfrappeur}{\mathsf{frappeur}}
%\newcommand{\PHcible}{\mathsf{cible}}
%\newcommand{\PHbond}{\mathsf{bond}}
%\newcommand{\PHsorte}{\mathsf{sorte}}
%\newcommand{\PHbloquant}{\mathsf{bloquante}}
%\newcommand{\PHbloque}{\mathsf{bloquee}}

%\newcommand{\PHfrappeR}{\textcolor{red}{\rightarrow}}
%\newcommand{\PHmonte}{\textcolor{red}{\Rsh}}

\newcommand{\PHhitA}{\rightarrow}
\newcommand{\PHhitB}{\Rsh}
%\newcommand{\PHfrappe}[3]{\mbox{$#1\PHhitA#2\PHhitB#3$}}
%\newcommand{\PHfrappebond}[2]{\mbox{$#1\PHhitB#2$}}
\newcommand{\PHhit}[3]{#1\PHhitA#2\PHhitB#3}
\newcommand{\PHfrappe}{\PHhit}
\newcommand{\PHhbounce}[2]{#1\PHhitB#2}
\newcommand{\PHobj}[2]{\mbox{$#1\PHhitB^*\!#2$}}
\newcommand{\PHconcat}{::}
%\newcommand{\PHneutralise}{\rtimes}

\def\PHget#1#2{{#1[#2]}}
%\newcommand{\PHchange}[2]{#1\langle #2 \rangle}
%\newcommand{\PHchange}[2]{(#1 \Lleftarrow #2)}
%\newcommand{\PHarcn}[2]{\mbox{$#1\PHneutralise#2$}}
\newcommand{\PHplay}{\cdot}

\newcommand{\PHstate}[1]{\mbox{$\langle #1 \rangle$}}

\input{macros/macros-ph}
\input{macros/macros-abstr}

% Macros
\newcommand{\sN}{\mathbb{N}}
\newcommand{\sNN}{\mathbb{N}^\bullet}
\def\DEF{\stackrel{\Delta}=}
\def\EQDEF{\stackrel{\Delta}\Leftrightarrow}
\newcommand{\segm}[2]{\llbracket #1; #2 \rrbracket}
\def\indexes#1{\mathbb{I}^{#1}}
\def\f#1{\mathsf{#1}}
\def\prio{\mathsf{prio}}
\newcommand{\bottom}{\perp}
\newcommand{\stable}{\mathsf{stable}}
\newcommand{\update}{\mathsf{update}}
\newcommand{\components}{\Gamma}
%\newcommand{\cs}{CS}
\newcommand{\cs}{\Delta}

% Restriction notation
\def\restriction#1#2{\mathchoice
              {\setbox1\hbox{${\displaystyle #1}_{\scriptstyle #2}$}
              \restrictionaux{#1}{#2}}
              {\setbox1\hbox{${\textstyle #1}_{\scriptstyle #2}$}
              \restrictionaux{#1}{#2}}
              {\setbox1\hbox{${\scriptstyle #1}_{\scriptscriptstyle #2}$}
              \restrictionaux{#1}{#2}}
              {\setbox1\hbox{${\scriptscriptstyle #1}_{\scriptscriptstyle #2}$}
              \restrictionaux{#1}{#2}}}
\def\restrictionaux#1#2{{#1\,\smash{\vrule height .8\ht1 depth .85\dp1}}_{\,#2}} 

%\renewcommand{\restriction}[2]{#1_{#2}}

% Commandes À FAIRE
\usepackage{color} % Couleurs du texte
%\definecolor{darkgreen}{rgb}{0,0.5,0}
%\newcommand{\towrite}[1]{\textcolor{darkgreen}{[#1]}}
\newcommand{\todo}[1]{\textcolor{red}{\textbf{[#1]}}}

% Ajouts et progression
\def\modMF#1{\textcolor{teal}{#1}}
\def\modLP#1{\textcolor{magenta}{#1}}
\def\modMM#1{\textcolor{blue}{#1}}
\def\modOR#1{\textcolor{olive}{#1}}
%\def\modMF#1{#1} \def\modLP#1{#1} \def\modMM#1{#1} \def\modOR#1{#1} 




% Special notations
\newcommand{\ie}{i.e.,\ }
\newcommand{\Ie}{I.e.,\ }
\newcommand{\eg}{e.g.,\ }
\newcommand{\Eg}{E.g.,\ }
\newcommand{\resp}{resp.\ }
\newcommand{\Resp}{Resp.\ }

% Césures
\hyphenation{pa-ra-me-tri-za-tion}
\hyphenation{pa-ra-me-tri-za-tions}



\def\lastname{Folschette \emph{et al.}}

\begin{document}
\begin{frontmatter}
\title{Sufficient Conditions for Reachability in Automata Networks with Priorities}

\author[irccyn]{Maxime Folschette}
\ead{Maxime.Folschette@irccyn.ec-nantes.fr}
\author[lri,amib]{Loïc Paulevé}
\author[irccyn]{Morgan Magnin}
\author[irccyn]{Olivier Roux}

\address[irccyn]{LUNAM Universit\'e, \'Ecole Centrale de Nantes, IRCCyN UMR CNRS 6597\\
(Institut de Recherche en Communications et Cybern\'etique de Nantes)\\
1 rue de la No\"e - B.P. 92101 - 44321 Nantes Cedex 3, France.}

\address[lri]{CNRS, Laboratoire de Recherche en Informatique (LRI)\\
		Université Paris-Sud - CNRS UMR 8623, France}
\address[amib]{AMIB group, Inria Saclay, France}{}



\begin{abstract}
The Process Hitting is a recently introduced framework designed for the modelling of concurrent systems.
Its originality lies in a compact representation of both components of the model and its corresponding actions:
each action can modify the status of a component, and is conditioned by the status of at most one other component.
This allowed to define very efficient static analysis based on local causality to compute reachability properties.
However, in the case of cooperations between components (for example, when two components are supposed to interact with a third one only when they are in a given configuration), the approach leads to an over-approximated interleaving between actions, because of the pure asynchronous semantics of the model.

To address this issue, we propose an extended definition of the framework, including priority classes for actions.
In this paper, we focus on a restriction of the Process Hitting with two classes of priorities and a specific behaviour of the components, that is sufficient to tackle the aforementioned problem of cooperations.
We develop a new refinement for the under-approximation of the static analysis to give accurate results for this class of Process Hitting models.
Then we show that this class of models is sufficient to represent any
Process Hitting model with an arbitrary number of classes of priorities,
and even any Asynchronous Discrete Networks, either Boolean or multivalued.
Our method thus allows to efficiently under-approximate reachability properties
in Asynchronous Discrete Networks,
as it is illustrated on the model-checking of a signalling network of 94
components, which is unprecedented.
\end{abstract}
\begin{keyword}
discrete networks \sep
abstract interpretation \sep
reachability \sep
qualitative models \sep
systems biology
\end{keyword}
\end{frontmatter}



% vi:spell spelllang=en:
\section{Introduction}
\label{sec:intro}

Discrete modelling frameworks for biological networks is an active research field where formal methods have proven that they were very
powerful~\cite{thomas1990biological,deJong02,snoussi_logical_1993}.
Such a work started in the seventies, with~\cite{kauffman69,Thomas73}.
It was later enriched in many directions and widely used to elucidate many biological questions.
Among these questions, a major one is to understand precisely how biological systems evolve and behave; why and how they change their usual behaviours, etc.
These questions are strongly linked to the (possible or inevitable) reachability of some states.
The ultimate goal is to discover how it could be possible to prevent biological systems from reaching some pathological conditions.

Of course, such formal models on which analyses are performed are abstract representations of the actual studied systems.
They are associated with parameters that have to be synthesised %so as to be as much as possible fitting with the real systems having some observed behaviors.
to give the most faithful representation of the real systems with their observed behaviours.
As a matter of fact, the abstractions we get are more or less rough or accurate.
Prevalent formal frameworks for such modelling activities are state-transition systems or process algebras. % Petri nets
We developed a quite similar framework named the Process Hitting~\cite{PMR10-TCSB},
consisting in a restriction of these frameworks where the evolution of a component is determined by the state of at most one other component that does not evolve.
In a sense, these kind of actions are of the form $X + Y \rightarrow X + Y'$ where $X$ behaves like a catalyst molecule that “hits” another molecule $Y$ and changes it into $Y'$, without being changed itself.
Assuming catalysts are always available, this can represent any biochemical system made of monomolecular reactions, and can also represent catalytic networks such as metabolic networks.
%, with the aim of avoiding to build the whole state space in order to be able to tackle very large systems
Our motivation behind this framework was to design a model and analysis techniques adapted to biological modelling.
These analyses avoid to build the whole state space, which allows to tackle very large systems (that would have led to a huge number of states, hopelessly too huge to be analysed).
They are based on the fact that most biological models have few levels of expression per component:
in Boolean networks~\cite{kauffman69,Thomas73} there are only two levels per component, and in their multivalued equivalent~\cite{deJong02}, components rarely have more than four levels.

Besides, one further objective of our work is now %to be more accurate in the description of the studied systems.
to improve the accuracy of the description of the studied systems dynamics.
The idea for this is to introduce timing features into models:
we are interested in taking into account some knowledge about the relative length of some phenomena as it is a way to refute some models (or parameters) that are inconsistent with the observed dynamic behaviours.
In this paper, we are dealing with these timing properties through priorities,
that are based on the simple founding idea that actions with higher priority have to be processed before the ones with lower priority.
Furthermore, due to the Process Hitting framework restrictions, multimolecular reactions were previously not immediately available, but one could simulate them with an encoding called “cooperative sort”.
That encoding however introduces extra reactions,
that produce a temporal shift between the presence of the reactants
and the playability of the reaction.
This is where the priorities become useful, if not necessary:
the extra reactions can for example be given “infinite speed” (highest priority) so that they do not affect the behaviour of “normal” (lower priority) reactions, including the multimolecular ones.

\modn{
The approach used in this paper, however,
consists in considering another class of models,
that we call Asynchronous Automata Network,
and that allows to naturally model these cooperations
by defining several requisites for a reaction.
Moreover, such automata networks are still compatible with the notion of priority,
that can also be used to model different reaction rates in the model.
Asynchronous Automata Networks
(and, a fortiori, their restriction, the Process Hitting framework)
can be considered as a subset of Communicating Finite State Machines
or safe Petri Nets.
%or Synchronous Automata Networks,
%is inspired from the $\pi$-calculus,
Our work is thus related to such semantic ramifications
of extending traditional process algebras with the concept of priority
that allows for some transitions to be given precedence over others.
We focus here on static priorities that allow to model
time constraints such as reaction rates or delays between regulations,
but can also model preemptions between evolutions branches.
}

Until now, such a priority scheduling of the actions was not studied extensively in the different formal modelling frameworks dedicated to systems biology.
Nevertheless, such an attempt has been carried out for Petri nets by F.~Bause~\cite{Bause97},
and the concept of priority relations among the transitions of a network has also more recently been introduced by A.~K.~Wagler \textit{et~al.}~\cite{waw,WaglerW12} in order to allow modelling deterministic systems for biological applications.
The concept of priority is much straightforward in the approach of process algebras as it was shown by R.~Cleaveland and M.~Hennessy in~\cite{Cleaveland199058,Cleaveland99prioritiesin} and their abstractions and equivalences were studied in~\cite{Cleaveland:2007:PAP:1282576.1282847}.
It was later extended for applications in the field of systems biology by M.~John \textit{et~al.}~\cite{jlnu2010}.

\subsection*{Contributions}
\modn{
This paper is an extension of~\cite{FPMR13-CS2Bio}.
Its main contribution consists in the development of a method
that allows to efficiently compute under-approximations of the dynamics
of Asynchronous Automata Networks (AANs),
based on a similar work carried out on Process Hitting~\cite{PMR12-MSCS}.
Rather than using brute force or symbolic model checking techniques,
our method focuses on static analysis by abstract interpretation.
Our work aims at checking reachability properties that,
for a given initial state $\ctx$ and a given local state $x$,
have the form:
“Starting from state $\ctx$, is it possible to reach
a state in which $x$ is active?”,
thus answering either “True”
(in which case the reachability is formally proven)
or “Inconclusive”
(which however does not stand for “False”, that can be proven by other analysis tools,
such as the over-approximation proposed in~\cite{PMR12-MSCS}).
Moreover, the successive or simultaneous reachability of several local states
can also be checked,
and in the case of a “True” response,
an execution path satisfying the property can be produced.
Our work thus allows to efficiently analyse the dynamics
of regulation networks,
especially the widespread Logical Networks \cite{Thomas95,deJong02},
which encompass variables with a limited number of discrete values
alongside with evolution functions or focal parameters.
To show the scalability of our method, we apply it to a
large-scale biological model containing 94 components;
the under-approximation turns out to be conclusive in
all cases and results have been computed in hundredths of seconds,
thus overtaking the efficiency of usual model-checkers.
}

\modn{
AANs (and, a fortiori, their restriction, the Process Hitting)
can be considered as a subset of Communicating Finite State Machines
or safe Petri Nets.
%or Synchronous Automata Networks,
%is inspired from the $\pi$-calculus,
Our work is thus related to such semantic ramifications
of extending traditional process algebras with the concept of priority
that allows for some transitions to be given precedence over others.
We focus here on static priorities that allow to model
time constraints such as reaction rates or delays between regulations,
but can also model preemptions between evolutions branches.
Thus, although the under-approximation method presented in this paper
is initially developed on AANs without classes of priorities,
we show that adding classes of priorities into AANs
does not increase their expressivity.
Thus, we exhibit an encoding that allows to extend the range
of application of our method to AANs with classes of priorities.
}

\modn{
\pref{sec:ph} presents the Asynchronous Automata Networks (AANs)
without any notion of classes of priorities.
In \pref{sec:sa},
we develop our under-approximation method allowing to efficiently
compute reachability analysis;
we also show how to extract a valid execution path if the response is positive,
and propose two refinements in case one needs to check
successive or simultaneous reachability properties.
The framework of AANs with classes of priorities is given in \pref{sec:flattening},
alongside with an encoding into AANs without classes of priorities
(or, equivalently, with one class of priority).
Finally, \pref{sec:example} provides a detailed example and a large-scale example
of the application of our method,
and \pref{sec:ccl} gives a conclusion and a discussion about our work.
}

\modn{
The main additions in this paper compared to~\cite{FPMR13-CS2Bio} are
\pref{ssec:concret} that allows the extraction of a concrete trace of execution
when our static analysis method is conclusive,
\pref{ssec:ordered-ua} that refines our approach in the case
of a successive reachability in order to increase the conclusiveness,
and \pref{sec:flattening} which states that any ANN
with any number of classes of priorities
can be represented into a simple ANN
(or, equivalently, with only one class of priority),
thus extending the scope of our method.
We note however that the use of AANs without priorities alone
instead of Process Hitting models
allows to simplify the notations, but does not increase
the range of applicability of the results.
}



\subsection*{Notations}
\label{notations}
%We denote: $\segm{a}{b} = \{ a, a+1, \dots, b-1, b \}$.

\paragraph*{Sets}
If $A$ is a finite set,
$|A|$ is the cardinality of $A$
and $\powerset(A)$ is the power set of $A$.
$\sN$ is the set of natural numbers,
$\sN^* = \sN \setminus \{ 0 \}$ is the set of positive natural numbers,
$\sNN = \sN \setminus \{ 0, 1 \}$ is the set of natural numbers strictly greater than 1,
and $\segm{x}{y} = \{ x, x+1, \dots, y-1, y \}$ is the set of natural numbers from $x$ to $y$ included.
The Cartesian product of sets is denoted by $\times$;
if $z$ is a tuple of $n$ components, $\toset{z}$ denotes the corresponding set:
$\toset{z} = \{z_1, \cdots, z_n\}$.

\paragraph*{Sequences}
We denote by $\emptyseq$ the empty sequence.
If $n \in \sN$ and
$x = (x_i)_{i \in \segm{1}{n}}$ is a sequence of elements indexed by $i \in \segm{1}{n}$,
%we denote %$|x| = (b-a)+1$ the size of this sequence and
then $\indexes{x} = \segm{1}{n}$ is the set of indexes of this sequence.
Furthermore, if $a, b \in \indexes{x}$ with $a \leq b$,
then $x_{a..b} = (x_i)_{i \in \segm{a}{b}}$ is a subsequence of $x$.
%from element $n$ to element $m$ inclusive.
Finally, if $x$ is a sequence, $\toset{x}$ also denotes the corresponding set:
$\toset{x} = \{x_1, \cdots, x_{\card{x}}\}$.

\paragraph*{Digraph}
If $(V,E)$ is a directed graph whose $V$ are the nodes and $E \subseteq V\times V$ the edges,
the children of a node $n$ are given by
$\childs: V \to \powerset(V)$, with
$\childs(n) \DEF \{ m\in V\mid (n,m)\in E\}$;
its parents are given by
$\parents: V \to \powerset(V)$ with
$\parents(n) \DEF \{ m\in N\mid (m,n)\in E\}$;
and its successors are given by
$\conn_{(V,E)}(n)$ which is the least fixed point containing $n$
of the function
$\f{f}\conn:\powerset(V)\to \powerset(V)$
with $\f{f}\conn(W) \DEF \bigcup_{m\in W} \childs(m)$.

%\paragraph*{Functions and least fixed point}
%If $A$ and $B$ are sets,
%$f : A \rightarrow B$ denotes a function $f$ that maps the elements of $A$ to elements of $B$.
%If $f$ is a monotonically increasing and bounded function, then
%$\lfp{x_0}{x}{x'}$ is the least fixed point of the function $x \mapsto x'$ which is greater than $x_0$.



\section{The Process Hitting framework}

\towrite{Transition.}

\subsection{Definition of the Process Hitting with $k$ priorities}
\label{ssec:PH}
\towrite{Reformuler (actuellement trop proche de CMSB)}
A PH (\pref{def:PH}) gathers a finite number of concurrent \emph{processes} grouped into a finite set of \emph{sorts}.
A process belongs to a unique sort and is noted $a_i$ where $a$ is the sort and $i$ the identifier of the process within the sort $a$.
At any time, one and only one process of each sort is present; a state of the PH thus corresponds to the set of such processes.

The concurrent interactions between processes are defined by a set of \emph{actions} divided into classes of priorities.
Actions describe the replacement of a process by another of the same sort conditioned by the presence of at most one other process and by the fact that no other action of higher priority is playable in the current state of the PH.
An action is denoted by $\PHhit{a_i}{b_j}{b_k}$ where $a_i,b_j,b_k$ are processes of sorts $a$ and $b$.
It is required that $b_j \neq b_k$ and that $a=b\Rightarrow a_i=b_j$.
An action $h=\PHfrappe{a_i}{b_j}{b_k}$ is read as ``$a_i$ \emph{hits} $b_j$ to make it bounce to $b_k$'', and $a_i,b_j,b_k$ are called respectively \emph{hitter}, \emph{target} and \emph{bounce} of the action, and can be referred to as $\PHhitter(h), \PHtarget(h), \PHbounce(h)$, respectively.

% Process Hitting sans priorités
\begin{comment}
\begin{definition}[Process Hitting]\label{def:PH}
A \emph{Process Hitting} is a triple $(\PHs,\PHl,\PHa)$:
\begin{itemize}
\item $\PHs \DEF \{a,b,\dots\}$ is the finite set of \emph{sorts};
\item $\PHl \DEF \prod_{a\in\PHs} \PHl_a$ is the set of states with $\PHl_a = \{a_0,\dots,a_{l_a}\}$
the finite set of \emph{processes} of sort $a\in\Sigma$ and $l_a$ a positive integer with
	$a\neq b\Rightarrow \forall(a_i,b_j)\in\PHl_a\times\PHl_b,a_i\neq b_j$;
\item $\PHa \DEF \{ \PHfrappe{a_i}{b_j}{b_k}, \dots \mid
					(a,b)\in\PHs^2 \wedge (a_i,b_j,b_k)\in \PHl_a\times\PHl_b\times\PHl_b$ \\
	\hspace*{2cm} $\wedge b_j\neq b_k \wedge a=b\Rightarrow a_i=b_j\}$
			is the finite set of \emph{actions}.
\end{itemize}
$\PHproc$ denotes the set of all processes ($\PHproc \DEF \{ a_i\mid a\in\PHs \wedge a_i\in\PHl_a\}$).
\end{definition}
\end{comment}

\begin{definition}[Process Hitting with $k$ priorities] \towrite{Revoir terminologie}
  A \emph{Process Hitting with $k$ priorities}, where $k \in \sN^*$, is a triplet $\PH = (\PHs; \PHl; \PHa^{\langle k \rangle})$, with $\PHa^{\langle k \rangle} = (\PHa^{(1)}; \dots; \PHa^{(k)})$ a $k$-uplet, where:
  \begin{itemize}
    \item $\PHs \DEF \{a,b,\dots\}$ is the finite set of \emph{sorts}.
    \item $\PHl \DEF \prod_{a \in \PHs} \PHl_a$ is the finite set of states, where $\PHl_a = \{a_0 ; \ldots ; a_{l_a}\}$ is the finite set of \emph{processes} of sort $a \in \PHs$ and $l_a \in \sN^*$. Each process belongs to a unique sort: $\forall (a_i; b_j) \in \PHl_a \times \PHl_b, a \neq b \Rightarrow a_i \neq b_j$.
    \item $\forall n \in \llbracket 1; k \rrbracket, \PHa^{(n)} \DEF \{\PHfrappe{a_i}{b_j}{b_k} \mid (a; b) \in \PHs^2 \wedge (a_i; b_j; b_k) \in \PHl_a \times \PHl_b \times \PHl_b \wedge b_j \neq b_k \wedge a = b \Rightarrow a_i = b_j \}$ is the finite set of \emph{actions of priority $n$}.
  \end{itemize}
  We call $\PHproc \DEF \bigcup_{a \in \PHs} \PHl_a$ the set of all processes, and $\PHh \DEF \bigcup_{n \in \segm{1}{k}} \PHh^{(n)}$ the set of all actions.
\end{definition}

\noindent
The sort of a process $a_i$ is referred to as $\PHsort(a_i)=a$ and the set of sorts present in an action $h\in\PHa$ as $\PHsort(h) = \{\PHsort(\PHhitter(h)),\PHsort(\PHtarget(h))\}$.
Given a state $s\in \PHl$, the process of sort $a\in\PHs$ present in $s$ is denoted by $\PHget{s}{a}$, that is the $a$-coordinate of the state $s$.
If $a_i\in \PHl_a$, we define the notation $a_i\in s \EQDEF \PHget{s}{a}=a_i$.

An action $h = \PHhit{a_i}{b_j}{b_k} \in \PHa^{(n)}$ of priority $n$ is \emph{playable} in $s \in \PHl$ iff $\PHget{s}{a} = a_i$, $\PHget{s}{b} = b_j$ and $\forall m < n, \forall g \in \PHa^{(m)}$, $g$ is not playable in $s$.
%Dans un tel cas, l'état résultant du jeu de l'action $h$ dans $\etat$ est dénoté $(\etat \PHjoue h)$, où $\PHget{(\etat \PHjoue h)}{b} = b_k$ et $\forall c \in \PHs, c \neq b \Rightarrow \PHget{(\etat \PHjoue h)}{c} = \PHget{\etat}{c}$.
%
%An action $h=\PHhit{a_i}{b_j}{b_k} \in\PHa$ is \emph{playable} in $s \in L$ if and only if $\PHget{s}{a} = a_i$ and $\PHget{s}{b} = b_j$.
In such a case, $(s \PHplay h)$ stands for the state resulting from the play of the action $h$ in $s$, that is $\PHget{(s \PHplay h)}{b} = b_k$ and $\forall c\in\PHs, c\neq b, \PHget{(s\PHplay h)}{c} = \PHget{s}{c}$.
For the sake of clarity, for all $h, h'\in\PHa$, $((s\PHplay h)\PHplay h')$ is abbreviated as $(s\PHplay h\PHplay h')$.

\begin{definition}[Process Hitting $n$-reduction] \towrite{Revoir terminologie : “reduction” ? “restriction” ?}
  If $\PH = (\PHs; \PHl; \PHa^{\langle k \rangle})$ is a Process Hitting with $k$ priorities and $n \in \segm{1}{k}$, we denote $\restriction{\PH}{n}$ \towrite{$\PH_{|n}$ $\PH'$} the $n$-reduction of $\PH$ the following Process Hitting with $n$ priorities, where:
  $$\restriction{\PH}{n} = (\PHs; \PHl; \restriction{\PHa}{n})$$
  and:
  $$\restriction{\PHa}{n} = (\PHa^{(1)}; \dots; \PHa^{(n)})$$
  
  Furthermore, we define: $\restriction{\PH}{0} = (\PHs; \PHl; \restriction{\PHa}{0})$, where $\restriction{\PHa}{0} = ()$ is a $0$-uplet.
\end{definition}

\begin{comment} %example*
\pref{fig:runningPH-1} represents a PH $(\PHs,\PHl,\PHa)$ with
$\PHs = \{a,b,c\}$,
$\PHl_a = \{a_0,a_1,a_2\}$,
$\PHl_b = \{b_0, b_1\}$,
$\PHl_c = \{c_0, c_1\}$, and
\begin{align*}
\PHa & = \{
	\PHfrappe{a_2}{b_1}{b_0},
&&  \PHfrappe{b_0}{a_2}{a_1},
&&	\PHfrappe{c_0}{a_2}{a_1},\\
&&& \PHfrappe{b_0}{a_1}{a_0},
&&	\PHfrappe{c_0}{a_1}{a_0},\\
&&& \PHfrappe{b_1}{a_0}{a_1},
&&	\PHfrappe{c_1}{a_0}{a_1},\\
&&& \PHfrappe{b_1}{a_1}{a_2},
&&	\PHfrappe{c_1}{a_1}{a_2} \}\enspace.
\end{align*}
The action $h=\PHfrappe{b_1}{a_1}{a_2}$ is playable in the state
$s = \PHstate{b_1,a_1,c_0}$; and $s\PHplay h=\PHstate{b_1,a_2,c_0}$.
%
\begin{figure}[t]
\centering
\scalebox{1.3}{
\begin{tikzpicture}
\TSort{(0,0)}{a}{3}{r}
\TSort{(-3,0.5)}{b}{2}{l}
\TSort{(3,0.5)}{c}{2}{r}
%
\THit{b_1}{very thick}{a_0}{.west}{a_1}
\THit{b_1}{very thick}{a_1}{.north west}{a_2}
\THit{b_0}{}{a_2}{.west}{a_1}
\THit{b_0}{}{a_1}{.west}{a_0}
%
\path[bounce, bend left=60]
\TBounce{a_1}{very thick}{a_2}{.south}
\TBounce{a_0}{very thick}{a_1}{.south}
;
\path[bounce, bend right=60]
\TBounce{a_2}{}{a_1}{.north}
\TBounce{a_1}{}{a_0}{.north}
;
%
\THit{c_1}{very thick}{a_0}{.east}{a_1}
\THit{c_1}{very thick}{a_1}{.north east}{a_2}
\THit{c_0}{}{a_2}{.east}{a_1}
\THit{c_0}{}{a_1}{.east}{a_0}
%
\path[bounce, bend right=60]
\TBounce{a_1}{very thick}{a_2}{.south east}
\TBounce{a_0}{very thick}{a_1}{.south east}
;
\path[bounce, bend left=60]
\TBounce{a_2}{}{a_1}{.north}
\TBounce{a_1}{}{a_0}{.north east}
;
%
\THit{a_2}{bend right}{b_1}{.north east}{b_0}
\path[bounce, bend left=80]
\TBounce{b_1}{out=100,in=140}{b_0}{.north}
;
\end{tikzpicture}
}
\caption{\label{fig:runningPH-1}
A Process Hitting (PH) example.
Sorts are represented by labeled boxes, and processes by circles (ticks are
the identifiers of the processes within the sort, for instance, $a_0$ is the
process ticked $0$ in the box $a$).
An action (for instance $\PHfrappe{b_1}{a_1}{a_2}$) is represented by a pair of
directed arcs, having the hit part ($b_1$ to $a_1$) in plain line and the bounce
part ($a_1$ to $a_2$) in dotted line.
Actions involving $b_1$ or $c_1$ are in thick lines.
%The current state is represented by the grayed processes:
%$\PHstate{a_0,b_1,c_0,d_0}$.
}
\end{figure}

This PH example actually models a BRN where the component $a$ has three qualitative
levels and components $b$ and $c$ are boolean.
In this BRN, $b$ and $c$ activate $a$, while $a$ inhibits $b$.
The inhibition of $b$ by $a$ is only effective when $a$ is at level $2$;
in the other cases, $b$ cannot evolve in any direction.
The activation of $a$ by $b$ ($c$) is encoded by the actions making the level of $a$ increase (resp.
decrease) when $b$ ($c$) is present (resp. absent).
It is worth noticing that the activation of $a$ by $b$ ($c$) is independent from $c$ ($b$).
This may express a lack of knowledge on the cooperation between these two regulators:
we thus model an over-approximation of the possible actions.
\end{example*}
\end{comment}

\subsection{Modeling cooperation.}
\towrite{Sortes coopératives avec mise à jour prioritaire (pas de décalage temporel)}
As described in \cite{PMR10-TCSB}, the cooperation between processes to make another bounce can be
expressed in PH by building a \emph{cooperative sort}.
\pref{fig:PH-cooperativity} shows an example of cooperation between processes $b_1$ and $c_1$ to
make $a_1$ bounce to $a_2$:
a cooperative sort $bc$ is defined with 4 processes (one for each sub-state of the presence of
processes $b_1$ and $c_1$).
For the sake of clarity, the $bc$ processes are indexed using the sub-state they represent.
Hence, $bc_{01}$ represents the sub-state $\PHstate{b_0,c_1}$, and so on.
Each process of sort $b$ and $c$ hit $bc$ to make it bounce to the process reflecting the status of the sorts $b$
and $c$ (e.g., $\PHfrappe{b_1}{bc_{00}}{bc_{10}}$ and $\PHfrappe{b_1}{bc_{01}}{bc_{11}}$).
Then, it is the process $bc_{11}$ which hits $a_1$ to make it bounce to $a_2$ instead of the
independent hits from $b_1$ and $c_1$.

\begin{figure}[p]
\centering
\scalebox{1.3}{
\begin{tikzpicture}
\TSort{(0,0)}{b}{2}{t}
\TSort{(0,-3.8)}{c}{2}{b}
\TSort{(4.5,-3)}{a}{3}{r}

\TSetTick{bc}{0}{00}
\TSetTick{bc}{1}{01}
\TSetTick{bc}{2}{10}
\TSetTick{bc}{3}{11}
% \TSetSortLbcel{bc}{$\neg a\wedge b$}
\TSort{(-0.5,-2)}{bc}{4}{b}

\THit{b_1}{very thick,bend right}{bc_0}{.north}{bc_2}
\THit{b_1}{very thick,bend right}{bc_1}{.north}{bc_3}
\THit{b_0}{}{bc_2}{.north west}{bc_0}
\THit{b_0}{}{bc_3}{.north west}{bc_1}

\THit{c_0}{}{bc_1}{.south}{bc_0}
\THit{c_0}{}{bc_3}{.south}{bc_2}
\THit{c_1}{very thick}{bc_0}{.south}{bc_1}
\THit{c_1}{very thick}{bc_2}{.south}{bc_3}

\path[bounce, bend right=25]
\TBounce{bc_2}{}{bc_0}{.north east}
\TBounce{bc_3}{}{bc_1}{.north east}
;
\path[bounce, bend left=80, distance=30]
\TBounce{bc_0}{very thick}{bc_2}{.north}
\TBounce{bc_1}{very thick}{bc_3}{.north}
;
\path[bounce, bend right]
\TBounce{bc_0}{very thick}{bc_1}{.west}
\TBounce{bc_2}{very thick}{bc_3}{.west}
;
\path[bounce, bend left]
\TBounce{bc_3}{}{bc_2}{.east}
\TBounce{bc_1}{}{bc_0}{.east}
;

\THit{bc_3}{}{a_1}{.west}{a_2}
\path[bounce, bend left=40]
\TBounce{a_1}{}{a_2}{.south west}
;

\end{tikzpicture}
}

\caption{\label{fig:PH-cooperativity}
\towrite{Ajouter priorités}
A PH modeling a cooperativity between $b_1$ and $c_1$ to make
$a_1$ bounce to $a_2$.
Actions involving $b_1$ or $c_1$ are in thick lines.
}
\end{figure}



\begin{figure}
  \centering
  \begin{tikzpicture}
    \TSort{(0,0)}{a}{2}{l}
    \TSort{(2,0)}{b}{2}{r}
    \TSort{(4,0)}{c}{2}{r}

    \THit{a_1}{prio}{b_1}{.west}{b_0}
    \THit{b_1}{}{c_0}{.north west}{c_1}

    \path[bounce, bend right]
    \TBounce{b_1}{}{b_0}{.north west}
    ;
    \path[bounce, bend left]
    \TBounce{c_0}{}{c_1}{.south west}
    ;
  \end{tikzpicture}
  \label{fig:ph-conca}
  \caption{A simple PH example depicting “backward concurrency” between two actions belonging to different classes of priorities.}
\end{figure}



\begin{figure}
  \centering
  \begin{tikzpicture}
    \TSort{(0,0)}{a}{2}{l}
    \TSort{(2,0)}{b}{2}{r}
    \TSort{(4,0)}{c}{3}{r}

    \THit{a_1}{}{b_0}{.west}{b_1}
    \THit{b_0}{prio}{c_1}{.south west}{c_0}
    \THit{b_1}{prio}{c_1}{.west}{c_2}

    \path[bounce, bend left]
    \TBounce{b_0}{}{b_1}{.south west}
    \TBounce{c_1}{}{c_2}{.south west}
    ;
    \path[bounce, bend right]
    \TBounce{c_1}{}{c_0}{.north west}
    ;
  \end{tikzpicture}
  \label{fig:ph-concb}
  \caption{A simple PH example depicting “forward concurrency” between two actions belonging to different classes of priorities.}
\end{figure}



\begin{comment}
\begin{figure}[p]
\centering
\scalebox{1.3}{
\begin{tikzpicture}
\path[use as bounding box] (-4,-1.9) rectangle (4.5,3.9);
%
\TSort{(0,0)}{a}{3}{l}
\TSort{(3, 3)}{b}{2}{t}
\TSort{(3,-1)}{c}{2}{b}
%
\TSetTick{bc}{0}{00}
\TSetTick{bc}{1}{01}
\TSetTick{bc}{2}{10}
\TSetTick{bc}{3}{11}
% \TSetSortLbcel{bc}{$\neg a\wedge b$}
\TSort{(-3,-0.5)}{bc}{4}{l}
%
\THit{bc_3}{}{a_1}{.north west}{a_2}
\THit{bc_0}{}{a_1}{.south west}{a_0}
\path[bounce]
\TBounce{a_1}{bend left}{a_2}{.south west}
\TBounce{a_1}{bend right}{a_0}{.north west}
;
%
\THit{b_0}{}{a_2}{.east}{a_1}
\THit{b_1}{}{a_0}{.north east}{a_1}
\path[bounce]
\TBounce{a_2}{bend left}{a_1}{.north east}
\TBounce{a_0}{bend right=20}{a_1}{.south}
;
%
\THit{c_0}{bend right}{a_2}{.south east}{a_1}
\THit{c_1}{bend right}{a_0}{.east}{a_1}
\path[bounce]
\TBounce{a_2}{bend left=20}{a_1}{.north}
\TBounce{a_0}{bend right=30}{a_1}{.south east}
;
%
\path[dashed,hit]
	(2,-1.3) edge[bend left=10] (-2.3,-0.7)
	(2.2, 3.3) edge[bend right=10] (-2.3,3)
;
%
\THit{a_2}{bend left,out=40,in=80}{b_1}{.north west}{b_0}
\path[bounce, bend right]
\TBounce{b_1}{}{b_0}{.east}
;
%
\end{tikzpicture}
}
%
\caption{\label{fig:runningPH-2}
PH resulting from the refinement of the one in \pref{fig:runningPH-1} by the
specification of several cooperations.
The actions from $b$ and $c$ to the cooperative sort $bc$ are identical to those defined in
\pref{fig:PH-cooperativity} and are represented here by a single dashed arc.
}
\end{figure}
\end{comment}

We note that cooperative sorts are standard PH sorts and do not involve any
special treatment regarding the semantics of related actions.

When the number of cooperating processes is large, it is possible to chain several cooperative sorts
to prevent the combinatoric explosion of the number of processes created within cooperative sorts.
For instance, if $b_1$, $c_1$, and $d_1$ cooperate, one can create a cooperative sort $bc$ with 4
processes reflecting the presence of $b_1$ and $c_1$, and a cooperative sort $bcd$ with 4 processes
reflecting the presence of $bc_{11}$ and $d_1$.  Such constructions are helpful in PH
as the static analysis of dynamics developed in \cite{PMR12-MSCS} does not suffer from the number of
sorts, but on the number of processes within a single sort.

While the construction of cooperation in PH allows to encode any boolean functions
between cooperating processes \cite{PMR10-TCSB}, it is worth noticing they introduce a temporal
shift in their application. \towrite{Adapter avec priorités}
This allows the existence of interleaving of actions leading to a cooperative sort representing a
past sub-state of the presence of the cooperative processes.
The resulting behavior is then an over-approximation
of the realization of an instantaneous cooperation.

\begin{example}
The PH in \pref{fig:runningPH-2} results from the refinement of the PH in \pref{fig:runningPH-1}
where several cooperations have been specified.
In particular, the bounce to $a_2$ is the result of a cooperation between $b_1$ and $c_1$; and the
bounce to $a_0$ of a cooperation between $b_0$ and $c_0$.
Hence, this PH expresses a BRN where $a$ requires both $b$ and $c$ active to reach its
highest level, and $a$ does not become inactive unless both $b$ and $c$ are inactive.
\end{example}

\subsection{Hypothesis} \towrite{Reformuler}

In this section, we describe several restrictions.
Although the definition of a Process Hitting with $k$ priorities allows a wide range of models, the method given in the next section only applies to Process Hitting models that regard these restrictions.

Let $\PH$ be a PH model on which we want to apply the static analysis.

We define the set of chains of actions $\PHh(b, a)$ between two sorts $b$ and $a$ as below:
\begin{align*}
\PHh(b, a) = \{ (h_i)_{i \in \segm{0}{s}} \in (\restriction{\PHh}{\prio(b)})^{s+1} &\mid s \in \sN, \PHsort(\PHhitter(h_0)) = b \wedge \PHsort(\PHtarget(h_n)) = a \\
  & \wedge \forall i \in \segm{1}{s}, \PHsort(\PHtarget(h_{i-1})) = \PHsort(\PHhitter(h_{i})) \}
\end{align*}

We define the neighbor \towrite{Revoir terminologie : “neighborhood” ? “vicinity” ?} sorts of any sort $a$ as :
$$
\Vs(a) = \{ b \in \PHs \mid \PHh(b, a) \neq \emptyset \}
$$

We define the neighbor actions of any sort $a$ as :
$$
\Vh(a) = \{ h \in \restriction{\PHh}{\prio(a)} \mid \PHsort(\PHhitter(h)) \in \Vs(a) \wedge \PHsort(\PHtarget(h)) \in \Vs(a) \cup \{ a \} \}
$$

This PH has to regard the two following restrictions.

\paragraph{Cycle-freeness of $\restriction{\PH}{k-1}$}
\towrite{Définition formelle ? + justification}
The model $\restriction{\PH}{k-1}$ contains no cycles. Cycles are still allowed in $\PH$ however.

\paragraph{Local priorities}
\towrite{Justification}
All actions hitting the same sort belong to the same class of priority:
$$\forall h_1, h_2 \in \PHh, \PHtarget(h_1) = \PHtarget(h_2) \Rightarrow \prio(h_1) = \prio(h_2)$$
Furthermore, for all sort $a$, we denote $\prio(a)$ the priority of the actions hitting $a$, if any:
$$\forall a \in \PHs, \exists h \in \PHh, \PHtarget(h) = a \Rightarrow \prio(a) = \prio(h)$$
\towrite{Ou : $\nexists h \Rightarrow \prio(a) = 0$}

These restrictions bring new interesting properties to the PH models regarding them.

\pref{th:vplay} tells that for any action $h = \PHhit{a_i}{b_j}{b_k}$, if no action can be played in the neighborhood of $a$ in a given state, then $h$ can be played after a series of hits that do not prevent it to be fired. Furthermore, if the requisites of \pref{th:vplay} are true, then no action of this series of hits belongs to $V(a)$, as stated in \pref{co:vplay}.
\begin{theorem}
\label{th:vplay}
  $\forall s \in \PHl, \forall h = \PHhit{a_i}{b_j}{b_k} \in \PHh, (\PHget{s}{a} = a_i \wedge \PHget{s}{b} = b_j \wedge $\\
  $\forall h' \in \Vh(a), \PHsort(\PHtarget(h')) \neq a \Rightarrow \text{$h'$ cannot be played in $s$}) \Rightarrow$\\
  $(\exists \delta \in \Sce, \text{$h$ can be played in $s \PHplay \delta$})$
\end{theorem}

\begin{proof}
  Because of the cycle-freeness of $\restriction{\PH}{k-1}$, $\restriction{\PH}{\prio(a)}$ is also cycle-free and there is a $\delta \in \Sce$ so that $\forall h' \in \PHh$, $h'$ cannot be played in $s \PHplay \delta$. Furthermore, $\forall h' \in \delta, \PHsort(\PHtarget(h')) \neq a$ by construction of $\Vs(a)$.
\end{proof}


\begin{corollary}
  \label{co:vplay}
  $\forall s \in \PHl, \forall h = \PHhit{a_i}{b_j}{b_k} \in \PHh, (\PHget{s}{a} = a_i \wedge \PHget{s}{b} = b_j \wedge $\\
  $(\forall h' \in \Vh(a), \PHsort(\PHtarget(h')) \neq a \Rightarrow \text{$h'$ cannot be played in $s$}) \Rightarrow$\\
  $\exists \delta \in \Sce, (\forall n \in \indexes{\delta}, \forall h' \in \delta_n, \PHsort(\PHtarget(h')) \notin \Vh(a)) \wedge (\text{$h$ can be played in $s \PHplay \delta$})$
\end{corollary}

\pref{th:totalss} states that from any state, a steady state is eventually reached. A local variant of this theorem, given in \pref{co:totalss}, can be derived,
where, for all $\Omega \subset \PHs$, $$\restriction{\PHl}{\Omega} = \underset{a \in \Omega}{\times} \PHl_a$$ and
for all $H \subset \PHh$, $$\restriction{\Sce}{H} = \{ \delta \in \Sce \mid \forall n \in \indexes{\delta}, \delta_n \in H \} \enspace .$$
\begin{theorem}
\label{th:totalss}
  $\forall s \in \PHl, \exists \delta \in \Sce$, no action can be played in $s \PHplay \delta$.
\end{theorem}

\begin{corollary}
\label{co:totalss}
  $\forall a \in \PHs, \forall s \in \restriction{\PHl}{V(a)}, \exists \delta \in \restriction{\Sce}{V(a)}$, no action can be played in $s \PHplay \delta$.
\end{corollary}



\pref{th:autohits} states that any sequence $\zeta$ of self-actions can be played eventually.
\begin{theorem}
\label{th:autohits}
  Let $a \in \PHs$, $s \in \PHl$ and $\zeta \in \BS(a)$, with $n = |\indexes{\zeta}|$ so that $\forall i \in \segm{1}{n}, \PHsort(\PHhitter(\zeta_i)) = a$.
  %If no action can be played in $V(a)$ in state $s$, then
  $\exists (\delta_i)_{i \in \segm{1}{n}} \in \Sce$ so that $\forall i \in \segm{1}{n}$, $\zeta_i$ can be played in $s \PHplay \delta_1 \PHplay \zeta_1 \PHplay \dots \PHplay \delta_i$. 
\end{theorem}

\begin{proof}
  With \pref{co:vplay} applied iteratively.
\end{proof}

\towrite{Théorème nécessaire pour la suite: une séquence d'actions dans $\BS(a)$ peut être jouée si précédée ou entrelacée par des $\zeta$ (?)}


\section{Static analysis}\label{sec:sa}

\towrite{Mettre de la colle}

\begin{definition}[Objective ($\Obj$)]
\label{def:obj}
  The reachability of a process $a_j$ prof a process $a_i$ is called an \emph{objective}, noted $\PHobj{a_i}{a_j}$.
  The set of all objectives is called $\Obj = \{ \PHobj{a_i}{a_j} \mid a \in \PHs \wedge (a_i, a_j) \in \PHl_a^2 \}$.
  For an objective $P \in \Obj$, where $P = \PHobj{a_i}{a_j}$, $\PHsort(P) = a$, $\PHtarget(P)=a_i$, $\PHbounce(P)=a_j$.
  An objective $P$ is \emph{trivial} is $\PHtarget(P)=\PHbounce(P)$.
\end{definition}

\begin{definition}[Context $\ctx$ ($\Ctx$)]
\label{def:context}
  A \emph{context} $\ctx$ associates to each sort in $\PHs$ a non-empty subset of its processes:
  $\forall a \in \PHs, \PHget{\ctx}{a} \subseteq \PHl_a \wedge \PHget{\ctx}{a} \neq \emptyset$.
  $\Ctx$ is the set of all contexts.
\end{definition}

For a given context $\ctx$, we note $a_i \in \ctx$ if and only if $a_i \in \PHget{\ctx}{a}$, and
$ps \in \powerset(\Proc), ps \subseteq \ctx \Leftrightarrow \forall a_i \in ps, a_i \in \ctx$.
%The override of a context $\ctx$ by a set of processes $ps$ is noted $\ctx \Cap ps$ (\defref{ctxcap}).
%Par exemple, $\state{a_1,a_2,b_1,c_1}\Cap\{ a_3, b_2, b_3 \} = \state{a_3,b_2,b_3,c_1}$.
\begin{definition}[$\Cap: \Ctx \times \powerset(\PHproc) \mapsto \Ctx$]
\label{def:ctxcap}
  For a given context $\ctx\in\Ctx$ and a set of processes $ps \in \powerset(\PHproc)$,
  the override of $\ctx$ by $ps$ is noted $\ctx \Cap ps$ and is defined by
  \[ \forall a \in \PHs, \PHget{(\ctx \Cap ps)}{a} =
  \begin{cases}
    \{ p \in ps \mid \PHsort(p)=a \} & \text{if } \exists p \in ps, \PHsort(p)=a,\\
    \PHget{\ctx}{a} & \text{else.}
  \end{cases}
  \]
\end{definition}

A scenario $\delta \in \Sce$ is \emph{playable} in a context $\ctx$ if and only if $\supp(\delta) \subseteq \ctx$. 
The play of $\delta$ in $\ctx$ is denoted $\ctx \PHplay \delta$ where $\ctx \PHplay \delta = \ctx \Cap \ceil(\delta)$.

\begin{definition}[Objective sequence ($\OS$)]
\label{def:OS}
\towrite{Simplifier / réécrire en dehors d'une def}
  An \emph{objective sequence} is a sequence $\w = P_1 \concat \dots \concat P_{|\w|}$,
  where $\forall n \in \indexes{\w}, \w_n \in \Obj$ and $a_i = \PHtarget(\w_n) \Rightarrow \last_a(\w_{1..n-1}) \in \{ \varnothing, a_i \}$.
  The set of all objective sequences is denoted by $\OS$.
  The definitions of $\last_a$ \todo{À définir}, $\first_a$ \todo{À définir}, $\supp$ \todo{À définir} and $\ceil$ \todo{À définir}
  are simply derived by omitting the case of hitters.
\end{definition}

\begin{definition}[Bounce sequence ($\BS$)]
\label{def:bs}
  A \emph{bounce sequence} $\zeta$ is a sequence of actions so that $\forall n \in \indexes{\zeta}, n < |\zeta|, \PHbounce(\zeta_{n}) = \PHtarget(\zeta_{n+1})$.
  $\BS$ denotes the set of all bounce sequences, and
  $\BS(P)$ denotes the set of bounce sequences \emph{solving} an objective $P$:
  \[
    \BS(\PHobj{a_i}{a_j}) = \{ \zeta \in \BS \mid \PHtarget(\zeta_1)= a_i \wedge \PHbounce(\zeta_{|\zeta|}) = a_j \} \enspace.
  \]
  Obviously, $\BS(\obj{a_i}{a_i}) = \{\emptyseq\}$; and $\BS(\obj{a_i}{a_j}) = \emptyset$ if there is no way to reach $a_j$ from $a_i$.
\end{definition}

\begin{definition}[$\aBS:\Obj \mapsto \powerset(\Proc)$]
\label{def:aBS}
  \[
    \aBS(P) = \{ \abstr{\zeta} \mid \zeta \in \BS(P), \nexists \zeta' \in \BS(P), \abstr{\zeta'} \subsetneq \abstr{\zeta} \} \enspace,
  \]
  where $\abstr{\zeta} = \{ \PHhitter(\zeta_n) \mid  n \in \indexes{\zeta} \wedge \PHsort(\PHhitter(\zeta_n)) \neq \PHsort(P) \}$.
\end{definition}



\subsection{Under-approximation}

\begin{definition}[$\concr: \OS \mapsto \powerset(\Sce)$]
\label{def:concr}
\towrite{Simplifier ? Supprimer la def formelle ?}
  For a given $\w \in \OS$, $\concr(\w)$ is the set of scenarios concretising $\w$ in the context $\ctx$:
  \begin{align*}
    \concr(\w) = \{ \delta \in \Sce \mid & (\w^\vartriangle = \emptyseq \wedge \delta = \emptyseq) 
      \vee (\w^\vartriangle \neq \emptyseq \wedge \supp(\delta) \subseteq \ctx
    \\ &
      \wedge \exists \phi:\indexes{\w} \mapsto \indexes{\delta}, (\forall n, m \in \indexes{\w}, n < m \Leftrightarrow \phi(n) \leq \phi(m)) 
    \\ &
      \wedge \forall n \in \indexes{\w}, \PHbounce(\w_n) \in \ctx \PHplay \delta_{1..\phi(n)})
    \}
    \enspace,
  \end{align*}
  where $\omega^\vartriangle$ refers to the sequence of objectives $\omega$ where the trivial objectives have been removed.
\end{definition}
%
\begin{definition}[$\concr: \powerset(\OS) \mapsto \powerset(\Sce)$]
\label{def:concr-set}
  $\concr(\W) = \{ \delta \in \concr(\w) \mid \w \in \W \} \enspace.$
\end{definition}

\begin{definition}[$\uconcr: \OS \mapsto \powerset(\Sce)$]
\label{def:uconcr}
  \[
  \uconcr(\w) = 
  \begin{cases}
    \concr(\w) & \text{if } \forall s \in \PHl, s \subseteq \ctx, \exists \delta \in \concr(\w), \supp(\delta) \subseteq s \\
    \emptyset & \text{else.}
  \end{cases}
  \]
\end{definition}
% 
\begin{theorem}
\label{th:uconcr-ctx}
  $\ctx' \subseteq \ctx \wedge \uconcr(\w) \neq \emptyset \Longrightarrow \muconcr_{\ctx'}(\w) \neq \emptyset \enspace.$
\end{theorem}
% 
\begin{definition}[$\uconcr: \powerset(\OS) \mapsto \powerset(\Sce)$]
\label{def:uconcr-set}
  $\uconcr(\Omega) = \{ \delta \in \uconcr(\w) \mid \w \in \Omega\}$
\end{definition}




%%% N'est plus utile (?)
%Plus haute priorité :
%$$\forall a \in \PHs, \priomax(a) = \max_{h \in \PHh, \PHtarget(h) = a}(\prio(h))$$

%%% N'est plus utile (?)
%Processus possibles :
%\begin{align*}
%\procs((\cwSol,\cwReq,\cwCont)) = \{ p \in \PHproc &\mid \exists (P,ps) \in \cwSol, p \in ps
%\\ & \vee p = \PHtarget(P)
%\\ & \vee (P \neq \omega \Rightarrow p = \PHbounce(P)) \}
%%\\ & \vee \exists h \in \BS(P), (p = \PHhitter(h) \vee p = \PHbounce(h)) \}
%\end{align*}

The set of all processes involved in an abstract structure $\cwB$ is given in \pref{def:allprocs}.
\begin{definition}
\label{def:allprocs}
  \begin{align*}
  &\allprocs((\cwSol,\cwReq,\cwRSP,\cwCont)) = \{ p \in \PHproc \mid \exists (P,ps) \in \cwSol, \\
    &\qquad\qquad p \in ps \vee p = \PHtarget(P) \vee p = \PHbounce(P) \} \\
  %  &\qquad\qquad \vee p = \PHtarget(P) \vee \exists h \in \BS(P), (p = \PHhitter(h) \vee p = \PHbounce(h)) \\
  %  &\qquad\qquad \vee \exists(Q, \PHobj{p}{\PHbounce(Q)} \in \cwPrioCont \}
  \end{align*}
\end{definition}

\subsection{Local fixed points / possible bounces}

%%% Pas utile
%Points fixes possibles : pour toute sorte $a$ et tout contexte partiel $\ctx$ sur $V^n(a)$:
%\begin{align*}
%  \pfp_\ctx: \PHs &\rightarrow \wp(\PHproc) \\
%  a &\mapsto \{ s \in \underset{b \in V^n(a)}{\times} \PHl_b \mid \text{$s$ est accessible depuis $\ctx$ et aucune action n'y est jouable} \}
%\end{align*}

\begin{comment}
Points fixes possibles : pour toute sorte $a$ et tout contexte $\ctx$:
\begin{align*}
  \pfp_\ctx: \PHs &\rightarrow \wp(\PHproc) \\
  a &\mapsto \{ s \PHplay \delta \in \restriction{\PHl}{\Vs(a)} \mid s \in \restriction{\ctx}{\Vs(a)} \wedge \delta \in \restriction{\Sce}{\Vs(a)} \\
  &\qquad\qquad\qquad\qquad \wedge \forall \PHhit{b_i}{c_j}{c_k} \in \Vh(a), \PHget{(s \PHplay \delta)}{b} \neq b_i \vee \PHget{(s \PHplay \delta)}{c} \neq c_j \}
  %\text{$h$ n'est pas jouable dans $s \PHplay \delta$} \}
\end{align*}
%
Processus rencontrés comme résultats d'un point fixe :
\begin{align*}
\pfpprocs_\ctx(a) = \{ \PHget{s}{a} \mid s \in \pfp_\ctx(a) \}
\end{align*}
%
Le nouvel ensemble de processus à prendre en compte à chaque itération du pppf :
\begin{align*}
\newprocs_\ctx(\myB) = %\allprocs(\myB) \cup \bigcup_{a \in \PHs} \pfpprocs_{\allprocs(\myB)}(a)
  \{ a_k \in \PHproc &\mid a_k \in \allprocs(\myB) \vee \\
  & (a_i, \PHobjp{a}{j}{i}) \in \cwReq \wedge a_k \in \pfpprocs_{\ctx \Cap \allprocs(\myB)}(a) \wedge \\
  & \quad (\exists (P, ps) \in \cwSol, a_i \in ps \wedge \prio(\PHsort(P)) > \prio(a) \\
  & \quad \vee \exists (\PHobjp{a}{j}{i}, ps) \in \cwSol, \exists p \in ps, \exists (p, P) \in \cwReq,\\
  & \qquad \text{$P$ is not trivial} \wedge \prio(\PHsort(P)) > \prio(a) \}
\end{align*}
\end{comment}

\begin{comment}
=================
%
Nouvelle version des états stables possibles: pour tout contexte $\ctx$:
\begin{align*}
  \pfp_\ctx = \{ s \PHplay \delta \in \PHl &\mid s \in \ctx \wedge \delta \in \restriction{\Sce}{k-1} \\
  & \qquad \wedge \forall \PHhit{b_i}{c_j}{c_k} \in \restriction{\PHh}{k-1}, \PHget{(s \PHplay \delta)}{b} \neq b_i \vee \PHget{(s \PHplay \delta)}{c} \neq c_j \}
  %\text{$h$ n'est pas jouable dans $s \PHplay \delta$} \}
\end{align*}
%
Processus rencontrés comme résultats d'un état stable :
\begin{align*}
\pfpprocs_\ctx = \{ a_i \in \PHproc \mid \exists s \in \pfp_\ctx, a_i \in s \}
\end{align*}
%
Nouvel ensemble $\newprocs$ :
\begin{align*}
\newprocs_\ctx(\myB) = \allprocs(\myB) \cup \pfpprocs_\ctx
\end{align*}
\end{comment}

\begin{comment}
==========================

Bonds possibles :
\begin{align*}
\bounceprocs_\ctx = \{ a_i \in \PHproc \mid \exists s \in \ctx, \exists \delta \in \restriction{\Sce}{k-1}, a_i \in (s \PHplay \delta) \}
\end{align*}
\end{comment}

%%% N'est plus adapté
%Points fixes possibles sur un ensemble de sortes :
%\begin{align*}
%  \pfp: \mathbb{A} &\rightarrow \wp(\PHproc) \\
%  \myB &\mapsto \bigcup_{a \in A} \pfp_{\allprocs(\myB)}(a)
%\end{align*}

\begin{comment}
Séquences de bonds abstraites :
$$\BS^\wedge(P) = \{ \zeta^\wedge \mid \zeta \in \BS(P), \nexists \zeta' \in \BS(P), \zeta'^\wedge \subsetneq \zeta^\wedge \}$$
where $\zeta^\wedge = (\zeta^\wedge_A, \zeta^\wedge_B, \zeta^\wedge_{max})$ with:
\begin{itemize}
  \item $\zeta^\wedge_A = \{ \PHhitter(\zeta_n) \mid n \in \indexes{\zeta} \wedge \PHsort(\PHhitter(\zeta_n)) \neq \PHsort(P) \}$ : ens. des requis d'autres sortes (frappeurs)
  \item $\zeta^\wedge_B = \{ \PHhitter(\zeta_n) \mid n \in \indexes{\zeta} \} \cup \{ \PHtarget(\zeta_n) \mid n \in \indexes{\zeta} \}$ : ens. des processus nécessaires (à ne pas perturber)
  \item $\zeta^\wedge_{max} = \max_{n \in \indexes{\zeta}}(\prio(\zeta_n))$ : plus faible priorité
\end{itemize}
\end{comment}

\subsection{Abstract structure}

\begin{definition}[$\gCont_\ctx : \Sigma \times \Obj \mapsto \powerset(\Proc)$]
  \label{def:maxCont}
  \begin{align*}
    \gCont_\ctx(a,P) = 
    \{ p \in \PHproc &\mid \exists ps \in \aBS(P), \exists b_i \in ps, b = a \wedge p = b_i \\
      & \vee b \neq a \wedge p \in \gCont_\ctx(a, \PHobj{b_j}{b_i}) \wedge b_j \in \PHget{\ctx}{b} \}
    \enspace.
  \end{align*}
\end{definition}

\begin{definition}
  \label{def:aS}
  The abstract structure $\cwB=(\Breq,\Bsol,\Brsp,\Bcont)$ is defined as
  $\cwB = \sfp{\aB^\w_\ctx}{\myB}{\aB^\w_{\ctx \Cap \allprocs_\ctx(\myB)}}$,\\
  with $\myB=(\myreq,\mysol,\myrsp,\mycont)$:
  \begin{align*}
    \myreq &= \{ (a_i,\PHobjp{a}{j}{i}) \in \PHproc \times \Obj \mid
      a \in \components \wedge a_j \in \PHget{\ctx}{a} \\ % \vee a_j \in \pfpprocs_\ctx(a) \\
      & \qquad \wedge ((\exists (P,ps) \in \mysol \vee \exists (b_j, ps) \in \myrsp), a_i \in ps \\
      & \qquad\qquad \vee \exists n \in \indexes{\w}, \PHbounce(\w_n)=a_i) \}
    \\
    \mysol &\subseteq \{ (P,ps) \in \Obj \times \powerset(\PHproc) \mid
            \exists (a_i, P) \in \myreq \wedge ps \in \aBS(P) \\
      & \qquad\qquad \vee \exists (Q, P) \in \mycont \wedge ps \in \aBS(P) \}
    \\
    \myrsp &= \{ (a_i,ps) \in \PHproc \times \powerset(\PHproc) \mid a \in \cs \\
      & \qquad \wedge (\exists (P,ps') \in \mysol \vee \exists (b_j,ps') \in \myrsp), \\
      & \qquad\qquad a_i \in ps' \wedge \csState(a_i) \in ps \}
    \\
    \mycont & = \{ (P, \PHobj{q}{\PHbounce(P)}) \in \Obj \times \Obj \mid
      \exists (P, ps) \in \mysol \\
      & \qquad\qquad \wedge q \in \gCont_\ctx(\PHsort(P),P) \}
  \end{align*}
\end{definition}

\begin{definition}[Coherent solution]
\label{def:coherent}
  A cooperative sort requisite $(a_i, ps)$ in $\myrsp$ is said coherent iff
  it has no successor $(b_k,\PHobjp{b}{j}{k})$ in $\myreq$ so that there exists $b_n \in ps$, $b_k \neq b_n$.
\end{definition}

\begin{theorem}[Approximation inf.]
\label{th:approxinf}
  If the graph $\cwB$ contains no cycle,
  all objectives have at least one solution
  and all cooperative sort requisites are coherent,
  then $\uconcr(\w) \neq \emptyset$.
\end{theorem}

\begin{proof}
We note $max\ctx = \ctx \Cap \allprocs(\cwB)$ the context supported by $\cwB$.

Let $(a_i, ps) \in \Brsp$ be a cooperative sort requirement whom all children are concretizable.
We label all processes of $ps$ by an integer: $ps = \{ p_n \}_{n \in \indexes{ps}}$.
Let us prove that for all $n \in \indexes{ps}$, there exists a scenario $\delta_n$ so that:
$\forall i \in \segm{1}{n}, \PHget{(s \PHplay \delta_n)}{\PHsort(p_i)} = p_i$.
\begin{itemize}
  \item It is straightforward for $\delta_0 = \varepsilon$.
  \item Let us suppose such $\delta_n$ exists. \todo{Il faut être plus précis pour prendre en compte les coopérations en cascade}
    By hypothesis, there exists $\delta' \in \muconcr_{s \PHplay \delta_n}(\PHobj{\any}{p_n})$.
%    so that: $\PHget{(s \PHplay \delta_n \PHplay \delta')}{\PHsort(p_{n+1})} = p_{n+1}$.
    By hypothesis, $(a_i, ps)$ is coherent (\pref{def:coherent}), which means that solving $p_{n+1}$ does not disturb the processes of $ps$.
    Therefore, $\forall i \in \segm{1}{n+1}$, $\PHget{(s \PHplay \delta_n \PHplay \delta')}{\PHsort(p_{i})} = p_{i}$.
    Finally, by \pref{th:update}, there exists a scenario $\delta'' \in \restriction{\Sce}{1}$
    that can be played in $s \PHplay \delta_n \PHplay \delta'$
    and so that, if we denote $\delta_{n+1} = \delta_n \PHplay \delta' \PHplay \delta''$,
    we have: $\update(s \PHplay \delta_n \PHplay \delta') = s \PHplay \delta_{n+1}$ and the same property about processes (by \pref{th:hcscomp}).
\end{itemize}
Therefore, $\delta = \delta_{\indexes{ps}}$ exists, and given its properties, we have: $\PHget{(s \PHplay \delta)}{a} = a_i$
and $\update(s \PHplay \delta) = s \PHplay \delta$.

As there is no cycle in $\cwB$, we show by induction that $\forall s\in L, s\subseteq max\ctx$, 
for all objective $P$ in $\cwB$ so that $\PHtarget(P) \in s$,
$\exists \delta \in \muconcr_s(P)$.% and $\ceil(\delta) \subseteq max\ctx$.

\begin{itemize}
  \item If $(P, \emptyset) \in \Bsol$, either $\PHtarget(P) = \PHbounce(P)$ and $\delta = \emptyseq$,
    or $\forall \zeta \in \BS(P), \zeta \in \Sce \wedge \PHsort(\zeta) = \{ \PHsort(P) \}$ and $\delta$ is given by \pref{th:autohits}.

  \item Suppose all children objectives of $P$ are concretizable.
  \begin{itemize}
    \item If $\exists (P, Q) \in \Bcont$, then by hypothesis,
      $\muconcr_{s}(\obj{\PHtarget(P)}{\PHtarget(Q)} \concat Q) \neq \emptyset$, thus
      $\muconcr_{s}(P) \neq \emptyset$.
    \item Else, by \pref{def:maxCont}, the concretizations of the children of $P$ require no process of sort $\PHsort(P)$.
      Furthermore, there exists $\zeta \in \BS(P)$ so that $(P, \aZ) \in \Bsol$.
      We show by induction that for all $n \in \indexes{\zeta}$, there is a scenario $\delta_n$ so that $\PHget{(s \PHplay \delta_n)}{\PHsort(P)} = \PHbounce(\zeta_n)$.
%      We build recursively a scenario $\delta$. Let $m = |\indexes{\zeta}|$.
      \begin{itemize}
        \item[*] \towrite{construire récurvivement ? Distinguer les cas $b_j$ dans $\components$ ou $\cs$ (stabilité requise et obtenue)}
        Suppose that $\delta_n$ exists and let $\zeta_n = \PHhit{b_i}{a_j}{a_k}$.
        By hypothesis there exists $\delta' \in \muconcr_{s \PHplay \delta_n}(\PHobj{\any}{b_i})$ with $\PHsort(P) \notin \PHsort(\delta')$ by \pref{def:maxCont}.
        By \pref{th:update} there exists $\delta'' \in \restriction{\Sce}{1}$ so that $\update(s \PHplay \delta') = s \PHplay \delta' \PHplay \delta''$.
        Furthermore, $\PHget{(s \PHplay \delta' \PHplay \delta'')}{b} = b_j$ (by \pref{th:hcompcomp} if $b \in \components$ or \pref{th:hcscomp} if $b \in \cs$).
        Therefore, $\delta_{n+1} = \delta_n \PHplay \delta' \PHplay \delta'' \PHplay \zeta_n$.
      \end{itemize}
      Thus, $\delta_{\indexes{\zeta}} \in \muconcr_s(P)$. % and $\ceil(\delta) \subseteq max\ctx$.
  \end{itemize}
\end{itemize}
Finally, as $\muconcr_{max\ctx}(\w) \neq \emptyset$, $\uconcr(\w) \neq \emptyset$ (\pref{th:uconcr-ctx}).
\end{proof}



%%% Ancien graphe avec PrioCont
\begin{comment}
\begin{definition}
  \label{def:aS}
  The abstract structure $\cwB=(\Breq,\Bsol,\Bcont,\Bsat)$ is defined as
  $\cwB = \sfp{\aB^\w_{\update(\ctx)}}{\myB}{\aB^\w_{\ctx \Cap \allprocs_\ctx(\myB)}}$,\\
  with $\myB=(\myreq,\mysol,\mycont,\mysat)$:
  \begin{align*}
    \myreq &= \{ (a_i,\PHobjp{a}{j}{i}) \in \PHproc \times \Obj \mid
      a_j \in \PHget{\ctx}{a} \\ % \vee a_j \in \pfpprocs_\ctx(a) \\
      & \qquad \wedge (\exists (P,ps) \in \mysol, a_i \in ps \vee \exists n \in \indexes{\w}, \PHbounce(\w_n)=a_i) \}
    \\
    \mysol &\subseteq \{ (P,ps) \in \Obj \times \powerset(\PHproc) \mid
            \exists (a_i, P) \in \myreq \wedge ps \in \aBS(P) \\
      & \qquad\qquad \vee \exists (Q, P) \in \mycont \wedge ps \in \aBS(P) \}
    \\
    \mycont & = \{ (P, \PHobj{q}{\PHbounce(P)}) \in \Obj \times \Obj \mid
      \exists (P, ps) \in \mysol \\
      & \qquad\qquad \wedge q \in \gCont_\ctx(\PHsort(P),P) \}
    \\
    \mysat & = \{ (\PHobjp{a}{j}{i}, \PHobjp{a}{k}{i}) \in \Obj \times \Obj \mid
      \exists (a_i, \PHobj{a_j}{a_i}) \in \myreq\\
      & \qquad\qquad \wedge a_j \neq a_k \wedge a_k \in \pfp_\ctx(a) \}
%%% Version états stables locaux
%    \mysat & = \{ (\PHobjp{a}{j}{i}, \PHobjp{a}{k}{i}) \in \Obj \times \Obj \mid
%      \exists (a_i, \PHobjp{a}{j}{i}) \in \myreq, \\
%      & \qquad\qquad a_j \neq a_k \wedge a_k \in \pfpprocs_\ctx(a) \\
%      & \qquad\qquad \wedge (\exists (P, ps) \in \mysol, a_i \in ps \wedge \prio(\PHsort(P)) > \prio(a) \\
%      & \qquad\qquad\qquad \vee \exists (\PHobjp{a}{j}{i}, ps) \in \mysol, \exists p \in ps, \exists (p, P) \in \myreq,\\
%      & \qquad\qquad\qquad \text{$P$ is not trivial} \wedge \prio(\PHsort(P)) > \prio(a) \}
  \end{align*}
\end{definition}

\begin{theorem}[Approximation inf.]\label{th:approxinf}
If the graph $\cwB$ contains no cycle and all objectives have at least one solution, then $\uconcr(\w) \neq \emptyset$.
\end{theorem}

\begin{proof}
We note $max\ctx = \update(\ctx) \Cap \allprocs(\cwB)$ the context supported by $\cwB$.
As there is no cycle in $\cwB$, we show by induction that $\forall s\in L, s\subseteq max\ctx$, 
for all objective $P$ in $\cwB$ so that $\PHtarget(P) \in s$,
$\exists \delta \in \muconcr_s(P)$.% and $\ceil(\delta) \subseteq max\ctx$.

\begin{itemize}
  \item If $(P, \emptyset) \in \Bsol$, either $\PHtarget(P) = \PHbounce(P)$ and $\delta = \emptyseq$,
    or $\forall \zeta \in \BS(P), \zeta \in \Sce \wedge \PHsort(\zeta) = \{ \PHsort(P) \}$ and $\delta$ is given by \pref{th:autohits}.

  \item Suppose all children objectives of $P$ are concretizable.
  \begin{itemize}
    \item If $\exists (P, Q) \in \Bcont$, then by hypothesis,
      $\muconcr_{s}(\obj{\PHtarget(P)}{\PHtarget(Q)} \concat Q) \neq \emptyset$, thus
      $\muconcr_{s}(P) \neq \emptyset$.
    \item If $\exists (P, Q) \in \Bpriocont$, then by hypothesis,
      $\muconcr_{s}(\obj{\PHtarget(P)}{\PHtarget(Q)} \concat Q) \neq \emptyset$, thus
      $\muconcr_{s}(P) \neq \emptyset$.
    \item Else, by \pref{def:maxCont}, the concretizations of the children of $P$ require no process of sort $\PHsort(P)$.
      Furthermore, there exists $\zeta \in \BS(P)$ so that $(P, \aZ) \in \Bsol$.
      We build recursively a scenario $\delta$. Let $m = |\indexes{\zeta}|$.
%      \todo{Idée avec les nouvelles defs : au rang 0, on se replace sur un processus de $\pfp$. Au rang n, il existe par saturation un objectif adéquat qui “reprend” la progression.}
      \begin{itemize}
%        \item[*] Let $\zeta_1 = \PHhit{b_i}{a_j}{a_k}$. By hypothesis, $\exists \delta_1 \in \muconcr_s(\PHobj{\any}{b_i}), \PHget{s \PHplay \delta_1}{a} \in \pfp_s(a)$.
%          If $\PHget{s \PHplay \delta_1}{a} \neq a_j$, then by construction of $\cwB$, $(\PHobj{a_j}{a_k}, \PHobj{\PHget{s \PHplay \delta_1}{a}}{a_k}) \in \Bsat$.
%          By hypothesis, $\muconcr_s(\PHobj{\any}{a_k}) \neq \emptyset$.
%          If $\PHget{s \PHplay \delta_1}{a} = a_j$, then from \pref{th:vplay}, $\exists \delta'_1 \in \Sce$, $\zeta_1$ can be played in $s \PHplay \delta_1 \PHplay \delta'_1$.
        \item[*] For $n \in \indexes{\zeta}$, let $s_n = s \PHplay \delta_1 \PHplay \delta'_1 \PHplay \delta''_1 \PHplay \zeta_1 \PHplay \dots \PHplay \delta_{n-1} \PHplay \delta'_{n-1} \PHplay \delta''_{n-1} \PHplay \zeta_{n-1}$ (or $s_1 = s$),
          and $\zeta_n = \PHhit{b_i}{a_j}{a_k}$.
          By hypothesis, $\exists \delta_n \in \muconcr_{s_n}(\PHobj{\any}{b_i})$.
          If $a \in \cs$, then $\prio(\zeta_n) = 1$ and $\zeta_n$ can be played in $s_n \PHplay \delta_n$ (\ie $\delta'_n = \delta''_n = \varepsilon$).
          If $a \notin \cs$ and $b \in \cs$, then there exists a scenario $\delta'_n$ so that $b_i \in \pfp_{s_n \PHplay \delta_n \PHplay \delta'_n}(b)$ (because of the $\Bpriocont$ relation).
          Then, by \pref{th:hcscomp}, there is a scenario $\delta'' \in \restriction{\Sce}{1}$ so that $\zeta_n$ is playable in $s_n \PHplay \delta_n \PHplay \delta'_n \PHplay \delta''_n$.
          Finally, if $a,b \in \components$, then by \pref{th:hcompcomp}, there is a scenario $\delta'' \in \restriction{\Sce}{1}$ so that $\zeta_n$ is playable in $s_n \PHplay \delta_n \PHplay \delta''_n$ (\ie $\delta'_n = \varepsilon$).
      \end{itemize}
      Thus, $\delta = \delta_m \in \muconcr_s(P)$. % and $\ceil(\delta) \subseteq max\ctx$.
  \end{itemize}
\end{itemize}

Finally, as $\muconcr_{max\ctx}(\w) \neq \emptyset$, $\uconcr(\w) \neq \emptyset$ (\pref{th:uconcr-ctx}).
\end{proof}
\end{comment}



\begin{comment}
Voisinage :
\begin{equation*}
\begin{split}
    V: \wp(\PHproc) \times \segm{1}{k} &\rightarrow \wp(\PHh) \\
    (ps; m) &\mapsto \sfp{\PHh^{(m)+}_{cibles}(ps)}{hs}{\PHh^{(m)+)}_{bonds}(\widehat{B}(hs)) \cup hs)}
  \end{split}
\end{equation*}
where:
\begin{equation*}
\begin{split}
    \widehat{B}: \wp(\PHh) &\rightarrow \wp(\PHproc) \\
    hs &\mapsto \{ \PHhitter(h) \mid h \in hs \} \cup \{ \PHtarget(h) \mid h \in hs \}
  \end{split}
\end{equation*}
\begin{equation*}
\begin{split}
    \PHh^{(m)+}_{\mathsf{ref}}: \wp(\PHproc) &\rightarrow \wp(\PHh) \quad,\quad m \in \segm{0}{k} \text{ and } \mathsf{ref} \in \{ \PHhitter, \PHtarget, \PHbounce \} \\
    ps &\mapsto \{ h \in \PHh \mid \mathsf{ref}(h) \in ps \wedge \prio(h) \leq m \}
  \end{split}
\end{equation*}
\end{comment}



\begin{figure}
  \centering
  \begin{tikzpicture}[aS,node distance=1.5cm]
    \node[Aproc] (c1) {$c_1$};
    \node[Aobj,below of=c1] (c01) {$\PHobj{c_0}{c_1}$};
    \node[Asol,below of=c01] (c01s) {};
    
    \node[Aproc,below of=c01s] (ab11) {$ab_{11}$};
    \node[Aobj,below right of=ab11] (ab0111) {$\PHobj{ab_{01}}{ab_{11}}$};
    \node[Asol,below of=ab0111] (ab0111s) {};
    \node[Aobj,below of=ab11,node distance=2.5cm] (ab0011) {$\PHobj{ab_{00}}{ab_{11}}$};
    \node[Asol,below of=ab0011] (ab0011s) {};
    \node[Aobj,below left of=ab11] (ab1011) {$\PHobj{ab_{10}}{ab_{11}}$};
    \node[Asol,below of=ab1011] (ab1011s) {};

    \node[Aproc,below of=ab0111s] (a1) {$a_1$};
    \node[Aobj,below of=a1] (a11) {$\PHobj{a_1}{a_1}$};
    \node[Asol,below of=a11] (a11s) {};
    \node[Aobj,below right of=a1] (a01) {$\PHobj{a_0}{a_1}$};
    \node[Asol,below of=a01] (a01s) {};
    \node[Aproc,below of=a01s] (b0) {$b_0$};
    \node[Aobj,below of=b0] (b00) {$\PHobj{b_0}{b_0}$};
    \node[Asol,below of=b00] (b00s) {};
    \node[Aobj,below right of=b0] (b10) {$\PHobj{b_1}{b_0}$};
    \node[Asol,below of=b10] (b10s) {};

    \node[Aproc,below of=ab1011s] (b1) {$b_1$};
    \node[Aobj,below of=b1] (b01) {$\PHobj{b_0}{b_1}$};
    \node[Asol,below of=b01] (b01s) {};
    \node[Aproc,below of=b01s] (a0) {$a_0$};
    \node[Aobj,below of=a0] (a10) {$\PHobj{a_1}{a_0}$};
    \node[Asol,below of=a10] (a10s) {};

    \path
    (c1) edge (c01)
    (c01) edge (c01s)
    (c01s) edge (ab11)
    (ab11) edge (ab0011) edge (ab0111) edge (ab1011)
    (ab0011) edge (ab0011s)
    (ab1011) edge (ab1011s)
    (ab0111) edge (ab0111s)

    (ab0011s) edge (a1) edge (b1)
    (ab1011s) edge (b1)
    (ab0111s) edge (a1)

    (a1) edge (a01) edge (a11)
    (a01) edge (a01s)
    (a01s) edge (b0)
    (a11) edge (a11s)
    (a0) edge (a10)
    (a10) edge (a10s)
    
    (b0) edge (b10) edge (b00)
    (b10) edge (b10s)
    (b00) edge (b00s)
    (b1) edge (b01)
    (b01) edge (b01s)
    (b01s) edge (a0)
    ;
    \path
    (ab0011) edge[double] (ab0111)
    (ab0011) edge[double,bend right=10] (ab1011)
    (ab1011) edge[double,bend right=10] (ab0011)
    (ab0111) edge[double,bend right=10] (ab1011)
    (ab1011) edge[double,bend right=10] (ab0111)
    ;
    \end{tikzpicture}
  \label{fig:sa-livelock}
  \caption{The local causality graph of the PH in \pref{fig:ph-livelock}.}
\end{figure}



\begin{comment}
\begin{figure}
  \centering
  \begin{tikzpicture}[aS,node distance=1.5cm]
    \node[Aproc] (c1) {$c_1$};
    \node[Aobj,right of=c1] (c01) {$\PHobj{c_0}{c_1}$};
    \node[Asol,right of=c01] (c01s) {};
    \node[Aproc,right of=c01s] (b1) {$b_1$};
    \node[Aobj,right of=b1] (b11) {$\PHobj{b_1}{b_1}$};
    \node[Asol,right of=b11] (b11s) {};
    \node[Aobj,below right of=b1] (b01) {$\PHobj{b_0}{b_1}$};
    \node[Anos,right of=b01] (b01nos) {$\bottom$};

    \path
    (c1) edge (c01)
    (c01) edge (c01s)
    (c01s) edge (b1)
    (b1) edge (b11) edge (b01)
    (b11) edge (b11s)
    ;
  \end{tikzpicture}
  \label{fig:sa-conca}
  \caption{The local causality graph of the PH in \pref{fig:ph-conca}.}
\end{figure}



\begin{figure}
  \centering
  \begin{tikzpicture}[aS,node distance=1.5cm]
    \node[Aproc] (c2) {$c_2$};
    \node[Aobj,right of=c2] (c12) {$\PHobj{c_1}{c_2}$};
    \node[Asol,right of=c12] (c12s) {};
    \node[Aproc,right of=c12s] (b1) {$b_1$};
    \node[Aobj,right of=b1] (b01) {$\PHobj{b_0}{b_1}$};
    \node[Asol,right of=b01] (b01s) {};
    \node[Aproc,right of=b01s] (a1) {$a_1$};
    \node[Aobj,right of=a1] (a11) {$\PHobj{a_1}{a_1}$};
    \node[Asol,right of=a11] (a11s) {};

    \node[Aobj,above right of=c2] (c22) {$\PHobj{c_2}{c_2}$};
    \node[Asol,right of=c22] (c22s) {};
    \node[Aobj,below right of=c2] (c02) {$\PHobj{c_0}{c_2}$};
    \node[Anos,right of=c02] (c02nos) {$\bottom$};

    \path
    (c2) edge (c12) edge (c22) edge (c02)
    (c22) edge (c22s)
    (c12) edge (c12s)
    (c12s) edge (b1)
    (b1) edge (b01)
    (b01) edge (b01s)
    (b01s) edge (a1)
    (a1) edge (a11)
    (a11) edge (a11s)
    ;
  \end{tikzpicture}
  \label{fig:sa-concb}
  \caption{The local causality graph of the PH in \pref{fig:ph-concb}.}
\end{figure}
\end{comment}







\section{\todo{Title?} Equivalence with other frameworks}

% Equivalence with ADNs
\subsection{Process Hitting with multiple classes of priorities}
\label{ssec:flattening}

Consider a PH with $k \in \sNN$ classes of priories
$\ov{\PH} = (\ov{\PHs}; \ov{\PHl}; \ov{\PHa}^{\angles{k}})$
that respects all Conditions given in \pref{ssec:hypothesis}.
If $\oPH$ does not respect these Conditions,
it is then sufficient to instead consider the PH model with $k+1$ classes of priorities
$\ov{\PH}' = (\ov{\PHs}; \ov{\PHl}; \ov{\PHa}'^{\angles{k+1}})$
where $\ov{\PHh}'^{(1)} = \emptyset$.
The aim of this section is to propose a translation of $\oPH$
into a PH model with $2$ classes of priorities $\PH = (\PHs; \PHl; \PHa^{\angles{2}})$
called \emph{flattening},
which is weakly bisimilar
and that also respects all Conditions of \pref{ssec:hypothesis}.

This translation is based on the notion of \emph{playability property} defined in \pref{def:pp}
which is a Boolean formula where the atoms are processes of $\ov{\PH}$.

\begin{definition}[Playability property language ($\F$)]
  \label{def:pp}
  A \emph{playability property} is an element of the language $\F$ inductively defined by:
  \begin{itemize}
    \item $\top$ and $\bot$ belong to $\F$;
    \item if $a \in \ov{\PHs}$ and $a_i \in \ov{\PHl}_a$, then $a_i \in \F$ and we call it an \emph{atom};
    \item if $P \in \F$ and $Q \in \F$, then $\neg P \in \F$, $P \wedge Q \in \F$ and $P \vee Q \in \F$.
  \end{itemize}
  If $P \in \F$ is a playability property and $\mysigma \in \PHsubl$ is a sub-state of $\oPH$,
  we note $\Fsem{P}{\mysigma}$ the \emph{evaluation} of $P$ in $\mysigma$:
  \begin{itemize}
    \item if $P = a_i \in \ov{\PHs}_a$ is an atom, with $a \in \ov{\PHs}$, then $\Fsem{a_i}{\mysigma}$ is true iff $a_i \in \mysigma$;
    \item if $P$ is not an atom, then $\Fsem{P}{\mysigma}$ is true iff all its subformulas are inductively true in $\mysigma$
      with the classical semantics for the logic operators $\top$, $\bot$, $\neg$, $\wedge$ and $\vee$.
  \end{itemize}
\end{definition}

Because we only use classical logic operators, the formulas of Boolean logic on 
distributivity, associativity and commutativity can be used, together with De Morgan's laws on negation.
We also have the following property for the negation of an atom:
\[\forall a \in \ov{\PHs}, \forall a_i \in \ov{\PHl}_a, \forall \mysigma \in \PHsubl,
  \Fsem{\neg a_i}{\mysigma} \Leftrightarrow \Fsem{\bigvee_{\substack{a_j \in \PHl_a\\a_j \neq a_i}} a_j}{\mysigma}\]
Indeed, if a process is not active in a state, this means that another process of the same sort is active.

In \pref{def:fop}, we define the the operator $\Fopsymbol$ which characterises the playability of an action
given the semantics of PH (\pref{def:play}).
This operator considers the \emph{virtual hitters} of an action,
which are the sub-states given by $\csState$ to activate the hitter of the given action if it is a process of cooperative sort.
Furthermore, the target of the action if not taken into account as it is not needed for the rest of the flattening.
%
\begin{definition}[Playability property of an action ($\Fopsymbol : \PHh \rightarrow \F$)]\label{def:fop}
  \[\forall h \in \ov{\PHh}, \Fop{h} \equiv \virtualhitters(h) \wedge
    \left( \bigwedge_{\substack{g \in \PHh^{(n)}\\n < \prio(h)}}
    \neg \left( \target{g} \wedge \hitter{g}\right) \right)\]
  where if we note $\hitter{h} = a_i$, then:
  \[\virtualhitters(h) =
    \begin{cases}
      a_i & \text{if } a \in \components \text{ or } h \in \ov{\PHh}^{(1)} \\
      \bigvee_{ps \in \csState(a_i)} \left( \bigwedge_{p \in ps} p \right) & \text{if } a \in \cs
    \end{cases}\]
\end{definition}
%
By construction of this operator and given the dynamics of a PH model,
an action $h$ is playable in a state $s \in \ov{\PHl}$ if and only if: $\Fsem{\Fop{h}}{s} \wedge \target{h} \in s$.

Because we only use classical logic operators, we can compute the Disjunctive Normal Form (DNF) of any playability property.
For any action $h \in \ov{\PHh}$, this DNF takes the form:
\[\Fop{h} \equiv \bigvee_{i \in \segm{1}{\n}} \left( \bigwedge_{j \in \segm{1}{\m}} p_{i,j} \right)\]
where $\n \in \sN$ and $\forall i \in \segm{1}{\n}, \m \in \sN^*$.
If $\n = 0$, then $\Fop{h} \equiv \bot$; this means that $h$ can never be played
due to preemptions by other actions with higher priorities.
If $\Fop{h} \not\equiv \bot$, on the other hand, then in this case $\Fop{h}$
can be seen as a disjunction of $\n$ smaller playability properties consisting only of conjunctions of atoms.
These $\n$ conjunctions can be translated to as many prioritised cooperative sorts,
thus creating a new PH model which is entailed in the restrictions depicted in \pref{ssec:hypothesis}.
In this second case, we denote, for any $i \in \segm{1}{\n}$:
$\PHdep{i}{h} = \{ \PHsort(p_{i,j}) \mid j \in \segm{1}{\m} \}$.

With \pref{lem:ppplaysubset}, we can then characterise the playability of an action in a state only with a sub-state.
This sub-state corresponds to one of the conjunctions of its playability property's DNF.
Finally, \pref{def:flattening} gives the construction of the flattening of $\oPH$:
for each action $h \in \ov{\PHh}$, several cooperative sorts $f^{h,i}$ are built to reflect each of the conjunction in $\Fop{h}$,
\ie for $i \in \segm{1}{\n}$.
This construction allows to obtain the same dynamics as $\oPH$, as stated by \pref{th:bisimPHP}.
A detailed proof of this theorem is available in \pref{suppl:demoflattening}.
%
\begin{lemma}
\label{lem:ppplaysubset}
  Let $h \in \ov{\PHh}$ and $s \in \ov{\PHl}$;
  $h$ is playable in $s$ if and only if:
  \[\target{h} \in s \wedge \exists \mysigma \subseteq s, \Fsem{\Fop{h}}{\mysigma} \enspace.\]
\end{lemma}
%
\begin{proof}
  ($\Rightarrow$)
    If $h$ is playable in $s$, then $\target{h} \in s$ and $\Fsem{\Fop{h}}{s}$.
    Thus, $\Fop{h} \not\equiv \bot$ and, by property of a DNF,
    at least one of the $\n$ conjunctions of $\Fop{h}$ is true in $s$.
    Suppose the $i$\textsuperscript{th} conjunction is true in $s$, with $i \in \segm{1}{\n}$;
    then we have: $\forall j \in \segm{1}{\m}, p_{i,j} \in s$.
    Let $\mysigma \in \PHsubl_{\PHdep{i}{h}}$
    with $\forall b \in \PHdep{i}{h}, \PHget{\mysigma}{b} = \PHget{s}{b}$.
    We immediately have: $\mysigma \subseteq s$,
    and, by construction of $\PHdep{i}{h}$, $\Fsem{\Fop{h}}{\mysigma}$.
  
  ($\Leftarrow$)
    $\Fsem{\Fop{h}}{\mysigma}$ therefore $\Fsem{\Fop{h}}{s}$; as $\target{h} \in s$, $h$ is playable in $s$.
\end{proof}

\begin{definition}
  \label{def:flattening}
  If $k \in \sNN$ and $\oPH = (\ov{\PHs}; \ov{\PHl}; \ov{\PHa}^{\angles{k}})$
  is a PH with $k$ classes of priorities that respects all Conditions of \pref{ssec:hypothesis},
  we note $\PHflat(\ov{\PHs}; \ov{\PHl}; \ov{\PHa}^{\angles{k}}) = (\PHs; \PHl; \PHa^{\angles{2}})$
  the \emph{flattening} of $\oPH$, where:
  \begin{itemize}
    \item $\PHs = \ov{\PHs} \cup \PHs_f$
      where $\PHs_f = \{ f^{h,i} \mid h \in \ov{\PHh} \wedge \n \geq 1 \wedge i \in \segm{1}{\n} \}$;
    \item $\PHl = \left( \bigtimes{a \in \ov{\PHs}} \PHl_{a} \right) \times
      \left(\bigtimes{f^{h,i} \in \PHs_f} \PHl_{f^{h,i}} \right)$,
      where $\forall a \in \ov{\PHs}, \PHl_{a} = \ov{\PHl}_{a}$, and
      $\forall f^{h,i} \in \PHs_f,
      \PHl_{f^{h,i}} = \{ f^{h,i}_\mysigma \mid \mysigma \in \PHsubl_{\PHdep{i}{h}} \}$;
    \item $\PHh^{(1)} = \{ \PHhit{a_k}{f^{h,i}_\mysigma}{f^{h,i}_{\mysigma'}} \mid
      h \in \ov{\PHh} \wedge
      f^{h,i} \in \PHs_f \wedge
      a \in \PHdep{i}{h} \wedge a_k \in \PHl_a \wedge
      f^{h,i}_\mysigma , f^{h,i}_{\mysigma'} \in \PHl_{f^{h,i}} \wedge
      \PHget{\mysigma}{a} \neq a_k \wedge \mysigma' = \mysigma \Cap \{ a_k \} \}$;
    \item $\PHh^{(2)}=\{ \PHhit{f^{h,i}_\mysigma}{\target{h}}{\bounce{h}} \mid
      h \in \ov{\PHh} \setminus \ov{\PHh}^{(1)} \wedge
      f^{h,i} \in \PHs_f \wedge
      f^{h,i}_\mysigma \in \PHl_{f^{h,i}} \wedge \Fsem{\Fop{h}}{\mysigma} \}$.
  \end{itemize}
  Given a state $s \in \PHl$, $\unflats{s} = \os$ is the corresponding state in $\ov{\PHl}$:
  $\forall a \in \ov{\PHs}, \PHget{\os}{a} = \PHget{s}{a}$.

  \noindent
  Given a state $\os \in \ov{\PHl}$, $\flats{\os} = s$ is the corresponding state in $\PHl$:
  $\forall a \in \ov{\PHs}, \PHget{s}{a} = \PHget{\os}{a}$
  and $\forall f^{h,i} \in \PHs_f, \PHget{s}{f^{h,i}} = f^{h,i}_\mysigma$ with $f^{h,i}_\mysigma \in \PHl_{f^{h,i}}$
  and $\forall b \in \PHdep{i}{h}, \PHget{\mysigma}{b} = \PHget{\os}{b}$.
\end{definition}

\begin{theorem}[$(\ov{\PHs}; \ov{\PHl}; \ov{\PHa}^{\angles{k}}) \approx \PHflat(\ov{\PHs}; \ov{\PHl}; \ov{\PHa}^{\angles{k}})$]
\label{th:bisimPHP}
  Let $\ov{\PH} = (\ov{\PHs}; \ov{\PHl}; \ov{\PHa}^{\angles{k}})$ a PH with $k$ classes of priorities,
  and $\PH = \PHflat(\ov{\PHs}; \ov{\PHl}; \ov{\PHa}^{\angles{k}}) = (\PHs; \PHl; \PHa^{\angles{2}})$ its flattening.
  \begin{enumerate}
    \item \label{php2ph} $\forall \os, \os' \in \ov{\PHl}$,
      $\os \PHPtrans[\oPH] \os' \Longrightarrow \flats{\os} \PHPtrans[\oPH]^* \flats{\os'}$,
      where $\PHPtrans[\oPH]^*$ is a finite sequence of $\PHPtrans[\oPH]$ transitions.
    \item \label{ph2php} $\forall s, s' \in \PHl$,
      $s \PHPtrans s' \Longrightarrow \unflats{s} = \unflats{s'} \vee
      \unflats{s} \PHPtrans \unflats{s'} \enspace.$
  \end{enumerate}
\end{theorem}

We showed in this subsection that it is possible to model a PH with $k$ classes of priorities as a PH
with $2$ classes of priorities and the same dynamics.
Furthermore, by construction, the cooperative sorts are well-formed, and $\PHflat(\oPH)$
thus satisfies the Conditions of \pref{ssec:hypothesis}.
We note however that the translation given in this section may not be minimal in terms of number of actions
or cooperative sorts.
We give here several ways to simplify the resulting model, but we do not give the proofs.

\paragraph{Playability properties simplifications}
The playability property $\Fop{h}$ of an action $h = \PHhit{a_i}{b_j}{b_k}$ can be simplified with the following properties,
which may remove the necessity to create some useless actions (that are never playable) or cooperations (that are always true):
\begin{itemize}
  \item it is always possible to simplify the target because its presence is checked besides: $b_j \equiv \top$;
  \item any process $b_l \neq b_j$ of the same sort than the target always prevents the playability, thus: $b_l \equiv \bot$;
  \item if $c_p, c_q$ are different processes ($c_p \neq c_q$) of the same sort $c$, then $c_p \wedge c_q \equiv \bot$.
\end{itemize}

\paragraph{Removal of the unnecessary cooperative sorts}
There are two cases where it is possible to remove a cooperative sort $f^{h,i}$ in the flattening:
\begin{itemize}
  \item if $\Fop{h} \equiv \top$, then the action $h$ can be translated as an auto-action
    (as it is always playable if the target is present);
  \item if the $i$\textsuperscript{th} conjunction of $\Fop{h}$ consists of only one element $p$,
    then $h$ can be translated as a regular action $\PHhit{p}{\target{h}}{\bounce{h}}$ instead of a cooperative sort
    (as, apart from the target, only one process is required).
\end{itemize}


% Flattening of a PH with k priorities
\subsection{Weak Bisimulation of Asynchronous Discrete Networks}
\label{sec:dn}

We exhibit in this section an encoding of Asynchronous Discrete Networks (ADN),
also called Logical Networks or equivalently René Thomas Models \cite{Thomas95,deJong02},
into AANs and prove a weak bisimulation relation.
This translation is important as it allows to efficiently study the dynamics of ADNs
by using the static analysis developed in \pref{sec:sa};
indeed, the bisimulation relation ensures that the dynamics of the
resulting AAN model are accurately the same than the original.

A \emph{Discrete Network} gathers a finite number of components $i\in\segm{1}{n}$ having a discrete finite domain
$\mathbb F^i$ that we note $\mathbb{F}^i = \segm{1}{l_i}$.
For each component $i\in\segm{1}{n}$, a map $\mathbb F \rightarrow \mathbb F^i$ is defined, where
$\mathbb F = \mathbb F^1 \times \cdots \times \mathbb F^n$, giving the next value of the component
with respect to the global state of the network.
Typically $f^i$ depends on a subset of components that we denote $\DNdep(f^i)$.
In the case of \emph{Asynchronous Discrete Networks} (ADN), a transition relation $\DNtrans\subseteq \mathbb
F\times \mathbb F$ is defined such that $x\DNtrans x'$ if and only if there exists a unique
$i\in\segm{1}{n}$ such that $\get{x'}{i}=f^i(x)$ and $\forall j\in\segm{1}{n}, j\neq i, \get{x'}{j}
=\get{x}{j}$, \ie one and only one component has been updated.
This is formalised in \pref{def:DN}.

\begin{definition}[Asynchronous Discrete Network]
\label{def:DN}
  An \emph{Asynchronous Discrete Networks} (ADN) is defined by a couple $(\mathbb F, \langle f^1, \dots, f^n \rangle)$
  where $\mathbb{F} = \mathbb{F}^1\times\dots\times\mathbb{F}^n$,
  and $\forall i\in\segm{1}{n}$,
  $f^i: \mathbb{F} \rightarrow \mathbb{F}^i$ with
  $\mathbb{F}^i = \segm{1}{l_i}$ and $l_i \in \sNN$.
  Given two states $x,x'\in\mathbb F$, the transition relation $\DNtrans$ is given by
  \[
  x\DNtrans x' \Longleftrightarrow
    \exists i\in\segm{1}{n}, f^i(x)=\get{x'}{i}
    \wedge \forall j\in\segm{1}{n}, j\neq i, \get{x}{j}=\get{x'}{j}
  \enspace,
  \]
  where $\get{x}{i}$ is the $i$\textsuperscript{th} component of $x$.
  %
  We note $\DNdep(f^i)\subseteq \segm{1}{n}$ the set of components on which the value of $f^i$
  depends: $\forall x,x'\in \mathbb F$ such that $\forall
  j\in\DNdep(f^i), \get{x}{j}=\get{x'}{j}$, necessarily $f^i(x)=f^i(x')$.
\end{definition}

Let us denote the encoding of a given ADN $\DNdef$ into an AAN by $\toPH\DNdef$ (\pref{def:DN2PH}).
For each component $i\in\segm{1}{n}$ of the ADN,
one automaton $a^i$ is built, acting for the component value.
Obviously, $a^i$ has one local state $a^i_k$ per element $k \in \mathbb F^i$.
It is then sufficient to build actions towards $a^i$
depending on any possible state of the components in $\DNdep(f^i)$
and the related value of the evolution function $f^i$.

\begin{definition}[Encoding ($\toPH$)]
\label{def:DN2PH}
  $\toPH\DNdef=(\Sigma,\PHl,\Hits)$ is the AAN encoding the ADN $\DNdef$, with:
  \begin{itemize}
    \item $\Sigma = \{ a^1, \dots, a^n \}$;
    
    \item $\PHl=\underset{i\in\segm{1}{n}}{\times} \PHl_{a^i}$, where
      $\PHl_{a^i}=\{a^i_0, \dots, a^i_{l_i}\}$;
    
    \item $\PHh = \{ \PHfrappe{A}{a^i_j}{a^i_k} \mid
      \exists \varsigma \PHsubl[\PHl]_{\DNdep(f^i)}, \exists x \in \mathbb{F},
      \varsigma \subseteq \encode{x} \wedge
      f^i(x) = k \wedge \PHget{x}{i} = j \wedge j \neq k \wedge
      A = \toset{\varsigma} \}$ where $\encode{x}$ is defined below.
  \end{itemize}
  Given a state $s \in \PHl$ of the AAN,
  $\decode s=x$ is the corresponding state in the ADN:
  $\forall i\in\segm{1}{n}, \get{s}{a^i}=a^i_k \Rightarrow \get{x}{i}=k$.
  Conversely, given a state $x\in \mathbb F$ of the ADN, 
  $\encode x=s$ is the corresponding state in the AAN:
  $\forall i\in\segm{1}{n}, \get{x}{i}=k \Rightarrow \get{s}{a^i}=a^i_k$.
\end{definition}

Finally,
\pref{th:bisimDN} states the weak bisimulation relation between an ADN and its encoding in AAN.
Intuitively, the actions follow strictly the possible transitions of the ADN.
A detailed proof is given in \pref{suppl:proofbisimADN}. \todo{Utile ?}

\begin{theorem}[$\DNdef \approx \toPH\DNdef$]
\label{th:bisimDN}
  If $\DNdef$ is an ADN
  and $\PH = \toPH\DNdef = (\PHs; \PHl; \PHa)$ is its AAN encoding, then:
  \begin{enumerate}
    \item \label{adn2ph} $\forall x,x'\in\mathbb F$,
      $x\DNtrans x' \Longrightarrow \encode x \PHPtrans \encode{x'}$,
    \item \label{ph2adn} $\forall s,s'\in \PHl$,
      $s\PHPtrans s' \Longrightarrow \decode s \DNtrans
      \decode{s'}\enspace.$
  \end{enumerate}
\end{theorem}



% vim:set spell spelllang=en:

\section{Biological Examples}\label{sec:example}

\subsection{\ldots (metazoan)}
\todo{Maxime}

\subsection{Large-scale Application}

In order to support the scalability and applicability of our under-approximation of reachability, we
apply our new approach for the analysis of large-scale model of the T-cell receptor (TCR)
signalling pathway \cite{tcrsig94}.
This model gathers 94 interacting components and is specified as a Boolean network.
The under-approximation presented in this paper has been implemented in the existing Pint
software\footnote{Pint is freely available at \url{http://process.hitting.free.fr}.}.

The Boolean model has been automatically encoded into a Process Hitting with 2 classes of priority%
\footnote{Model and scripts are available at
\url{http://www.irccyn.ec-nantes.fr/~folschet/underapprox-tcrsig94.zip}.}.
Then, we verified the reachability for the independent activation of 4 outputs of the signalling
cascade (SRE, AP1, NFkB, NFAT) from all possible input combinations (CD45, CD28, TCRlib) using our
new reachability under-approximation (answering either \emph{yes} or \emph{inconclusive}) and a 
previously defined reachability over-approximation \cite{PMR12-MSCS} (answering either \emph{no} or
\emph{inconclusive}).
All result in conclusive decisions, and the under-approximation has been satisfied in 12 cases (over
32) proving the satisfiability of the concerned reachability property in the encoded Boolean network
(and non-satisfiability in the other cases).

Computations times are in the order of a few hundredths of a second on a 2.4GHz processor with 2GB
of RAM.
To give a comparison, we did the same experiments with a standard symbolic model-checker, libDDD
\cite{libddd}, known for its good performances, the input model being the Boolean network expressed
as a Petri net.
However, due to the large scale of the model, the program runs out of memory for all the experiments.

While ensuring a low complexity for the analysis of reachability in Boolean and discrete networks, our
under-approximation method reveals to be conclusive in numerous cases when applied to real
large-scale biological models, which were not tractable with exact model-checking.



% vi:spell spelllang=en:
\section{Discussion \& Conclusion}\label{sec:ccl}

We introduced a new semantics to include priorities into the Process Hitting framework, which prove especially useful to model cooperations.
Then, we developed a method to efficiently perform a reachability analysis of a sequence of objectives in a restricted class of Process Hitting models,
but it is also useful to establish the reachability of a partial state.
This analysis is based on an under-approximation of the true reachability solutions.
%; however, the most usual cases can be handled.

We showed that the class of Process Hitting models that can be handled by the aforementioned method are equivalent to Asynchronous Discrete Networks, and therefore to Asynchronous Boolean Networks.
This allows to efficiently compute reachability results on large biological models provided that they are equivalent to Asynchronous Discrete Networks and that a translation from the original framework into a Process Hitting model is possible.
Such a translation for interaction graphs of Thomas modelling was proposed in~\cite{PMR10-TCSB}. % \towrite{[À garder ?] and is made possible by the Pint software}.

We also demonstrated that general Process Hitting models with arbitrary priority
classes can also be encoded in the particular class of two-prioritised Process Hitting.
Therefore, our static analysis is applicable to a broad range of automata
networks with priorities.

Further work can be derived from what have been presented in this paper.
The over-approximation on Process Hitting models without priorities proposed in~\cite{PMR12-MSCS}
is still accurate in the framework with priorities (by “merging” all actions),
but may be refined given the restrictions proposed in this paper,
and a specific search of key processes or cut sets may be derived.
%but turns out to be too wide even in some obvious cases that are consequently not conclusive.
%This approximation may be refined in order to better fit the introduction of priorities, and mak the overall approximation approach 
%and mode precisely the class of models studied in this paper.
Furthermore, we are investigating alternative under-approximations that can be
applied directly to the whole class of Process Hitting models with priorities,
and not only to a sub-class with particular restrictions;
such improvement may allow to increase the conclusiveness of the static analysis
while allowing to analyse any model without the need of a translation.
Finally, in order to take into account quantitative data in transition delays, the overall approximation method could be extended to handle evolutions that are chronometric instead of only chronologic.


\section*{References}
\bibliographystyle{elsarticle-num}
\bibliography{biblio}

\appendix

% Fix appendix references
\renewcommand*{\thesection}{\Alph{section}}



% Proof of under-approximation
\section{Proof of Under-approximation (\pref{ssec:ua})}
\label{suppl:demoapproxinf}

In the following, we denote:
$\Bee{X}{Y} = \Be \cap (X \times Y)$, with $X, Y$
amongst $\PHproc$, $\Obj$, $\sSol$ and $\Sol$.

\newproof{proofapproxinf}{Proof of \pref{th:approxinf}}
\begin{proofapproxinf}
We note $max\ctx = \ctx \Cap \allprocs(\cwB)$ the context supported by $\cwB$.

Let $ps \in \Bv \cap \sSol$ be a set of hitters
and suppose all of its successors are concretisable.
We first want to demonstrate that
there exists a scenario that activate all the local states it contains.
We label all local states of $ps$ by an integer: $ps = \{ p_n \}_{n \in \indexes{ps}}$.
Let us prove by induction that for all $n \in \indexes{ps}$, there exists a scenario $\delta_n$ so that:
$\forall i \in \segm{1}{n}, \PHget{(s \PHplay \delta_n)}{\PHsort(p_i)} = p_i$.
\begin{itemize}
  \item It is straightforward for $\delta_0 = \varepsilon$.
  \item Suppose such $\delta_n$ exists and let $q = \PHget{(s \PHplay \delta_n)}{\PHsort(p_{n+1})}$.
    By construction, $p_{n+1} \in \Bv \cap \Proc$ is a direct successor of $ps$.
    Furthermore, by hypothesis, $ps$ is coherent (see \pref{def:coherent}).
    This means that amongst all the successors in $\Proc$ of $p_{n+1}$,
    there does not exist a local state $b_j$ to that
    $\exists b_k \in ps, \PHsort(b_j) = \PHsort(b_k) \wedge b_j \neq b_k$;
    in other words, the resolution of $p_{n+1}$ does not require a local state
    that may change the other local states of the set $ps$.
    Therefore, there exists $\delta' \in \muconcr_{s \PHplay \delta_n}(\PHobj{q}{p_{n+1}})$,
    so that $\forall i \in \segm{1}{n+1}$, $\PHget{(s \PHplay \delta_n \PHplay \delta')}{\PHsort(p_{i})} = p_{i}$.
    We denote then: $\delta_{n+1} = \delta_n \play \delta'$.
\end{itemize}
Therefore, $\delta = \delta_{\card{ps}}$ exists, and given its properties, we have:
$\forall i \in \segm{1}{\card{ps}}, \PHget{(s \PHplay \delta)}{\PHsort(p_{i})} = p_{i}$.

As there is no cycle in $\cwB$, we show by induction in the following that
$\forall s\in L, s\subseteq max\ctx, \forall P \in \Bv \cap \Obj,
\PHtarget(P) \in s \Longrightarrow \exists \delta \in \muconcr_s(P)$.
\begin{itemize}
  \item If $(P, \{ \emptyset \}) \in \Bee{\Obj}{\Sol}$,
    either $\PHtarget(P) = \PHbounce(P)$ and $\delta = \emptyseq$;
    or $\exists \zeta \in \BS(P), \zeta \in \Sce(s) \wedge
      \forall i \in \indexes{\zeta}, \hitter{\zeta_i} = \emptyset$
    and in this case, $\delta = \zeta$ is a valid scenario in $s$.

  \item Suppose all successors objectives of $P$ are concretisable.
    If $\exists (P, Q) \in \Bee{\Obj}{\Obj}$, then by hypothesis,
      $\muconcr_{s}(\obj{\PHtarget(P)}{\PHtarget(Q)} \concat Q) \neq \emptyset$, thus
      $\muconcr_{s}(P) \neq \emptyset$.
    Else, by \pref{def:maxCont}, the concretisations of the successors of $P$ require no local state of automaton $\PHsort(P)$.
      Furthermore, there exists $\zeta \in \BS(P)$ so that $(P, \aZ) \in \Bee{\Obj}{\Sol}$.
      We show by induction that for all $n \in \indexes{\zeta}$, there is a scenario $\delta_n$ so that $\PHget{(s \PHplay \delta_n)}{\PHsort(P)} = \PHbounce(\zeta_n)$.
      \begin{itemize}
%        \item[$\circ$] If $\zeta = \emptyseq$, then trivially, $\delta = \emptyseq$.
        \item[$\circ$] Suppose that $\delta_n$ exists and let $\zeta_n = \PHhit{A}{a_j}{a_k}$.
        By construction, there exists $A \in \Bv \cap \sSol$
        amongst the direct successors of $\aZ$.
        By the first result of this demonstration,
        there exists a scenario $\delta'$ in $s \play \delta_n$ so that
        $\forall a_i \in A, \PHget{(s \play \delta_n \PHplay \delta')}{a} = a_i$.
        Therefore, $\zeta_n$ is playable in $s \play \delta_n \PHplay \delta'$,
        and $\delta_{n+1} = \delta_n \concat \delta' \concat \zeta_n$.
      \end{itemize}
      Thus, $\delta_{|\zeta|} \in \muconcr_s(P)$. % and $\ceil(\delta) \subseteq max\ctx$.
\end{itemize}
Finally, as $\muconcr_{max\ctx}(\w) \neq \emptyset$, $\uconcr(\w) \neq
\emptyset$ (\pref{lem:uconcr-ctx}).
\end{proofapproxinf}


% Proofs for flattening
\section{Proof of the Weak Bisimulation of the Flattening (\pref{ssec:flattening})}
\label{suppl:demoflattening}

\newproof{proofbisimPHP}{Proof of \pref{th:bisimPHP}}
\begin{proofbisimPHP}
  (\ref{php2ph}) Let $\flats{\os} = s$.
    Given the dynamics of a PH (\pref{def:play}), if $\os \PHPtrans[\PH] \os'$,
    then there exists $h \in \ov{\PHh}$ playable in $\os$;
    therefore, $\os' = \os \PHplay h$.
    Thus, $\Fsem{\Fop{h}}{\os}$ and $\target{h} \in \os$,
    and given \pref{lem:ppplaysubset}, there exists $\mysigma \subseteq s$ so that $\Fsem{\Fop{h}}{\mysigma}$.
    Therefore, by construction of $\PH$ (\pref{def:flattening}) there exists
    $g = \PHhit{f^{h,i}_\mysigma}{\target{h}}{\bounce{h}} \in \PHh^{(2)}$.
    By construction of $s$ (\pref{def:flattening}), $\PHget{s}{f^{h,i}} = f^{h,i}_\mysigma$ and $\target{h} = \target{g} \in s$.
    Therefore, $g$ is playable in $s$.
    In $s \PHplay g$, the only playable actions are those in $\PHh^{(1)}$ having $\bounce{h} = \bounce{g}$ as hitter
    and updating cooperative sorts, allowing to reach $\flats{\os'}$ in a finite number of actions.
    Thus, $\flats{\os} \PHPtrans[\oPH]^* \flats{\os'}$.
  
  (\ref{ph2php}) Let $\os = \unflats{s}$.
    Given the dynamics of a PH (\pref{def:play}), if $s \PHPtrans s'$,
    then there exists $g \in \PHh$ playable in $s$; therefore, $s' = s \PHplay g$ and $\target{g} \in s$.
    If $\prio(g) = 1$ then only the active process of a cooperative sort has evolved, and $\unflats{s} = \unflats{s'}$.
    Otherwise, $\prio(g) = 2$; we note in this case: $g = \PHhit{f^{h,i}_\mysigma}{b_j}{b_k}$.
    By construction of the flattening (\pref{def:flattening}), there exists $h \in \ov{\PHh}$ so that
    $\target{h} = b_j$, $\bounce{h} = b_k$ and $\Fsem{\Fop{h}}{\mysigma}$.
    If $g$ is playable, this means that no other action in $\PHh^{(1)}$ is playable, and especially the cooperative sort
    $f^{h,i}$ is already updated; therefore, $\mysigma \subseteq \os$.
    Furthermore, $b_j \in \os$.
    Thus, by \pref{lem:ppplaysubset}, it comes that $h$ is playable in $\os$,
    and $\unflats{s} \PHPtrans \unflats{s'}$.
\end{proofbisimPHP}


% Proof of ADN equivalence
%\todo{to be removed}
%\section{Proof of the Weak Bisimulation of Asynchronous Discrete Networks (\pref{sec:dn})}
\label{suppl:proofbisimADN}

\newproof{proofbisimADN}{Proof of \pref{th:bisimDN}}
\begin{proofbisimADN}
(\ref{adn2ph}) From \pref{def:DN}, $x\DNtrans x'\Rightarrow \exists i\in\segm{1}{n},
f^i(x)=\get{x'}{i} \wedge \forall j\in\segm{1}{n},i\neq j, \get{x}{j}=\get{x'}{j}$.
Let us assume (without loss of generality) that $f^i(x)=k'$, $\get{x}{i}=k$ and
$\varsigma\in\underset{j\in\DNdep(f^i)}{\times} \PHl_{a^j}$ such that
$\forall j\in\DNdep(f^i), \get{\varsigma}{j}=a^j_{\get{x}{j}}$.
From \pref{def:DN2PH}, $h=\hit{f^i_\varsigma}{a^i_k}{a^i_{k'}}\in\Hits^{(2)}$.
From the definition of $\encode x$,
$a^i_k\in \encode x$ and $f^i_\varsigma\in \encode x$;
moreover, as there is no action in $\Hits^{(1)}$ applicable in $\encode x$,
$h$ is applicable in $\encode x$:
$\encode x\PHPtrans \encode x\play h$.
In $\encode x\play h$, the only applicable actions of priority $1$ are those having
$a^i_{k'}$ as hitter and hitting cooperative sorts, giving a finite number of transitions towards
$\encode{x'}$.

(\ref{ph2adn}) $s\PHPtrans s'$ only if there exists an action $h$ applicable in $s$ such that
$s\play h=s'$.
If $\prio(h)=1$, then, by definition of $\Hits^{(1)}$, 
$\decode s=\decode {s'}$.
If $\prio(h)=2$, then $\forall i\in\segm{1}{n}$,
if $\get{s}{f^i} = f^i_\varsigma$, then, $\forall j\in\DNdep(f^i),
\get{\varsigma}{a^j}=\get{s}{a^j}$.
Let $i\in\segm{1}{n}$ such that $\get{s}{a^i}\neq\get{s'}{a^i}$ ($i$ is unique for this
transition).
By \pref{def:DN2PH}, if $\get{s'}{a^i}=a^i_{k'}$, necessarily $f^i(\decode s)=k'$, hence
$\decode s\DNtrans \decode{s'}$.
\end{proofbisimADN}



\end{document}
