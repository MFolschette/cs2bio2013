\documentclass{elsarticle}

\usepackage[british]{babel}
\usepackage[utf8]{inputenc}
\usepackage[T1]{fontenc}

\usepackage{amsmath}  % Maths
\usepackage{amsfonts} % Maths
\usepackage{amssymb}  % Maths
\usepackage{stmaryrd} % Maths (crochets doubles)

\usepackage{url}     % Mise en forme + liens pour URLs
\usepackage{hyperref}
\usepackage{array}   % Tableaux évolués
\usepackage{color, colortbl}  % Cellules de couleur dans les tableaux

\usepackage{comment}

% Theorems and definitions
\newdefinition{definition}{Definition}
\newdefinition{condition}{Condition}
\newtheorem{theorem}{Theorem}
\newtheorem{lemma}{Lemma}
\newproof{example}{Example}
\newproof{remark}{Remark}
\newproof{rawproof}{Proof}

\newenvironment{proof}{\begin{rawproof}}{\hfill$\Box$\end{rawproof}}

% Pretty references
\usepackage{prettyref}
\newrefformat{def}{Def.~\ref{#1}}
\newrefformat{fig}{Fig.~\ref{#1}}
\newrefformat{tab}{Table~\ref{#1}}
\newrefformat{lem}{Lemma~\ref{#1}}
\newrefformat{th}{Theorem~\ref{#1}}
\newrefformat{thm}{Theorem~\ref{#1}}
\newrefformat{co}{Corollary~\ref{#1}}
\newrefformat{sec}{Sect.~\ref{#1}}
\newrefformat{ssec}{Subsect.~\ref{#1}}
\newrefformat{suppl}{Appendix~\ref{#1}}
\newrefformat{cr}{Condition~\ref{#1}}
\newrefformat{eq}{Eq.~\eqref{#1}}
\def\pref{\prettyref}

\usepackage{tikz}
\newdimen\pgfex
\newdimen\pgfem
\usetikzlibrary{arrows,shapes,shadows,scopes}
\usetikzlibrary{positioning}
\usetikzlibrary{matrix}
\usetikzlibrary{decorations.text}
\usetikzlibrary{decorations.pathmorphing}

% Macros relatives à la traduction de PH avec arcs neutralisants vers PH à k-priorités fixes

% Macros générales
%\newcommand{\ie}{\textit{i.e.} }

% Notations générales pour PH
\newcommand{\PH}{\mathcal{PH}}
%\newcommand{\PHs}{\mathcal{S}}
\newcommand{\PHs}{\Sigma}
%\newcommand{\PHp}{\mathcal{P}}
\newcommand{\PHp}{\textcolor{red}{\mathcal{P}}}
%\newcommand{\PHproc}{\mathcal{P}}
\newcommand{\PHproc}{\mathbf{Proc}}
\newcommand{\Proc}{\PHproc}
\newcommand{\PHh}{\mathcal{H}}
\newcommand{\PHa}{\PHh}
%\newcommand{\PHa}{\mathcal{A}}
\newcommand{\PHl}{\mathcal{L}}
\newcommand{\PHn}{\mathcal{N}}

\newcommand{\PHhitter}{\mathsf{hitter}}
\newcommand{\PHtarget}{\mathsf{target}}
\newcommand{\PHbounce}{\mathsf{bounce}}
%\newcommand{\PHsort}{\Sigma}
\newcommand{\PHsort}{\PHs}

\newcommand{\hitter}[1]{\PHhitter(#1)}
\newcommand{\target}[1]{\PHtarget(#1)}
\newcommand{\bounce}[1]{\PHbounce(#1)}

%\newcommand{\PHfrappeur}{\mathsf{frappeur}}
%\newcommand{\PHcible}{\mathsf{cible}}
%\newcommand{\PHbond}{\mathsf{bond}}
%\newcommand{\PHsorte}{\mathsf{sorte}}
%\newcommand{\PHbloquant}{\mathsf{bloquante}}
%\newcommand{\PHbloque}{\mathsf{bloquee}}

%\newcommand{\PHfrappeR}{\textcolor{red}{\rightarrow}}
%\newcommand{\PHmonte}{\textcolor{red}{\Rsh}}

\newcommand{\PHhitA}{\rightarrow}
\newcommand{\PHhitB}{\Rsh}
%\newcommand{\PHfrappe}[3]{\mbox{$#1\PHhitA#2\PHhitB#3$}}
%\newcommand{\PHfrappebond}[2]{\mbox{$#1\PHhitB#2$}}
\newcommand{\PHhit}[3]{#1\PHhitA#2\PHhitB#3}
\newcommand{\PHfrappe}{\PHhit}
\newcommand{\PHhbounce}[2]{#1\PHhitB#2}
\newcommand{\PHobj}[2]{\mbox{$#1\PHhitB^*\!#2$}}
\newcommand{\PHconcat}{::}
%\newcommand{\PHneutralise}{\rtimes}

\def\PHget#1#2{{#1[#2]}}
%\newcommand{\PHchange}[2]{#1\langle #2 \rangle}
%\newcommand{\PHchange}[2]{(#1 \Lleftarrow #2)}
%\newcommand{\PHarcn}[2]{\mbox{$#1\PHneutralise#2$}}
\newcommand{\PHplay}{\cdot}

\newcommand{\PHstate}[1]{\mbox{$\langle #1 \rangle$}}

\def\supp{\mathsf{support}}
\def\first{\mathsf{first}}
\def\last{\mathsf{last}}

\def\DNtrans{\rightarrow_{ADN}}
\def\DNdef{(\mathbb F, \langle f^1, \dots, f^n\rangle)}
\def\DNdep{\f{dep}}
%\def\PHPtrans{\rightarrow_{PH}}
\newcommand{\PHPtrans}[1][\PH]{\rightarrow_{#1}}
\def\get#1#2{#1[{#2}]}
\def\encodeF#1{\mathbf{#1}}
\def\toPH{\encodeF{PH}}
\def\card#1{|#1|}
\def\decode#1{\llbracket#1\rrbracket}
\def\encode#1{\llparenthesis#1\rrparenthesis}
\def\Hits{\PHa}
\def\hit{\PHhit}
\def\play{\cdot}

\newcommand{\angles}[1]{{\langle #1 \rangle}}
\newcommand{\trans}[1]{\rightarrow_{#1}}
\newcommand{\PHdep}[2]{\f{dep}^{#1}(#2)}
\newcommand{\PHflat}{\f{flat}}

\usepackage{ifthen}
\usepackage{tikz}
\usetikzlibrary{arrows,shapes}

\definecolor{lightgray}{rgb}{0.8,0.8,0.8}
\definecolor{lightgrey}{rgb}{0.8,0.8,0.8}

\tikzstyle{boxed ph}=[]
\tikzstyle{sort}=[fill=lightgray,rounded corners]
\tikzstyle{process}=[circle,draw,minimum size=15pt,fill=white,
font=\footnotesize,inner sep=1pt]
\tikzstyle{black process}=[process, fill=black,text=white, font=\bfseries]
\tikzstyle{gray process}=[process, draw=black, fill=lightgray]
\tikzstyle{current process}=[process, draw=black, fill=lightgray]
\tikzstyle{process box}=[white,draw=black,rounded corners]
\tikzstyle{tick label}=[font=\footnotesize]
\tikzstyle{tick}=[black,-]%,densely dotted]
\tikzstyle{hit}=[->,>=stealth']
\tikzstyle{selfhit}=[min distance=30pt,curve to]
\tikzstyle{bounce}=[densely dotted,>=stealth',->]
\tikzstyle{hl}=[font=\bfseries,very thick]
\tikzstyle{hl2}=[hl]
\tikzstyle{nohl}=[font=\normalfont,thin]

\tikzstyle{prio}=[draw,very thick,-stealth]
\tikzstyle{coopupdate}=[dashed]
\tikzstyle{superprio}=[draw,very thick,double,-stealth]

\tikzstyle{apdot}=[circle, fill=black, inner sep=1.2pt, draw=transparent]
\tikzstyle{apdotsimple}=[]
\tikzstyle{aan}=[apdotsimple/.style={apdot}]

\newcommand{\currentScope}{}
\newcommand{\currentSort}{}
\newcommand{\currentSortLabel}{}
\newcommand{\currentAlign}{}
\newcommand{\currentSize}{}

\newcounter{la}
\newcommand{\TSetSortLabel}[2]{
  \expandafter\repcommand\expandafter{\csname TUserSort@#1\endcsname}{#2}
}
\newcommand{\TSort}[4]{
  \renewcommand{\currentScope}{#1}
  \renewcommand{\currentSort}{#2}
  \renewcommand{\currentSize}{#3}
  \renewcommand{\currentAlign}{#4}
  \ifcsname TUserSort@\currentSort\endcsname
    \renewcommand{\currentSortLabel}{\csname TUserSort@\currentSort\endcsname}
  \else
    \renewcommand{\currentSortLabel}{\currentSort}
  \fi
  \begin{scope}[shift={\currentScope}]
  \ifthenelse{\equal{\currentAlign}{l}}{
    \filldraw[process box] (-0.5,-0.5) rectangle (0.5,\currentSize-0.5);
    \node[sort] at (-0.2,\currentSize-0.4) {\currentSortLabel};
   }{\ifthenelse{\equal{\currentAlign}{r}}{
     \filldraw[process box] (-0.5,-0.5) rectangle (0.5,\currentSize-0.5);
     \node[sort] at (0.2,\currentSize-0.4) {\currentSortLabel};
   }{
    \filldraw[process box] (-0.5,-0.5) rectangle (\currentSize-0.5,0.5);
    \ifthenelse{\equal{\currentAlign}{t}}{
      \node[sort,anchor=east] at (-0.3,0.2) {\currentSortLabel};
    }{
      \node[sort] at (-0.6,-0.2) {\currentSortLabel};
    }
   }}
  \setcounter{la}{\currentSize}
  \addtocounter{la}{-1}
  \foreach \i in {0,...,\value{la}} {
    \TProc{\i}
  }
  \end{scope}
}

\newcommand{\TTickProc}[2]{ % pos, label
  \ifthenelse{\equal{\currentAlign}{l}}{
    \draw[tick] (-0.6,#1) -- (-0.4,#1);
    \node[tick label, anchor=east] at (-0.55,#1) {#2};
   }{\ifthenelse{\equal{\currentAlign}{r}}{
    \draw[tick] (0.6,#1) -- (0.4,#1);
    \node[tick label, anchor=west] at (0.55,#1) {#2};
   }{
    \ifthenelse{\equal{\currentAlign}{t}}{
      \draw[tick] (#1,0.6) -- (#1,0.4);
      \node[tick label, anchor=south] at (#1,0.55) {#2};
    }{
      \draw[tick] (#1,-0.6) -- (#1,-0.4);
      \node[tick label, anchor=north] at (#1,-0.55) {#2};
    }
   }}
}
\newcommand{\TSetTick}[3]{
  \expandafter\repcommand\expandafter{\csname TUserTick@#1_#2\endcsname}{#3}
}

\newcommand{\myProc}[3]{
  \ifcsname TUserTick@\currentSort_#1\endcsname
    \TTickProc{#1}{\csname TUserTick@\currentSort_#1\endcsname}
  \else
    \TTickProc{#1}{#1}
  \fi
  \ifthenelse{\equal{\currentAlign}{l}\or\equal{\currentAlign}{r}}{
    \node[#2] (\currentSort_#1) at (0,#1) {#3};
  }{
    \node[#2] (\currentSort_#1) at (#1,0) {#3};
  }
}
\newcommand{\TSetProcStyle}[2]{
  \expandafter\repcommand\expandafter{\csname TUserProcStyle@#1\endcsname}{#2}
}
\newcommand{\TProc}[1]{
  \ifcsname TUserProcStyle@\currentSort_#1\endcsname
    \myProc{#1}{\csname TUserProcStyle@\currentSort_#1\endcsname}{}
  \else
    \myProc{#1}{process}{}
  \fi
}

\newcommand{\repcommand}[2]{
  \providecommand{#1}{#2}
  \renewcommand{#1}{#2}
}
\newcommand{\THit}[5]{
  \path[hit] (#1) edge[#2] node[pos=.5,apdotsimple] {} (#3#4);
  \expandafter\repcommand\expandafter{\csname TBounce@#3@#5\endcsname}{#4}
}
\newcommand{\TBounce}[4]{
  (#1\csname TBounce@#1@#3\endcsname) edge[#2] (#3#4)
}

\newcommand{\TState}[1]{
  \foreach \proc in {#1} {
    \node[current process] (\proc) at (\proc.center) {};
  }
}



% ex : \TAction{c_1}{a_1.west}{a_0.north west}{}{right}
% #1 = frappeur
% #2 = cible
% #3 = bond
% #4 = style frappe
% #5 = style bond
\newcommand{\TAction}[5]{
  \THit{#1}{#4}{#2}{}{#3}
  \path[bounce, bend #5=50] \TBounce{#2}{}{#3}{};
}

\newcommand{\TCoopHit}[6]{
  \node[#2, apdot] at (#3) {};
  \foreach \proc in {#1} {
    \draw[#2,-] (#3) edge (\proc);
  }
  \path[hit] (#3) edge[#2] (#4#5);
  \expandafter\repcommand\expandafter{\csname TBounce@#4@#6\endcsname}{#5}
}



% ex : \TActionPlur{f_1, c_0}{a_0.west}{a_1.south west}{}{3.5,2.5}{left}
% #1 = frappeur
% #2 = cible
% #3 = bond
% #4 = style frappe
% #5 = coordonnées point central
% #6 = direction bond
\newcommand{\TActionPlur}[6]{
  \TCoopHit{#1}{#4}{#5}{#2}{}{#3}
  \path[bounce, bend #6=50] \TBounce{#2}{}{#3}{};
}

% procedure, abstractions and dependencies
\newcommand{\abstr}[1]{#1^\wedge}%\text{\textasciicircum}}
\def\BS{\mathbf{BS}}
\def\aBS{\abstr{\BS}}
\def\abeta{\abstr{\beta}}
\def\aZ{\abstr{\zeta}}
\def\aY{\abstr{\xi}}

\def\beforeproc{\vartriangleleft}

\def\powerset{\wp}

\def\Sce{\mathbf{Sce}}
\def\OS{\mathbf{OS}}
\def\Obj{\mathbf{Obj}}
%\def\Proc{\mathbf{Proc}}
%\def\Sol{\mathbf{Sol}}
\newcommand{\Sol}{\mathbf{Sol}}

\usepackage{galois}
\newcommand{\theOSabstr}{toOS}
\newcommand{\OSabstr}[1]{\theOSabstr(#1)}
\newcommand{\theOSconcr}{toSce}
\newcommand{\OSconcr}[1]{\theOSconcr(#1)}

% \def\gO{\mathbb{O}}
% \def\gS{\mathbb{S}}
\def\aS{\mathcal{A}}
\def\Req{\mathrm{Req}}
%\def\Sol{\mathrm{Sol}}
\def\Cont{\mathrm{Cont}}
\def\cBS{\BS_\ctx}
\def\caBS{\aBS_\ctx}
\def\caS{\aS_\ctx}
\def\cSol{\Sol_\ctx}
\def\cReq{\Req_\ctx}
\def\cCont{\Cont_\ctx}

\def\any{\star}

% \def\gProc{\mathrm{maxPROC}}
\def\mCtx{\mathrm{maxCtx}}

\def\procs{\f{procs}}
\def\objs{\f{objs}}
\def\sat#1{\lceil #1\rceil}

\def\gCont{\f{maxCont}}
\def\lCont{\f{minCont}}
\def\lProc{\f{minProc}}
\def\gProc{\f{maxProc}}

\def\join{\oplus}
\def\concat{\!::\!}
\def\emptyseq{\varepsilon}
\def\ltw{\preccurlyeq_{\OS}}
\def\indexes#1{\mathbb{I}^{#1}}
%\def\indexes#1{\{1..|#1|\}}
\def\supp{\f{support}}
\def\w{\omega}
\def\W{\Omega}
\def\ctx{\varsigma}
\def\Ctx{\mathbf{Ctx}}
\def\mconcr{\gamma}
\def\concr{\mconcr_\ctx}
\def\obj#1#2{{#1\!\Rsh^*\!\!#2}}
\def\objp#1#2#3{\obj{{#1}_{#2}}{{#1}_{#3}}}
\def\A{\mathcal{A}}
\def\cwA{\A_\ctx^\w}
\def\cwReq{\Req_\ctx^\w}
\def\cwSol{\Sol_\ctx^\w}
\def\cwCont{\Cont_\ctx^\w}
\def\gCtx{\f{maxCtx}}
\def\endCtx{\f{endCtx}}
\def\ceil{\f{end}}

%\def\lfp{\mathrm{lfp}\;}
%\def\mlfp#1{\mathrm{lfp}\{#1\}\;}
\newcommand{\lfp}[3]{\mathbf{lfp}\{#1\}\left(#2\mapsto#3\right)}
\def\maxobjs{{\f{maxobjs}}}
\def\maxprocs{{\f{maxprocs}_\ctx}}
\def\objends{{\f{ends}}}

\def\ra{\rho}
\def\rb{\rho^\wedge}
\def\rc{\widetilde{\rho}}
\def\interleave{\f{interleave}}

\def\join{\concat}

\tikzstyle{aS}=[every edge/.style={draw,->,>=stealth}]
\tikzstyle{Asol}=[draw,circle,minimum size=5pt,inner sep=0,node distance=1cm]
\tikzstyle{Aproc}=[draw,node distance=1.2cm]
\tikzstyle{Aobj}=[node distance=1.5cm]
\tikzstyle{Anos}=[font=\Large]
%\tikzstyle{AprocPrio}=[Aproc,double]
\tikzstyle{AsolPrio}=[Asol,double]



\def\procs{\mathsf{procs}}
%\def\allprocs{\mathsf{allProcs}}
\def\allprocs{\procs}
%\def\pfp{\mathsf{pfp}}
\def\pfp{\mathsf{lst}}
\def\pfpprocs{\mathsf{pfpProcs}}
\def\bounceprocs{\mathsf{bounceProcs}}
\def\newprocs{\mathsf{newProcs}}

\def\aB{\mathcal{B}}
\def\sat#1{\lceil #1\rceil}
\def\cwB{\sat{\aB_\ctx^\w}}
\def\mycwB#1#2{\sat{\aB_{#1}^{#2}}}
\def\Bsol{\sat{\Sol^\w_\ctx}}
\def\Breq{\sat{\Req^\w_\ctx}}
\def\Bcont{\sat{\Cont^\w_\ctx}}

\def\myB{\aB^\w_\ctx}
\def\mysol{\overline{\Sol^\w_\ctx}}
\def\myreq{\overline{\Req^\w_\ctx}}
\def\mycont{\overline{\Cont^\w_\ctx}}

\begin{comment}
\def\PrioCont{\textcolor{red}{\mathrm{PrioCont}}}
\def\mypriocont{\overline{\PrioCont^\w_\ctx}}
\def\cwPrioCont{\PrioCont_\ctx^\w}
\def\Bpriocont{\sat{\PrioCont^\w_\ctx}}
\def\Sat{\PrioCont}
\def\mysat{\overline{\Sat^\w_\ctx}}
\def\cwSat{\Sat_\ctx^\w}
\def\Bsat{\sat{\Sat^\w_\ctx}}

\def\ReqSolPrio{\textcolor{blue}{\mathrm{ReqSolPrio}}}
\def\RSP{\ReqSolPrio}
\def\myrsp{\overline{\RSP^\w_\ctx}}
\def\cwRSP{\RSP_\ctx^\w}
\def\Brsp{\sat{\RSP^\w_\ctx}}
\end{comment}

\newcommand{\csState}{\mathsf{procState}}

\newcommand{\V}{V}
\newcommand{\E}{E}
\newcommand{\cwV}{\V_\ctx^\w}
\newcommand{\cwE}{\E_\ctx^\w}
\newcommand{\VProc}{\V_\PHproc}
\newcommand{\VObj}{\V_\Obj}
%\newcommand{\VSol}{\V_{Sol}}
\newcommand{\VSol}{\V_{\Sol}}

\def\Bv{\sat{\cwV}}
\def\Be{\sat{\cwE}}
\def\BvProc{\textcolor{red}{\sat{\cwV}^\PHproc}}
\def\BvObj{\textcolor{red}{\sat{\cwV}^\Obj}}
%\def\BvSol{\sat{\cwV}^{Sol}}
\def\BvSol{\textcolor{red}{\sat{\cwV}^{\Sol}}}

\newcommand{\Bee}[2]{\Be^{#1}_{#2}}

%\def\mlfp#1{\f{pppf}\{#1\}}

\def\PHobjp#1#2#3{\PHobj{{#1}_{#2}}{{#1}_{#3}}}
\def\Obj{\mathbf{Obj}}
\def\powerset{\wp}
\def\gCont{\f{maxCont}}

\def\muconcr{\ell}
\def\uconcr{\muconcr_\ctx}

\begin{comment}
%\newcommand{\abstr}[1]{#1^\wedge}%\text{\textasciicircum}}
%\def\priomax{\mathsf{prio}_{max}}
\def\procs{\mathsf{procs}}
\def\allprocs{\mathsf{allProcs}}
\def\pfp{\mathsf{pfp}}
\def\pfpprocs{\mathsf{pfpProcs}}
%
\def\ctx{\varsigma}
\def\w{\omega}
%\def\aBS{\abstr{\BS}}
%
\def\Req{\mathrm{Req}}
\def\Sol{\mathrm{Sol}}
\def\Cont{\mathrm{Cont}}
\def\A{\mathcal{A}}
\def\cwA{\A_\ctx^\w}
\def\cwReq{\Req_\ctx^\w}
\def\cwSol{\Sol_\ctx^\w}
\def\cwCont{\Cont_\ctx^\w}
%
%
%
\end{comment}

% Macros
\newcommand{\sN}{\mathbb{N}}
\newcommand{\sNN}{\mathbb{N}^\bullet}
\def\DEF{\stackrel{\Delta}=}
\def\EQDEF{\stackrel{\Delta}\Leftrightarrow}
\newcommand{\segm}[2]{\llbracket #1; #2 \rrbracket}
\def\indexes#1{\mathbb{I}^{#1}}
\def\f#1{\mathsf{#1}}
\def\prio{\mathsf{prio}}
\newcommand{\bottom}{\perp}
\newcommand{\stable}{\mathsf{stable}}
\newcommand{\update}{\mathsf{update}}
\newcommand{\components}{\PHs}
%\newcommand{\cs}{CS}
\newcommand{\cs}{\textcolor{red}{\Delta}}

% Restriction notation
\def\restriction#1#2{\mathchoice
              {\setbox1\hbox{${\displaystyle #1}_{\scriptstyle #2}$}
              \restrictionaux{#1}{#2}}
              {\setbox1\hbox{${\textstyle #1}_{\scriptstyle #2}$}
              \restrictionaux{#1}{#2}}
              {\setbox1\hbox{${\scriptstyle #1}_{\scriptscriptstyle #2}$}
              \restrictionaux{#1}{#2}}
              {\setbox1\hbox{${\scriptscriptstyle #1}_{\scriptscriptstyle #2}$}
              \restrictionaux{#1}{#2}}}
\def\restrictionaux#1#2{{#1\,\smash{\vrule height .8\ht1 depth .85\dp1}}_{\,#2}} 

%\renewcommand{\restriction}[2]{#1_{#2}}

% Commandes À FAIRE
\usepackage{color} % Couleurs du texte
%\definecolor{darkgreen}{rgb}{0,0.5,0}
%\newcommand{\towrite}[1]{\textcolor{darkgreen}{[#1]}}
\newcommand{\todo}[1]{\textcolor{red}{\textbf{[#1]}}}



% Special notations
\newcommand{\ie}{i.e.,\ }
\newcommand{\Ie}{I.e.,\ }
\newcommand{\eg}{e.g.,\ }
\newcommand{\Eg}{E.g.,\ }
\newcommand{\resp}{resp.\ }
\newcommand{\Resp}{Resp.\ }

% Césures
\hyphenation{pa-ra-me-tri-za-tion}
\hyphenation{pa-ra-me-tri-za-tions}



\def\lastname{Folschette \emph{et al.}}

\begin{document}
\begin{frontmatter}
\title{Sufficient Conditions for Reachability in Automata Networks with Priorities}

\author[irccyn,unikassel]{Maxime Folschette}
\ead{Maxime.Folschette@irccyn.ec-nantes.fr}
\author[lri]{Loïc Paulevé}
\author[irccyn]{Morgan Magnin}
\author[irccyn]{Olivier Roux}

\address[irccyn]{LUNAM Universit\'e, \'Ecole Centrale de Nantes, IRCCyN UMR CNRS 6597\\
(Institut de Recherche en Communications et Cybern\'etique de Nantes)\\
1 rue de la No\"e - B.P. 92101 - 44321 Nantes Cedex 3, France.}

\address[unikassel]{School of Electrical Engineering and Computer Science,\\
  University of Kassel, Germany}

\address[lri]{CNRS, Laboratoire de Recherche en Informatique (LRI)\\
		Université Paris-Sud - CNRS UMR 8623, France}


\begin{abstract}
In this paper,
we develop a framework for an efficient under-approximation of the dynamics of
Asynchronous Automata Networks (AANs).
An AAN is an Automata Network with synchronised transitions between automata,
where each transition changes the local state of exactly one automaton
(but any number of synchronizing local states are allowed).
The work we propose here
is based on static analysis by abstract interpretation,
which allows to prove that reaching a state with a given property
is possible,
without the same computational cost of usual model checkers:
the complexity is polynomial with the total number of local states and exponential with the number of
local states within a single automaton.
Furthermore, we address AANs with classes of priorities,
and give an encoding into AANs without priorities,
thus extending the application range of our under-approximation.
Finally, we illustrate our method for the model checking of
large-scale biological networks.
\end{abstract}
\begin{keyword}
discrete networks \sep
abstract interpretation \sep
reachability \sep
qualitative models \sep
systems biology
\end{keyword}
\end{frontmatter}



% vi:spell spelllang=en:
\section{Introduction}
\label{sec:intro}

Discrete modelling frameworks for biological networks is an active research field where formal
methods have be proven very
powerful~\cite{thomas1990biological,deJong02,snoussi_logical_1993}.
Such a work started in the seventies,
with the emergence of the notion of Boolean Network~\cite{kauffman69}
and its use to represent biological phenomena~\cite{Thomas73}.
It was later enriched in many directions and widely used to elucidate many biological questions.
Among these questions, a major one is to understand precisely how biological systems evolve and behave; why and how they change their usual behaviour, etc.
These questions are strongly linked to the (possible or inevitable) reachability of some states.
The ultimate goal is to discover how it could be possible to prevent biological systems from reaching some pathological conditions.

Of course, such formal models on which analyses are performed are abstract representations of the actual studied systems.
They are associated with parameters that have to be synthesised %so as to be as much as possible fitting with the real systems having some observed behaviors.
to give the most faithful representation of the real systems with their observed behaviours.
As a matter of fact, the abstractions we get are more or less rough or accurate.
Prevalent formal frameworks for such modelling activities are state-transition systems or process algebras. % Petri nets
We developed a quite similar framework named the Process Hitting~\cite{PMR10-TCSB},
consisting in a restriction of these frameworks where the evolution of a component is determined by the state of at most one other component that does not evolve.
This is modelled by actions of the form $X + Y \rightarrow X + Y'$, where $X$ behaves like a catalyst molecule that “hits” another molecule $Y$ and changes it into $Y'$, without being changed itself.
Assuming catalysts are always available, this can represent any biochemical system made of monomolecular reactions, and can also represent catalytic networks such as metabolic networks.
%, with the aim of avoiding to build the whole state space in order to be able to tackle very large systems
Our motivation behind this framework was to design a model and analysis techniques adapted to biological modelling.
These analyses avoid building the whole state space, which allows to tackle very large systems (that would have led to a huge number of states, hopelessly too huge to be analysed).
They are based on the fact that most biological models have few levels of expression per component:
in Boolean networks~\cite{kauffman69,Thomas73} there are only two levels per component, and in their multivalued equivalent~\cite{deJong02}, components rarely have more than four levels.

Besides, one further objective of our work is now %to be more accurate in the description of the studied systems.
to improve the accuracy of the description of the studied systems dynamics.
The idea for this is to introduce timing features into models:
we are interested in taking into account some knowledge about the relative length of some phenomena as it is a way to refute some models (or parameters) that are inconsistent with the observed dynamic behaviours.
In this paper, we are dealing with these timing properties through priorities,
that are based on the simple founding idea that actions with higher priority have to be processed before the ones with lower priority.
Furthermore, due to the Process Hitting framework restrictions, multimolecular reactions were previously not immediately available, but one could simulate them with an encoding called “cooperative sort”.
That encoding however introduces extra reactions,
that produce a temporal shift between the presence of the reactants
and the playability of the reaction.
This is where the priorities become useful, if not necessary:
the extra reactions can for example be given “infinite speed” (highest priority) so that they do not affect the behaviour of “normal” (lower priority) reactions, including the multimolecular ones.

The approach used in this paper
consists in considering a broader class of models,
that we call Asynchronous Automata Network,
and that allows to naturally model these cooperations
by defining several requisites for a transition.
Moreover, such automata networks are still compatible with the notion of priority,
that can also be used to model different reaction rates in the model.
Asynchronous Automata Networks
(and, a fortiori, their restriction, the Process Hitting framework)
can be considered as a subset of Communicating Finite State Machines
or safe Petri Nets~\cite{PMR-PetriNets}.
%or Synchronous Automata Networks,
%is inspired from the $\pi$-calculus,
Our work is thus related to such semantic ramifications
of extending traditional process algebras with the concept of priority
that allows for some transitions to be given precedence over others.
We focus here on static priorities that allow to model
time constraints such as reaction rates or delays between regulations,
but can also model preemptions between evolution branches.

Until now, such a priority scheduling of the actions was not studied extensively in the different formal modelling frameworks dedicated to systems biology.
Nevertheless, such an attempt has been carried out for Petri nets by F.~Bause~\cite{Bause97},
and the concept of priority relations among the transitions of a network has also more recently been introduced by A.~K.~Wagler \textit{et~al.}~\cite{waw,WaglerW12} in order to allow modelling deterministic systems for biological applications.
The concept of priority is rather straightforward in the approach of process algebras as it was shown by R.~Cleaveland and M.~Hennessy in~\cite{Cleaveland199058} and their abstractions and equivalences were studied in~\cite{Cleaveland:2007:PAP:1282576.1282847}.
It was later extended for applications in the field of systems biology by M.~John \textit{et~al.}~\cite{jlnu2010}.

\subsection*{Contributions}
In this paper, we develop a method
that allows to efficiently compute under-approximations of the dynamics
of Asynchronous Automata Networks (AANs),
generalising a prior work carried out on Process Hitting~\cite{PMR12-MSCS}.
Rather than using brute force or symbolic model checking techniques,
our method focuses on static analysis by abstract interpretation.
Our work aims at checking reachability properties that,
for a given initial state $s$ and the discrete expression level $n$
of a given component $x$,
have the form:
“Starting from state $s$, is it possible to reach
a state in which $x$ is at level $n$?”,
thus answering either “True”
(in which case the reachability is formally proven)
or “Inconclusive”
(which however does not stand for “False”, that can be proven by other analysis tools,
such as the over-approximation proposed in~\cite{PMR12-MSCS}).
Moreover, the successive or simultaneous reachability of several local states
can also be checked,
and in the case of a “True” response,
an execution path satisfying the property can be produced.
Extensions of AANs where transitions are split in priority classes are addressed with an encoding in
AANs without classes of priorities.
Our work thus allows to efficiently analyse the dynamics
of regulation networks,
especially the widespread Logical Networks \cite{Thomas95,deJong02},
which encompass variables with a limited number of discrete values
alongside with evolution functions or focal parameters.
To show the scalability of our method, we apply it to two
large-scale biological models with around 100 components.

\pref{sec:ph} presents the Asynchronous Automata Networks (AANs)
without any notion of classes of priorities.
In \pref{sec:sa},
we develop our under-approximation method allowing to efficiently
compute reachability analyses;
we also show how to extract a valid execution path if the response is positive,
and propose two refinements in case one needs to check
successive or simultaneous reachability properties.
The framework of AANs with classes of priorities is given in \pref{sec:flattening},
alongside with an encoding into AANs without classes of priorities
(or, equivalently, with one class of priority).
Finally, \pref{sec:example} provides a detailed example and two large-scale examples
of the application of our method,
and \pref{sec:ccl} gives a conclusion and a discussion about our work.

The main additions in this paper compared to~\cite{FPMR13-CS2Bio} are
\pref{ssec:concret} that allows the extraction of a concrete trace of execution
when our static analysis method is conclusive,
\pref{ssec:ordered-ua} that refines our approach in the case
of a successive reachability in order to increase the conclusiveness,
and \pref{sec:flattening} which states that any ANN
with any number of classes of priorities
can be represented into a simple ANN
(or, equivalently, with only one class of priority),
thus extending the scope of our method.
We note however that the use of AANs without priorities alone
instead of Process Hitting models
allows to simplify the notations, but does not increase
the range of applicability of the results.
Finally, \pref{sec:example} has been improved with a detailed example
and a new large-scale example with a quantitative examination of the results.



\subsection*{Notations}
\label{notations}
%We denote: $\segm{a}{b} = \{ a, a+1, \dots, b-1, b \}$.

\paragraph*{Sets}
If $A$ is a finite set,
$\card{A}$ is the cardinality of $A$
and $\powerset(A)$ is the power set of $A$.
$\sN$ is the set of natural numbers,
$\sN^* = \sN \setminus \{ 0 \}$ is the set of positive natural numbers
and $\segm{x}{y} = \{ x, x+1, \dots, y-1, y \}$ is the set of natural numbers from $x$ to $y$ included.
The Cartesian product of sets is denoted by $\times$ and
if $z$ is a tuple of $n$ components, $\toset{z}$ denotes the corresponding set:
$\toset{z} = \{z_1, \cdots, z_n\}$.

\paragraph*{Sequences}
We denote by $\emptyseq$ the empty sequence.
If $n \in \sN$ and
$x = (x_i)_{i \in \segm{1}{n}}$ is a sequence of elements indexed by $i \in \segm{1}{n}$,
then $\card{x} = n$ is the length of $x$
%we denote %$|x| = (b-a)+1$ the size of this sequence and
and $\indexes{x} = \segm{1}{n}$ is the set of indexes of this sequence.
Furthermore, if $a, b \in \indexes{x}$ with $a \leq b$,
then $x_{a..b} = (x_i)_{i \in \segm{a}{b}}$ is the subsequence of $x$
between indexes $a$ and $b$ included.
%from element $n$ to element $m$ inclusive.
Finally, if $x$ is a sequence, $\toset{x}$ also denotes the corresponding set:
$\toset{x} = \{x_1, \cdots, x_{\card{x}}\}$.

\paragraph*{Digraphs}
If $(V,E)$ is a directed graph whose set of nodes is $V$ and whose set of edges is $E \subseteq V\times V$,
the children of a node $n$ are given by
$\childs: V \to \powerset(V)$, with
$\childs(n) = \{ m\in V\mid (n,m)\in E\}$;
its parents are given by
$\parents: V \to \powerset(V)$ with
$\parents(n) = \{ m\in N\mid (m,n)\in E\}$;
and its successors are given by
$\conn_{(V,E)}(n)$ which is the least fixed point containing $n$
of the function
$\f{f}\conn:\powerset(V)\to \powerset(V)$
with $\f{f}\conn(W) = \bigcup_{m\in W} \childs(m)$.

%\paragraph*{Functions and least fixed point}
%If $A$ and $B$ are sets,
%$f : A \rightarrow B$ denotes a function $f$ that maps the elements of $A$ to elements of $B$.
%If $f$ is a monotonically increasing and bounded function, then
%$\lfp{x_0}{x}{x'}$ is the least fixed point of the function $x \mapsto x'$ which is greater than $x_0$.



\section{Asynchronous Automata Networks}
\label{sec:ph}

We give in this section the definition and the semantics of the Asynchronous Automata Networks (AANs).
It is a restriction of the classical (synchronous) Automata Networks
where each set of transitions sharing the same label
can only change one local state at a time.
We also discuss how it is related to the Process Hitting framework
(with or without classes of priorities).
Another definition of AANs introducing classes of priorities
is proposed in \pref{sec:flattening},
where we also show that they have the same expressivity as AANs.

\medskip



We consider an AAN (\pref{def:ph})
which gathers a finite number of \emph{automata},
each one containing a finite number of \emph{local states}.
A local state is noted $a_i$, where $a$ is the name of the automaton it belongs to,
and $i$ is the identifier of this local state within automaton $a$.
A \emph{global state} of the system is a Cartesian product
with exactly one local state from each automata.

The concurrent interactions between local states are defined by a set of \emph{actions}.
An action stands for a set of transitions sharing the same label,
so that playing them changes exactly one local state.
Therefore, an action is denoted by $\PHhit{A}{b_j}{b_k}$,
where $A$ is a set of local states
and $b_j$ and $b_k$ are local states of a same automaton $b$;
it is moreover required that $b_j \neq b_k$ and that
$A$ does not contain a local state of $b$ or two local states of the same automaton.
An action $h=\PHfrappe{A}{b_j}{b_k}$ is read as
``$A$ \emph{hits} $b_j$ to make it bounce to $b_k$'',
and $A$, $b_j$, $b_k$ are called respectively the (set of)
\emph{hitters}, the \emph{target} and the \emph{bounce} of the action,
and can be referred to as $\PHhitter(h)$, $\PHtarget(h)$ and $\PHbounce(h)$, respectively.

\begin{definition}[Asynchronous Automata Networks]
\label{def:ph}
  An \emph{Asynchronous Automata Network} (AAN) is a triplet $\PH = (\PHs; \PHl; \PHa)$,
  where:
  \begin{itemize}
    \item $\PHs \DEF \{a, b, \dots, z\}$ is the finite set of \emph{automata};
    \item $\PHl \DEF \underset{a \in \PHs}{\times} \PHl_a$ is the finite set of
      (global) \emph{states},
      where $\PHl_a = \{a_0, \ldots, a_{l_a}\}$ is the finite set of \emph{local states}
      of automaton $a \in \PHs$, with $l_a \in \sN^*$,
      and so that:
      $\forall (a_i; b_j) \in \PHl_a \times \PHl_b, a \neq b \Rightarrow a_i \neq b_j$;
    \item $\PHa \DEF \{\PHfrappe{A}{b_j}{b_k} \mid
      b \in \PHs \wedge (b_j; b_k) \in \PHl_b \times \PHl_b \wedge
      b_j \neq b_k \wedge
      \forall a \in \PHs, \card{A \cap \PHl_a} \leq 1 \wedge
      A \cap \PHl_b = \emptyset \}$ is the finite set of \emph{actions}.
  \end{itemize}
  Furthermore,
  we call $\PHproc \DEF \bigcup_{a \in \PHs} \PHl_a$ the set of all local states in the model.
\end{definition}

\paragraph{Notations}
The automaton that a local state $a_i$ belongs to is referred to as $\PHsort(a_i) = a$,
and if $A \subseteq \PHproc$, we denote:
$\PHsortf{A} = \{ \PHsortf{a_i} \in \PHs \mid a_i \in A \}$.
Given a state $s\in \PHl$, the local state of automaton $a \in \PHs$ present in $s$ is denoted by $\PHget{s}{a}$, that is, the $a$-coordinate of the state $s$.
If $a_i \in \PHproc$, we denote $a_i \in s \Leftrightarrow \PHget{s}{a} = a_i$;
and if $A\subseteq\PHproc$, $A\subseteq s \Leftrightarrow \forall a_i\in A, \PHget{s}{a} = a_i$.

\begin{definition}[Semantics of an AAN ($\PHPtrans$)]
\label{def:play}
  If $\PH = (\PHs; \PHl; \PHa)$ is an AAN,
  an action $h = \PHhit{A}{b_j}{b_k} \in \PHa$ is \emph{playable} in $s \in \PHl$
  if and only if $A \subseteq s$ and $\PHget{s}{b} = b_j$.
  In such a case, $(s \PHplay h)$ stands for the state resulting from the play of the action $h$ in $s$, which is defined by:
    $\PHget{(s \PHplay h)}{b} = b_k$
  and
    $\forall a \in \PHs, a \neq b, \PHget{(s\PHplay h)}{a} = \PHget{s}{a}$.
  Moreover, we denote: $s \PHPtrans (s \PHplay h)$.

  If $s \in \PHl$,
  a \emph{scenario} $\delta$ from $s$ is a (possibly empty) sequence of actions of $\PHh$
  that can be played successively in $s$.
  The set of all scenarios from $s$ is noted $\Sce(s)$.
\end{definition}



\begin{example}
  \pref{fig:ph-livelock} gives an example of AAN where:
  \begin{align*}
    \PHs &= \{ a, b, c \} \enspace,
      & \PHl_a &= \{ a_0, a_1 \} \enspace, \\
    \PHl_b &= \{ b_0, b_1 \} \enspace,
      & \PHl_c &= \{ c_0, c_1 \} \enspace, \\
    \PHh = \{ \quad
      & \PHfrappem{a_1}{a_0} \enspace , \quad
        \PHfrappes{b_0}{a_0}{a_1} \enspace , \\
      & \PHfrappem{b_1}{b_0} \enspace , \quad
        \PHfrappes{a_0}{b_0}{b_1} \enspace , \\
      & \PHfrappes{a_1, b_1}{c_0}{c_1} \enspace \quad \qquad \qquad
      \quad \}
  \end{align*}

\begin{figure}[tb]
  \centering
  \scalebox{1.3}{
  \begin{tikzpicture}[aan]
%    \path[use as bounding box] (-.2,-.5) rectangle (7.2,5.7);
    \TSort{(0,3)}{a}{2}{b}
    \TSort{(0,0)}{b}{2}{t}
    \TSort{(4,1)}{c}{2}{r}
    
    \TAction{a_0.south west}{b_0.north west}{b_1.north west}{bend right=20}{left}
    \TAction{b_0.north east}{a_0.south east}{a_1.south west}{bend right=20}{right}
    \TActionPlur{}{a_1.north}{a_0.north east}{}{1.5,4}{right}
    \TActionPlur{}{b_1.south}{b_0.south east}{}{1.5,-1}{left}
    \TActionPlur{a_1, b_1}{c_0.west}{c_1.south west}{}{2.5,1.5}{left}
    
    \TState{a_1, b_0, c_0}
  \end{tikzpicture}
  }
  \caption{%
  \label{fig:ph-livelock}%
    An example of AAN.
    This model represents the interaction of two components $a$ and $b$,
    whose production is mutually exclusive and that degrade over time.
    Moreover, these two components can cooperate to activate $c$
    if their “active” states ($a_1$ and $b_1$)
    are present in the same state.
    Automata are represented by labelled boxes
    and local states by circles with their identifier on the side.
    Actions are represented by a dot connected by an edge to the set of hitters
    and by an arrow to the target, followed by another dotted arrow towards the bounce.
    Greyed local states stand for the following possible global state:
    $\PHstate{a_1, b_0, c_0}$.
  }
\end{figure}

\end{example}



\begin{remark}[Comparison with the Process Hitting]
  The Process Hitting framework
  previously introduced in~\cite{PMR10-TCSB}
  is a restriction of the AAN formalism;
  indeed, a Process Hitting model is an AAN so that
  $\forall h \in \PHh, 0 \leq \card{\hitter{h}} \leq 1$.
  However, the AANs defined in this paper have the same expressivity than
  the Process Hitting with classes of priorities, as previously
  introduced in~\cite{FPMR13-CS2Bio}.
  Indeed, in the Process Hitting with at least 2 classes of priorities,
  it is possible to use additional automata called “cooperative sorts”
  in order to model the actions in an AAN that have more than one hitter.
\end{remark}


% vi:spell spelllang=en:
\section{Under-approximation of Reachability}\label{sec:sa}

We present a static analysis that takes as input an AAN
$\PH = (\PHs; \PHl; \PHa)$,
an initial state in $\PHl$, and a local state in $\Proc$.
Its objective it to identify sufficient conditions
that ensure the existence of a scenario starting from the initial state,
and leading to a state where the given local state is present.

A classical approach for determining if a local state of an automaton can be
reached from a given initial state is to build sequences of transitions from the initial state
until reaching a state in which the given local state is active.
%containing the given local state is reached.
This is essentially what model-checkers do, and the complexity of such analysis (PSPACE-complete
\cite{Harel02}) makes it intractable on large systems, even with advanced symbolic approaches
\cite{PMR12-MSCS}.

Our approach relies on abstractions of scenarios (\pref{ssec:abstr-sce}) that have been introduced in
\cite{PMR12-MSCS} for the static analysis of reachability in the Process Hitting framework,
a particular sub-class of AAN (see the Remark at the end of the previous section).
In this paper, we generalize the static analysis for the case of the under-approximation of
reachability in any AAN (\pref{ssec:ua}).

In \pref{ssec:concret}, we detail a procedure to extract a witness scenario
concretizing a given local state reachability property.
Finally, we discuss in \pref{ssec:ordered-ua} how the static analysis can be
extended to sequential (sub)states reachability properties.


\subsection{Abstractions for Scenarios}
\label{ssec:abstr-sce}

The approach presented in this section is based on two complementary notions,
the \emph{objectives} and their \emph{local causality},
that are intertwined in so-called \emph{Local Causality Graphs}.

\subsubsection{Objectives}

An \emph{objective} (\pref{def:obj}) denotes the reachability of a local state (e.g., $a_j$) of a given automaton $a$
from the initial local state of that automaton (e.g., $a_i$).
Such an objective is written $\PHobjp{a}{i}{j}$.
Successive objectives are described with objective sequences.

\begin{definition}[Objective ($\Obj$) \& Objective Sequence ($\OS$)]
\label{def:obj}
  If $a \in \components$, the reachability of a local state $a_j$ from a local state $a_i$ is called an \emph{objective}, noted $\PHobj{a_i}{a_j}$.
  The set of all objectives is noted:
  \[\Obj \DEF \{ \PHobj{a_i}{a_j} \mid a \in \components \wedge (a_i, a_j) \in \PHl_a \times \PHl_a \} \enspace.\]
  For an objective $P = \PHobj{a_i}{a_j} \in \Obj$, we define: $\PHsort(P) \DEF
  a$, $\PHtarget(P)\DEF a_i$, $\PHbounce(P)\DEF a_j$.
  Finally, $P$ is said \emph{trivial} iff $a_i = a_j$.

  We define an \emph{objective sequence} as a sequence of objectives in which each objective target must be equal to the previous objective bounce of the same automaton, if it exists.
  The set of all objective sequences is denoted by $\OS$.
  Given $\w\in\OS$, $\PHsort(\w) \DEF \{\PHsort(\w_i)\mid i\in\indexes{\w}\}$.
For each automaton $a\in\PHsort(\w)$,  the first local state of $a$ referenced in
$\w$ is denoted by $\first_a(\w) \DEF \target{\w_m}$, where $m=\min\{n\in\indexes{\w}\mid \PHsort(\w)=a\}$.
  The set of objective sequences starting in a state $s\in\PHl$ are denoted by
  $\OS(s)\DEF \{\w\in\OS\mid\forall a\in\PHsort(\w), \first_a(\w)\in s\}$.
\end{definition}


We define the partial ordering relation $\ltw$ between two objective sequences (\pref{def:ltw}) as
follows:
$w\ltw w'$ if and only if there exists a mapping between all the objective bounces of $w'$ in $w$
that preserves the sequentiality, provided that each automaton starts in the same
local state.

\begin{definition}[$\ltw \subset \OS\times\OS$]\label{def:ltw}
$\w\ltw\w'$ if and only if
$|\w|\geq|\w'|$,
$\forall a\in\PHsort(\w'), a\in\PHsort(\w) \wedge \first_a(\w')=\first_a(\w)$;
and there exists a mapping $\phi: \indexes{\w'}\mapsto \indexes{\w}$ such that
$\forall n,m \in\indexes{\w'}, n<m \Leftrightarrow \phi(n)<\phi(m)$,
and
$\forall n\in\indexes{\w'}, \bounce{\w'_n}=\bounce{\w_{\phi(n)}}$.
\end{definition}

\begin{example}
\[\obj{b_0}{b_1}\concat\obj{a_0}{a_1}\concat\obj{b_1}{b_2}
\ltw
\obj{a_0}{a_1}\concat\obj{b_0}{b_2}
\ltw
\obj{b_0}{b_2}\]
\[\obj{b_0}{b_1}\not\ltw\obj{b_0}{b_2}
\quad\text{and}\quad
\obj{b_0}{b_2}\not\ltw\obj{b_0}{b_1}\]
\end{example}

An objective sequence can be seen as an abstract representation of a set of scenarios that describe
(part of) the successive state changes of the automata.
We denote by $\concr(\w)$ (\pref{def:concr}) the set of scenarios matching with an objective sequence
$\w$ in the state $s\in\PHl$.
It is essentially all the scenarios for which there exists a mapping from the bounces of each
objective to the bounce of an action in the scenario which preserve the sequential ordering.

\begin{definition}[$\concr: \OS \to \powerset(\Sce)$]\label{def:concr}
Given a state $s\in\PHl$ and an objective sequence
$\w\in\OS(s)$, $\concr(\w)$ is the set of scenarios matching with $\w$:
\begin{align*}
\concr(\w) \DEF \{ \delta\in\Sce(s)\mid &\ 
\w^\vartriangle\neq \emptyseq\Rightarrow
\bounce{\delta_{\card\delta}} = \bounce{\w_{\card\w}}
\\ &
 \wedge \exists \phi:\indexes{\w}\mapsto\indexes{\delta},
    (\forall n,m\in\indexes{\w}, n<m \Leftrightarrow \phi(n)\leq\phi(m)
\\ & \qquad
	\wedge \forall n\in\indexes{\w},
	  \bounce{\w_n} \in s\play\delta_{1..\phi(n)})
\}
\enspace,
\end{align*}
in which $\omega^\vartriangle$ refers to the objective sequence $\omega$ where
trivial objectives have been removed.
The notation $\delta_{j..k}$
denotes the subsequence of $\delta$ between indexes $j$ and $k$,
as defined on page \pageref{notations}.
\end{definition}

From \pref{def:ltw} and \pref{def:concr}, we derive that if
$\w \ltw \w'$, then the scenarios matching $\w$ also match $\w'$ (\pref{lem:ltw}).
%$\concr(\w)\subseteq\concr(\w')$ (\pref{lem:ltw}).
\begin{lemma}\label{lem:ltw}
$\forall \w, \w' \in \OS, \forall s \in \PHl,
\w\ltw\w' \Longrightarrow \concr(\w)\subseteq\concr(\w')$\enspace.
\end{lemma}

Given a non-empty set $\Delta\subseteq\Sce(s)$ of non-empty scenarios that have a common last bounce
($\exists a_i\in\Proc, \forall\delta\in\Delta,
\bounce{\delta_{\card\delta}}=a_i$), one can define an abstraction $\alpha_s$ of
such a set as the smallest (according to $\ltw$) objective sequence $\w\in\OS(s)$ such that
$\bounce{\w_{\card\w}} = \bounce{\delta_{\card\delta}}, \delta\in\Delta$
and
$\forall\delta\in\Delta, \exists \phi:\indexes{\w}\mapsto\indexes{\delta}
    (\forall n,m\in\indexes{\w}, n<m \Leftrightarrow \phi(n)\leq\phi(m))
	\wedge \forall n\in\indexes{\w},
	  \bounce{\w_n} \in s\play\delta_{1..\phi(n)})$.
In such a setting, $\alpha_s$ and $\concr$ form a Galois connection between sets
of scenarios and objective sequences.


\subsubsection{Local Causality for Objectives}

The existence of a scenario from a state $s$ leading to a state
where a given local state $a_j$ is present
can be reformulated as the checking for the non-emptiness of $\concr(\obj{a_i}{a_j})$, where
$a_i=\get{s}{a}$.
A first hint for checking this emptiness is to look for actions that have to be played in order to
reach the state $j$ from $i$ within the automaton $a$.

Given an objective $P = \obj{a_i}{a_j} \in\Obj$, we define $\BS(P)$ the \emph{bounce sequences} of
$P$ as the set of minimal sequences of actions hitting $a$ in which the bounce of each action is
the target of the following action (\pref{def:bs}).

\begin{definition}[Bounce Sequence ($\BS$)]\label{def:bs}
A \emph{bounce sequence} $\zeta$  is a sequence of actions such that
$\forall n\in\indexes{\zeta}, n<|\zeta|,
\bounce{\zeta_{n}} = \target{\zeta_{n+1}}$.
$\BS$ denotes the set of minimal bounce sequences.
We refer to the set of bounce sequences \emph{resolving} the objective $P$ as
$\BS(P)$:
\begin{align*}
\BS(\obj{a_i}{a_j}) \DEF \{ \zeta\in\BS\mid & \target{\zeta_1}=a_i\wedge
			    \bounce{\zeta_{|\zeta|}}=a_j \\
& \wedge \forall m,n\in\indexes{\zeta}, n>m, \bounce{\zeta_n}\neq\target{\zeta_m}
				\}\enspace.
\end{align*}
\end{definition}

Therefore,
$\BS(\obj{a_i}{a_i}) = \{\emptyseq\}$; and
$\BS(\obj{a_i}{a_j}) = \emptyset$ if there is no possibility to reach $a_j$ from
$a_i$.

\begin{example}
Given the AAN of \pref{fig:ph-livelock},
$\BS(\obj{c_0}{c_1}) = \{\PHfrappes{a_1,b_1}{c_0}{c_1}\}$;
and 
$\BS(\obj{a_0}{a_1}) = \{\PHfrappes{b_0}{a_0}{a_1}\}$.
\end{example}

From \pref{def:bs}, we can derive that any scenario matching with an objective includes all the
actions of one of the bounce sequences of the objective (\pref{lem:bs-concr}).

\begin{lemma}
Given a state $s\in \PHl$ and an objective $\obj{a_i}{a_j}$ with $\get{s}{a}=a_i$,
$\forall\delta\in\concr(\obj{a_i}{a_j})$,
$\exists\zeta\in\BS(\obj{a_i}{a_j})$ such that
$\exists \phi: \indexes{\zeta}\to\indexes{\delta}$
with
$\forall n,m\in\indexes{\zeta}, n<m \Leftrightarrow \phi(n)<\phi(m)$
and
$\forall n\in\indexes{\zeta}$, $\zeta_n = \delta_{\phi(n)}$.
\label{lem:bs-concr}
\end{lemma}



\subsubsection{Local Causality Graph}

\modmf{%
The previous subsections permitted to define all the tools needed to build the Local Causality Graph (LCG).
This graph is denoted $\cwB$ where $u$ is the local state to be reached
from the initial state $s$.
It will be the basis of our static analysis,
as it represents the causality links between the different objectives involved
in the solving of the reachability of $u$.
This LCG is built recursively by considering some required local states (\eg $u$),
linking them to objectives (\eg $\PHobj{t}{u}$, if $t \in s$),
and locally refining these objectives in order to include new required local states
from other automata.
An example of LCG is depicted in \pref{fig:sa-livelock}.
}

Technically,
one can iteratively refine a given objective into (possibly several) objective
sequences extracted from its bounce sequences.
Indeed, the refinement of an objective $P$ with $\zeta\in\BS(P)$
consists in prepending to $P$ the
objectives leading to the activation of all hitters of the actions in $\zeta$,
in the same sequential order.
The generalization of this refinement to an objective sequence would
naturally require to consider all possible
interleaving between the refinements~\cite{PMR12-MSCS}.

For example, given the AAN of \pref{fig:ph-livelock},
in the state $\langle a_0,b_0,c_0\rangle$,
the objective $\obj{a_0}{a_1}$ can be refined in the objective sequence
$\obj{b_0}{b_0}\concat\obj{a_0}{a_1}$
and
the objective $\obj{c_0}{c_1}$ can be refined in the objective sequence
$\obj{a_0}{a_1}\concat\obj{b_0}{b_1}\concat\obj{c_0}{c_1}$
or
$\obj{b_0}{b_1}\concat\obj{a_0}{a_1}\concat\obj{c_0}{c_1}$
(which can then be further refined).

\medskip

Formally, a Local Causality Graph (LCG)
is a digraph where $\cwV \subseteq \Proc\cup \Obj \cup \Sol \cup \sSol$ is the set of vertices,
with $\sSol=\powerset(\Proc)$ and $\Sol=\Obj\times\powerset(\sSol)$,
and $\cwE \subseteq \cwV \times \cwV$ is the set of oriented edges
(\pref{def:lcg}).
% 
% These refinements can be summarized with a so-called Local Causality Graph (LCG).
% Given an initial context $s$ and a local state $\myp\in\Proc$,
% an LCG $\cwB = (\cwV, \cwE)$
% is a digraph where $\cwV \subseteq \Proc\cup \Obj \cup \Sol \cup \sSol$ is the set of vertices
% where $\sSol=\powerset(\Proc)$ and $\Sol=\Obj\times\powerset(\sSol)$
% and $\cwE \subseteq \cwV \times \cwV$ is the set of oriented edges
% (\pref{def:lcg}).
A node in $\cwV\cap\Obj$ represents an objective to refine.
Such a node is linked to nodes in $\Sol$ which represent
sets of hitter sets extracted from a
bounce sequence of the objective
(that is, one node for each bounce sequence in $\BS(P)$); thus,
given an objective $P$, if $\zeta\in\BS(P)$
and $\abstr\zeta$ is the set of all hitters of $\zeta$,
% and $\abstr\zeta \DEF \{\PHhitter(\zeta_n) \mid
% n\in\indexes\zeta \}$,
then $P$ is linked to a node $\langle P, \abstr\zeta\rangle\in\Sol$
(\pref{def:lcg}\nobreakdash-\ref{lcg-obj-sol}).
For each hitter set $ps\in\abstr\zeta$,
the node $\langle P,\abstr\zeta\rangle\in\cwV\cap\Sol$ is
linked to the node $ps\in\sSol$
(\pref{def:lcg}\nobreakdash-\ref{lcg-sol-sync}).
Then, a node $ps\in\cwV\cap\sSol$ is linked to each local state $a_i \in ps$ is contains
(\pref{def:lcg}\nobreakdash-\ref{lcg-sync-ls}).
A local state node $a_i\in\cwV\cap\Proc$
is linked to each objective $\obj{a_j}{a_i} \in \Obj$, where
$a_j$ is either in the initial state $s$ or
is a local state node in the LCG
(\pref{def:lcg}\nobreakdash-\ref{lcg-ls-obj}).
Finally, given an objective node $\obj{a_j}{a_i}\in\cwV\cap\Obj$,
if a local state $a_k\neq a_i$ is in its successors (which is given by
$\conn_{(\cwBNodes,\cwBEdges)}(\obj{a_j}{a_i})$), then
$\obj{a_j}{a_i}$ is linked to $\obj{a_k}{a_i}$ in the LCG
(\pref{def:lcg}\nobreakdash-\ref{lcg-conn}).

\begin{definition}
\label{def:lcg}
Given a state $s \in \PHl$ and a local state $\myp\in \Proc$,
the Local Causality Graph (LCG) for reachability under-approximation
$\cwB \DEF (\Bv, \Be)$,
with
$\cwBNodes \subseteq \Proc \cup \Obj \cup \NSol \cup \sSol$
and
$\cwBEdges \subseteq \cwBNodes \times \cwBNodes$,
is the smallest graph such that:
\begin{enumerate}
\item
$\myp \in \cwBNodes$
\item
$a_i\in\cwBNodes\cap\Proc \Leftrightarrow \{ (a_i,\obj{a_j}{a_i}) \mid
a_j \neq \myp \wedge (a_j\in s \vee a_j\in\cwBNodes\cap\Proc)
\}\subseteq\cwBEdges$
\label{lcg-ls-obj}
\item
$P\in\cwBNodes\cap\Obj \Leftrightarrow 
	\{ (P,\langle P,\abstr \zeta\rangle) \mid \zeta\in\BS(P) \}\subseteq\cwBEdges$
\label{lcg-obj-sol}
\item
$\langle P, pps \rangle\in\cwBNodes\cap\NSol \Leftrightarrow
	\{ (\langle P,pps\rangle, ps) \mid ps\in pps \}\subseteq\cwBEdges$
\label{lcg-sol-sync}
\item
$ps \in\cwBNodes\cap\sSol \Leftrightarrow
	\{ (ps, a_i) \mid a_i\in ps \}\subseteq\cwBEdges$
\label{lcg-sync-ls}
\item
$\obj{a_i}{a_j}\in \cwBNodes\cap\Obj \Rightarrow 
	\{(\obj{a_i}{a_j},\obj{a_k}{a_j}) \mid a_k\neq a_j,$
\label{lcg-conn}
\\
\hspace*{4cm}
$a_k\in \conn_{(\cwBNodes,\cwBEdges)}(\langle\obj{a_i}{a_j},\abstr\zeta\rangle),$
\\
\hspace*{4cm}
$\zeta\in\BS(\obj{a_i}{a_j}) \}\subseteq\cwBEdges$
\end{enumerate}
with $\abstr\zeta \DEF \{\PHhitter(\zeta_n) \mid n\in\indexes\zeta \}$.
\end{definition}

\modmf{%
The definition of an LCG tackles two cases
where the target of an objective may be changed.
First,
the addition of objectives based on the local states already mentioned anywhere in the LCG,
as performed in \pref{def:lcg}\nobreakdash-\ref{lcg-ls-obj},
ensures to take into account the possible changes in the active local states
made by other objectives.
We thus try to ensure that a required local state is reachable even when
starting from a local state of the same automaton that is not in the initial state,
but that may be active at some point of the solving.
Second, \pref{def:lcg}\nobreakdash-\ref{lcg-conn},
allows to “re-target” objectives whose own solving changes their target.
This happens when playing the actions that activate the local states required to solve
an objective $P$ also changes the active local state of automaton $\PHsort(P)$.
In this case, the initial objective $P$ is re-targeted to another
objective $\PHobj{p}{\bounce{P}}$, where $p$ is the new active local state in $\PHsort(P)$.
Linking to all objectives of this kind ensures that all possible disturbances
are taken into account.
}

\modmf{%
Finally, we note that
the LCG of \pref{def:lcg} contains synchronizing nodes ($\sSol$)
that allow to express the need for several local states simultaneously
in order to play a given action.
This is the main difference regarding~\cite{PMR12-MSCS}
which tackled with Process Hitting, whose actions are limited to at most one hitter
and thus did not require this kind of synchronization.
However, \pref{def:lcg} is equivalent to the definition of the structure
given in~\cite{FPMR13-CS2Bio},
although it was adapted for AANs which notably allowed to simplify its formalism.
}


\subsection{Sufficient condition for reachability of a local state}
\label{ssec:ua}

Given a state $s \in \PHl$ and a local state $\myp\in\Proc$, the LCG $\cwB$ contains a set of objectives
that can be used to build a concrete scenario reaching $\myp$.
Because we are focused on sufficient conditions for reachability, we do not require that all
possible scenarios can be derived from the LCG.

In this section, we prove that if the LCG has no cycle, all its nodes in $\Obj$ have at least one
child, and all its nodes in $\sSol$ satisfy a
particular criteria, so-called independence,
then there exists a scenario that reaches $\myp$ from state $s$ (\pref{th:approxinf}).

A node $ps \in \sSol$ of the LCG is \emph{independent} (\pref{def:coherent}) if for each local state
$a_i\in ps$, none of the other local states $b_j\in ps$ have in their successors
a local state of automaton $a$ but that is different than $a_i$.
This criteria ensures that once a local state in $ps$ has been reached, reaching another local state
in $ps$ should not impact the first.

\begin{definition}[Independent synchronizations]
\label{def:coherent}
  In a LCG $\cwB = (\Bv, \Be)$,
  a node $ps \in \Bv\cap\sSol$ is \emph{independent} if and only if
  for each $a_i\in ps$,
  for each $b_j\in ps, b_j\neq a_i$,
  $a_k \in\conn_{\cwB}(b_j) \cap \Proc \Rightarrow a_k = a_i$.
\end{definition}

The intuition of the under-approximation of \pref{th:approxinf} is the following.
Given the initial state $s$, we recursively refine the initial objective
(which is: reaching $\myp$ from the initial state) according to its children.
As the LCG is acyclic, by hypothesis, such a recursion always terminates, and as all the objective nodes have at
least one child, it never gets stuck.
The refinement of an objective node $\obj{a_i}{a_j}$ acts as follows:
if the node has another objective node $\obj{a_k}{a_j}$ as child, we first refine the objective
$\obj{a_i}{a_k}$ (which, by construction, is necessarily in the LCG)
and then we refine the objective $\obj{a_k}{a_j}$.
If the objective node has only successors in $\Sol$ (bounce sequences), one is picked arbitrarily.
If $\langle P, pps\rangle$ is the chosen node,
by construction there exists $\zeta\in\BS(P)$ such that $\abstr\zeta = pps \in \powerset(\sSol)$.
If $\zeta = \emptyseq$ (for instance in the case where $a_i = a_j$), the recursion stops and we
continue to the next stage.
Otherwise, for each $n\in\indexes\zeta$,
for each $b_i \in\PHhitter(\zeta_n)$,
we refine the objective $\obj{b_j}{b_i}$, where $b_j$ is the state of $b$ in the current state.
By induction, $\obj{b_j}{b_i}$ is a child of $b_i$ in the LCG of \pref{def:lcg}.
Thus, we know that the current state of $b$ is $b_i$.
After having repeated this procedure for each $b_i\in\PHhitter(\zeta_n)$,
because all the nodes in $\sSol$, and in particular $\PHhitter(\zeta_n)$, are independent,
we know that all the local states in $\PHhitter(\zeta_n)$ are in the current state.
In addition, we know that the state of automaton $a$ has remained unchanged, otherwise
$\obj{a_i}{a_j}$ would have an objective child.
Hence, the action $\zeta_n$ is playable in the current state.
We can thus apply this action, modifying the state of $a$ to $a_j$ and continue to the next stage.
In the end, this recursive procedures builds a scenario from $s$ to a state containing $\myp$.

\begin{theorem}[Under-approximation]
\label{th:approxinf}
  Given an AAN $(\PHs; \PHl; \PHa)$,
  a state $s$ and local state $\myp$,
  if the LCG $\cwB$ contains no cycle,
  all nodes in $\Obj$ have at least one child,
  and all nodes in $\sSol$ are independent,
  then there exists a scenario $\delta\in\Sce$ such as $\myp\in s\play\delta$.
\end{theorem}

Regarding the complexity of the method,
computing the LCG is polynomial in the number of automata in $\PH$ and exponential in the number of local states in one automaton.
Checking the properties allowing to apply \pref{th:approxinf} is polynomial in the size of the graph.
Therefore, the building and checking processes can be considered as polynomial in the size
of the AAN, provided that each automaton only contains a few local states.
We note that this is particularly true for biological models, where
each component usually contains a limited number of expression levels.

We note furthermore that the method does not require any completeness of the bounce sequences
$\BS$.
Therefore, in order to reduce the number of bounce sequences to consider, and potentially remove
cycles and non-satisfying nodes,
one can consider only a sub-set of bounce sequences
(and in particular: a unique bounce sequence) for each objective.
However, such an approach requires to enumerate all possible combinations of bounce sequences
subsets, hence being exponential in the number of considered bounce sequences.
%objectives solved by at least two bounce sequences.



\begin{example}
  We consider in this example the AAN of \pref{fig:ph-livelock},
  and the initial state $\PHstate{a_1, b_0, c_0}$
  that is also represented.
  The under-approximation given in \pref{th:approxinf}
  concludes that $a_1$ is reachable from this initial state, as well as $b_1$.
  Nevertheless, it does not conclude regarding the reachability of $c_1$.
  This is due to the fact that the node $\{ a_1, b_1 \} \in \Bv \cap \sSol$
  is not independent because of its successor $a_0$ (and $b_0$)
  as we can see in the LCG of \pref{fig:sa-livelock}.
  (However, from the inconclusiveness of \pref{th:approxinf},
  one cannot conclude about the unreachability of $c_1$.
  Such analysis should be driven for instance
  with over-approximation methods developed in~\cite{PMR12-MSCS}.)
  
  This result is new compared to the method proposed in~\cite{PMR12-MSCS}.
  Indeed, the representation based on the Process Hitting that was proposed
  in this paper only allowed to represent “over-approximated” Boolean gates
  with the use of so-called cooperative sorts.
  This especially did not allow to model the fact that $a_1$ and $b_1$ could not
  be activated in the same state, but only in successive states.
  Thus, when using Process Hitting, $c_1$ was indeed reachable,
  contrary to the behaviour expected from an accurate Boolean gate.
  
  Finally, we note however that
  if actions $\PHhits{a_0}{b_0}{b_1}$ and $\PHhits{b_0}{a_0}{a_1}$
  are replaced respectively by
  $\PHhitm{a_0}{a_1}$ and $\PHhitm{b_0}{b_1}$,
  then the resulting saturated graph of local causality changes, and
  \pref{th:approxinf} concludes that $c_1$ is reachable from $\PHstate{a_1, b_0, c_0}$.
  The reader can also refer to \pref{ssec:ex-metazoan}
  for a detailed conclusive example.

\begin{figure}[tp]
  \centering
  \begin{tikzpicture}[aS]
    \node[Aproc] (c1) {$c_1$};
    \node[Aobj,below of=c1] (c01) {$\PHobj{c_0}{c_1}$};
%    \node[Aobj] (c01) {$\PHobj{c_0}{c_1}$};
    \node[Asol,below of=c01] (c01s) {};

    \node[Assol,below of=c01s] (a1a1ss) {$\{ a_1, b_1 \}$};
    \node[Aproc,below left of=a1a1ss] (a1) {$a_1$};
    \node[Aobj,below of=a1] (a11) {$\PHobj{a_1}{a_1}$};
    \node[Asol,below of=a11] (a11s) {};
    \node[Assol,below of=a11s] (na11s) {$\emptyset$};
    \node[Aobj,below left of=a1] (a01) {$\PHobj{a_0}{a_1}$};
    \node[Asol,below of=a01] (a01s) {};
    \node[Assol,below of=a01s] (a01ss) {$\{b_0\}$};
    \node[Aproc,below of=a01ss] (b0) {$b_0$};
    \node[Aobj,below of=b0] (b00) {$\PHobj{b_0}{b_0}$};
    \node[Asol,below of=b00] (b00s) {};
    \node[Assol,below of=b00s] (nb00s) {$\emptyset$};
    \node[Aobj,below left of=b0] (b10) {$\PHobj{b_1}{b_0}$};
    \node[Asol,below of=b10] (b10s) {};
    \node[Assol,below of=b10s] (nb10s) {$\emptyset$};

    \node[Aproc,below right of=a1a1ss] (b1) {$b_1$};
    \node[Aobj,below of=b1] (b11) {$\PHobj{b_1}{b_1}$};
    \node[Asol,below of=b11] (b11s) {};
    \node[Assol,below of=b11s] (nb11s) {$\emptyset$};
    \node[Aobj,below right of=b1] (b01) {$\PHobj{b_0}{b_1}$};
    \node[Asol,below of=b01] (b01s) {};
    \node[Assol,below of=b01s] (b01ss) {$\{a_0\}$};
    \node[Aproc,below of=b01ss] (a0) {$a_0$};
    \node[Aobj,below of=a0] (a00) {$\PHobj{a_0}{a_0}$};
    \node[Asol,below of=a00] (a00s) {};
    \node[Assol,below of=a00s] (na00s) {$\emptyset$};
    \node[Aobj,below right of=a0] (a10) {$\PHobj{a_1}{a_0}$};
    \node[Asol,below of=a10] (a10s) {};
    \node[Assol,below of=a10s] (na10s) {$\emptyset$};

    \path
    (c1) edge (c01)
    (c01) edge (c01s)
    (c01s) edge (a1a1ss)
    (a1a1ss) edge (a1) edge (b1)

    (a1) edge (a01) edge (a11)
    (a01) edge (a01s)
    (a01s) edge (a01ss)
    (a01ss) edge (b0)
    (a11) edge (a11s)
    (a11s) edge (na11s)
    (a0) edge (a10) edge (a00)
    (a10) edge (a10s)
    (a10s) edge (na10s)
    (a00) edge (a00s)
    (a00s) edge (na00s)

    (b0) edge (b10) edge (b00)
    (b10) edge (b10s)
    (b10s) edge (nb10s)
    (b00) edge (b00s)
    (b00s) edge (nb00s)
    (b1) edge (b01) edge (b11)
    (b01) edge (b01s)
    (b01s) edge (b01ss)
    (b01ss) edge (a0)
    (b11) edge (b11s)
    (b11s) edge (nb11s)
    ;
    \end{tikzpicture}
  \caption{
  \label{fig:sa-livelock}
    The local causality graph $\cwB$ on the AAN in \pref{fig:ph-livelock}
    for the reachability of $\myp = c_1$
    from the initial state $s = \PHstate{a_1, b_0, c_0}$.
    Elements in $\Proc$ are represented by rectangular nodes,
    elements in $\Sol$ are represented by small circles,
    and elements in $\sSol$ and $\Obj$ are the remaining borderless nodes.
    \pref{th:approxinf} is inconclusive on this example as
    node $\{ a_1, b_1 \} \in \sSol$
    is not independent (see \pref{def:coherent}).
    Indeed, $a_0$ is a successor of $b_1$, but $a_0 \neq a_1$
    (and the same also stands for $b_0$, which is a successor of $a_1$).
  }
\end{figure}
\end{example}



\subsection{Extraction of a Scenario}
\label{ssec:concret}

This section gives a recursive method to find a scenario that concretizes
the reachability of a given local state $\myp \in \Proc$,
from a given initial state $s_0 \in \PHl$,
provided that \pref{th:approxinf} answered positively on this couple of inputs.
The algorithm proposed in what follows relies on a traversal of the Local Causality Graph
$\cwBz$ that has been used for this conclusion.
The correctness of this extraction can be demonstrated
in the same fashion as \pref{th:approxinf}.

This algorithm consists in visiting all required nodes in a given order,
to build a sequence of actions that concretizes the objective.
The actions to perform on a given node,
listed below,
depend on the current state, the type of node, the markers on this node
and the state of the traversal which can be either “descending” (D) or “ascending” (A).
Nodes in $\Sol$ can be marked with a sequence of actions
and nodes in $\sSol$ can be marked with a set of local states.
When ascending, the traversal always ascend into the node it previously descended from
(in order to cover the same path backwards).
The traversal starts in node $\myp$ in descending mode,
the starting state is $s = s_0$
and the initial output is the empty sequence $\emptyseq$.

\begin{itemize}
  \item In a node $a_k \in \Proc$:
    \begin{itemize}
      \item[D)] Descend in the node $\PHobj{\PHget{s}{a}}{a_k} \in \Obj$.
      \item[A)] Ascend in the parent $\sSol$ node, if any,
        and remove the element $a_k$ from its marking set.
        If the node has no $\sSol$ parent, the traversal is complete
        and the output of the algorithm is the current output.
    \end{itemize}
  
  \item In a node $\PHobj{a_j}{a_k} \in \Obj$:
    \begin{itemize}
      \item[D)] Descend in an arbitrarily chosen node $\langle P, pps \rangle \in \Sol$,
        and mark it with an arbitrarily chosen sequence
        $\zeta \in \BS(P)$ so that $\abstr{\zeta} = pps$.
      \item[A)] Ascend in the parent $\Proc$ or $\Obj$ node.
    \end{itemize}
  
  \item In a node $\langle P, pps \rangle \in \Sol$ with the current marking $\zeta$:
    \begin{itemize}
      \item If $\zeta \neq \emptyseq$,
        descend in the node $ps \in \sSol$
        so that $ps = \hitter{\zeta_1}$
        and mark it with the set $ps$.
      \item If $\zeta = \emptyseq$,
        ascend in the parent node $P \in \Obj$ and carry on ascending.
    \end{itemize}
  
  \item In a $\sSol$ node with the current marking $m$:
    \begin{itemize}
      \item If $m \neq \emptyset$,
        descend in a child node $a_k \in \Proc$ arbitrarily chosen,
        so that $a_k \in m$ and carry on descending.
      \item If $m = \emptyset$,
        ascend in the parent $\Sol$ node.
        Let $\zeta$ be the current marking of this node.
        If $\zeta_1$ is playable in $s$:
          append $\zeta_1$ to the output,
          change the current state to $s \play \zeta_1$
          and mark this node with $\zeta_{2..|\zeta|}$.
        If $\zeta_1$ is not playable
        (which means that $\target{\zeta_1}$ has changed),
          then go to the node $\PHobj{\PHget{s}{\PHsort(\bounce{\zeta_1})}}{\bounce{\zeta_1}} \in \Obj$
          and start descending.
    \end{itemize}
\end{itemize}

The execution of such a traversal outputs a scenario from $s_0$.
In the following, we denote by $\Delta(\cwBz)$ the set of all scenarios extracted from the
local causality graph $\cwBz$.
Indeed, several scenarios may exist for a same graph,
generated by the arbitrary choices that exist in this algorithm.
An example of such a traversal is given on a small example
in \pref{ssec:ex-metazoan}.



\subsection{Addressing Sequential and Sub-state Reachability}
\label{ssec:ordered-ua}
\label{ssec:simult-ua}

\newcommand{\total}{\tau}
\newcommand{\reach}{\sigma}

In this section, we briefly discuss how the presented static analysis for local
state reachability properties can be extended to sequential (sub)states
reachability properties.

\subsubsection*{Sub-state reachability}
Given an AAN $\PH = (\PHs, \PHl, \PHh)$,
an initial state $s \in \PHl$ and
a sub-set of local states
$\total \subseteq \Proc$ % = \{ a_i \in\Proc\mid a\in\PHs, a_i\in\PHl_a\}$
so that $\forall a \in \PHs, \card{\total \cap \PHl_a} \leq 1$,
the reachability of a state containing all local states of $\total$
from the initial state $s\in\PHl$
can be tackled by our method.
Indeed,
consider the AAN
$\PH' = (\PHs', \PHl', \PHh')$ with:
$\PHs' = \PHs \cup \{ \reach \}$, $\PHl' = \PHl \times \PHl_\reach$,
where $\PHl_\reach = \{ \reach_0, \reach_1 \}$,
and $\PHh' = \PHh \cup \{ \PHhit{\total}{\reach_0}{\reach_1} \}$.
Obviously,
the reachability in $\PH$
of any state $s'\in\PHl$ such that
$\total \subset s'$ from the initial state $s$
is equivalent to the reachability in $\PH'$ of the local state
$\reach_1$ from state $s \times \{ \reach_0 \}$.

Such analysis was not possible with the Process Hitting framework,
because of the lack of the notion of simultaneity for more than two components.

\subsubsection*{Sequential reachability}
Given a sequence of local states to reach (\eg reach $a_i$, then $b_j$, etc.),
one can use the same approach as for sub-state reachability by introducing
a new automaton $\reach$ having $n+1$ local states, where $n$ is the size of the
sequence of reachability, and $n$ actions making it bounce gradually from $\reach_0$ to
$\reach_n$ with hitters corresponding the successive local states
(\eg $\PHfrappes{a_i}{\reach_0}{\reach_1}$,
$\PHfrappes{b_j}{\reach_1}{\reach_2}$, etc.).

An alternative approach is to use the extraction of a scenario of
\pref{ssec:concret}:
given an initial state $s_0$, and a first local state reachability property $\myp_1$,
one can compute a possible scenario $\delta \in \Delta(\mycwB{s_0}{\myp})$
witnessing this reachability;
then the next local state reachability properties ($\myp_2$, $\myp_3$, etc.)
are computed from the
state $s_0 \play \delta$ that outcomes from the latter scenario.


\section{Asynchronous Automata Networks with classes of priorities}
\label{sec:flattening}

In this section, we define the notion of AANs with classes of priorities,
and give a transformation from these into
AANs without priorities, as defined in \pref{sec:ph}.

The idea behind AANs with classes of priorities (\pref{def:php})
is to split the set of actions into several subsets assigned to priorities,
and to constrain the behaviour of the model to make any action unplayable
until no other action of higher priority is playable (\pref{def:playp}).
Such a framework allows to model preemptions between sets of actions,
which can be helpful to abstract time or duration properties under certain conditions.

\begin{definition}[AAN with $k$ classes of priorities]
\label{def:php}
  If $k \in \sN^*$,
  an \emph{Asynchronous Automata Network with $k$ classes of priorities} (AAN$k$)
  is a triplet $\PH = (\PHs; \PHl; \PHa^{\angles{k}})$,
  where $\PH^{\angles{k}} = (\PHh^{(1)} \dots; \PHh^{(k)})$,
  and:
  \begin{itemize}
    \item $\PHs \DEF \{a, b, \dots, z\}$ is the finite set of \emph{automata};
    \item $\PHl \DEF \underset{a \in \PHs}{\times} \PHl_a$ is the finite set of
      (global) \emph{states},
      where $\PHl_a = \{a_0, \ldots, a_{l_a}\}$ is the finite set of \emph{local states}
      of automaton $a \in \PHs$, with $l_a \in \sN^*$,
      and so that:
      $\forall (a_i; b_j) \in \PHl_a \times \PHl_b, a \neq b \Rightarrow a_i \neq b_j$;
    \item $\forall n \in \llbracket 1; k \rrbracket,
      \PHa^{(n)} \DEF \{\PHfrappe{A}{b_j}{b_k} \mid
      b \in \PHs \wedge (b_j; b_k) \in \PHl_b \times \PHl_b \wedge
      b_j \neq b_k \wedge
      \forall a \in \PHs, \card{A \cap \PHl_a} \leq 1 \wedge
      A \cap \PHl_b = \emptyset \}$ is the finite set of \emph{actions of priority $n$}.
  \end{itemize}
\end{definition}

\paragraph{Notations}
We use the same notations as those defined in \pref{sec:ph}, when applicable.
Furthermore,
we denote by $\PHh = \bigcup_{n \in \segm{1}{k}} \PHh^{(n)}$ the set of all actions
and, for all $h \in \PHh$,
by $\prio(h) = \min\{ n \in \segm{1}{k} \mid h \in \PHh^{(n)} \}$
the priority of action $h$.

\begin{definition}[Semantics of an AAN$k$ ($\PHPtrans$)]
\label{def:playp}
  An action $h = \PHhit{A}{b_j}{b_k} \in \PHa^{(n)}$ of priority $n$
  is \emph{playable} in $s \in \PHl$
  if and only if $A \subseteq s$, $\PHget{s}{b} = b_j$ and
  $\forall m < n, \forall g \in \PHa^{(m)},
    \neg (\PHhitter(g) \subseteq s \wedge \PHtarget(g) \in s)$.
  In such a case, $(s \PHplay h)$ stands for the state resulting from the play
  of the action $h$ in $s$, which is defined by: 
    $\PHget{(s \PHplay h)}{b} = b_k$
  and
    $\forall a \in \PHs, a \neq b, \PHget{(s\PHplay h)}{a} = \PHget{s}{a}$.
  Moreover, we denote: $s \PHPtrans (s \PHplay h)$.
\end{definition}

The translation given in the following relies on the notion of \emph{sub-state}
(\pref{def:substate}),
which is a set of local states containing at most one local state of each automata.
Thus, a sub-state can be considered as a partial state.

\begin{definition}[Sub-state ($\PHsublize{\PHl}$)]
\label{def:substate}
  If $S \subseteq \PHs$ is a set of automata, a sub-state on $S$ is an element of:
  $\PHsubl[\PHl]_S \DEF \{ \toset{\rho} \subseteq \Proc \mid
    \rho \in \bigtimes{a \in S} \PHl_a \}$
  (where the notation $\toset{\rho}$ represents
  the set of components of the Cartesian product $\rho$,
  as defined on page~\pageref{notations}).
  The set of all sub-states is denoted by:
  $\PHsubl[\PHl] \DEF \bigcup_{S \in \powerset(\PHs)} \PHsubl[\PHl]_S$.
  Furthermore, we recall the notation from \pref{sec:ph},
  if $\mysigma \in \PHsubl[\PHl]$ and $s \in \PHl$:
    \[\mysigma \subseteq s \Leftrightarrow \forall a_i \in \mysigma, \PHget{s}{a} = a_i\]
%    \[\mysigma \subseteq s \EQDEF \forall a_i \in \Proc, a_i \in \mysigma \Rightarrow a_i \in s\]
\end{definition}

We consider in the following an AAN$k$:
$\ov{\PH} = (\ov{\PHs}; \ov{\PHl}; \ov{\PHa}^{\angles{k}})$
with $k \in \sN$, $k > 1$.
The aim of the rest this section is to propose a translation of $\oPH$
into an AAN with $1$ class of priority $\PH = (\PHs; \PHl; \PHa^{\angles{1}})$
called \emph{flattening}, which is bisimilar.
As an AAN with 1 class of priority is equivalent to a regular AAN without priorities,
such a translation is particularly useful to be able to study the dynamics of
any kind of AAN with priorities by using the static analysis developed in \pref{sec:sa}.

The translation of an AAN$k$ into an AAN
is based on the notion of \emph{playability property} (\pref{def:pp})
which is a Boolean formula where the atoms are local states of $\ov{\PH}$.

\begin{definition}[Playability property language ($\F$)]
  \label{def:pp}
  A \emph{playability property} is an element of the language $\F$ inductively defined by:
  \begin{itemize}
    \item $\top$ and $\bot$ belong to $\F$;
    \item if $a \in \ov{\PHs}$ and $a_i \in \ov{\PHl}_a$, then $a_i \in \F$ and we call it an \emph{atom};
    \item if $P \in \F$ and $Q \in \F$, then $\neg P \in \F$, $P \wedge Q \in \F$ and $P \vee Q \in \F$.
  \end{itemize}
  If $P \in \F$ is a playability property and $\mysigma \in \PHsubl$ is a sub-state of $\oPH$,
  we note $\Fsem{P}{\mysigma}$ the \emph{evaluation} of $P$ in $\mysigma$
  in a three-valued Kleene logic (\emph{true}, \emph{undecided} or \emph{false})
  which is given by:
  \begin{itemize}
    \item if $P = a_i \in \ov{\PHl}_a$ is an atom,
%      with $a \in \ov{\PHs}$, 
      then $\Fsem{a_i}{\mysigma}$ is
    $\begin{cases}
       \text{true}      & \text{if } a_i \in \mysigma \\
       \text{undecided} & \text{if } \mysigma \cap \ov{\PHl}_a = \emptyset \\
       \text{false}     & \text{otherwise }
     \end{cases}$ \enspace;
%    true iff $a_i \in \mysigma$;
    \item if $P$ is not an atom, then $\Fsem{P}{\mysigma}$ is evaluated in $\mysigma$
      with the classical semantics for the logic operators $\top$, $\bot$, $\neg$, $\wedge$ and $\vee$, which is recalled in \pref{fig:kleene}.
  \end{itemize}
  Furthermore, in what follows, we will use $\Fsem{\Fop{h}}{s}$ as a shorthand for
  $\Fsem{\Fop{h}}{\toset{s}}$.
\end{definition}

\begin{figure}[ht]
  For all playability properties $P, Q \in \F$ that are not atoms,
  and for all sub-state $\mysigma \in \PHsubl$:
  \begin{itemize}
    \item $\Fsem{\top}{\mysigma}$ is always true;
    \item $\Fsem{\bot}{\mysigma}$ is always false;
    \item $\Fsem{\neg P}{\mysigma}$ is
      $\begin{cases}
        \text{true} & \text{if $\Fsem{P}{\mysigma}$ is false} \\
        \text{false} & \text{if $\Fsem{P}{\mysigma}$ is true} \\
        \text{undecided} & \text{if $\Fsem{P}{\mysigma}$ is undecided}
      \end{cases}$ \enspace;
    \item $\Fsem{P \wedge Q}{\mysigma}$ is
      $\begin{cases}
        \text{true} & \text{if both $\Fsem{P}{\mysigma}$ and $\Fsem{Q}{\mysigma}$ are true} \\
        \text{false} & \text{if $\Fsem{P}{\mysigma}$ is false or $\Fsem{Q}{\mysigma}$ is false} \\
        \text{undecided} & \text{otherwise}
      \end{cases}$ \enspace;
    \item $\Fsem{P \vee Q}{\mysigma}$ is
      $\begin{cases}
        \text{true} & \text{if $\Fsem{P}{\mysigma}$ is true or $\Fsem{Q}{\mysigma}$ is true} \\
        \text{false} & \text{if both $\Fsem{P}{\mysigma}$ and $\Fsem{Q}{\mysigma}$ are false} \\
        \text{undecided} & \text{otherwise}
      \end{cases}$ \enspace.
  \end{itemize}
  \caption{\label{fig:kleene}%
    Explicit semantics of the evaluation of playability properties in the
    three-valued Kleene logic.
  }
\end{figure}



Because we only use classical logic operators, the formulas of Boolean logic on
distributivity, associativity and commutativity can be used, together with De Morgan's laws on negation.
We also have the following property for the negation of an atom:
\[\forall a \in \ov{\PHs}, \forall a_i \in \ov{\PHl}_a,
  \neg a_i \Longleftrightarrow \bigvee_{\substack{a_j \in \ov{\PHl}_a\\a_j \neq a_i}} a_j\]
Indeed, if a local state is not active in a state, this means that another local state of the same automaton is active.
Moreover,
provided that in what follows we are only interested in properties that are true
and thus we make no distinction between false and undecided results,
playability properties can be simplified with the following result:
\[\forall a \in \ov{\PHs}, \forall a_i, a_j \in \ov{\PHl}_a,
  a_i \neq a_j \Longrightarrow a_i \wedge a_j \equiv \bot \]
because two different local states can never be active simultaneously.

In \pref{def:fop}, we define the the operator $\Fopsymbol$ which characterises the playability of an action
given the semantics of AAN$k$s (see \pref{def:playp}).
This operator simply states that the hitters of an action have to be active,
and every other action of higher priority must not be playable.

\begin{definition}[Playability property operator ($\Fopsymbol : \PHh \rightarrow \F$)]\label{def:fop}
  For all $h = \PHfrappe{A}{b_j}{b_m} \in \ov{\PHh}$, we define:
  \[
    \Fop{h} \equiv
    b_j \wedge
    \left(\bigwedge_{a_i \in A} a_i\right)
    %\hitter{h}
    \wedge
      \left( \bigwedge_{\substack{g \in \ov{\PHh}^{(n)}\\1 \leq n < \prio(h)}}
      \neg \left( \target{g} \wedge \left(\bigwedge_{c_l \in \hitter{g}} c_l\right)\right) \right)
  \]
\end{definition}
%
By construction of this operator and given the dynamics of a AAN$k$,
an action $h$ is playable in a state $s \in \ov{\PHl}$ if and only if
$\Fsem{\Fop{h}}{s}$ is true.

Because we only use classical logic operators, we can compute the Disjunctive Normal Form (DNF) of any playability property.
For any action $h \in \ov{\PHh}$, this DNF takes the form:
\[\Fop{h} \equiv \bigvee_{i \in \segm{1}{\n}} \left( \bigwedge_{j \in \segm{1}{\m}} p_{i,j} \right)\]
where $\n \in \sN$ and $\forall i \in \segm{1}{\n}, \m \in \sN^*$.
If $\n = 0$, then $\Fop{h} \equiv \bot$; this means that $h$ can never be played
due to preemptions by other actions with higher priorities.
If $\Fop{h} \not\equiv \bot$, on the other hand, then in this case $\Fop{h}$
can be seen as a disjunction of $\n$ smaller playability properties consisting only of conjunctions of atoms.
These $\n$ conjunctions can be translated to as many actions,
thus creating a new AAN$1$.
In this case, we denote, for any $i \in \segm{1}{\n}$:
$\PHdep{i}{h} = \{ \PHsort(p_{i,j}) \mid j \in \segm{1}{\m} \}$.

With \pref{lem:ppplaysubset}, we can then characterise the playability of an action in a state only with a sub-state.
This sub-state corresponds to one of the conjunctions of its playability property's DNF.
Finally, \pref{def:flattening} gives the construction of the flattening of $\oPH$:
for each action $h \in \ov{\PHh}$, several actions $f^{h,i}$ are built to reflect each of the conjunctions in $\Fop{h}$,
\ie for $i \in \segm{1}{\n}$.
This construction allows to obtain the same dynamics as $\oPH$, as stated by \pref{th:bisimPHP}.

\begin{lemma}
\label{lem:ppplaysubset}
  Let $h \in \ov{\PHh}$ and $s \in \ov{\PHl}$;
  $h$ is playable in $s$ if and only if:
  \[\target{h} \in s \wedge \exists \n \in \sN, \exists \mysigma \in \PHsubl_{\PHdep{i}{h}},
    \mysigma \subseteq s \wedge \Fsem{\Fop{h}}{\mysigma} \text{ is true} \enspace.\]
\end{lemma}
%
\begin{proof}
  ($\Rightarrow$)
    If $h$ is playable in $s$, then $\target{h} \in s$ and $\Fsem{\Fop{h}}{s}$ is true.
    Thus, $\Fop{h} \not\equiv \bot$ and, by property of a DNF,
    at least one of the $\n$ conjunctions of $\Fop{h}$ is true in $s$.
    Suppose the $i$\textsuperscript{th} conjunction is true in $s$, with $i \in \segm{1}{\n}$;
    then we have: $\forall j \in \segm{1}{\m}, p_{i,j} \in s$.
    Let $\mysigma \in \PHsubl_{\PHdep{i}{h}}$
    with $\forall b \in \PHdep{i}{h}, \PHget{\mysigma}{b} = \PHget{s}{b}$.
    We immediately have: $\mysigma \subseteq s$,
    and, by construction of $\PHdep{i}{h}$, $\Fsem{\Fop{h}}{\mysigma}$ is true.
  
  ($\Leftarrow$)
    $\Fsem{\Fop{h}}{\mysigma}$ is true, and therefore $\Fsem{\Fop{h}}{s}$ is true;
    as $\target{h} \in s$, $h$ is playable in $s$.
\end{proof}

\begin{definition}[Flattening ($\PHflat$)]
  \label{def:flattening}
  If $k \in \sN$, $k > 1$ and $\oPH = (\ov{\PHs}; \ov{\PHl}; \ov{\PHa}^{\angles{k}})$ is an AAN$k$,
  we denote by
  $\PHflat(\ov{\PHs}; \ov{\PHl}; \ov{\PHa}^{\angles{k}}) = (\PHs; \PHl; \PHa)$
  the \emph{flattening} of $\oPH$, where:
  \begin{itemize}
    \item $\PHs = \ov{\PHs}$;
    
    \item $\PHl = \ov{\PHl}$;
    
    \item $\PHh = \{
      \PHfrappe{(\mysigma \setminus \{ \target{h} \})}{\target{h}}{\bounce{h}} \mid
      h \in \ov{\PHh} \wedge \n \geq 1 \wedge i \in \segm{1}{\n} \wedge
      \mysigma \in \PHsubl_{\PHdep{i}{h}} \wedge
      \Fsem{\Fop{h}}{\mysigma} \text{ is true} \}$.
  \end{itemize}
\end{definition}

We note that the set of global states of an AAN$k$
and the set of global states of its flattening are the same.

\begin{theorem}[$(\ov{\PHs}; \ov{\PHl}; \ov{\PHa}^{\angles{k}}) \approx \PHflat(\ov{\PHs}; \ov{\PHl}; \ov{\PHa}^{\angles{k}})$]
\label{th:bisimPHP}
  If $\ov{\PH} = (\ov{\PHs}; \ov{\PHl}; \ov{\PHa}^{\angles{k}})$ is an AAN$k$
  and $\PH = \PHflat(\ov{\PHs}; \ov{\PHl}; \ov{\PHa}^{\angles{k}}) =
    (\PHs; \PHl; \PHa)$ is its flattening, then:
  \[ \forall s, s' \in \PHl, s \PHPtrans[\oPH] s' \Longleftrightarrow s \PHPtrans[\PH] s' \]
\end{theorem}

\begin{proof}
  By definition of $\PHflat$.
\end{proof}



We showed in this subsection that it is possible to model any AAN$k$
as an AAN (or, equivalently, as an AAN$1$).
This translation thus extends the applicability of the static analysis developed in
\pref{sec:sa} to any AAN$k$, with $k \in \sN^*$.
Moreover, it allows to represent
any Process Hitting model with classes of priorities~\cite{FPMR13-CS2Bio}
under the form of an AAN
(or, equivalently, of a Process Hitting model with $2$ classes of priorities).
The translation given in this section
is exponential in the number of actions of higher priority for each action.


% vim:set spell spelllang=en:

\section{Biological Examples}\label{sec:example}

This section aims at giving application examples of the static analysis method
that we developed in \pref{sec:sa}.
In \pref{ssec:ex-metazoan}, we apply our method to a small model of the
metazoan segmentation process,
and demonstrate how classes of priorities help in the modelling process,
and how the flattening of \pref{sec:flattening}
can be used in such a case.
In \pref{ssec:ex-tcrsig}, we apply our method to two large-scale models
in order to show the scalability of our method.



\subsection{Under-approximation of a Model with Priorities: Metazoan Segmentation}
\label{ssec:ex-metazoan}

We give here a detailed example of the use of classes of priorities
in order to model a system with temporal constraints.
This model also allows us to give a detailed example of the application of the sequential
under-approximation proposed in \pref{ssec:ordered-ua},
which consists of several applications of the method developed in
\pref{ssec:ua}.
For this, we first have to use the flattening method presented in \pref{sec:flattening}
because the considered model contains 2 classes of priorities.

Let us consider a model of metazoan segmentation
inspired from a first translation to Process Hitting model given in~\cite{PMR10-TCSB}.
This model was originally established in silico in~\cite{MSB:MSB4100192}
in a differential equations framework.
It is composed of a wavefront gene $f$ that activates the gap-gene $a$ whose products are responsible for stripes formation.
Gene $f$ also activates a gene $c$ whose products represses the gene $a$.
The auto-inhibition of $c$ generalises a chain of repressors on $a$.
The auto-inhibition of $f$, which normally terminates
the stripes formation in the original model,
has been removed in order to focus on the stationary dynamics of the model.

The actions of the original model are split into $2$ classes of priorities, as represented in \pref{fig:metazoan-php}:
\begin{align*}
  \ov{\PHh}^{(1)} = \{ \quad
    & \PHfrappes{c_1}{a_1}{a_0} \quad , \quad
    \PHfrappes{f_1, c_0}{a_0}{a_1}
  \quad \} \\
  \ov{\PHh}^{(2)} = \{ \quad
    & \PHfrappem{c_1}{c_0} \quad , \quad
    \PHfrappes{f_1}{c_0}{c_1}
  \quad \} \\
\end{align*}
Indeed, without this use of priorities,
some unwanted behaviours emerge, allowing the formation of irregular stripes.
In order to fix this, a high priority is affected to the actions hitting $a$
and a low priority to the actions hitting $c$,
in order to model the fact that the switch of $c$ has to be immediately followed by
a switch of $a$.
This forces the evolution of genes $a$ and $c$
to alternate in order not to miss a stripe;
$a$ and $c$ thus have intertwined oscillations.
We can thus consider that these two classes of priorities
are derived from known relative reaction rates,
the evolution of the clock $c$ being slower and regular,
while the evolution of $a$ has to follow the changes of $c$.

\begin{figure}[p]
  \centering
  \scalebox{1}{
  \begin{tikzpicture}[aan]
    \TSort{(0,4)}{c}{2}{l}
    \TSort{(1,0)}{f}{2}{l}
    \TSort{(5,4)}{a}{2}{r}
    
    \TAction{f_1}{c_0.west}{c_1.south west}{bend left=30, in=90}{left}
    \TActionPlur{}{c_1.west}{c_0.north west}{}{-1,5.5}{right}
    
    \TAction{c_1}{a_1.west}{a_0.north west}{prio}{right}
    \TActionPlur{f_1, c_0}{a_0.west}{a_1.south west}{prio}{2.5,2.5}{left}
    
    \TState{f_1, a_0, c_0}
  \end{tikzpicture}
  }
  \caption{
  \label{fig:metazoan-php}
    An example of AAN$2$
    modelling the process of metazoan segmentation.
    Component $a$ models the pigment production, and is influenced by
    component $c$ that has the role of a clock,
    while $f$ represents the wavefront propagation.
    If component $a$ oscillates (that is, its active local state changes regularly)
    then regular stripes are created on the metazoan.
    Actions of $\ov{\PHh}^{(2)}$ (low priority) are represented in thin lines
    and actions of $\ov{\PHh}^{(1)}$ (high priority) are in thick lines.
    The greyed local states represent a possible initial state:
    $\PHstate{f_1, a_0, c_0}$.
  }
\end{figure}

\pref{fig:metazoan-ph} gives the flattening of this model,
that is, an AAN with the equivalent dynamics
(but only one class of priority).
Its actions are:
\begin{align*}
  \PHh = \{ \quad
    & \PHfrappes{c_1}{a_1}{a_0} \quad , \quad
    \PHfrappes{f_1, c_0}{a_0}{a_1} \quad , \\
    & \PHfrappes{a_0}{c_1}{c_0} \quad , \quad
    \PHfrappes{f_1, a_1}{c_0}{c_1}
  \quad \}
\end{align*}
We note that the two actions in $\ov{\PHh}^{(2)}$ have been replaced by
the equivalent actions
$\PHfrappes{a_0}{c_1}{c_0}$ and $\PHfrappes{f_1, a_1}{c_0}{c_1}$,
in order to model the preemption of the actions in $\ov{\PHh}^{(1)}$.

\begin{figure}[p]
  \centering
  \scalebox{1}{
  \begin{tikzpicture}[aan]
    \TSort{(-3,4)}{c}{2}{l}
    \TSort{(0,1)}{f}{2}{l}
    \TSort{(3,4)}{a}{2}{r}
    
%    \TAction{f_1}{c_0.west}{c_1.south west}{bend left=30, in=90}{left}
    \TActionPlur{f_1, a_1}{c_0.south east}{c_1.south east}{}{-1.5,3.3}{right}
    \TAction{a_0.west}{c_1.east}{c_0.north east}{}{left}
    
    \TAction{c_1.north east}{a_1.north west}{a_0.north west}{}{right}
    \TActionPlur{f_1, c_0}{a_0.south west}{a_1.south west}{}{1.5,3.5}{left}
    
    \TState{f_1, a_0, c_0}
  \end{tikzpicture}
  }
  \caption{
  \label{fig:metazoan-ph}
    An example of AAN, which is the flattening of the AAN$2$ in \pref{fig:metazoan-php};
    in other words, this model has exactly the same dynamics as
    \pref{fig:metazoan-php}, but its actions make only one class of priority.
    The greyed local states represent the same initial state:
    $\PHstate{f_1, a_0, c_0}$.
  }
\end{figure}

\medskip

At this point, the static analysis results presented in
\pref{ssec:ua} can be used to check if the model is functional,
\ie if gene $a$ can oscillate, thus leading to the formation of stripes.
Starting from context $\ctx^1 = \PHstate{f_1, a_0, c_0}$,
we thus want to check the reachability of $\myp^1 = a_1$;
then, starting from any new state obtained,
we want to check the reachability of $\myp^2 = a_0$,
and finally $\myp^3 = a_1$ once again to ensure that we entered a cycle.
%This consists in using 
For this, we apply the method proposed in \pref{ssec:ordered-ua}:
we consider each reachability step independently
and we use \pref{th:approxinf} three times;
the initial states of steps 2 and 3 are computed with the
extraction method of \pref{ssec:concret}.

We first build the local causality graph $\thisB{\myp^1}{\ctx^1}$
related to the reachability of $\myp^1 = a_1$ from the initial state $\ctx^1$.
This graph is depicted in \pref{fig:metazoan-sa}(left).
\pref{th:approxinf} allows to conclude that this reachability is true.
One can thus extract a concretizing scenario from this local causality graph
with a traversal of the graph, as explained in \pref{ssec:concret}.
Such a traversal is detailed in \pref{tab:concret-metazoan};
the scenario extracted from this example consists of only one action:
$\PHfrappes{c_0, f_1}{a_0}{a_1}$.
Another traversal of the same graph
would consist in visiting node $f_1$ before node $c_0$
when first descending from node $\{ c_0, f_1 \}$;
however, the same scenario would be extracted.
As there exists no other traversal, we thus have in this case:
$\Delta(\thisB{\myp^1}{\ctx^1}) = \{ \PHfrappes{c_0, f_1}{a_0}{a_1} \}$.

We denote in the following:
$\ctx^2 = \PHstate{f_1, a_0, c_0} \play \PHfrappes{c_0, f_1}{a_0}{a_1} =
\PHstate{f_1, a_1, c_0}$
the state resulting from the play of the scenario concretizing
$\myp^1$ in $\ctx^1$.
As explained in \pref{ssec:ordered-ua},
one can use this resulting state in order to carry on with another
successive reachability,
such as the reachability of $\myp^2 = a_0$ in our case.
The local causality graph used to check the reachability of $\myp^2$
from $\ctx^2$ is given in \pref{fig:metazoan-sa}(middle).
The same reasoning allows to conclude that this reachability is true, and
to extract the following set containing only one concretizing scenario:
$\Delta(\thisB{\myp^2}{\ctx^2}) =
\{ \PHfrappes{a_1, f_1}{c_0}{c_1} \concat \PHfrappes{c_1}{a_1}{a_0} \}$.
We note that once again, two traversals are possible
(by visiting node $a_1$ before or after note $f_1$)
but both output the same scenario, and therefore end up in the same sate
$\ctx^3 = \PHstate{f_1, a_0, c_1}$.
Finally, the last local causality graph $\thisB{\myp^3}{\ctx^3}$,
depicted in \pref{fig:metazoan-sa}(right),
allows to conclude that this final reachability is true.

In conclusion, by following \pref{ssec:ordered-ua},
we showed that it is possible to reach successively
$a_1$, $a_0$ and $a_1$
and thus that the AAN of \pref{fig:metazoan-ph} is functional.
This result can be extended to the model of \pref{fig:metazoan-php}
because they have the same dynamics (given \pref{th:bisimPHP}).

\begin{figure}[tp]
  \centering
  \begin{tikzpicture}[aS]
    % STEP 1
    \node[Aproc] (a1) {$a_1$};
    \node[Aobj,below of=a1] (a01) {$\PHobj{a_0}{a_1}$};
    \node[Asol,below of=a01] (a01s) {};
    \node[Assol,below of=a01s] (a01ss) {$\{ c_0, f_1 \}$};

    \node[Aproc,below right of=a01ss] (f1) {$f_1$};
    \node[Aobj,below of=f1] (f11) {$\PHobj{f_1}{f_1}$};
    \node[Asol,below of=f11] (f11s) {};
    \node[Assol,below of=f11s] (nf11s) {$\emptyset$};

    \node[Aproc,below left of=a01ss] (c0) {$c_0$};
    \node[Aobj,below of=c0] (c00) {$\PHobj{c_0}{c_0}$};
    \node[Asol,below of=c00] (c00s) {};
    \node[Assol,below of=c00s] (nc00s) {$\emptyset$};

    \path
    (a1) edge (a01)
    (a01) edge (a01s)
    (a01s) edge (a01ss)
    (a01ss) edge (f1) edge (c0)

    (f1) edge (f11)
    (f11) edge (f11s)
    (f11s) edge (nf11s)

    (c0) edge (c00)
    (c00) edge (c00s)
    (c00s) edge (nc00s)
    ;
    
    % STEP 2
    \node[Aproc,right of=a1,node distance=3.5cm] (a0) {$a_0$};
    \node[Aobj,below of=a0] (a10) {$\PHobj{a_1}{a_0}$};
    \node[Asol,below of=a10] (a10s) {};
    \node[Assol,below of=a10s] (a10ss) {$\{ c_1 \}$};

    \node[Aproc,below of=a10ss] (c1) {$c_1$};
    \node[Aobj,below right of=c1] (c01) {$\PHobj{c_0}{c_1}$};
    \node[Asol,below of=c01] (c01s) {};
    \node[Assol,below of=c01s] (c01ss) {$\{ a_1, f_1 \}$};
    \node[Aobj,below left of=c1] (c11) {$\PHobj{c_1}{c_1}$};
    \node[Asol,below of=c11] (c11s) {};
    \node[Assol,below of=c11s] (nc11s) {$\emptyset$};

    \node[Aproc,below left of=c01ss] (a1) {$a_1$};
    \node[Aobj,below of=a1] (a11) {$\PHobj{a_1}{a_1}$};
    \node[Asol,below of=a11] (a11s) {};
    \node[Assol,below of=a11s] (na11s) {$\emptyset$};

    \node[Aproc,below right of=c01ss] (f1) {$f_1$};
    \node[Aobj,below of=f1] (f11) {$\PHobj{f_1}{f_1}$};
    \node[Asol,below of=f11] (f11s) {};
    \node[Assol,below of=f11s] (nf11s) {$\emptyset$};

    \path
    (a0) edge (a10)
    (a10) edge (a10s)
    (a10s) edge (a10ss)
    (a10ss) edge (c1)

    (c1) edge (c01) edge (c11)
    (c01) edge (c01s)
    (c01s) edge (c01ss)
    (c01ss) edge (f1) edge (a1)
    (c11) edge (c11s)
    (c11s) edge (nc11s)

    (f1) edge (f11)
    (f11) edge (f11s)
    (f11s) edge (nf11s)

    (a1) edge (a11)
    (a11) edge (a11s)
    (a11s) edge (na11s)
    ;
    
    % STEP 3
    \node[Aproc,right of=a0,node distance=4cm] (ta1) {$a_1$};
    \node[Aobj,below of=ta1] (ta01) {$\PHobj{a_0}{a_1}$};
    \node[Asol,below of=ta01] (ta01s) {};
    \node[Assol,below of=ta01s] (ta01ss) {$\{ c_0, f_1 \}$};

    \node[Aproc,below left of=ta01ss] (tf1) {$f_1$};
    \node[Aobj,below of=tf1] (tf11) {$\PHobj{f_1}{f_1}$};
    \node[Asol,below of=tf11] (tf11s) {};
    \node[Assol,below of=tf11s] (tnf11s) {$\emptyset$};

    \node[Aproc,below right of=ta01ss] (tc0) {$c_0$};
    \node[Aobj,below of=tc0] (tc10) {$\PHobj{c_1}{c_0}$};
    \node[Asol,below of=tc10] (tc10s) {};
    \node[Assol,below of=tc10s] (tnc10s) {$\{ a_0 \}$};
    \node[Aobj,right of=tc10] (tc00) {$\PHobj{c_0}{c_0}$};
    \node[Asol,below of=tc00] (tc00s) {};
    \node[Assol,below of=tc00s] (tnc00s) {$\emptyset$};

    \node[Aproc,below of=tnc10s] (ta0) {$a_0$};
    \node[Aobj,below of=ta0] (ta00) {$\PHobj{a_0}{a_0}$};
    \node[Asol,below of=ta00] (ta00s) {};
    \node[Assol,below of=ta00s] (tna00s) {$\emptyset$};

    \path
    (ta1) edge (ta01)
    (ta01) edge (ta01s)
    (ta01s) edge (ta01ss)
    (ta01ss) edge (tf1) edge (tc0)

    (tf1) edge (tf11)
    (tf11) edge (tf11s)
    (tf11s) edge (tnf11s)

    (tc0) edge (tc00)
    (tc00) edge (tc00s)
    (tc00s) edge (tnc00s)
    (tc0) edge (tc10)
    (tc10) edge (tc10s)

    (tc10s) edge (tnc10s)
    (tnc10s) edge (ta0)
    (ta0) edge (ta00)
    (ta00) edge (ta00s)
    (ta00s) edge (tna00s)
    ;
  \end{tikzpicture}
  \caption{%
  \label{fig:metazoan-sa}%
    The three successive saturated graphs of local causality
    of the AAN in \pref{fig:metazoan-ph}
    for the successive reachability of $a_0$, $a_1$ and $a_0$
    from the initial context
    $\ctx = \PHstate{f_1, a_0, c_0}$.
    The (left) graph allows to check the reachability of $a_1$
    from the initial context $\ctx$.
    The (middle) graph is for the reachability of $a_0$
    and the context $\PHstate{f_1, a_1, c_0}$.
    The (right) graph is for the last reachability, $a_1$,
    and the context $\PHstate{f_1, a_0, c_1}$.
  }
\end{figure}

\newcommand{\xproc}[1]{#1}
%\newcommand{\xsol}[1]{\circ^{{#1}}}
\newcommand{\xsol}{\circ}

\begin{table}[p]
  \centering
  \begin{tabular}{|c|c|c|l|l|l|}
    \hline
    \# & Dir. & Node & Marking & Output & State \\\hline\hline
    1  & $\searrow$ & $\xproc{a_1}$ &  &  & $\PHstate{f_1, a_0, c_0}$ \\\hline
    2  & $\searrow$ & $\PHobj{a_0}{a_1}$ &  &  & $\PHstate{f_1, a_0, c_0}$ \\\hline
    3  & $\searrow$ & $\xsol$ & $\PHfrappes{c_0, f_1}{a_0}{a_1}$ &  & $\PHstate{f_1, a_0, c_0}$ \\\hline
    4  & $\searrow$ & $\{ c_0, f_1 \}$ & $\{ c_0, f_1 \}$ &  & $\PHstate{f_1, a_0, c_0}$ \\\hline
    \phantom{*}%
    5* & $\searrow$ & $\xproc{c_0}$ &  &  & $\PHstate{f_1, a_0, c_0}$ \\\hline
    6  & $\searrow$ & $\PHobj{c_0}{c_0}$ &  &  & $\PHstate{f_1, a_0, c_0}$ \\\hline
    7  & $\searrow$ & $\xsol$ & $\emptyseq$ &  & $\PHstate{f_1, a_0, c_0}$ \\\hline
    8  & $\nearrow$ & $\PHobj{c_0}{c_0}$ &  &  & $\PHstate{f_1, a_0, c_0}$ \\\hline
    9  & $\nearrow$ & $\xproc{c_0}$ &  &  & $\PHstate{f_1, a_0, c_0}$ \\\hline
    10 & $\nearrow$ & $\{ c_0, f_1 \}$ & $\{ f_1 \}$ &  & $\PHstate{f_1, a_0, c_0}$ \\\hline
%    \phantom{*}%
    11 & $\searrow$ & $\xproc{f_1}$ &  &  & $\PHstate{f_1, a_0, c_0}$ \\\hline
    12 & $\searrow$ & $\PHobj{f_1}{f_1}$ &  &  & $\PHstate{f_1, a_0, c_0}$ \\\hline
    13 & $\searrow$ & $\xsol$ & $\emptyseq$ &  & $\PHstate{f_1, a_0, c_0}$ \\\hline
    14 & $\nearrow$ & $\PHobj{f_1}{f_1}$ &  &  & $\PHstate{f_1, a_0, c_0}$ \\\hline
    15 & $\nearrow$ & $\xproc{f_1}$ &  &  & $\PHstate{f_1, a_0, c_0}$ \\\hline
    16 & $\nearrow$ & $\{ c_0, f_1 \}$ & $\emptyset$ &  & $\PHstate{f_1, a_0, c_0}$ \\\hline
    17 & $\nearrow$ & $\xsol$ & $\emptyseq$ & $\PHfrappes{c_0, f_1}{a_0}{a_1}$ & $\PHstate{f_1, a_1, c_0}$ \\\hline
    18 & $\nearrow$ & $\PHobj{a_0}{a_1}$ &  &  & $\PHstate{f_1, a_1, c_0}$ \\\hline
    19 & $\nearrow$ & $\xproc{a_1}$ &  &  & $\PHstate{f_1, a_1, c_0}$ \\\hline
  \end{tabular}
  \caption{\label{tab:concret-metazoan}%
    Example of the extraction of a concretizing scenario
    from the local causality graph of \pref{fig:metazoan-sa}(left),
    by using the algorithm of \pref{ssec:concret}.
    The first column denotes the step number in the traversal,
    the second depicts the direction of traversal,
    either “$\searrow$” for descending or “$\nearrow$” for ascending,
    the third one is the name of the current node,
    the fourth one gives the marking of this node when it is left,
    the fifth one gives the actions output by the algorithm
    and the last column gives the current state of the model when leaving each node.
    In step \#5, marked with an asterisk,
    the traversal visits node $c_0$, but this was arbitrarily chosen
    amongst the set $\{ c_0, f_1 \}$.
    Another traversal thus consists in visiting node $f_1$ first,
    although in this example it does not change the result.
  }
\end{table}



\subsection{Large-scale Applications}
\label{ssec:ex-tcrsig}

In order to support the scalability and applicability of our under-approximation of reachability, we
apply our new approach to the analysis of two large-scale models:
a T-cell receptor (TCR) signalling pathway~\cite{tcrsig94}
and an epidermal growth factor receptor (EGFR) signalling pathway~\cite{Samaga2009}.
These models both gather about a hundred components and are detailed below.
They are originally specified as Boolean networks,
and have been automatically encoded into
a Process Hitting model with two classes of priorities,
whose dynamics is equivalent to an AAN%
\footnote{Files are available at
\url{http://maxime.folschette.name/underapprox-aan.zip}.}.
Boolean networks being a subclass of AANs, this translation is straightforward.
In the rest of this section, we first present the Pint implementation,
then the two models that were used for our tests,
and the finally summarise the quantitative results of these tests.

\subsubsection*{The Pint Implementation}

The under-approximation presented in \pref{sec:sa} has been implemented
in the existing Pint software%
\footnote{Pint gathers tools related to the Process Hitting
and is freely available at \url{http://loicpauleve.name/pint}.
The release “2015-02-11” was used for these experiments.}.
It comes in two versions:
\begin{itemize}
  \item One version consists in building the local causality graph once
    and checking \pref{th:approxinf} on this graph only.
    This version is polynomial in the size of the model,
    and therefore scalable by nature by its low complexity.
    However, it may be non-conclusive due to the presence of cycles
    or non-independent synchronizations that could be avoided.
  \item The other version consists in checking \pref{th:approxinf}
    on every sub-graph obtained by considering sub-sets of the nodes in $\Sol$
    of the local causality graph.
    This enumeration of “sub-solutions” may allow to find conclusive
    responses by removing cycles or unnecessary local states in the graph.
    However, due to its exhaustive nature,
    it is exponential in the number of solutions of each objective,
    and thus not as scalable.
\end{itemize}
In other words, when this implementation is unable to conclude with the
local causality graph alone, it tries to enumerate sub-solutions
in order to reach a conclusion.
Although this search is guided in order to remove in priority
the solutions that could create cycles or depended synchronizations,
in the worst case all combinations have to be checked.
Therefore, for the following tests, we capped the execution time
of each call to Pint to 3 seconds,
considering that the implementation is inconclusive above this threshold.

The under-approximation presented in this paper can therefore answer
either \emph{True} or \emph{Inconclusive} regarding the reachability
of a given local state.
We used it together with a previously defined
reachability over-approximation~\cite{PMR12-MSCS}
that allows to answer either \emph{False} or \emph{Inconclusive},
but which was not tailored specifically, but still valid, for AANs.

\subsubsection*{The T-cell Receptor Model}

The TCR signalling pathway consists of 94 components.
We checked the reachability for the independent activation of
the 4 outputs of the signalling cascade (SRE, AP1, NFkB, NFAT)
for all possible combinations of the 3 inputs (CD4, CD28, TCRlig).
All result in conclusive decisions,
and our under-approximation has been satisfied in 12 cases (over
32) proving the satisfiability of the concerned reachability properties in the encoded Boolean network
(and non-satisfiability in the other cases, using the previously existing over-approximation).

By analysing the detail of the results, one can find out that one of the outputs, NFkB, can never be activated.
In other words, FkB$_1$ is never reachable, whatever the initial state of the inputs,
while the other outputs (SRE, AP1, NFAT) can be independently activated
for some configurations of inputs.
We thus wanted to check if these three outputs could be activated together,
that is, if the activation of one of the outputs did not prevent the activation another one.
For this, as proposed in \pref{ssec:simult-ua},
we have added a new automaton $\reach$ into the model
with two local states ($\PHl_\reach = \{ \reach_0, \reach_1 \}$)
and the action:
\[ \PHhits{\text{SRE}_1, \text{AP1}_1, \text{NFAT}_1}{\reach_0}{\reach_1} \enspace. \]
Finally, checking the reachability of $\reach_1$ in all configurations of the inputs
has always been conclusive,
and the existence of 4 positive answers
(amongst all 8 possible initial configurations of the inputs) allows to conclude that
it is possible to reach a state where SRE, AP1 and NFAT are simultaneously active.

\subsubsection*{The Epidermal Growth Factor Receptor Model}

The EGFR signalling pathway gathers 104 components, amongst which one can distinguish
21 inputs and 12 outputs.
We selected 13 of all inputs
(erbb1, erbb2, erbb3, erbb4, bir, btc, egf, epr, nrg1a, nrg1b, nrg2b, nrg4, tgfa)
and observed the impact of their variations on all outputs
(elk1, creb, ap1, hsp27, actin\_reorg, cmyc, pro\_apoptotic, p70s6\_2, pkc, stat1, stat3, stat5).
We note that some of the experiments trigger the exponential search
of sub-solutions described above, and are cut after 3 seconds of computation
(thus leading to \emph{Inconclusive} cases).
These “interrupted” experiments represent about 10\% of all tests.
%amongst the remaining 90\%, while all the others are conclusive.
We note that amongst all tests that terminated “normally” (without being cut after 3 seconds)
%none of the tests terminated normally (without being cut after 3 seconds)
none of them responded with an inconclusive answer, which is however theoretically possible.

\subsubsection*{Synthesis of the Results}

The results of all tests performed with the implementation of our under-approximation
are summarised in \pref{tab:results} (columns “AAN”).
We held all these experiments on a personal computer,
and regarding the experiments that were not cut at the 3 seconds limit,
computations times were in the order of a few tenths of a second to about one second.
To give a comparison, we did the same experiments with a standard symbolic model-checker, LibDDD
\cite{libddd}, known for its good performances, the input model being the Boolean network expressed
as a Petri net.
However, due to the large scale of the model,
this program takes at least several minutes to terminate,
and runs out of memory for the majority of all experiments.
On the other hand,
our method is able to conclude with limited memory and computation time usage
in the majority of the cases,
and is expected to be scalable to models that are even larger,
even by orders of magnitude.

Finally, we note that similar experiments were conducted in~\cite{PMR12-MSCS}.
However, if these experiments allowed to obtain some information
on the TCR and EGFR models to some extent,
they did not provide a formal “True” response regarding the reachabilities that have been
checked in the examples above.
Indeed, the semantics of Process Hitting (without classes of priorities)
does not allow to model accurate Boolean gates,
thus leading to some unwanted spurious behaviours
that especially take the form of temporal shifts.
The adding of classes of priorities allows to remove these temporal shifts,
as explained in~\cite{FPMR13-CS2Bio},
with a construction that is equivalent to the AANs presented in \pref{sec:ph}.
Therefore, only the method presented in this paper provides a formal proof that
the observed behaviours are the result of the true dynamics of the systems.
Thus, additionally to the experiments previously mentioned,
we conducted similar experiments with the previous version
of the under-approximation, that are reported alongside in \pref{tab:results}
(in columns “PH”).
We note that indeed, about 15\% of the positive results regarding the EGFR model
could not be proven with our new under-approximation,
and are thus still formally uncertain.

\definecolor{lightgraycell}{gray}{0.85}
\newcommand{\grcl}{\cellcolor{lightgraycell}}

\begin{table}[htp]
  \centering
  \begin{tabular}{|c||c|c||c|c|}
  \hline
    & \multicolumn{2}{c||}{TCR} & \multicolumn{2}{c|}{EGFR} \\
  \hline
  \hline
    Inputs & \multicolumn{2}{c||}{3} & \multicolumn{2}{c|}{13} \\
  \hline
    Outputs & \multicolumn{2}{c||}{4 + $\reach$} & \multicolumn{2}{c|}{12} \\
  \hline
    Total tests & \multicolumn{2}{c||}{40} & \multicolumn{2}{c|}{98'304} \\
  \hline
  \hline
               & PH        & AAN       & PH               & AAN              \\
  \hline
  \hline
    True       & 16 (40\%) & \grcl 16 (40\%) & 74'268 (75,55\%) & \grcl 64'282 (65,39\%) \\
  \hline
    Inconclusive & 0 (0\%) & \grcl 0 (0\%)   & 0 (0\%)          & \grcl 9'986 (10,16\%)  \\
  \hline
    False      & \multicolumn{2}{c||}{24 (60\%)} & \multicolumn{2}{c|}{24'036 (24,45\%)} \\
  \hline
  \hline
    Max time   & 0.043s    & \grcl 0.20s     & 0.37s            & \grcl 0.87s            \\
  \hline
    Total time & <1s       & \grcl <1s       & 45min            & \grcl 9h50min          \\
  \hline
  \end{tabular}
  \caption{\label{tab:results}%
    Results of the tests on large-scale examples.
    The “AAN” column gives the related results on AAN models,
    using the under-approximation presented in this paper,
    while the “PH” column gives the results for PH models
    using cooperative sorts to model actions with multiple hitters,
    and using the under-approximation of~\cite{PMR12-MSCS}.
    The lines labelled “True”, “Inconclusive” and “False” give respectively
    the number of positive answers,
    experiments cut after 3 seconds and negative answers;
    while “Max time” and “Total time” depict respectively
    the maximum time of the individual computations (except those cut at 3 seconds)
    and the overall execution time of all tests (including those cut at 3 seconds).
    The greyed cells highlight the results that are proper to the
    under-approximation presented in this paper.
  }
\end{table}

In conclusion,
while ensuring a low complexity for the analysis of reachability in Boolean and discrete networks,
our under-approximation method turns out to be conclusive in numerous cases when applied to real
large-scale biological models, which were not tractable in most cases with exact model-checking.


% vi:spell spelllang=en:
\section{Discussion \& Conclusion}\label{sec:ccl}

In this paper, we focused on Asynchronous Automata Networks (AANs),
which are a restriction of classical Automata Networks
and are equivalent to Process Hitting models with classes of priorities~\cite{FPMR13-CS2Bio}.
This formalism proves useful to model accurate Boolean gates, which was not possible
with the standard form of the Process Hitting,
and avoid an unwanted over-approximation of the dynamics.
Furthermore, we proposed an extension of this formalism with classes of priorities,
which prove convenient to abstract time parameters into these kinds of models
or more simply to add preemption relations between actions,
and showed that any AAN with priorities can be translated into
an equivalent AAN without priorities.

Then, we developed a method to perform a reachability analysis
of a local state in an AAN,
based on an under-approximation of the true reachability solutions.
We also extended this analysis to global and
partial states, and to the successive reachability of several local states.
The method can be considered efficient,
because it is polynomial in the size of the model.
A more conclusive analysis also exists,
but at the price of being exponential in the number of local solutions.
Finally, AANs with classes of priorities can also be studied in this way,
at the cost of an exponential translation that we gave in this paper.
We applied it to a large number of experiments on two large-scale
biological models, and obtained in the worst case a ratio of
%65\% of positively responding cases, growing to
90\% of conclusive cases
with the joint use of a previously proposed over-approximation,
although limiting the computation time to 3 seconds for each test.

AANs are also equivalent to Logical Networks, that is,
either multivalued or Boolean networks
with evolution functions or focal parameters,
such as Thomas' models with Snoussi parameters.
This especially allows to efficiently compute reachability results
on large biological models,
provided that they are equivalent to Logical Networks,
which ensures that a translation to AANs is possible.
For example,
such a translation for generalized Interaction Graphs,
that is, Discrete Networks without evolution functions or parameters,
was proposed in~\cite{PMR10-TCSB}.

\modmf{%
The static analysis method proposed in this paper can theoretically be applied to other
kinds of formalisms that are not mentioned in the rest of this paper.
For example, AANs are expressive enough to represent Asynchronous Finite Cellular Automata,
that is, randomly updated and thus completely asynchronous Finite Cellular Automata
\cite{cornforth_artificial_2003,harvey_time_1997}.
Another possibility is to represent turn-based multiplayer games,
such as arena \cite{gradel_infinite_2002}
where a specific player can move in each state,
or games in which players move alternately,
with the use of classes of priorities and special automata to model these rules.
% in AANs with 3 classes of priorities, with the addition of several automata
% to tackle the rules of the game, that is, the sequentiality of the turns of each player.
In each case, the considered formalism can be represented as an AAN,
making the application of our analysis method possible in theory.
Furthermore, the addition of new automata does not significantly impact its overall complexity.
However, more study and experiments are required to evaluate
the effectiveness of our method on these kinds of models,
especially regarding its conclusiveness, given their particular form.
% However, in practice, the conclusiveness on such “unusual” models
% can be potentially lowered, as the static analysis is not specifically tailored for them.
% Indeed, some particular structures may create loops or non-independent nodes
% in the Local Causality Graph,
% making the whole method inconclusive.
% Therefore, more study and experiments are required to evaluate
% the effectiveness of our method on these kinds of models.
}

Further work can also be directly derived to improve the method presented in this paper.
The over-approximation on Process Hitting models without priorities proposed in~\cite{PMR12-MSCS}
and that was used in this work
is still accurate on AANs (by over-approximating dynamics)
but may be refined given the particular for of AANs proposed in this paper.
A specific search of key local states or cut sets \cite{PAK13-CAV} may especially be derived.
%but turns out to be too wide even in some obvious cases that are consequently not conclusive.
%This approximation may be refined in order to better fit the introduction of priorities, and mak the overall approximation approach 
%and mode precisely the class of models studied in this paper.
Furthermore, we are investigating alternative under-approximations that can be
applied directly to the whole class of AANs or Process Hitting models with priorities,
and not only to a sub-class with particular restrictions;
such improvement may permit to increase the conclusiveness of the static analysis
while allowing to analyse any model without the need of a translation.
Finally, in order to take into account quantitative data in transition delays, the overall approximation method could be extended to handle evolutions that are chronometric instead of only chronologic.
This may require the addition of information such as time delays in the model,
that would be exploited during the solving.


{\small
\paragraph{Acknowledgement}
The European Research Council has provided financial support
under the European Community's Seventh Framework Programme (FP7/2007--2013)~/
ERC grant agreement no.~259267.
}

\section*{References}
\bibliographystyle{elsarticle-num}
\bibliography{biblio}

\appendix

% Fix appendix references
\renewcommand*{\thesection}{\Alph{section}}

% Proof of under-approximation
\section{Proof of Under-approximation (\pref{ssec:ua})}
\label{suppl:demoapproxinf}

In the following, we denote:
$\Bee{X}{Y} = \Be \cap (X \times Y)$, with $X, Y$
amongst $\PHproc$, $\Obj$, $\sSol$ and $\Sol$.

\newproof{proofapproxinf}{Proof of \pref{th:approxinf}}
\begin{proofapproxinf}
Given the LCG $\cwB=(\cwV,\cwE)$,
we note $max\ctx = \{a\mapsto \cwV\cap\PHl_a \mid a\in\PHs\}$ the context supported by $\cwB$.

Let $ps \in \Bv \cap \sSol$ be a set of hitters
and suppose all of its successors are concretizable.
We first want to demonstrate that
there exists a scenario that activates all the local states it contains.
We label all local states of $ps$ by an integer: $ps = \{ p_m \}_{m \in \indexes{ps}}$.
Let us prove by induction that for all $n \in \{ 0 \} \cup \indexes{ps}$,
there exists a scenario $\delta_n$ so that:
$\forall i \in \segm{1}{n}, \PHget{(s \PHplay \delta_n)}{\PHsort(p_i)} = p_i$.
\begin{itemize}
  \item It is straightforward for $\delta_0 = \varepsilon$.
  \item Suppose such $\delta_n$ exists and let $q = \PHget{(s \PHplay \delta_n)}{\PHsort(p_{n+1})}$.
    By construction, $p_{n+1} \in \Bv \cap \Proc$ is a child of $ps$.
    Furthermore, by hypothesis, $ps$ is independent (see \pref{def:coherent}).
    This means that amongst all the successors in $\Proc$ of $p_{n+1}$,
    there does not exist a local state $b_j$ to that
    $\exists b_k \in ps, \PHsort(b_j) = \PHsort(b_k) \wedge b_j \neq b_k$;
    in other words, the resolution of $p_{n+1}$ does not require a local state
    that may change the other local states of the set $ps$.
    Therefore, there exists $\delta' \in \mconcr_{s \PHplay \delta_n}(\PHobj{q}{p_{n+1}})$,
    so that $\forall i \in \segm{1}{n+1}$, $\PHget{(s \PHplay \delta_n \PHplay \delta')}{\PHsort(p_{i})} = p_{i}$.
    We denote then: $\delta_{n+1} = \delta_n \play \delta'$.
\end{itemize}
Therefore, $\delta = \delta_{\card{ps}}$ exists, and given its properties, we have:
$\forall i \in \segm{1}{\card{ps}}, \PHget{(s \PHplay \delta)}{\PHsort(p_{i})} = p_{i}$.

As there is no cycle in $\cwB$, we show by induction in the following that
$\forall s\in \PHl, s\subseteq max\ctx, \forall P \in \Bv \cap \Obj,
\PHtarget(P) \in s \Longrightarrow \exists \delta \in \mconcr_s(P)$.
\begin{itemize}
  \item If $(P, \langle P, \{ \emptyset \} \rangle) \in \Bee{\Obj}{\Sol}$,
    either $\PHtarget(P) = \PHbounce(P)$ and $\delta = \emptyseq$;
    or $\exists \zeta \in \BS(P), \zeta \in \Sce(s) \wedge
      \forall i \in \indexes{\zeta}, \hitter{\zeta_i} = \emptyset$
    and in this case, $\delta = \zeta$ is a valid scenario in $s$.

  \item Suppose all successors objectives of $P$ are concretizable.
    If $\exists (P, Q) \in \Bee{\Obj}{\Obj}$, then by hypothesis,
      $\mconcr_{s}(\obj{\PHtarget(P)}{\PHtarget(Q)} \concat Q) \neq \emptyset$, thus
      $\mconcr_{s}(P) \neq \emptyset$.
    Else, by \pref{def:lcg}-\ref{lcg-conn}, the concretizations of the successors of $P$ require no local state of automaton $\PHsort(P)$.
      Furthermore, there exists $\zeta \in \BS(P)$ so that $(P, \aZ) \in \Bee{\Obj}{\Sol}$.
      We show by induction that for all $n \in \indexes{\zeta}$, there is a scenario $\delta_n$ so that $\PHget{(s \PHplay \delta_n)}{\PHsort(P)} = \PHbounce(\zeta_n)$.
      \begin{itemize}
%        \item[$\circ$] If $\zeta = \emptyseq$, then trivially, $\delta = \emptyseq$.
        \item[$\circ$] Suppose that $\delta_n$ exists and let $\zeta_n = \PHhit{A}{a_j}{a_k}$.
        By construction, there exists $A \in \Bv \cap \sSol$
        amongst the children of $\aZ$.
        By the first result of this demonstration,
        there exists a scenario $\delta'$ in $s \play \delta_n$ so that
        $\forall a_i \in A, \PHget{(s \play \delta_n \PHplay \delta')}{a} = a_i$.
        Therefore, $\zeta_n$ is playable in $s \play \delta_n \PHplay \delta'$,
        and $\delta_{n+1} = \delta_n \concat \delta' \concat \zeta_n$.
      \end{itemize}
      Thus, $\delta_{|\zeta|} \in \mconcr_s(P)$. % and $\ceil(\delta) \subseteq max\ctx$.
\end{itemize}
\end{proofapproxinf}


% % Proof of ADN equivalence
% % vim:set spell spelllang=en:

\section{Weak Bisimulation of Asynchronous Discrete Networks}

\def\DNtrans{\rightarrow_{ADN}}
\def\DNdef{(\mathbb F, \langle f^1, \dots, f^n\rangle)}
\def\DNdep{\f{dep}}
\def\PHPtrans{\rightarrow_{PHP}}
\def\get#1#2{#1[{#2}]}
\def\encodeF#1{\mathbf{#1}}
\def\toPH{\encodeF{PH}}
\def\card#1{|#1|}
\def\decode#1{\llbracket#1\rrbracket}
\def\encode#1{\llparenthesis#1\rrparenthesis}
\def\Hits{\PHa}
\def\hit{\PHhit}
\def\play{\cdot}

\todo{synchronise notations and terminology with the main text}
\todo{define $\PHPtrans$? or defined in main text?}

We exhibit an encoding of Asynchronous Discrete Networks (ADN) with the Process
Hitting using two priority classes, and prove a weak bisimulation relation.

A Discrete Network gathers a finite number of components $i\in[1;n]$ having a discrete finite domain
$\mathbb F^i$ that we note $\mathbb{F}^i = [0;l_i]$.
For each component $i\in[1;n]$, a map $\mathbb F \mapsto F^i$ is defined, where
$\mathbb F = \mathbb F^1 \times \cdots \times \mathbb F^n$, giving the next value of the component
with respect to the global state of the network.
Typically $f^i$ depends on a subset of components that we denote $\DNdep(f^i)$.
In the case of Asynchronous Discrete Networks (ADN), a transition relation $\DNtrans\subset \mathbb
F\times \mathbb F$ is defined such that $x\DNtrans x'$ if and only if there exists a unique
$i\in[1;n]$ such that $\get{x'}{i}=f^i(x)$ and $\forall j\in[1;n], j\neq i, \get{x'}{j}
=\get{x}{j}$, i.e. one and only component has been updated.
This is formalised in \pref{def:DN}.

\begin{definition}[Asynchronous Discrete Network (ADN)]
\label{def:DN}
An ADN is defined by a couple $(\mathbb F, \langle f^1, \dots, f^n \rangle)$
where $\mathbb{F} = \mathbb{F}^1\times\dots\times\mathbb{F}^n$,
and $\forall i\in[1;n]$,
$f^i: \mathbb{F} \mapsto \mathbb{F}^i$ with
$\mathbb{F}^i = [0;l_i]$.
Given two states $x,x'\in\mathbb F$, the transition relation $\DNtrans$ is given by
\[
x\DNtrans x' \Longleftrightarrow
  \exists i\in[1;n], f^i(x)=\get{x'}{i}
  \wedge \forall j\in[1;n], j\neq i, \get{x}{j}=\get{x'}{j}
\enspace,
\]
where $\get{x}{i}$ is the $i$-th component of $x$.
%
We note $\DNdep(f^i)\subseteq \{1,\dots,n\}$ the set of components on which the value of $f^1$
depends: $\forall x,x'\in \mathbb F$ such that $\forall
j\in\DNdep(f^i), \get{x}{j}=\get{x'}{j}$, necessarily $f^i(x)=f^i(x')$.
\end{definition}

Let us denote by $\toPH\DNdef$ the encoding of the ADN $\DNdef$ in Process Hitting with $2$ priority
classes (\pref{def:DN2PH}).
For each component $i\in[1;n]$ of the ADN, two sorts are built: $a^i$ acting for the component
value, and $f^i$ acting for a cooperative sort between the components $\DNdep(f^i)$.
Sorts $a^i$ have one process $a^i_k$ per element in $k\in\mathbb F^i$.
Sorts $f^i$ have one process $f^i_\varsigma$ per state $\varsigma \in \prod_{j\in\DNdep(f^i)}
L_{a^j}$.
Two classes of actions are then defined:
$\Hits^1$ is the set of actions updating the cooperative sorts according to the actual state of the
components:
if $j\in\DNdep(f^i)$, $a^j_k$ hits each process $f^i_\varsigma$ where $\get{\varsigma}{a^j}\neq
a^j_k$ to make it bounce to the process $f^i_{\varsigma'}$ where $\get{\varsigma'}{a^j}=a^j_k$.
$\Hits^2$ is the set of actions encoding the transitions in the ADN:
$f^i_\varsigma$ hits the processes of sort $a^i$ to make them bounce to the process
$a^i_{k'}$ if and only if $k'=f^i(\decode \varsigma)$;
$\decode \varsigma$ being the ADN state correspond to the PH (partial) state $\varsigma$ (note that
$f^i(\decode \varsigma)$ is fully defined because $\decode \varsigma$ specifies the state for all
the components in $\DNdep(f^i)$).

\begin{definition}
\label{def:DN2PH}
$\toPH\DNdef=(\Sigma,L,\Hits^1,\Hits^2)$ is the Process Hitting with 2 priority classes encoding the
ADN $\DNdef$, with:
\begin{itemize}
\item $\Sigma = \{ a^1, \dots, a^n \} \cup \{ f^1, \dots, f^n \}$,
the sorts for components ($a^i$) and cooperative sorts ($f^i$).

\item $L=\prod_{i\in[1;n]} L_{a^i} \times \prod_{i\in[1;n]} L_{f^i}$, where
$L_{a^i}=\{a^i_0, \dots, a^i_{l_i}\}$, and
$L_{f^i}=\{f^i_\varsigma \mid \varsigma\in\prod_{j\in\DNdep(f^i)} L_{a^i} \}$
if $\DNdep(f^i)\neq\emptyset$, otherwise
$L_{f^i}=\{f^i_\emptyset\}$;
  
\item $\Hits^1= \{ \hit{a^j_k}{f^i_\varsigma}{f^i_{\varsigma'}}
 \mid i\in[1;n] \wedge
j\in\DNdep(f^i) \wedge a^j_k\in L_{a^j} \wedge f^i_\varsigma\in L_{f^i}
\wedge
\get{\varsigma}{a^j}\neq a^j_k 
\wedge \get{\varsigma'}{a^j}=a^j_k
\wedge (\get{\varsigma'}{a^{l}}=\get{\varsigma}{a^l},
\forall l\in[1;n],l\neq j)
\}$, 
the set of actions with priority $1$ for updating cooperative sorts.

\item $\Hits^2=\{ \hit{f^i_\varsigma}{a^i_k}{a^i_{k'}} \mid i\in[1;n] \wedge
      f^i_\varsigma\in L_{f^i} \wedge
      a^i_k\in L_{a^i}\wedge k\neq k' \wedge f^i(\decode \varsigma) = k'
    \}$,
the set of actions with priority $2$ for updating the components using their respective discrete
maps.
$\decode \varsigma$ is defined below.
\end{itemize}
Given a state $s\in L$ of the Process Hitting, 
$\decode s=x$ is the corresponding state in the ADN:
$\forall i\in[1;n], \get{s}{a^i}=a^i_k \Rightarrow \get{x}{i}=k$.

\noindent
Given a state $x\in \mathbb F$ of the ADN, 
$\encode x=s$ is the corresponding state in the Process Hitting:
$\forall i\in[1;n], \get{x}{i}=k \Rightarrow \get{s}{a^i}=a^i_k$
and
$\forall i\in[1;n], \get{s}{f^i}=f^i_\varsigma$ with $f^i_\varsigma\in L_{f^i}$
and $\forall j\in\DNdep(f^i), \get{\varsigma}{j}=\get{s}{a^j}$.
\end{definition}

\pref{thm:bisimDN} states the weak bisimulation relation between an ADN and its encoding in
Process Hitting with $2$ priority classes.
Intuitively, actions updating cooperative sorts being of higher priority, actions updating component
sorts follow strictly the possible transitions of the ADN.

\begin{theorem}[$\DNdef \approx \toPH\DNdef$]~
\label{thm:bisimDN}
\begin{enumerate}
\item $\forall x,x'\in\mathbb F$,
$x\DNtrans x' \Longrightarrow \encode x \PHPtrans^* \encode{x'}$,
where $\PHPtrans^*$ is a finite sequence of $\PHPtrans$ transitions.

\item $\forall s,s'\in L$,
$s\PHPtrans s' \Longrightarrow \decode s = \decode {s'} \vee \decode s \DNtrans
\decode{s'}\enspace.$
\end{enumerate}
\end{theorem}
\begin{proof}
(1) From \pref{def:DN} $x\DNtrans x'\Rightarrow \exists i\in[1;n],
f^i(x)=\get{x'}{i} \wedge \forall j\in[1;n],i\neq j, \get{x}{j}=\get{x'}{j}$.
Let us assume (without loss of generality) that $f^i(x)=k'$, $\get{x}{i}=k$ and
$\varsigma\in\prod_{j\in\DNdep(f^i)} L_{a^j}$ such that
$\forall j\in\DNdep(f^i), \get{\varsigma}{j}=a^j_{\get{x}{j}}$.
From \pref{def:DN2PH}, $h=\hit{f^i_\varsigma}{a^i_k}{a^i_{k'}}\in\Hits^2$.
From $\encode x$ definition,
$a^i_k\in \encode x$ and $f^i_\varsigma\in \encode x$;
moreover, as there is no action in $\Hits^1$ applicable in $\encode x$,
$h$ is applicable in $\encode x$:
$\encode x\PHPtrans \encode x\play h$.
In $\encode x\play h$, the only applicable actions of priority $1$ are those having
$a^i_{k'}$ as hitter and hitting cooperative sorts, giving a finite number of transitions towards
$\encode{x'}$.

(2) $s\PHPtrans s'$ only if there exists an action $h$ applicable in $s$ such that
$s\play h=s'$.
If $\prio(h)=1$, then, by $\Hits^1$ definition, 
$\decode s=\decode {s'}$.
If $\prio(h)=2$, then $\forall i\in[1;n]$,
if $\get{s}{f^i} = f^i_\varsigma$, then, $\forall j\in\DNdep(f^i),
\get{\varsigma}{a^j}=\get{s}{a^j}$.
Let us define $i\in[1;n]$ such that $\get{s}{a^i}\neq\get{s'}{a^i}$ ($i$ is unique for this
transition).
By \pref{def:DN2PH}, if $\get{s'}{a^i}=a^i_{k'}$, necessarily $f^i(\decode s)=k'$, hence
$\decode s\DNtrans \decode{s'}$.
\end{proof}



% 
% % Proofs for flattening
% \section{Proof of the Weak Bisimulation of the Flattening (\pref{ssec:flattening})}
\label{suppl:demoflattening}

\newproof{proofbisimPHP}{Proof of \pref{th:bisimPHP}}
\begin{proofbisimPHP}
  (\ref{php2ph}) Let $\flats{\os} = s$.
    Given the dynamics of a PH (\pref{def:play}), if $\os \PHPtrans[\PH] \os'$,
    then there exists $h \in \ov{\PHh}$ playable in $\os$;
    therefore, $\os' = \os \PHplay h$.
    Thus, $\Fsem{\Fop{h}}{\os}$ and $\target{h} \in \os$,
    and given \pref{lem:ppplaysubset}, there exists $\mysigma \subseteq s$ so that $\Fsem{\Fop{h}}{\mysigma}$.
    Therefore, by construction of $\PH$ (\pref{def:flattening}) there exists
    $g = \PHhit{f^{h,i}_\mysigma}{\target{h}}{\bounce{h}} \in \PHh^{(2)}$.
    By construction of $s$ (\pref{def:flattening}), $\PHget{s}{f^{h,i}} = f^{h,i}_\mysigma$ and $\target{h} = \target{g} \in s$.
    Therefore, $g$ is playable in $s$.
    In $s \PHplay g$, the only playable actions are those in $\PHh^{(1)}$ having $\bounce{h} = \bounce{g}$ as hitter
    and updating cooperative sorts, allowing to reach $\flats{\os'}$ in a finite number of actions.
    Thus, $\flats{\os} \PHPtrans[\oPH]^* \flats{\os'}$.
  
  (\ref{ph2php}) Let $\os = \unflats{s}$.
    Given the dynamics of a PH (\pref{def:play}), if $s \PHPtrans s'$,
    then there exists $g \in \PHh$ playable in $s$; therefore, $s' = s \PHplay g$ and $\target{g} \in s$.
    If $\prio(g) = 1$ then only the active process of a cooperative sort has evolved, and $\unflats{s} = \unflats{s'}$.
    Otherwise, $\prio(g) = 2$; we note in this case: $g = \PHhit{f^{h,i}_\mysigma}{b_j}{b_k}$.
    By construction of the flattening (\pref{def:flattening}), there exists $h \in \ov{\PHh}$ so that
    $\target{h} = b_j$, $\bounce{h} = b_k$ and $\Fsem{\Fop{h}}{\mysigma}$.
    If $g$ is playable, this means that no other action in $\PHh^{(1)}$ is playable, and especially the cooperative sort
    $f^{h,i}$ is already updated; therefore, $\mysigma \subseteq \os$.
    Furthermore, $b_j \in \os$.
    Thus, by \pref{lem:ppplaysubset}, it comes that $h$ is playable in $\os$,
    and $\unflats{s} \PHPtrans \unflats{s'}$.
\end{proofbisimPHP}



\end{document}
