% vi:spell spelllang=en:
\section{Discussion \& Conclusion}\label{sec:ccl}

In this paper, we focused on Asynchronous Automata Networks (AANs),
which are a restriction of classical Automata Networks
and are equivalent to Process Hitting models with classes of priorities~\cite{FPMR13-CS2Bio}.
This formalism proves useful to model accurate Boolean gates, which was not possible
with the standard form of the Process Hitting,
and avoid an unwanted over-approximation of the dynamics.
Furthermore, we proposed an extension of this formalism with classes of priorities,
which prove convenient to abstract time parameters into these kinds of models
or more simply to add preemption relations between actions,
and showed that any AAN with priorities can be translated into
an equivalent AAN without priorities.

Then, we developed a method to perform a reachability analysis
of a local state in an AAN,
based on an under-approximation of the true reachability solutions.
We also extended this analysis to global and
partial states, and to the successive reachability of several local states.
The method can be considered efficient,
because it is polynomial in the size of the model.
A more conclusive analysis also exists,
but at the price of being exponential in the number of local solutions.
Finally, AANs with classes of priorities can also be studied in this way,
at the cost of an exponential translation that we gave in this paper.
We applied it to a large number of experiments on two large-scale
biological models, and obtained in the worst case a ratio of
%65\% of positively responding cases, growing to
90\% of conclusive cases
with the joint use of a previously proposed over-approximation,
although limiting the computation time to 3 seconds for each test.

AANs are also equivalent to Logical Networks, that is,
either multivalued or Boolean networks
with evolution functions or focal parameters,
such as Thomas' models with Snoussi parameters.
This especially allows to efficiently compute reachability results
on large biological models,
provided that they are equivalent to Logical Networks,
which ensures that a translation to AANs is possible.
For example,
such a translation for generalized Interaction Graphs,
that is, Discrete Networks without evolution functions or parameters,
was proposed in~\cite{PMR10-TCSB}.

\modmf{%
The static analysis method proposed in this paper can theoretically be applied to other
kinds of formalisms that are not mentioned in the rest of this paper.
For example, AANs are expressive enough to represent Asynchronous Finite Cellular Automata,
that is, randomly updated and thus completely asynchronous Finite Cellular Automata.
Another possibility is to represent multiplayer games, which is possible
in AANs with 3 classes of priorities, with the addition of several automata
to tackle the rules of the game, and especially the sequentiality of the turns of each player.
In both cases, the considered formalism can be represented as an AAN,
making the application of our analysis method possible in theory.
However, in practice, the conclusiveness on such “unusual” models
can be potentially lowered, as the static analysis is not specifically tailored for them.
Indeed, some particular structures may create loops or non-independent nodes
in the Local Causality Graph,
making the whole method inconclusive.
However, more study and experiments are required to evaluate
the effectiveness of our method on these kinds of models.
}

Further work can be directly derived to improve the method presented in this paper.
The over-approximation on Process Hitting models without priorities proposed in~\cite{PMR12-MSCS}
and that was used in this work
is still accurate on AANs (by over-approximating dynamics)
but may be refined given the particular for of AANs proposed in this paper.
A specific search of key processes or cut sets \cite{PAK13-CAV} may especially be derived.
%but turns out to be too wide even in some obvious cases that are consequently not conclusive.
%This approximation may be refined in order to better fit the introduction of priorities, and mak the overall approximation approach 
%and mode precisely the class of models studied in this paper.
Furthermore, we are investigating alternative under-approximations that can be
applied directly to the whole class of AANs or Process Hitting models with priorities,
and not only to a sub-class with particular restrictions;
such improvement may permit to increase the conclusiveness of the static analysis
while allowing to analyse any model without the need of a translation.
Finally, in order to take into account quantitative data in transition delays, the overall approximation method could be extended to handle evolutions that are chronometric instead of only chronologic.
This may require the addition of information such as time delays in the model,
that would be exploited during the solving.
