% vi:spell spelllang=en:
\section{Under-approximation of Reachability}\label{sec:sa}

We present a static analysis that takes as input an AAN
$\PH = (\PHs; \PHl; \PHa)$,
an initial state in $\PHl$, and a local state in $\Proc$.
Its objective it to identify sufficient conditions
that ensure the existence of a scenario starting from the initial state,
and leading to a state where the given local state is present.

A classical approach for determining if a local state of an automaton can be
reached from a given initial state is to build sequences of transitions from the initial state
until reaching a state in which the given local state is active.
%containing the given local state is reached.
This is essentially what model-checkers do, and the complexity of such analysis (PSPACE-complete
\cite{Harel02}) makes it intractable on large systems, even with advanced symbolic approaches
\cite{PMR12-MSCS}.

Our approach relies on abstractions of scenarios (\pref{ssec:abstr-sce}) that have been introduced in
\cite{PMR12-MSCS} for the static analysis of reachability in the Process Hitting framework,
a particular sub-class of AAN (see the Remark at the end of the previous section).
In this paper, we generalize the static analysis for the case of the under-approximation of
reachability in any AAN (\pref{ssec:ua}).

In \pref{ssec:concret}, we detail a procedure to extract a witness scenario
concretizing a given local state reachability property.
Finally, we discuss in \pref{ssec:ordered-ua} how the static analysis can be
extended to sequential (sub)states reachability properties.


\subsection{Abstractions for Scenarios}
\label{ssec:abstr-sce}

The approach presented in this section is based on two complementary notions,
the \emph{objectives} and their \emph{local causality},
that are intertwined in so-called \emph{Local Causality Graphs}.

\subsubsection{Objectives}

An \emph{objective} (\pref{def:obj}) denotes the reachability of a local state (e.g., $a_j$) of a given automaton $a$
from the initial local state of that automaton (e.g., $a_i$).
Such an objective is written $\PHobjp{a}{i}{j}$.
Successive objectives are described with objective sequences.

\begin{definition}[Objective ($\Obj$) \& Objective Sequence ($\OS$)]
\label{def:obj}
  If $a \in \components$, the reachability of a local state $a_j$ from a local state $a_i$ is called an \emph{objective}, noted $\PHobj{a_i}{a_j}$.
  The set of all objectives is noted:
  \[\Obj \DEF \{ \PHobj{a_i}{a_j} \mid a \in \components \wedge (a_i, a_j) \in \PHl_a \times \PHl_a \} \enspace.\]
  For an objective $P = \PHobj{a_i}{a_j} \in \Obj$, we define: $\PHsort(P) \DEF
  a$, $\PHtarget(P)\DEF a_i$, $\PHbounce(P)\DEF a_j$.
  Finally, $P$ is said \emph{trivial} iff $a_i = a_j$.

  We define an \emph{objective sequence} as a sequence of objectives in which each objective target must be equal to the previous objective bounce of the same automaton, if it exists.
  The set of all objective sequences is denoted by $\OS$.
  Given $\w\in\OS$, $\PHsort(\w) \DEF \{\PHsort(\w_i)\mid i\in\indexes{\w}\}$.
For each automaton $a\in\PHsort(\w)$,  the first local state of $a$ referenced in
$\w$ is denoted by $\first_a(\w) \DEF \target{\w_m}$, where $m=\min\{n\in\indexes{\w}\mid \PHsort(\w)=a\}$.
  The set of objective sequences starting in a state $s\in\PHl$ are denoted by
  $\OS(s)\DEF \{\w\in\OS\mid\forall a\in\PHsort(\w), \first_a(\w)\in s\}$.
\end{definition}


We define the partial ordering relation $\ltw$ between two objective sequences (\pref{def:ltw}) as
follows:
$w\ltw w'$ if and only if there exists a mapping between all the objective bounces of $w'$ in $w$
that preserves the sequentiality, provided that each automaton starts in the same
local state.

\begin{definition}[$\ltw \subset \OS\times\OS$]\label{def:ltw}
$\w\ltw\w'$ if and only if
$|\w|\geq|\w'|$,
$\forall a\in\PHsort(\w'), a\in\PHsort(\w) \wedge \first_a(\w')=\first_a(\w)$;
and there exists a mapping $\phi: \indexes{\w'}\mapsto \indexes{\w}$ such that
$\forall n,m \in\indexes{\w'}, n<m \Leftrightarrow \phi(n)<\phi(m)$,
and
$\forall n\in\indexes{\w'}, \bounce{\w'_n}=\bounce{\w_{\phi(n)}}$.
\end{definition}

\begin{example}
\[\obj{b_0}{b_1}\concat\obj{a_0}{a_1}\concat\obj{b_1}{b_2}
\ltw
\obj{a_0}{a_1}\concat\obj{b_0}{b_2}
\ltw
\obj{b_0}{b_2}\]
\[\obj{b_0}{b_1}\not\ltw\obj{b_0}{b_2}
\quad\text{and}\quad
\obj{b_0}{b_2}\not\ltw\obj{b_0}{b_1}\]
\end{example}

An objective sequence can be seen as an abstract representation of a set of scenarios that describe
(part of) the successive state changes of the automata.
We denote by $\concr(\w)$ (\pref{def:concr}) the set of scenarios matching with an objective sequence
$\w$ in the state $s\in\PHl$.
It is essentially all the scenarios for which there exists a mapping from the bounces of each
objective to the bounce of an action in the scenario which preserve the sequential ordering.

\begin{definition}[$\concr: \OS \to \powerset(\Sce)$]\label{def:concr}
Given a state $s\in\PHl$ and an objective sequence
$\w\in\OS(s)$, $\concr(\w)$ is the set of scenarios matching with $\w$:
\begin{align*}
\concr(\w) \DEF \{ \delta\in\Sce(s)\mid &\ 
\w^\vartriangle\neq \emptyseq\Rightarrow
\bounce{\delta_{\card\delta}} = \bounce{\w_{\card\w}}
\\ &
 \wedge \exists \phi:\indexes{\w}\mapsto\indexes{\delta},
    (\forall n,m\in\indexes{\w}, n<m \Leftrightarrow \phi(n)\leq\phi(m)
\\ & \qquad
	\wedge \forall n\in\indexes{\w},
	  \bounce{\w_n} \in s\play\delta_{1..\phi(n)})
\}
\enspace,
\end{align*}
in which $\omega^\vartriangle$ refers to the objective sequence $\omega$ where
trivial objectives have been removed.
The notation $\delta_{j..k}$
denotes the subsequence of $\delta$ between indexes $j$ and $k$,
as defined on page \pageref{notations}.
\end{definition}

From \pref{def:ltw} and \pref{def:concr}, we derive that if
$\w \ltw \w'$, then the scenarios matching $\w$ also match $\w'$ (\pref{lem:ltw}).
%$\concr(\w)\subseteq\concr(\w')$ (\pref{lem:ltw}).
\begin{lemma}\label{lem:ltw}
$\forall \w, \w' \in \OS, \forall s \in \PHl,
\w\ltw\w' \Longrightarrow \concr(\w)\subseteq\concr(\w')$\enspace.
\end{lemma}

Given a non-empty set $\Delta\subseteq\Sce(s)$ of non-empty scenarios that have a common last bounce
($\exists a_i\in\Proc, \forall\delta\in\Delta,
\bounce{\delta_{\card\delta}}=a_i$), one can define an abstraction $\alpha_s$ of
such a set as the smallest (according to $\ltw$) objective sequence $\w\in\OS(s)$ such that
$\bounce{\w_{\card\w}} = \bounce{\delta_{\card\delta}}, \delta\in\Delta$
and
$\forall\delta\in\Delta, \exists \phi:\indexes{\w}\mapsto\indexes{\delta}
    (\forall n,m\in\indexes{\w}, n<m \Leftrightarrow \phi(n)\leq\phi(m))
	\wedge \forall n\in\indexes{\w},
	  \bounce{\w_n} \in s\play\delta_{1..\phi(n)})$.
In such a setting, $\alpha_s$ and $\concr$ form a Galois connection between sets
of scenarios and objective sequences.


\subsubsection{Local Causality for Objectives}

The existence of a scenario from a state $s$ leading to a state
where a given local state $a_j$ is present
can be reformulated as the checking for the non-emptiness of $\concr(\obj{a_i}{a_j})$, where
$a_i=\get{s}{a}$.
A first hint for checking this emptiness is to look for actions that have to be played in order to
reach the state $j$ from $i$ within the automaton $a$.

Given an objective $P = \obj{a_i}{a_j} \in\Obj$, we define $\BS(P)$ the \emph{bounce sequences} of
$P$ as the set of minimal sequences of actions hitting $a$ in which the bounce of each action is
the target of the following action (\pref{def:bs}).

\begin{definition}[Bounce Sequence ($\BS$)]\label{def:bs}
A \emph{bounce sequence} $\zeta$  is a sequence of actions such that
$\forall n\in\indexes{\zeta}, n<|\zeta|,
\bounce{\zeta_{n}} = \target{\zeta_{n+1}}$.
$\BS$ denotes the set of minimal bounce sequences.
We refer to the set of bounce sequences \emph{resolving} the objective $P$ as
$\BS(P)$:
\begin{align*}
\BS(\obj{a_i}{a_j}) \DEF \{ \zeta\in\BS\mid & \target{\zeta_1}=a_i\wedge
			    \bounce{\zeta_{|\zeta|}}=a_j \\
& \wedge \forall m,n\in\indexes{\zeta}, n>m, \bounce{\zeta_n}\neq\target{\zeta_m}
				\}\enspace.
\end{align*}
\end{definition}

Therefore,
$\BS(\obj{a_i}{a_i}) = \{\emptyseq\}$; and
$\BS(\obj{a_i}{a_j}) = \emptyset$ if there is no possibility to reach $a_j$ from
$a_i$.

\begin{example}
Given the AAN of \pref{fig:ph-livelock},
$\BS(\obj{c_0}{c_1}) = \{\PHfrappes{a_1,b_1}{c_0}{c_1}\}$;
and 
$\BS(\obj{a_0}{a_1}) = \{\PHfrappes{b_0}{a_0}{a_1}\}$.
\end{example}

From \pref{def:bs}, we can derive that any scenario matching with an objective includes all the
actions of one of the bounce sequences of the objective (\pref{lem:bs-concr}).

\begin{lemma}
Given a state $s\in \PHl$ and an objective $\obj{a_i}{a_j}$ with $\get{s}{a}=a_i$,
$\forall\delta\in\concr(\obj{a_i}{a_j})$,
$\exists\zeta\in\BS(\obj{a_i}{a_j})$ such that
$\exists \phi: \indexes{\zeta}\to\indexes{\delta}$
with
$\forall n,m\in\indexes{\zeta}, n<m \Leftrightarrow \phi(n)<\phi(m)$
and
$\forall n\in\indexes{\zeta}$, $\zeta_n = \delta_{\phi(n)}$.
\label{lem:bs-concr}
\end{lemma}



\subsubsection{Local Causality Graph}

The previous subsections introduced the necessary definitions to build the
so-called Local Causality Graph (LCG).
This graph is denoted $\cwB$ where $u$ is the local state to be reached
from the initial state $s$.
It will be the basis of our static analysis,
as it represents the causality links between the different objectives involved
in the solving of the reachability of $u$.
This LCG is built recursively by considering some required local states (\eg $u$),
linking them to objectives (\eg $\PHobj{t}{u}$, if $t \in s$),
and locally refining these objectives in order to include new required local states
from other automata.
An example of LCG is depicted in \pref{fig:sa-livelock}.

Technically,
one can iteratively refine a given objective into (possibly several) objective
sequences extracted from its bounce sequences.
Indeed, the refinement of an objective $P$ with $\zeta\in\BS(P)$
consists in prepending to $P$ the
objectives leading to the activation of all hitters of the actions in $\zeta$,
in the same sequential order.
The generalization of this refinement to an objective sequence would
naturally require to consider all possible
interleaving between the refinements~\cite{PMR12-MSCS}.

For example, given the AAN of \pref{fig:ph-livelock},
in the state $\langle a_0,b_0,c_0\rangle$,
the objective $\obj{a_0}{a_1}$ can be refined in the objective sequence
$\obj{b_0}{b_0}\concat\obj{a_0}{a_1}$
and
the objective $\obj{c_0}{c_1}$ can be refined in the objective sequence
$\obj{a_0}{a_1}\concat\obj{b_0}{b_1}\concat\obj{c_0}{c_1}$
or
$\obj{b_0}{b_1}\concat\obj{a_0}{a_1}\concat\obj{c_0}{c_1}$
(which can then be further refined).

\medskip

Formally, a Local Causality Graph (LCG)
is a digraph where $\cwV \subseteq \Proc\cup \Obj \cup \Sol \cup \sSol$ is the set of vertices,
with $\sSol=\powerset(\Proc)$ and $\Sol=\Obj\times\powerset(\sSol)$,
and $\cwE \subseteq \cwV \times \cwV$ is the set of oriented edges
(\pref{def:lcg}).
% 
% These refinements can be summarized with a so-called Local Causality Graph (LCG).
% Given an initial context $s$ and a local state $\myp\in\Proc$,
% an LCG $\cwB = (\cwV, \cwE)$
% is a digraph where $\cwV \subseteq \Proc\cup \Obj \cup \Sol \cup \sSol$ is the set of vertices
% where $\sSol=\powerset(\Proc)$ and $\Sol=\Obj\times\powerset(\sSol)$
% and $\cwE \subseteq \cwV \times \cwV$ is the set of oriented edges
% (\pref{def:lcg}).
A node in $\cwV\cap\Obj$ represents an objective to refine.
Such a node is linked to nodes in $\Sol$ which represent
sets of hitter sets extracted from a
bounce sequence of the objective
(that is, one node for each bounce sequence in $\BS(P)$); thus,
given an objective $P$, if $\zeta\in\BS(P)$
and $\abstr\zeta$ is the set of all hitters of $\zeta$,
% and $\abstr\zeta \DEF \{\PHhitter(\zeta_n) \mid
% n\in\indexes\zeta \}$,
then $P$ is linked to a node $\langle P, \abstr\zeta\rangle\in\Sol$
(\pref{def:lcg}\nobreakdash-\ref{lcg-obj-sol}).
For each hitter set $ps\in\abstr\zeta$,
the node $\langle P,\abstr\zeta\rangle\in\cwV\cap\Sol$ is
linked to the node $ps\in\sSol$
(\pref{def:lcg}\nobreakdash-\ref{lcg-sol-sync}).
Then, a node $ps\in\cwV\cap\sSol$ is linked to each local state $a_i \in ps$ is contains
(\pref{def:lcg}\nobreakdash-\ref{lcg-sync-ls}).
A local state node $a_i\in\cwV\cap\Proc$
is linked to each objective $\obj{a_j}{a_i} \in \Obj$, where
$a_j$ is either in the initial state $s$ or
is a local state node in the LCG
(\pref{def:lcg}\nobreakdash-\ref{lcg-ls-obj}).
Finally, given an objective node $\obj{a_j}{a_i}\in\cwV\cap\Obj$,
if a local state $a_k\neq a_i$ is in its successors (which is given by
$\conn_{(\cwBNodes,\cwBEdges)}(\obj{a_j}{a_i})$), then
$\obj{a_j}{a_i}$ is linked to $\obj{a_k}{a_i}$ in the LCG
(\pref{def:lcg}\nobreakdash-\ref{lcg-conn}).

\begin{definition}
\label{def:lcg}
Given a state $s \in \PHl$ and a local state $\myp\in \Proc$,
the Local Causality Graph (LCG) for reachability under-approximation
$\cwB \DEF (\Bv, \Be)$,
with
$\cwBNodes \subseteq \Proc \cup \Obj \cup \NSol \cup \sSol$
and
$\cwBEdges \subseteq \cwBNodes \times \cwBNodes$,
is the smallest graph such that:
\begin{enumerate}
\item
$\myp \in \cwBNodes$
\item
$a_i\in\cwBNodes\cap\Proc \Leftrightarrow \{ (a_i,\obj{a_j}{a_i}) \mid
a_j \neq \myp \wedge (a_j\in s \vee a_j\in\cwBNodes\cap\Proc)
\}\subseteq\cwBEdges$
\label{lcg-ls-obj}
\item
$P\in\cwBNodes\cap\Obj \Leftrightarrow 
	\{ (P,\langle P,\abstr \zeta\rangle) \mid \zeta\in\BS(P) \}\subseteq\cwBEdges$
\label{lcg-obj-sol}
\item
$\langle P, pps \rangle\in\cwBNodes\cap\NSol \Leftrightarrow
	\{ (\langle P,pps\rangle, ps) \mid ps\in pps \}\subseteq\cwBEdges$
\label{lcg-sol-sync}
\item
$ps \in\cwBNodes\cap\sSol \Leftrightarrow
	\{ (ps, a_i) \mid a_i\in ps \}\subseteq\cwBEdges$
\label{lcg-sync-ls}
\item
$\obj{a_i}{a_j}\in \cwBNodes\cap\Obj \Rightarrow 
	\{(\obj{a_i}{a_j},\obj{a_k}{a_j}) \mid a_k\neq a_j,$
\label{lcg-conn}
\\
\hspace*{4cm}
$a_k\in \conn_{(\cwBNodes,\cwBEdges)}(\langle\obj{a_i}{a_j},\abstr\zeta\rangle),$
\\
\hspace*{4cm}
$\zeta\in\BS(\obj{a_i}{a_j}) \}\subseteq\cwBEdges$
\end{enumerate}
with $\abstr\zeta \DEF \{\PHhitter(\zeta_n) \mid n\in\indexes\zeta \}$.
\end{definition}

The definition of an LCG tackles two cases
where the target of an objective may be changed.
First,
the addition of objectives based on the local states already mentioned anywhere in the LCG,
as performed in \pref{def:lcg}\nobreakdash-\ref{lcg-ls-obj},
ensures to take into account the possible changes in the active local states
made by other objectives.
We thus try to ensure that a required local state is reachable even when
starting from a local state of the same automaton that is not in the initial state,
but that may be active at some point of the solving.
Second, \pref{def:lcg}\nobreakdash-\ref{lcg-conn},
allows to “re-target” objectives whose own solving changes their target.
This happens when playing the actions that activate the local states required to solve
an objective $P$ also changes the active local state of automaton $\PHsort(P)$.
In this case, the initial objective $P$ is re-targeted to another
objective $\PHobj{p}{\bounce{P}}$, where $p$ is the new active local state in $\PHsort(P)$.
Linking to all objectives of this kind ensures that all possible disturbances
are taken into account.

Finally, we note that
the LCG of \pref{def:lcg} contains synchronizing nodes ($\sSol$)
that allow to express the need for several local states simultaneously
in order to play a given action.
This is the main difference regarding~\cite{PMR12-MSCS,FPMR13-CS2Bio}
which tackled with Process Hitting, whose actions are limited to at most one hitter
and thus did not require this kind of synchronization.


\subsection{Sufficient condition for reachability of a local state}
\label{ssec:ua}

Given a state $s \in \PHl$ and a local state $\myp\in\Proc$, the LCG $\cwB$ contains a set of objectives
that can be used to build a concrete scenario reaching $\myp$.
Because we are focused on sufficient conditions for reachability, we do not require that all
possible scenarios can be derived from the LCG.

In this section, we prove that if the LCG has no cycle, all its nodes in $\Obj$ have at least one
child, and all its nodes in $\sSol$ satisfy a
particular criteria, so-called independence,
then there exists a scenario that reaches $\myp$ from state $s$ (\pref{th:approxinf}).

A node $ps \in \sSol$ of the LCG is \emph{independent} (\pref{def:coherent}) if for each local state
$a_i\in ps$, none of the other local states $b_j\in ps$ have in their successors
a local state of automaton $a$ but that is different than $a_i$.
This criteria ensures that once a local state in $ps$ has been reached, reaching another local state
in $ps$ should not impact the first.

\begin{definition}[Independent synchronizations]
\label{def:coherent}
  In a LCG $\cwB = (\Bv, \Be)$,
  a node $ps \in \Bv\cap\sSol$ is \emph{independent} if and only if
  for each $a_i\in ps$,
  for each $b_j\in ps, b_j\neq a_i$,
  $a_k \in\conn_{\cwB}(b_j) \cap \Proc \Rightarrow a_k = a_i$.
\end{definition}

The intuition of the under-approximation of \pref{th:approxinf} is the following.
Given the initial state $s$, we recursively refine the initial objective
(which is: reaching $\myp$ from the initial state) according to its children.
As the LCG is acyclic, by hypothesis, such a recursion always terminates, and as all the objective nodes have at
least one child, it never gets stuck.
The refinement of an objective node $\obj{a_i}{a_j}$ acts as follows:
if the node has another objective node $\obj{a_k}{a_j}$ as child, we first refine the objective
$\obj{a_i}{a_k}$ (which, by construction, is necessarily in the LCG)
and then we refine the objective $\obj{a_k}{a_j}$.
If the objective node has only successors in $\Sol$ (bounce sequences), one is picked arbitrarily.
If $\langle P, pps\rangle$ is the chosen node,
by construction there exists $\zeta\in\BS(P)$ such that $\abstr\zeta = pps \in \powerset(\sSol)$.
If $\zeta = \emptyseq$ (for instance in the case where $a_i = a_j$), the recursion stops and we
continue to the next stage.
Otherwise, for each $n\in\indexes\zeta$,
for each $b_i \in\PHhitter(\zeta_n)$,
we refine the objective $\obj{b_j}{b_i}$, where $b_j$ is the state of $b$ in the current state.
By induction, $\obj{b_j}{b_i}$ is a child of $b_i$ in the LCG of \pref{def:lcg}.
Thus, we know that the current state of $b$ is $b_i$.
After having repeated this procedure for each $b_i\in\PHhitter(\zeta_n)$,
because all the nodes in $\sSol$, and in particular $\PHhitter(\zeta_n)$, are independent,
we know that all the local states in $\PHhitter(\zeta_n)$ are in the current state.
In addition, we know that the state of automaton $a$ has remained unchanged, otherwise
$\obj{a_i}{a_j}$ would have an objective child.
Hence, the action $\zeta_n$ is playable in the current state.
We can thus apply this action, modifying the state of $a$ to $a_j$ and continue to the next stage.
In the end, this recursive procedures builds a scenario from $s$ to a state containing $\myp$.

\begin{theorem}[Under-approximation]
\label{th:approxinf}
  Given an AAN $(\PHs; \PHl; \PHa)$,
  a state $s$ and local state $\myp$,
  if the LCG $\cwB$ contains no cycle,
  all nodes in $\Obj$ have at least one child,
  and all nodes in $\sSol$ are independent,
  then there exists a scenario $\delta\in\Sce$ such as $\myp\in s\play\delta$.
\end{theorem}

Regarding the complexity of the method,
computing the LCG is polynomial in the number of automata in $\PH$ and exponential in the number of local states in one automaton.
Checking the properties allowing to apply \pref{th:approxinf} is polynomial in the size of the graph.
Therefore, the building and checking processes can be considered as polynomial in the size
of the AAN, provided that each automaton only contains a few local states.
We note that this is particularly true for biological models, where
each component usually contains a limited number of expression levels.

We note furthermore that the method does not require any completeness of the bounce sequences
$\BS$.
Therefore, in order to reduce the number of bounce sequences to consider, and potentially remove
cycles and non-satisfying nodes,
one can consider only a sub-set of bounce sequences
(and in particular: a unique bounce sequence) for each objective.
However, such an approach requires to enumerate all possible combinations of bounce sequences
subsets, hence being exponential in the number of considered bounce sequences.
%objectives solved by at least two bounce sequences.



\begin{example}
  We consider in this example the AAN of \pref{fig:ph-livelock},
  and the initial state $\PHstate{a_1, b_0, c_0}$
  that is also represented.
  The under-approximation given in \pref{th:approxinf}
  concludes that $a_1$ is reachable from this initial state, as well as $b_1$.
  Nevertheless, it does not conclude regarding the reachability of $c_1$.
  This is due to the fact that the node $\{ a_1, b_1 \} \in \Bv \cap \sSol$
  is not independent because of its successor $a_0$ (and $b_0$)
  as we can see in the LCG of \pref{fig:sa-livelock}.
  (However, from the inconclusiveness of \pref{th:approxinf},
  one cannot conclude about the unreachability of $c_1$.
  Such analysis should be driven for instance
  with over-approximation methods developed in~\cite{PMR12-MSCS}.)
  
  This result is new compared to the method proposed in~\cite{PMR12-MSCS}.
  Indeed, the representation based on the Process Hitting that was proposed
  in this paper only allowed to represent “over-approximated” Boolean gates
  with the use of so-called cooperative sorts.
  This especially did not allow to model the fact that $a_1$ and $b_1$ could not
  be activated in the same state, but only in successive states.
  Thus, when using Process Hitting, $c_1$ was indeed reachable,
  contrary to the behaviour expected from an accurate Boolean gate.
  
  Finally, we note however that
  if actions $\PHhits{a_0}{b_0}{b_1}$ and $\PHhits{b_0}{a_0}{a_1}$
  are replaced respectively by
  $\PHhitm{a_0}{a_1}$ and $\PHhitm{b_0}{b_1}$,
  then the resulting saturated graph of local causality changes, and
  \pref{th:approxinf} concludes that $c_1$ is reachable from $\PHstate{a_1, b_0, c_0}$.
  The reader can also refer to \pref{ssec:ex-metazoan}
  for a detailed conclusive example.

\begin{figure}[tp]
  \centering
  \begin{tikzpicture}[aS]
    \node[Aproc] (c1) {$c_1$};
    \node[Aobj,below of=c1] (c01) {$\PHobj{c_0}{c_1}$};
%    \node[Aobj] (c01) {$\PHobj{c_0}{c_1}$};
    \node[Asol,below of=c01] (c01s) {};

    \node[Assol,below of=c01s] (a1a1ss) {$\{ a_1, b_1 \}$};
    \node[Aproc,below left of=a1a1ss] (a1) {$a_1$};
    \node[Aobj,below of=a1] (a11) {$\PHobj{a_1}{a_1}$};
    \node[Asol,below of=a11] (a11s) {};
    \node[Assol,below of=a11s] (na11s) {$\emptyset$};
    \node[Aobj,below left of=a1] (a01) {$\PHobj{a_0}{a_1}$};
    \node[Asol,below of=a01] (a01s) {};
    \node[Assol,below of=a01s] (a01ss) {$\{b_0\}$};
    \node[Aproc,below of=a01ss] (b0) {$b_0$};
    \node[Aobj,below of=b0] (b00) {$\PHobj{b_0}{b_0}$};
    \node[Asol,below of=b00] (b00s) {};
    \node[Assol,below of=b00s] (nb00s) {$\emptyset$};
    \node[Aobj,below left of=b0] (b10) {$\PHobj{b_1}{b_0}$};
    \node[Asol,below of=b10] (b10s) {};
    \node[Assol,below of=b10s] (nb10s) {$\emptyset$};

    \node[Aproc,below right of=a1a1ss] (b1) {$b_1$};
    \node[Aobj,below of=b1] (b11) {$\PHobj{b_1}{b_1}$};
    \node[Asol,below of=b11] (b11s) {};
    \node[Assol,below of=b11s] (nb11s) {$\emptyset$};
    \node[Aobj,below right of=b1] (b01) {$\PHobj{b_0}{b_1}$};
    \node[Asol,below of=b01] (b01s) {};
    \node[Assol,below of=b01s] (b01ss) {$\{a_0\}$};
    \node[Aproc,below of=b01ss] (a0) {$a_0$};
    \node[Aobj,below of=a0] (a00) {$\PHobj{a_0}{a_0}$};
    \node[Asol,below of=a00] (a00s) {};
    \node[Assol,below of=a00s] (na00s) {$\emptyset$};
    \node[Aobj,below right of=a0] (a10) {$\PHobj{a_1}{a_0}$};
    \node[Asol,below of=a10] (a10s) {};
    \node[Assol,below of=a10s] (na10s) {$\emptyset$};

    \path
    (c1) edge (c01)
    (c01) edge (c01s)
    (c01s) edge (a1a1ss)
    (a1a1ss) edge (a1) edge (b1)

    (a1) edge (a01) edge (a11)
    (a01) edge (a01s)
    (a01s) edge (a01ss)
    (a01ss) edge (b0)
    (a11) edge (a11s)
    (a11s) edge (na11s)
    (a0) edge (a10) edge (a00)
    (a10) edge (a10s)
    (a10s) edge (na10s)
    (a00) edge (a00s)
    (a00s) edge (na00s)

    (b0) edge (b10) edge (b00)
    (b10) edge (b10s)
    (b10s) edge (nb10s)
    (b00) edge (b00s)
    (b00s) edge (nb00s)
    (b1) edge (b01) edge (b11)
    (b01) edge (b01s)
    (b01s) edge (b01ss)
    (b01ss) edge (a0)
    (b11) edge (b11s)
    (b11s) edge (nb11s)
    ;
    \end{tikzpicture}
  \caption{
  \label{fig:sa-livelock}
    The local causality graph $\cwB$ on the AAN in \pref{fig:ph-livelock}
    for the reachability of $\myp = c_1$
    from the initial state $s = \PHstate{a_1, b_0, c_0}$.
    Elements in $\Proc$ are represented by rectangular nodes,
    elements in $\Sol$ are represented by small circles,
    and elements in $\sSol$ and $\Obj$ are the remaining borderless nodes.
    \pref{th:approxinf} is inconclusive on this example as
    node $\{ a_1, b_1 \} \in \sSol$
    is not independent (see \pref{def:coherent}).
    Indeed, $a_0$ is a successor of $b_1$, but $a_0 \neq a_1$
    (and the same also stands for $b_0$, which is a successor of $a_1$).
  }
\end{figure}
\end{example}



\subsection{Extraction of a Scenario}
\label{ssec:concret}

This section gives a recursive method to find a scenario that concretizes
the reachability of a given local state $\myp \in \Proc$,
from a given initial state $s_0 \in \PHl$,
provided that \pref{th:approxinf} answered positively on this couple of inputs.
The algorithm proposed in what follows relies on a traversal of the Local Causality Graph
$\cwBz$ that has been used for this conclusion.
The correctness of this extraction can be demonstrated
in the same fashion as \pref{th:approxinf}.

This algorithm consists in visiting all required nodes in a given order,
to build a sequence of actions that concretizes the objective.
The actions to perform on a given node,
listed below,
depend on the current state, the type of node, the markers on this node
and the state of the traversal which can be either “descending” (D) or “ascending” (A).
Nodes in $\Sol$ can be marked with a sequence of actions
and nodes in $\sSol$ can be marked with a set of local states.
When ascending, the traversal always ascend into the node it previously descended from
(in order to cover the same path backwards).
The traversal starts in node $\myp$ in descending mode,
the starting state is $s = s_0$
and the initial output is the empty sequence $\emptyseq$.

\begin{itemize}
  \item In a node $a_k \in \Proc$:
    \begin{itemize}
      \item[D)] Descend in the node $\PHobj{\PHget{s}{a}}{a_k} \in \Obj$.
      \item[A)] Ascend in the parent $\sSol$ node, if any,
        and remove the element $a_k$ from its marking set.
        If the node has no $\sSol$ parent, the traversal is complete
        and the output of the algorithm is the current output.
    \end{itemize}
  
  \item In a node $\PHobj{a_j}{a_k} \in \Obj$:
    \begin{itemize}
      \item[D)] Descend in an arbitrarily chosen node $\langle P, pps \rangle \in \Sol$,
        and mark it with an arbitrarily chosen sequence
        $\zeta \in \BS(P)$ so that $\abstr{\zeta} = pps$.
      \item[A)] Ascend in the parent $\Proc$ or $\Obj$ node.
    \end{itemize}
  
  \item In a node $\langle P, pps \rangle \in \Sol$ with the current marking $\zeta$:
    \begin{itemize}
      \item If $\zeta \neq \emptyseq$,
        descend in the node $ps \in \sSol$
        so that $ps = \hitter{\zeta_1}$
        and mark it with the set $ps$.
      \item If $\zeta = \emptyseq$,
        ascend in the parent node $P \in \Obj$ and carry on ascending.
    \end{itemize}
  
  \item In a $\sSol$ node with the current marking $m$:
    \begin{itemize}
      \item If $m \neq \emptyset$,
        descend in a child node $a_k \in \Proc$ arbitrarily chosen,
        so that $a_k \in m$ and carry on descending.
      \item If $m = \emptyset$,
        ascend in the parent $\Sol$ node.
        Let $\zeta$ be the current marking of this node.
        If $\zeta_1$ is playable in $s$:
          append $\zeta_1$ to the output,
          change the current state to $s \play \zeta_1$
          and mark this node with $\zeta_{2..|\zeta|}$.
        If $\zeta_1$ is not playable
        (which means that $\target{\zeta_1}$ has changed),
          then go to the node $\PHobj{\PHget{s}{\PHsort(\bounce{\zeta_1})}}{\bounce{\zeta_1}} \in \Obj$
          and start descending.
    \end{itemize}
\end{itemize}

The execution of such a traversal outputs a scenario from $s_0$.
In the following, we denote by $\Delta(\cwBz)$ the set of all scenarios extracted from the
local causality graph $\cwBz$.
Indeed, several scenarios may exist for a same graph,
generated by the arbitrary choices that exist in this algorithm.
An example of such a traversal is given on a small example
in \pref{ssec:ex-metazoan}.



\subsection{Addressing Sequential and Sub-state Reachability}
\label{ssec:ordered-ua}
\label{ssec:simult-ua}

\newcommand{\total}{\tau}
\newcommand{\reach}{\sigma}

In this section, we briefly discuss how the presented static analysis for local
state reachability properties can be extended to sequential (sub)states
reachability properties.

\subsubsection*{Sub-state reachability}
Given an AAN $\PH = (\PHs, \PHl, \PHh)$,
an initial state $s \in \PHl$ and
a sub-set of local states
$\total \subseteq \Proc$ % = \{ a_i \in\Proc\mid a\in\PHs, a_i\in\PHl_a\}$
so that $\forall a \in \PHs, \card{\total \cap \PHl_a} \leq 1$,
the reachability of a state containing all local states of $\total$
from the initial state $s\in\PHl$
can be tackled by our method.
Indeed,
consider the AAN
$\PH' = (\PHs', \PHl', \PHh')$ with:
$\PHs' = \PHs \cup \{ \reach \}$, $\PHl' = \PHl \times \PHl_\reach$,
where $\PHl_\reach = \{ \reach_0, \reach_1 \}$,
and $\PHh' = \PHh \cup \{ \PHhit{\total}{\reach_0}{\reach_1} \}$.
Obviously,
the reachability in $\PH$
of any state $s'\in\PHl$ such that
$\total \subset s'$ from the initial state $s$
is equivalent to the reachability in $\PH'$ of the local state
$\reach_1$ from state $s \times \{ \reach_0 \}$.

Such analysis was not possible with the Process Hitting framework,
because of the lack of the notion of simultaneity for more than two components.

\subsubsection*{Sequential reachability}
Given a sequence of local states to reach (\eg reach $a_i$, then $b_j$, etc.),
one can use the same approach as for sub-state reachability by introducing
a new automaton $\reach$ having $n+1$ local states, where $n$ is the size of the
sequence of reachability, and $n$ actions making it bounce gradually from $\reach_0$ to
$\reach_n$ with hitters corresponding the successive local states
(\eg $\PHfrappes{a_i}{\reach_0}{\reach_1}$,
$\PHfrappes{b_j}{\reach_1}{\reach_2}$, etc.).

An alternative approach is to use the extraction of a scenario of
\pref{ssec:concret}:
given an initial state $s_0$, and a first local state reachability property $\myp_1$,
one can compute a possible scenario $\delta \in \Delta(\mycwB{s_0}{\myp})$
witnessing this reachability;
then the next local state reachability properties ($\myp_2$, $\myp_3$, etc.)
are computed from the
state $s_0 \play \delta$ that outcomes from the latter scenario.
