% vi:spell spelllang=en:
\section{Introduction}
\label{sec:intro}

Discrete modelling frameworks for biological networks is an active research field where formal
methods have be proven very
powerful~\cite{thomas1990biological,deJong02,snoussi_logical_1993}.
Such a work started in the seventies, with~\cite{kauffman69,Thomas73}.
It was later enriched in many directions and widely used to elucidate many biological questions.
Among these questions, a major one is to understand precisely how biological systems evolve and behave; why and how they change their usual behaviours, etc.
These questions are strongly linked to the (possible or inevitable) reachability of some states.
The ultimate goal is to discover how it could be possible to prevent biological systems from reaching some pathological conditions.

Of course, such formal models on which analyses are performed are abstract representations of the actual studied systems.
They are associated with parameters that have to be synthesised %so as to be as much as possible fitting with the real systems having some observed behaviors.
to give the most faithful representation of the real systems with their observed behaviours.
As a matter of fact, the abstractions we get are more or less rough or accurate.
Prevalent formal frameworks for such modelling activities are state-transition systems or process algebras. % Petri nets
We developed a quite similar framework named the Process Hitting~\cite{PMR10-TCSB},
consisting in a restriction of these frameworks where the evolution of a component is determined by the state of at most one other component that does not evolve.
This is modelled by actions are of the form $X + Y \rightarrow X + Y'$ where $X$ behaves like a catalyst molecule that “hits” another molecule $Y$ and changes it into $Y'$, without being changed itself.
Assuming catalysts are always available, this can represent any biochemical system made of monomolecular reactions, and can also represent catalytic networks such as metabolic networks.
%, with the aim of avoiding to build the whole state space in order to be able to tackle very large systems
Our motivation behind this framework was to design a model and analysis techniques adapted to biological modelling.
These analyses avoid to build the whole state space, which allows to tackle very large systems (that would have led to a huge number of states, hopelessly too huge to be analysed).
They are based on the fact that most biological models have few levels of expression per component:
in Boolean networks~\cite{kauffman69,Thomas73} there are only two levels per component, and in their multivalued equivalent~\cite{deJong02}, components rarely have more than four levels.

Besides, one further objective of our work is now %to be more accurate in the description of the studied systems.
to improve the accuracy of the description of the studied systems dynamics.
The idea for this is to introduce timing features into models:
we are interested in taking into account some knowledge about the relative length of some phenomena as it is a way to refute some models (or parameters) that are inconsistent with the observed dynamic behaviours.
In this paper, we are dealing with these timing properties through priorities,
that are based on the simple founding idea that actions with higher priority have to be processed before the ones with lower priority.
Furthermore, due to the Process Hitting framework restrictions, multimolecular reactions were previously not immediately available, but one could simulate them with an encoding called “cooperative sort”.
That encoding however introduces extra reactions,
that produce a temporal shift between the presence of the reactants
and the playability of the reaction.
This is where the priorities become useful, if not necessary:
the extra reactions can for example be given “infinite speed” (highest priority) so that they do not affect the behaviour of “normal” (lower priority) reactions, including the multimolecular ones.

The approach used in this paper
consists in considering a broader class of models,
that we call Asynchronous Automata Network,
and that allows to naturally model these cooperations
by defining several requisites for a transition.
Moreover, such automata networks are still compatible with the notion of priority,
that can also be used to model different reaction rates in the model.
Asynchronous Automata Networks
(and, a fortiori, their restriction, the Process Hitting framework)
can be considered as a subset of Communicating Finite State Machines
or safe Petri Nets~\cite{PMR-PetriNets}.
%or Synchronous Automata Networks,
%is inspired from the $\pi$-calculus,
Our work is thus related to such semantic ramifications
of extending traditional process algebras with the concept of priority
that allows for some transitions to be given precedence over others.
We focus here on static priorities that allow to model
time constraints such as reaction rates or delays between regulations,
but can also model preemptions between evolution branches.

Until now, such a priority scheduling of the actions was not studied extensively in the different formal modelling frameworks dedicated to systems biology.
Nevertheless, such an attempt has been carried out for Petri nets by F.~Bause~\cite{Bause97},
and the concept of priority relations among the transitions of a network has also more recently been introduced by A.~K.~Wagler \textit{et~al.}~\cite{waw,WaglerW12} in order to allow modelling deterministic systems for biological applications.
The concept of priority is much straightforward in the approach of process algebras as it was shown by R.~Cleaveland and M.~Hennessy in~\cite{Cleaveland199058} and their abstractions and equivalences were studied in~\cite{Cleaveland:2007:PAP:1282576.1282847}.
It was later extended for applications in the field of systems biology by M.~John \textit{et~al.}~\cite{jlnu2010}.

\subsection*{Contributions}
In this paper, we develop a method
that allows to efficiently compute under-approximations of the dynamics
of Asynchronous Automata Networks (AANs),
generalising a prior work carried out on Process Hitting~\cite{PMR12-MSCS}.
Rather than using brute force or symbolic model checking techniques,
our method focuses on static analysis by abstract interpretation.
Our work aims at checking reachability properties that,
for a given initial state $s$ and the discrete expression level $n$
of a given component $x$,
have the form:
“Starting from state $s$, is it possible to reach
a state in which $x$ is at level $n$?”,
thus answering either “True”
(in which case the reachability is formally proven)
or “Inconclusive”
(which however does not stand for “False”, that can be proven by other analysis tools,
such as the over-approximation proposed in~\cite{PMR12-MSCS}).
Moreover, the successive or simultaneous reachability of several local states
can also be checked,
and in the case of a “True” response,
an execution path satisfying the property can be produced.
Extensions of AANs where transitions are split in priority classes are addressed with an encoding in
AANs without classes of priorities.
Our work thus allows to efficiently analyse the dynamics
of regulation networks,
especially the widespread Logical Networks \cite{Thomas95,deJong02},
which encompass variables with a limited number of discrete values
alongside with evolution functions or focal parameters.
To show the scalability of our method, we apply it to two
large-scale biological models with around 100 components.

\pref{sec:ph} presents the Asynchronous Automata Networks (AANs)
without any notion of classes of priorities.
In \pref{sec:sa},
we develop our under-approximation method allowing to efficiently
compute reachability analyses;
we also show how to extract a valid execution path if the response is positive,
and propose two refinements in case one needs to check
successive or simultaneous reachability properties.
The framework of AANs with classes of priorities is given in \pref{sec:flattening},
alongside with an encoding into AANs without classes of priorities
(or, equivalently, with one class of priority).
Finally, \pref{sec:example} provides a detailed example and two large-scale examples
of the application of our method,
and \pref{sec:ccl} gives a conclusion and a discussion about our work.

The main additions in this paper compared to~\cite{FPMR13-CS2Bio} are
\pref{ssec:concret} that allows the extraction of a concrete trace of execution
when our static analysis method is conclusive,
\pref{ssec:ordered-ua} that refines our approach in the case
of a successive reachability in order to increase the conclusiveness,
and \pref{sec:flattening} which states that any ANN
with any number of classes of priorities
can be represented into a simple ANN
(or, equivalently, with only one class of priority),
thus extending the scope of our method.
We note however that the use of AANs without priorities alone
instead of Process Hitting models
allows to simplify the notations, but does not increase
the range of applicability of the results.
Finally, \pref{sec:example} has been improved with a detailed example
and a new large-scale example with a quantitative examination of the results.



\subsection*{Notations}
\label{notations}
%We denote: $\segm{a}{b} = \{ a, a+1, \dots, b-1, b \}$.

\paragraph*{Sets}
If $A$ is a finite set,
$\card{A}$ is the cardinality of $A$
and $\powerset(A)$ is the power set of $A$.
$\sN$ is the set of natural numbers,
$\sN^* = \sN \setminus \{ 0 \}$ is the set of positive natural numbers
and $\segm{x}{y} = \{ x, x+1, \dots, y-1, y \}$ is the set of natural numbers from $x$ to $y$ included.
The Cartesian product of sets is denoted by $\times$ and
if $z$ is a tuple of $n$ components, $\toset{z}$ denotes the corresponding set:
$\toset{z} = \{z_1, \cdots, z_n\}$.

\paragraph*{Sequences}
We denote by $\emptyseq$ the empty sequence.
If $n \in \sN$ and
$x = (x_i)_{i \in \segm{1}{n}}$ is a sequence of elements indexed by $i \in \segm{1}{n}$,
then $\card{x} = n$ is the size of $x$
%we denote %$|x| = (b-a)+1$ the size of this sequence and
and $\indexes{x} = \segm{1}{n}$ is the set of indexes of this sequence.
Furthermore, if $a, b \in \indexes{x}$ with $a \leq b$,
then $x_{a..b} = (x_i)_{i \in \segm{a}{b}}$ is the subsequence of $x$
between indexes $a$ and $b$ included.
%from element $n$ to element $m$ inclusive.
Finally, if $x$ is a sequence, $\toset{x}$ also denotes the corresponding set:
$\toset{x} = \{x_1, \cdots, x_{\card{x}}\}$.

\paragraph*{Digraphs}
If $(V,E)$ is a directed graph whose set of nodes is $V$ and whose set of edges is $E \subseteq V\times V$,
the children of a node $n$ are given by
$\childs: V \to \powerset(V)$, with
$\childs(n) = \{ m\in V\mid (n,m)\in E\}$;
its parents are given by
$\parents: V \to \powerset(V)$ with
$\parents(n) = \{ m\in N\mid (m,n)\in E\}$;
and its successors are given by
$\conn_{(V,E)}(n)$ which is the least fixed point containing $n$
of the function
$\f{f}\conn:\powerset(V)\to \powerset(V)$
with $\f{f}\conn(W) = \bigcup_{m\in W} \childs(m)$.

%\paragraph*{Functions and least fixed point}
%If $A$ and $B$ are sets,
%$f : A \rightarrow B$ denotes a function $f$ that maps the elements of $A$ to elements of $B$.
%If $f$ is a monotonically increasing and bounded function, then
%$\lfp{x_0}{x}{x'}$ is the least fixed point of the function $x \mapsto x'$ which is greater than $x_0$.

