\section{Proof of Under-approximation (\pref{ssec:ua})}
\label{suppl:demoapproxinf}

In the following, we denote:
$\Bee{X}{Y} = \Be \cap (X \times Y)$, with $X, Y$ amongst $\PHproc$, $\Obj$ and $\Sol$.

\newproof{proofapproxinf}{Proof of \pref{th:approxinf}}
\begin{proofapproxinf}
We note $max\ctx = \ctx \Cap \allprocs(\cwB)$ the context supported by $\cwB$.

Let $(a_i, ps) \in \Bee{\Proc}{\Sol}$ be an edge linking the required process of a cooperative sort to a solution set and suppose all children of $ps$ are concretisable.
We label all processes of $ps$ by an integer: $ps = \{ p_n \}_{n \in \indexes{ps}}$.
Let us prove by induction that for all $n \in \indexes{ps}$, there exists a scenario $\delta_n$ so that:
$\forall i \in \segm{1}{n}, \PHget{(s \PHplay \delta_n)}{\PHsort(p_i)} = p_i$.
\begin{itemize}
  \item It is straightforward for $\delta_0 = \varepsilon$.
  \item Suppose such $\delta_n$ exists and let $q = \PHget{(s \PHplay \delta_n)}{\PHsort(p_{n+1})}$.
    By hypothesis, $(a_i, ps)$ is coherent (\pref{def:coherent}) and all processes of $ps$ are processes of components;
    this means that none of the processes needed to solve $p_{n+1}$ is another process of the same sort than another process of $ps$.
    Therefore, there exists $\delta' \in \muconcr_{s \PHplay \delta_n}(\PHobj{q}{p_{n+1}})$,
    so that $\forall i \in \segm{1}{n+1}$, $\PHget{(s \PHplay \delta_n \PHplay \delta')}{\PHsort(p_{i})} = p_{i}$.
    Finally, by \pref{lem:update}, there exists a scenario $\delta'' \in \restriction{\Sce}{1}(s \PHplay \delta_n \PHplay \delta')$
    so that, if we denote $\delta_{n+1} = \delta_n \PHplay \delta' \PHplay \delta''$,
    we have: $\update(s \PHplay \delta_n \PHplay \delta') = s \PHplay
	\delta_{n+1}$ and the same property about processes (by \pref{lem:hcscomp}).
\end{itemize}
Therefore, $\delta = \delta_{|ps|}$ exists, and given its properties, we have: $\PHget{(s \PHplay \delta)}{a} = a_i$
and $\update(s \PHplay \delta) = s \PHplay \delta$.

As there is no cycle in $\cwB$, we show by induction that $\forall s\in L, s\subseteq max\ctx$, 
for all objective $P$ in $\Bv \cap \Obj$ so that $\PHtarget(P) \in s$,
$\exists \delta \in \muconcr_s(P)$.
\begin{itemize}
  \item If $(P, \emptyset) \in \Bee{\Obj}{\Sol}$, either $\PHtarget(P) = \PHbounce(P)$ and $\delta = \emptyseq$;
    or $\forall \zeta \in \BS(P), \zeta \in \Sce(s) \wedge \PHsort(\zeta) = \{ \PHsort(P) \}$
    and $\delta = \delta_1 \PHplay \zeta_1 \PHplay \dots \PHplay
	\delta_{|\zeta|} \PHplay \zeta_{|\zeta|}$ is a valid sequence given by
	\pref{lem:hcompcomp}.

  \item Suppose all children objectives of $P$ are concretisable.
    If $\exists (P, Q) \in \Bee{\Obj}{\Obj}$, then by hypothesis,
      $\muconcr_{s}(\obj{\PHtarget(P)}{\PHtarget(Q)} \concat Q) \neq \emptyset$, thus
      $\muconcr_{s}(P) \neq \emptyset$.
    Else, by \pref{def:maxCont}, the concretisations of the children of $P$ require no process of sort $\PHsort(P)$.
      Furthermore, there exists $\zeta \in \BS(P)$ so that $(P, \aZ) \in \Bee{\Obj}{\Sol}$.
      We show by induction that for all $n \in \indexes{\zeta}$, there is a scenario $\delta_n$ so that $\PHget{(s \PHplay \delta_n)}{\PHsort(P)} = \PHbounce(\zeta_n)$.
      \begin{itemize}
        \item[$\circ$] Suppose that $\delta_n$ exists and let $\zeta_n = \PHhit{b_i}{a_j}{a_k}$.
        By hypothesis there exists $\delta' \in \muconcr_{s \PHplay \delta_n}(\PHobj{\any}{b_i})$ with $\PHsort(P) \notin \PHsort(\delta')$ (by \pref{def:maxCont}).
        By \pref{lem:update} there exists $\delta'' \in \restriction{\Sce}{1}(s \PHplay \delta')$ so that $\update(s \PHplay \delta') = s \PHplay \delta' \PHplay \delta''$.
        Furthermore, $\PHget{(s \PHplay \delta' \PHplay \delta'')}{b} = b_j$ (by
		\pref{lem:hcompcomp} if $b \in \components$ or \pref{lem:hcscomp} if $b \in \cs$).
        Therefore, $\delta_{n+1} = \delta_n \PHplay \delta' \PHplay \delta'' \PHplay \zeta_n$.
      \end{itemize}
      Thus, $\delta_{|\zeta|} \in \muconcr_s(P)$. % and $\ceil(\delta) \subseteq max\ctx$.
\end{itemize}
Finally, as $\muconcr_{max\ctx}(\w) \neq \emptyset$, $\uconcr(\w) \neq
\emptyset$ (\pref{lem:uconcr-ctx}).
\end{proofapproxinf}
