% vi:spell spelllang=en:
\section{Discussion \& Conclusion}\label{sec:ccl}

We introduced a new semantics to include priorities into the Process Hitting framework, which prove especially useful to model cooperations.
Then, we developed a method to efficiently perform a reachability analysis of a sequence of objectives in a restricted class of Process Hitting models,
but it is also useful to establish the reachability of a partial state.
This analysis is based on an under-approximation of the true reachability solutions.
%; however, the most usual cases can be handled.

We showed that the class of Process Hitting models that can be handled by the aforementioned method are equivalent to Asynchronous Discrete Networks, and therefore to Asynchronous Boolean Networks.
This allows to efficiently compute reachability results on large biological models provided that they are equivalent to Asynchronous Discrete Networks and that a translation from the original framework into a Process Hitting model is possible.
Such a translation for interaction graphs of Thomas modelling was proposed in~\cite{PMR10-TCSB}. % \towrite{[À garder ?] and is made possible by the Pint software}.

We also demonstrated that general Process Hitting models with arbitrary priority
classes can also be encoded in the particular class of two-prioritised Process Hitting.
Therefore, our static analysis is applicable to a broad range of automata
networks with priorities.

Further work can be derived from what have been presented in this paper.
The over-approximation on Process Hitting models without priorities proposed in~\cite{PMR12-MSCS}
is still accurate in the framework with priorities (by “merging” all actions),
but may be refined given the restrictions proposed in this paper,
and a specific search of key processes or cut sets may be derived.
%but turns out to be too wide even in some obvious cases that are consequently not conclusive.
%This approximation may be refined in order to better fit the introduction of priorities, and mak the overall approximation approach 
%and mode precisely the class of models studied in this paper.
Furthermore, we are investigating alternative under-approximations that can be
applied directly to the whole class of Process Hitting models with priorities,
and not only to a sub-class with particular restrictions;
such improvement may allow to increase the conclusiveness of the static analysis
while allowing to analyse any model without the need of a translation.
Finally, in order to take into account quantitative data in transition delays, the overall approximation method could be extended to handle evolutions that are chronometric instead of only chronologic.
