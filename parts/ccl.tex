\section{Discussion \& Conclusion}\label{sec:ccl}
\towrite{Pistes :
\begin{itemize}
  \item Plus de classes de priorités ?
  \item Modèles Zeno (système de précédences et prise en compte des états globaux)
\end{itemize}
}

In this paper we introduced a new semantics to include priorities into the Process Hitting framework, which are especially useful to model cooperations.
Then, we developed a method to efficiently perform a reachability analysis of a sequence of processes on a restricted class of Process Hitting models.
This analysis is based on an under-approximation of the true reachability solutions; however, the most usual cases can be handled.
Furthermore, our method can also be used to determine the reachability of a set of states of a model.
The method has been implemented as an extension of the existing “ph-reach” tool into the existing Pint\footnote{\url{http://processhitting.wordpress.com/}} library.

Finally, we showed that the class of Process Hitting models that can be handled by the aforementioned method are equivalent to discrete networks, which include boolean networks.
This allows to efficiently compute reachability results on large biological models provided that they are equivalent to discrete networks and that a translation from the original framework into a Process Hitting model is possible.
Such a translation for interaction graphs of Thomas modelling was proposed in~\cite{PMR10-TCSB} and is made possible by the Pint software.

Further work can be derived from what have been presented in this paper.
An over-approximation on Process Hitting models without priorities was proposed in~\cite{PMR12-MSCS}.
Although it is still accurate on the Process Hitting framework with priorities (by “flattening” all actions),
it turns out to be too wide even in some obvious cases that are consequently not conclusive.
This over-approximation may be refined in order to better fit the introduction of priorities, and mode precisely the class of models studied in this paper.
Furthermore, a more general under-approximation could be developed in order to handle a larger class of Process Hitting models, that is,
models with more than two classes of priorities, that do not only contain components of cooperative sorts or whose behaviour may contain cycles or cyclic attractors.
Finally, in order to take into account quantitative data on transition delays, the overall approximation method could be extended to handle evolution that are chronometric instead of only chronologic.
