% vi:spell spelllang=en:
\section{Discussion \& Conclusion}\label{sec:ccl}

In this paper, we focused on the Asynchronous Automata Networks (AANs),
which is a restriction of classical Automata Networks
and is equivalent to the Process Hitting with classes of priorities~\cite{FPMR13-CS2Bio}.
This formalism proves useful to model accurate Boolean gates, which was not possible
with the standard form of the Process Hitting,
and avoid an unwanted over-approximation of the dynamics.
Furthermore, we proposed an extension of this formalism with classes of priorities,
which prove convenient to abstract time parameters into these kinds of models,
or more simply to add preemption relations between actions.

Then, we developed a method to perform a reachability analysis
of a sequence of objectives in an AAN,
based on an under-approximation of the true reachability solutions.
The checked objectives can take the form of local states, global states
or partial states, making the analysis more convenient.
This method can be considered efficient,
because it is exponential in the number of possible solutions,
and each step is only polynomial in the size of the model.
Moreover, AANs with classes of priorities can also be studied in this way,
at the cost of a translation that we gave in this paper.

Finally, we also showed that AANs are also equivalent to Asynchronous Discrete Networks,
that is, the multivalued version of the classical Asynchronous Boolean Networks
with discrete expression levels and evolution functions.
This framework is also equivalent to Thomas' models with Snoussi parameters
instead of evolution functions.
This especially allows to efficiently compute reachability results
on large biological models,
provided that they are equivalent to Asynchronous Discrete Networks,
which ensures that a translation to AANs is possible.
For example,
such a translation for generalized Interaction Graphs,
that is, Discrete Networks without evolution functions or parameters,
was proposed in~\cite{PMR10-TCSB}.

Further work can be derived from what have been presented in this paper.
The over-approximation on Process Hitting models without priorities proposed in~\cite{PMR12-MSCS}
is still accurate in the framework with priorities (by “merging” all actions),
but may be refined given the restrictions proposed in this paper,
and a specific search of key processes or cut sets \cite{PAK13-CAV} may be derived.
%but turns out to be too wide even in some obvious cases that are consequently not conclusive.
%This approximation may be refined in order to better fit the introduction of priorities, and mak the overall approximation approach 
%and mode precisely the class of models studied in this paper.
Furthermore, we are investigating alternative under-approximations that can be
applied directly to the whole class of Process Hitting models with priorities,
and not only to a sub-class with particular restrictions;
such improvement may allow to increase the conclusiveness of the static analysis
while allowing to analyse any model without the need of a translation.
Finally, in order to take into account quantitative data in transition delays, the overall approximation method could be extended to handle evolutions that are chronometric instead of only chronologic.
