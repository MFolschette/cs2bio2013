\section{\todo{Title?} Equivalence with other frameworks}

\subsection{Asynchronous Discrete Networks}
\todo{Detail the equivalence with ADNs (cf. the annex)}

\subsection{PH with multiple classes of priorities}
\todo{Reduce and reuse the flattening method}

\newcommand{\ov}{\overline}
\newcommand{\F}{F}
\newcommand{\Fopsymbol}{\F}
\newcommand{\Fop}[1]{\Fopsymbol(#1)}
\newcommand{\Fsem}[2]{\left[#1\right](#2)}
%\newcommand{\Fall}{\F}

Consider a PH with $k \in \sN^*$ classes of priories:
$\ov{\PH} = (\ov{\PHs}; \ov{\PHl}; \ov{\PHa}^{\angles{k}})$.
We want to translate it to a PH with $2$ classes of priorities
$\PH = (\PHs; \PHl; \PHa^{\angles{2}})$
that respects all Criteria given in \pref{ssec:hypothesis}.

A playability property (PP) as defined in \pref{def:pp} is a formula where the atoms are processes of $\ov{\PH}$.

\begin{definition}[Playability property language ($\F$)]
  \label{def:pp}
  A \emph{playability property} (PP) is an element of the language $\F$ inductively defined by:
  \begin{itemize}
    \item $\top$ and $\bot$ belong to $\F$;
    \item if $a \in \ov{\PHs}$ and $a_i \in \ov{\PHl}_a$, then $a_i \in \F$ and we call it an \emph{atom};
    \item if $P \in \F$ and $Q \in \F$, then $\neg P \in \F$, $P \wedge Q \in \F$ and $P \vee Q \in \F$.
  \end{itemize}
  We note $\Fsem{P}{s}$ the \emph{evaluation} of a PP $P \in \F$ in a state $s \in \ov{\PHl}$.
  The semantics of this evaluation is the following for any state $s \in \ov{\PHl}$:
  \begin{itemize}
    \item if $P = a_i \in \ov{\PHs}_a$ is an atom, with $a \in \ov{\PHs}$, then $\Fsem{a_i}{s}$ is true iff $a_i \in s$;
    \item if $P$ is not an atom, then $\Fsem{P}{s}$ is true iff all its subformulas are inductively true in $s$
      with the classical semantics for the logic operators $\top$, $\bot$, $\neg$, $\wedge$ and $\vee$.
  \end{itemize}
\end{definition}

Because we only use classical logic operators, the formulas of Boolean logic on distribution and negation (de Morgan's laws) can be used.
We also have the following property for the negation of an atom:
\[\forall a \in \ov{\PHs}, \forall a_i \in \ov{\PHl}_a, \Fsem{\neg a_i}{s}
  \Leftrightarrow \Fsem{\bigvee_{\substack{a_j \in \PHl_a\\a_j \neq a_i}} a_j}{s}\]
Indeed, if a process is not active in a state, this means that another process of the same sort is active.

In \pref{def:fop}, we finally define the the operator $\Fopsymbol$ which characterizes the playability of an action
given the semantics of PH (\pref{def:play}).
\begin{definition}[Playability property of an action ($\Fopsymbol : \PHh \rightarrow \F$)]\label{def:fop}
  $$\forall h \in \ov{\PHh}, \Fop{h} \equiv \hitter{h} \wedge
    \left( \bigwedge_{\substack{g \in \PHh^{(n)}\\n < \prio(h)}}
    \neg \left( \target{g} \wedge \hitter{g}\right) \right)$$
\end{definition}
%
\begin{theorem}
\label{th:ppplay}
  $h \in \ov{\PHh}$ is playable in $s \in \ov{\PHl}$ if and only if $\Fsem{\Fop{h}}{s}$.
\end{theorem}
%
\begin{proof}
  By definition of the semantics of a PH given in \pref{def:play}.
\end{proof}

\newcommand{\n}[1][h]{n^{#1}}
\newcommand{\m}[1][h,i]{m^{#1}}

Because we only use classical logic operators, we can compute the Disjunctive Normal Form (DNF) of any PP.
For any action $h \in \ov{\PHh}$, this DNF takes the form:
\[\Fop{h} \equiv \bigvee_{i \in \segm{1}{\n}} \left( \bigwedge_{j \in \segm{1}{\m}} p_{i,j} \right)\]
where $\n \in \sN$ and $\forall i \in \segm{1}{\n}, \m \in \sN^*$.
If $\n = 0$, then $\Fop{h} \equiv \bot$; this means that $h$ can never be played
due to preemptions by other actions with higher priorities.
If $\Fop{h} \not\equiv \bot$, on the other hand, then in this case $\Fop{h}$
can be seen as a disjunction of $\n$ smaller playability properties consisting only of conjunctions of atoms.
These $\n$ conjunctions can be translated to as many prioritized cooperative sorts,
thus creating a new PH model which is entailed in the restrictions depicted in \pref{ssec:hypothesis}.
%Indeed, for any $i \in \segm{1}{\n}$, the PP $\bigwedge_{j \in \segm{1}{\m}} p_{i,j}$
In this second case, we denote, for any $i \in \segm{1}{\n}$:
$\PHdep{i}{h} = \{ \PHsort(p_{i,j}) \mid j \in \segm{1}{\m} \}$.

\newcommand{\flats}[1]{\lfloor #1 \rfloor}
\newcommand{\unflats}[1]{\lceil #1 \rceil}

\newcommand{\os}{\ov{s}}
\newcommand{\oPH}{\ov{\PH}}

\begin{definition}
  \label{def:flattening}
  If $k \in \sN^*$, we note $\PHflat(\ov{\PHs}; \ov{\PHl}; \ov{\PHa}^{\angles{k}}) = (\PHs; \PHl; \PHa^{\angles{2}})$
  the flattening of the PH with $k$ classes of priorities $\ov{\PHs}; \ov{\PHl}; \ov{\PHa}^{\angles{k}}$, with:
  \begin{itemize}
    \item $\PHs = \ov{\PHs} \cup \PHs_f$
      where $\PHs_f = \{ f^{h,i} \mid h \in \ov{\PHh} \wedge \n \geq 1 \wedge i \in \segm{1}{\n} \}$;
    \item $\PHl = \left( \underset{a \in \ov{\PHs}}{\times} \PHl_{a} \right) \times
      \left(\underset{f^{h,i} \in \PHs_f}{\times} \PHl_{f^{h,i}} \right)$,
      where $\forall a \in \ov{\PHs}, \PHl_{a} = \ov{\PHl}_{a}$, and
      $\forall f^{h,i} \in \PHs_f,
      \PHl_{f^{h,i}} = \{ f^{h,i}_\varsigma \mid \varsigma \in \underset{b \in \PHdep{i}{h}}{\times} \PHl_{b} \}$;
    \item $\PHh^{(1)} = \{ \PHhit{a_k}{f^{h,i}_\varsigma}{f^{h,i}_{\varsigma'}} \mid
      %h \in \ov{\PHh} \wedge %\n \geq 1 \wedge i \in \segm{1}{\n} \wedge
      f^{h,i} \in \PHs_f \wedge
      a \in \PHdep{i}{h} \wedge a_k \in \PHl_a \wedge
      f^{h,i}_\varsigma , f^{h,i}_{\varsigma'} \in \PHl_{f^{h,i}} \wedge
      \PHget{\varsigma}{a} \neq a_k \wedge \varsigma' = \varsigma \Cap \{ a_k \} \}$;
%      \todo{From here: ...} \PHget{\varsigma'}{a} = a_k \wedge
%      (\forall b \in \PHdep{i}{h} \setminus \{ a \}, \PHget{\varsigma'}{b} = \PHget{\varsigma}{b}) \}$;
%      \todo{... can be replaced by: $\varsigma' = \varsigma \Cap \{ a_k \}$}
    \item $\PHh^{(2)}=\{ \PHhit{f^{h,i}_\varsigma}{\target{h}}{\bounce{h}} \mid
      f^{h,i} \in \PHs_f \wedge
      %h \in \ov{\PHh} \wedge 
      f^{h,i}_\varsigma \in \PHl_{f^{h,i}} \wedge \Fsem{h}{\varsigma} \}$.
  \end{itemize}
  Given a state $s \in \PHl$, $\unflats{s} = \os$ is the corresponding state in $\ov{\PHl}$:
  $\forall a \in \ov{\PHs}, \PHget{\os}{a} = \PHget{s}{a}$.

  \noindent
  Given a state $\os \in \ov{\PHl}$, $\flats{\os} = s$ is the corresponding state in $\PHl$:
  $\forall a \in \ov{\PHs}, \PHget{s}{a} = \PHget{\os}{a}$
  and $\forall f^{h,i} \in \PHs_f, \PHget{s}{f^{h,i}} = f^{h,i}_\varsigma$ with $f^{h,i}_\varsigma \in \PHl_{f^{h,i}}$
  and $\forall b \in \PHdep{i}{h}, \PHget{\varsigma}{b} = \PHget{\os}{b}$.
\end{definition}



\begin{theorem}[$(\ov{\PHs}; \ov{\PHl}; \ov{\PHa}^{\angles{k}}) \approx \PHflat(\ov{\PHs}; \ov{\PHl}; \ov{\PHa}^{\angles{k}})$]
\label{thm:bisimPHP}
  Let $\ov{\PH} = (\ov{\PHs}; \ov{\PHl}; \ov{\PHa}^{\angles{k}})$ a PH with $k$ classes of priorities,
  and $\PH = \PHflat(\ov{\PHs}; \ov{\PHl}; \ov{\PHa}^{\angles{k}}) = (\PHs; \PHl; \PHa^{\angles{2}})$ its flattening.
  \begin{enumerate}
    \item \label{php2ph} $\forall \os, \os' \in \ov{\PHl}$,
      $\os \PHPtrans[\oPH] \os' \Longrightarrow \flats{\os} \PHPtrans[\oPH]^* \flats{\os'}$,
      where $\PHPtrans[\oPH]^*$ is a finite sequence of $\PHPtrans[\oPH]$ transitions.

    \item \label{ph2php} $\forall s, s' \in \PHl$,
      $s \PHPtrans s' \Longrightarrow \unflats{s} = \unflats{s'} \vee
      \unflats{s} \PHPtrans \unflats{s'}\enspace.$
  \end{enumerate}
\end{theorem}
%
\begin{proof}
  (\ref{php2ph}) Let $\flats{\os} = s$.
    From \pref{def:play}, if $\os \PHPtrans[\PH] \os'$,
    then there exists $h \in \ov{\PHh}$ so that $\os' = \os \PHplay h$.
    Furthermore, given \pref{th:ppplay}, $\Fsem{\F(h)}{\os}$.
    Finally, given \pref{def:flattening}, $\PHget{s}{f^{h,i}}$
  
  
  \todo{-- À continuer --}
\end{proof}




\todo{--- Le reste de la traduction ici ---}



\begin{definition}[Equivalent Process Hitting]\label{def:equiv}
  Two PH models $\ov{\PH} = (\ov{\PHs}; \ov{\PHl}; \ov{\PHh}^\angles{k})$ and $\PH = (\PHs; \PHl; \PHh^\angles{n})$,
  where $k, n \in \sN^*$,
  are \emph{equivalent} iff:
  $$\forall s \in \ov{\PHl}, ( \exists h = \PHhit{A}{b_i}{b_j} \in \PHh, \Fsem{\F(h)}{s}
    \Longleftrightarrow \exists g = \PHhit{\ov{A}}{b_i}{b_j} \in \ov{\PHh}, \Fsem{\F(g)}{s} )$$
\end{definition}

We note that several PH models can be equivalent to a same PH model.
For example, adding actions that are never playable (due to priorities) allows to create an equivalent model.

\begin{theorem}[Bisimilarity of equivalent PH]
  Two equivalent PH models are bisimilar.
\end{theorem}

\begin{proof}
  Let $\PH = (\PHs; \PHl; \PHh^\angles{k})$ and $\ov{\PH} = (\PHs; \PHl; \ov{\PHh}^\angles{n})$
  two equivalent PH models, with $k, n \in \sN^*$.
  We want to show:
  $$\forall s, s' \in \PHl, s \neq s', s \trans{\PH} s' \Longleftrightarrow s \trans{\ov{\PH}} s' $$
  Let $s, s' \in \PHl$ with $s \neq s'$.
  ($\Rightarrow$) Suppose $s \trans{\PH} s'$.
  It means that there exists $h = \PHhit{A}{b_i}{b_j} \in \PHh$ so that $s' = s \PHplay h = s \Cap b_j$
  in $\PH$ (\pref{def:play}), \ie $\Fsem{\F(h)}{s}$ and $b_i \in s$.
  As $\ov{\PH}$ is equivalent to $\PH$,
  there exists $g = \PHhit{\ov{A}}{b_i}{b_j} \in \ov{\PHh}$ so that $\Fsem{\F(g)}{s}$ (\pref{def:equiv}).
  Therefore, $g$ is playable in $s$ and $s \PHplay g = s \Cap b_j = s'$;
  which means: $s \trans{\ov{\PH}} s'$
  ($\Leftarrow$) Same reasoning.
\end{proof}

\begin{definition}[Flattening]
  Given a PH model $\PH = (\PHs; \PHl; \PHh^\angles{k})$, where $k \in \sN^*$,
  we call \emph{flattening} of $\PH$ any PH model $\ov{\PH} = (\PHs; \PHl; \ov{\PHh}^\angles{1})$
  with 1 class of priority and cooperative actions that is equivalent to $\PH$.
\todo{Is this definition really useful?}
\end{definition}

\begin{theorem}[Existence of a flattening]\label{th:exists-flattening}
  For all $k \in \sN^*$, there exists a flattening of any PH model with $k$ priorities and cooperative actions.
\todo{Is this theorem really useful?}
\end{theorem}

The demonstration of this theorem is given in the following subsection.
