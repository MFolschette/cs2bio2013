\section{The Process Hitting Framework}
\label{sec:ph}

We give in this section the definition and the semantics of the Process Hitting (PH) with priorities, which is an extension of the basic semantics given in~\cite{PMR10-TCSB}.
Then we describe the modelling of cooperation between components and discuss how the new aforementioned semantics makes this modelling more accurate.
Finally, in order to perform a static analysis adapted to this new semantics, we give several criteria to restrict the class of models that we can study,
and give several lemmas that follow.
This class of models is equivalent to Asynchronous Discrete Networks.

\subsection{Definition of the Process Hitting with $k$ classes of priorities}
\label{ssec:PH}

A PH with $k \in \sN^*$ classes of priorities (\pref{def:ph}), also simply called “PH” in the following when it is not ambiguous, gathers a finite number of concurrent \emph{processes} divided into a finite set of \emph{sorts}.
A process belongs to a unique sort and is noted $a_i$ where $a$ is the sort and $i$ the identifier of the process within the sort $a$.
Each process stands for a kind of “activity level” of its sort; a state of the PH thus corresponds to a set of processes containing exactly one process of each sort.

The concurrent interactions between processes are defined by a set of \emph{actions} divided into classes of priorities.
Actions describe the replacement of a process by another of the same sort conditioned by the presence of at most one other process and by the fact that no other action of higher priority can be played in the considered state of the PH.
An action is denoted by $\PHhit{a_i}{b_j}{b_k}$ where $a_i,b_j,b_k$ are processes of sorts $a$ and $b$.
It is required that $b_j \neq b_k$ and that $a=b\Rightarrow a_i=b_j$.
An action $h=\PHfrappe{a_i}{b_j}{b_k}$ is read as ``$a_i$ \emph{hits} $b_j$ to make it bounce to $b_k$'', and $a_i,b_j,b_k$ are called respectively \emph{hitter}, \emph{target} and \emph{bounce} of the action, and can be referred to as $\PHhitter(h), \PHtarget(h), \PHbounce(h)$, respectively.

\begin{definition}[Process Hitting with $k$ classes of priorities]
\label{def:ph}
  If $k \in \sN^*$, a \emph{Process Hitting with $k$ classes of priorities} is a triplet $\PH = (\PHs; \PHl; \PHa^{\langle k \rangle})$,
  where $\PHa^{\langle k \rangle} = (\PHa^{(1)}; \dots; \PHa^{(k)})$ is a $k$-tuple and:
  \begin{itemize}
    \item $\PHs \DEF \{a, b, \dots, z\}$ is the finite set of \emph{sorts},
    \item $\PHl \DEF \underset{a \in \PHs}{\times} \PHl_a$ is the finite set of \emph{states}, where $\PHl_a = \{a_0, \ldots, a_{l_a}\}$ is the finite set of \emph{processes} of sort $a \in \PHs$ and $l_a \in \sN^*$. Each process belongs to a unique sort: $\forall (a_i; b_j) \in \PHl_a \times \PHl_b, a \neq b \Rightarrow a_i \neq b_j$,
    \item $\forall n \in \llbracket 1; k \rrbracket, \PHa^{(n)} \DEF \{\PHfrappe{a_i}{b_j}{b_l} \mid (a; b) \in \PHs^2 \wedge (a_i; b_j; b_l) \in \PHl_a \times \PHl_b \times \PHl_b \wedge b_j \neq b_l \wedge a = b \Rightarrow a_i = b_j \}$ is the finite set of \emph{actions of priority $n$}.
  \end{itemize}
  We call $\PHproc \DEF \bigcup_{a \in \PHs} \PHl_a$ the set of all processes, and $\PHh \DEF \bigcup_{n \in \segm{1}{k}} \PHh^{(n)}$ the set of all actions.
  Furthermore, for all $h \in \PHh$,
  %for all $n \in \sN^*$ and $h \in \PHh^{(n)}$,
  $\prio(h) \DEF \min\{ n \in \segm{1}{k} \mid h \in \PHh^{(n)} \}$.
\end{definition}
%
\noindent
The sort of a process $a_i$ is referred to as $\PHsort(a_i) = a$.
Given a state $s\in \PHl$, the process of sort $a \in \PHs$ present in $s$ is denoted by $\PHget{s}{a}$, that is the $a$-coordinate of the state $s$.
If $a_i \in \PHl_a$, we define the notation $a_i \in s \EQDEF \PHget{s}{a} = a_i$.
\pref{def:substate} defines the notion of sub-state on a set of sorts, that is used to focus on the interesting part of a complete state.
Note that a state is \textit{a fortiori} a sub-state.
The override of a state $s$ by a process $a_i$ (or by a set of processes)
is defined in \pref{def:statecap} as the same state in which the process of sort $a$ has been replaced by $a_i$,
which then allows to define the dynamics of a PH in \pref{def:play}.
%
\begin{definition}[Sub-states ($\PHsublize{\PHl}$)]
\label{def:substate}
  If $S \subseteq \PHs$ is a set of sorts, a sub-state on $S$ is an element of:
  $\PHsubl[\PHl]_S \DEF \bigtimes{a \in S} \PHl_a$.
  The set of all sub-states is:
  $\PHsubl[\PHl] \DEF \bigcup_{S \in\powerset(\PHs)} \PHsubl[\PHl]_S$.
  
  \noindent
  Furthermore, if $\mysigma \in \PHsubl[\PHl]$ and $s \in \PHl$, we note:
    \[\mysigma \subseteq s \EQDEF \forall a_i \in \Proc, a_i \in \mysigma \Rightarrow a_i \in s \enspace.\]
\end{definition}
%
\begin{definition}[$\Cap : \PHl \times \PHproc \rightarrow \PHl$]
\label{def:statecap}
  Given a state $s \in \PHl$ and a process $a_i \in \PHproc$, $(s \Cap a_i)$ is the state defined by:
  $\PHget{(s \Cap a_i)}{a} = a_i \wedge \forall b \neq a, \PHget{(s \Cap a_i)}{b} = \PHget{s}{b}$.
  Under the condition that all processes are from different sorts,
  we can also extend this definition to a set of processes
  by the override of each process independently:
  $\forall ps \in \PHsubl[\PHl], s \Cap \toset{ps} = s \underset{a_i \in \toset{ps}}{\Cap} a_i$.
\end{definition}
The notation $\toset{ps}$ represents the set of all elements in $ps$
(which is not a set but a tuple)
as defined on page~\pageref{notations}.
%
\begin{definition}[Dynamics of a PH ($\PHPtrans$)]
\label{def:play}
  An action $h = \PHhit{a_i}{b_j}{b_k} \in \PHa^{(n)}$ of priority $n$ is \emph{playable} in $s \in \PHl$
  if and only if $\PHget{s}{a} = a_i$, $\PHget{s}{b} = b_j$ and $\forall m < n, \forall g \in \PHa^{(m)}, \PHhitter(g) \notin s \vee \PHtarget(g) \notin s$.
  In such a case, $(s \PHplay h)$ stands for the state resulting from the play of the action $h$ in $s$ and is defined by: $(s \PHplay h) = s \Cap b_k$.
  Moreover, we denote: $s \PHPtrans (s \PHplay h)$.

  If $s \in \PHl$,
  a \emph{scenario} $\delta$ from $s$ is a (possibly empty) sequence of actions of $\PHh$
  that can be played successively in $s$.
  The set of all scenarios from $s$ is noted $\Sce(s)$.
\end{definition}

\modMF{
In \pref{def:restriction}, we define the $1$-reduction of a given PH as the model
in which only actions of highest priority lower are considered.
\begin{definition}[PH $1$-reduction]
\label{def:restriction}
  If $\PH = (\PHs; \PHl; \PHa^{\langle k \rangle})$ is a Process Hitting with $k$ classes of priorities, we denote by $\restriction{\PH}{1}$ the $1$-reduction of $\PH$:
  $\restriction{\PH}{1} = (\PHs; \PHl; \PHa'^{\langle 1 \rangle})$
  with $\PHa'^{\langle 1 \rangle} = (\PHa^{(1)})$.
  Furthermore, we denote by $\restriction{\Sce}{1}(s)$ the set of scenarios from $s$ in $\restriction{\PH}{1}$.
\end{definition}
}



\begin{example}
  \pref{fig:ph-livelock} gives an example of PH with $2$ classes of priorities where:
  \begin{align*}
    \PHs &= \{ a, b, c, ab \} \enspace,
      & \PHa^{\langle 2 \rangle} &= (\PHa^{(1)}; \PHa^{(2)}) \enspace, \\
    \PHl_a &= \{ a_0, a_1 \} \enspace,
      & \PHl_b &= \{ b_0, b_1 \} \enspace, \\
    \PHl_c &= \{ c_0, c_1 \} \enspace,
      & \PHl_{ab} &= \{ ab_{00}, ab_{01}, ab_{10}, ab_{11} \}
  \end{align*}
  and the sets of actions are:
  \begin{align*}
    \PHh^{(1)} = \{ \quad
      & \PHfrappe{a_1}{ab_{00}}{ab_{10}} \enspace , \quad
      \PHfrappe{a_1}{ab_{01}}{ab_{11}} \enspace , \\
      & \PHfrappe{a_0}{ab_{10}}{ab_{00}} \enspace , \quad
      \PHfrappe{a_0}{ab_{11}}{ab_{01}} \enspace , \\
      & \PHfrappe{b_1}{ab_{00}}{ab_{01}} \enspace , \quad
      \PHfrappe{b_1}{ab_{10}}{ab_{11}} \enspace , \\
      & \PHfrappe{b_0}{ab_{01}}{ab_{00}} \enspace , \quad
      \PHfrappe{b_0}{ab_{11}}{ab_{10}}
    \quad \} \\
    \PHh^{(2)} = \{ \quad
      & \PHfrappe{a_1}{a_1}{a_0} \enspace , \quad
      \PHfrappe{b_0}{a_0}{a_1} \enspace , \\
      & \PHfrappe{b_1}{b_1}{b_0} \enspace , \quad
      \PHfrappe{a_0}{b_0}{b_1} \enspace , \\
      & \PHfrappe{ab_{11}}{c_0}{c_1}
    \quad \} \\
  \end{align*}
%   Note that: $\{ \PHhit{ab_{11}}{c_0}{c_1}, \PHhit{a_1}{a_1}{a_0}, \PHhit{a_0}{b_0}{b_1} \} \subseteq \PHh^{(2)}$.

\begin{figure}[tb]
  \centering
  \scalebox{1.3}{
  \begin{tikzpicture}
    \path[use as bounding box] (-.2,-.5) rectangle (7.2,5.7);

    \TSort{(0,0)}{a}{2}{l}
    \TSort{(0,4)}{b}{2}{l}
    \TSort{(7,2.5)}{c}{2}{r}

    \TSetTick{ab}{0}{00}
    \TSetTick{ab}{1}{01}
    \TSetTick{ab}{2}{10}
    \TSetTick{ab}{3}{11}
    \TSort{(2,2.5)}{ab}{4}{t}

    \THit{a_0.-20}{bend right,in=270,prio}{ab_3}{.south}{ab_1}
    \THit{a_0.20}{bend right,in=250,prio}{ab_2}{.south}{ab_0}
    \THit{a_1.-20}{bend right,in=240,prio}{ab_1}{.south}{ab_3}
    \THit{a_1.20}{bend right,in=220,prio}{ab_0}{.south}{ab_2}

    \THit{b_0}{bend right,in=70,out=40,prio}{ab_3}{.north}{ab_2}
    \THit{b_0.south west}{bend right,in=140,out=40,prio}{ab_1}{.north}{ab_0}
    \THit{b_1}{bend right,in=90,out=40,prio}{ab_2}{.north}{ab_3}
    \THit{b_1}{bend right,in=170,out=10,prio}{ab_0}{.north}{ab_1}
    
    \THit{a_1}{selfhit}{a_1}{.west}{a_0}
    \THit{b_1}{selfhit}{b_1}{.west}{b_0}
    \THit{a_0.110}{bend left}{b_0}{.210}{b_1}
    \THit{b_0.-10}{bend right=60}{a_0}{.west}{a_1}

    \THit{ab_3}{}{c_0}{.west}{c_1}

	\path[bounce, bend right=90] %55]
      \TBounce{ab_0}{in=-110}{ab_2}{.south west}
      \TBounce{ab_1}{in=-110}{ab_3}{.south west}
	;
	\path[bounce, bend left=25]
      \TBounce{ab_3}{}{ab_1}{.south east}
      \TBounce{ab_2}{}{ab_0}{.south east}
	;
    \path[bounce, bend right=30]
      \TBounce{ab_3}{}{ab_2}{.30}
      \TBounce{ab_1}{}{ab_0}{.30}
    ;
    \path[bounce, bend left=40]
      \TBounce{ab_2}{}{ab_3}{.150}
      \TBounce{ab_0}{}{ab_1}{.150}
    ;
    \path[bounce, bend right=60]
      \TBounce{a_1}{}{a_0}{.north west}
      \TBounce{b_1}{}{b_0}{.north west}
    ;
    \path[bounce, bend left=60]
      \TBounce{a_0}{}{a_1}{.south west}
      \TBounce{b_0}{}{b_1}{.south west}
    ;
    \path[bounce, bend left]
      \TBounce{c_0}{}{c_1}{.south west}
    ;
    \TState{a_1, b_0, ab_2, c_0}
  \end{tikzpicture}
  }
  \caption{%
  \label{fig:ph-livelock}%
    An example of PH with $2$ classes of priorities.
    This model represents the interaction of two exclusive components $a$ and $b$,
    that cannot be active simultaneously, and that degrade over time.
    Furthermore, the cooperative sort $ab$ allows to encode a conjunction between
    the processes $a_1$ and $b_1$ in order to activate the third component $c$.
    Sorts are represented by labelled boxes and processes by circles with their identifier on the side.
    Actions of $\PHh^{(1)}$ are represented by thick arrows and actions of $\PHh^{(2)}$ are represented by thin arrows;
    the hit part of each action is drawn in plain line and the bounce part is in dotted line.
    Greyed processes stand for the following possible state: $\PHstate{a_1, b_0, c_0, ab_{10}}$.
  }
\end{figure}

\end{example}



\subsection{Modelling cooperation}
\label{ssec:cooperation}

Cooperation between processes to make another process bounce can be expressed in PH by building a \emph{cooperative sort},
which allows to encode any kind of Boolean function \cite{PMR10-TCSB}.
\pref{fig:ph-livelock} shows an example of cooperation between processes $a_1$ and $b_1$
in order to allow $c_0$ to bounce to $c_1$.
In this model, a cooperative sort $ab$ is defined with 4 processes
in order to represent each possible configuration
of the presence of processes $a_1$ and $b_1$
(“only $a_1$ is active”, “only $b_1$ is active”, “both are active” and “both are not active”).
For the sake of clarity, the processes of $ab$ are indexed using the sub-state they represent.
Hence, $ab_{10}$ represents the sub-state $\PHstate{a_1,b_0}$, and so on.
Each process of sort $a$ and $b$ hit $ab$ to make it bounce to the process reflecting the status of the sorts $a$ and $b$
(\eg $\PHfrappe{a_1}{ab_{00}}{ab_{10}}$ and $\PHfrappe{a_1}{ab_{01}}{ab_{11}}$).
Then, to represent the cooperation between $a_1$ and $b_1$, the process $ab_{11}$ hits $c_0$ to make it bounce to $c_1$.
Thus, this method based on a cooperative sort
replaces independent actions, \eg $\PHhit{a_1}{c_0}{c_1}$ and $\PHhit{b_1}{c_0}{c_1}$,
that would not be sufficient to model the conjunction
between $a_1$ and $b_1$.

We note that cooperative sorts are standard PH sorts and do not involve any
special treatment regarding the semantics of related actions.
Furthermore, it is possible to “factorise” cooperative sorts in order to decrease the number of processes created within each cooperative sort.
For example, if three processes $x_1$, $y_1$ and $z_1$ cooperate,
it is preferable to create a cooperative sort $xy$ with 4 processes to state the presence of $x_1$ and $y_1$
and a second cooperative sort $xyz$ with 4 processes to state the presence of $xy_{11}$ and $z_1$,
rather than a unique cooperative sort with 8 processes stating the presence of $x_1$, $y_1$ and $z_1$.
This “factorisation” allows to prevent the combinatorial explosion of the number of processes in cooperative sorts,
especially for cooperations between more than three processes.
It may have computational consequences as the static analysis method developed in~\pref{sec:sa} does not suffer from the number of sorts but from the number of processes in each sort.



\modMF{
\subsection{Justification of the priorities}
}
\label{ssec:priorities}

As explained previously,
the construction of cooperation in PH allows to encode any Boolean function between cooperating processes \cite{PMR10-TCSB}.
However, in the original version of the PH framework,
which contained no classes of priorities
(or, equivalently, only one class of priority),
the representation of such Boolean functions was not accurate.
Indeed, the lack of priorities did not allow cooperative sorts
to always encode the \emph{current state} of several sorts, as expected,
but only a \emph{combination of past states}.

Let us for example consider the PH model
$\PH = (\PHs; \PHl; \PHh^{\angles{2}})$ from \pref{fig:ph-livelock},
and the “merged” version of this model:
$\PH' = (\PHs; \PHl; \PHh'^{\angles{1}})$,
that is, where $\PHh'^{\angles{1}} = \PHh = \PHh^{(1)} \cup \PHh^{(2)}$.
In other words, $\PH'$ has the same sorts and processes
(and therefore the same states) than $\PH$,
but the actions of the two classes of priories of the latter
are united into one only class in the former.

Starting from state $\PHstate{a_1, b_0, c_0, ab_{10}}$,
it is not possible to reach a state where the processes $a_1$ and $b_1$
are simultaneously active, in either $\PH$ and $\PH'$.
Indeed, in both cases the actions hitting sorts $a$ and $b$ require that both
$a_0$ and $b_0$ are active in order to activate $b_1$ (\resp $a_1$).
Therefore, given the fact that the cooperative sort $ab$ represents
the combined states of $a$ and $b$,
and that the action $\PHhit{ab_{11}}{c_0}{c_1}$
represents a Boolean “conjunction” between $a$ and $b$
(that is, requires that both have process name “1” active),
the expected behaviour is that $c_1$ cannot be reached.

However, starting from state $\PHstate{a_1, b_0, c_0, ab_{10}}$,
it is still possible to activate $c_1$ in $\PH'$
by visiting the following sequence of states:
\begin{align*}
  & \PHstate{a_1, b_0, c_0, ab_{10}} \concat
    \PHstate{a_0, b_0, c_0, ab_{10}} \concat
    \PHstate{a_0, b_1, c_0, ab_{10}} \concat \\
  & \qquad\qquad
    \PHstate{a_0, b_1, c_0, ab_{11}} \concat
    \PHstate{a_0, b_1, c_1, ab_{11}}
\end{align*}
This is due to the fact that after playing the action
$\PHhit{a_1}{a_1}{a_0}$,
the absence of priorities does not constrain the cooperative sort to be updated.
We clearly see in this example that without priorities,
the cooperative sort $ab$ only represents a combination of the past
states of $a$ and $b$, without any notion of simultaneity.

Adding classes of priorities allows to remove this undesired behaviour,
especially by prioritizing all actions updating the cooperative sorts.
Indeed, the previous behaviour is not possible in the original model $\PH$,
where the actions of highest priority ($\PHh^{(1)}$)
force the update of the cooperative sort before any other action of
low priority ($\PHh^{(2)}$) can be played.
The closest possible sequence of states is the following,
where $ab_{11}$ can never be activated:
\begin{align*}
  & \PHstate{a_1, b_0, c_0, ab_{10}} \concat
    \PHstate{a_0, b_0, c_0, ab_{10}} \concat
    \PHstate{a_0, b_0, c_0, ab_{00}} \\
  & \qquad\qquad
    \PHstate{a_0, b_1, c_0, ab_{00}} \concat
    \PHstate{a_0, b_1, c_0, ab_{01}}
\end{align*}

We conclude that the addition of priorities,
by allowing to represent accurate Boolean gates in the form of cooperative sorts,
thus allows to reach the same expressivity as Asynchronous Discrete Networks
(see \pref{sec:dn}).
Therefore, the next sections \todo{nouvelle section ?}
formalize this use of the highest class of priority
to force the update of cooperative sorts.
The aim of the rest of this paper is to allow the static analysis
of the dynamics on PH with priorities,
and for this we rely on models involving
such prioritised actions to update cooperative sorts.
However, despite focusing on a sub-class of PH models,
we show in \pref{ssec:flattening}
that this sub-class is in fact as expressive as the class
of all PH with classes of priorities.



\subsection{Conditions}
\label{ssec:hypothesis}

In the scope of this paper, we focus on a specific class of PH models
in which actions of $\PHh^{(1)}$, called \emph{prioritised actions},
are only used to model cooperations in the form of strict logic gates.
\modMF{%
For this, we formally define in this section the notion of well-formed cooperative sorts,
and we require that prioritized actions can only be used to update them.
This particular form in the actions allows to apply the static analysis
developed in \pref{sec:sa}, which is not directly possible in the general case.
However, we note here that this class of models is not less expressive than
the general PH framework as given in \pref{def:ph};
indeed, the translation given in \pref{ssec:flattening} proves that
this class of models can reproduce any desired dynamics of the general case.
}%

We consider in this section a PH model with $k$ classes of priorities: $\PH = (\PHs; \PHl; \PHa^{\langle k \rangle})$, with $k \in \sNN$
and we define here the restrictions that lead to this class of models.

\pref{cr:bounded} states that the dynamics of the studied model $\PH$ contain no infinite sequence of prioritised actions.
As these actions can be considered as non-biological and therefore instantaneous, we thus prevent the existence of any Zeno-like behaviour
which would allow the play of an infinite sequence of prioritised actions in “zero time”.
As a consequence, the set $\restriction{\Sce}{1}$,
\modMF{
that is, the set of scenarios playable only with actions of $\PHh^{(1)}$,
(see \pref{def:restriction}),
}
is finite.

\begin{condition}[Bounded termination]
\label{cr:bounded}
  The dynamics of $\restriction{\PH}{1}$ contains no cycles:
  $\exists N \in \sN, \forall s \in \PHl, \forall \delta \in \restriction{\Sce}{1}(s), |\delta| \leq N$.
\end{condition}

\modMF{%
\begin{example}
  The model of \pref{fig:ph-livelock} respects \pref{cr:bounded}:
  indeed, whatever are the active processes of $a$ and $b$,
  it is never possible to indefinitely play actions of $\PHh^{(1)}$.
\end{example}
}%

Given a sort $a\in\Sigma$ and a state $s\in \PHl$, 
we denote by $\pfp_s(a)$ (\pref{def:pfp}) the processes of sort $a$ that can be present after
playing all actions of priority $1$ in $s$,
which is always defined because of \pref{cr:bounded}.
\modMF{%
This definition will be useful for cooperative sorts,
in order to determine their local stable state in every configuration,
and thus which configuration each of their processes represent.
}%

\begin{definition}[$\pfp : \PHl \times \PHs \rightarrow \powerset(\PHproc)$]
\label{def:pfp}
  For all $s \in \PHl$ and $a \in \PHs$,
  \begin{align*}
    \pfp_s(a) = \{ \get{(s\play\delta)}{a} \in \PHl_a &\mid \delta \in \restriction{\Sce}{1}(s)
					\wedge\nexists h\in\PHh^{(1)}, (\delta; h) \in\restriction{\Sce}{1}(s) \}
  \end{align*}
\end{definition}

\modMF{%
\begin{example}
  In the model of \pref{fig:ph-livelock}, we have for instance:
  $\pfp_{s}(ab) = \{ab_{10}\}$ for the state $s = \PHstate{a_1, b_0, ab_{00}, z_0}$.
  Indeed, starting from $\PHstate{a_1, b_0, ab_{00}, z_0}$,
  it is only possible to play the action $\PHfrappe{a_1}{ab_{00}}{ab_{10}} \in \PH^{(1)}$,
  and no other action in $\PH^{(1)}$ is playable in the resulting state
  $\PHstate{a_1, b_0, ab_{10}, z_0}$.
  Thus, process $ab_{10}$ can be used to represent the configuration $\PHstate{a_1, b_0}$.
\end{example}
}%

Our second condition (\pref{cr:compcs}) imposes that any sort in the Process
Hitting is either a well-formed component (\pref{def:component}) or a well-formed cooperative sort
(\pref{def:cs}).
The former is a sort that is hit only by actions with priority greater than or equal to $2$,
\modMF{%
and thus stands for a regular biological component.
}%
The latter is a sort that is hit only by actions with priority $1$ and which
always converges to the same process with respect to the state of the components
that have an influence on it (\pref{def:conn}).
\modMF{%
Indeed, we require that each process of a cooperative sort corresponds
to exactly one configuration of the sorts it represents.
}%
% We denote by $\components\subseteq\Sigma$ the set of well-formed components,
% and $\cs\subseteq\Sigma$ the set of well-formed cooperative sorts.
\modMF{
We note that some non-determinism may be involved during the update of
cooperative sorts.
However, this non-determinism is only confined to the choice of the order of the
actions played or the update of one cooperative sort before another.
Thus, there is no non-determinism in the focal process of each cooperative sort,
that is, from the point of view of lower-priority actions.
}

\begin{definition}[Well-formed component ($\components$)]
\label{def:component}
  A sort $a \in \PHs$ is a \emph{well-formed component} if and only if:
    $\forall h \in \PHh, \PHsort(\PHtarget(h)) = a \Rightarrow \prio(h) \geq 2$.
  We denote by $\components$ the set of all well-formed components.
\end{definition}

\begin{definition}[Components influence $(\compin: \PHs\to\powerset(\components))$]
\label{def:conn}
  Given a sort $a$, $\compin(a)\DEF \conn(a)\cap\components$ where
  $\conn(a) \subseteq \PHs$ is the smallest set of sorts satisfying the following properties:
  \begin{align*}
    a\in\conn(a) &
    \\
    \forall h\in  \PHh^{(1)},
      \Sigma(\target{h})\in\conn(a) & \Longrightarrow \Sigma(\hitter{h})\in\conn(a)
  \end{align*}
\end{definition}

\modMF{%
\begin{example}
  For the PH model of \pref{fig:ph-livelock}, we have:
  \begin{align*}
    \conn(a) &= \{a\} &
    \conn(ab) &= \{a, b, ab\} \\
    \conn(b) &= \{b\} &
    \conn(z) &= \{z\}
  \end{align*}
  Therefore, $\compin(ab) = \{a, b\}$.
\end{example}
}%

\begin{definition}[Well-formed cooperative sorts ($\cs$)]
\label{def:cs}
  A sort $a \in \PHs$ is a \emph{well-formed cooperative} sort if and only if:
  \begin{enumerate}[(i)]
    \item $\forall h\in\PHh, \Sigma(\PHtarget(h))=a \Longrightarrow \prio(h) = 1$;
    \item\label{csai} $\forall s\in\PHl, \card{\pfp_s(a)} = 1$;
    \item\label{css} $\forall a_i \in \PHl_a, \exists s \in \PHl, a_i\in\pfp_s(a)$;
    \item $\forall \sigma\in\PHsubl[\PHl]_{\compin(a)},
      \forall s,s'\in\PHl,
      \sigma\subseteq s\wedge \sigma\subseteq s'\Rightarrow 
      \pfp_s(a) = \pfp_{s'}(a)$\enspace.
  \end{enumerate}
  We denote by $\cs$ the set of all well-formed cooperative sorts.
\end{definition}

\begin{condition}[Components \& cooperative sorts partition]
\label{cr:compcs}
  $$\PHs = \components \cup \cs \wedge \components \cap \cs = \emptyset$$
\end{condition}

\modMF{%
\begin{example}
  The sorts $a$, $b$ and $z$ of \pref{fig:ph-livelock} are well-formed components
  regarding \pref{def:component}.
  Furthermore, $ab$ of is a well-formed cooperative sort,
  with respect to \pref{def:cs}, which models cooperation between $a$ and $b$
  because $\compin(ab) = \{a, b\}$ and:
  \begin{enumerate}[(i)]
    \item it is only hit by actions in $\PH^{(1)}$;
    \item in every state, it has exactly one focal process;
    \item each of its processes corresponds to at least one state of the model;
    \item its focal processes only depend on $a$ and $b$.
  \end{enumerate}
  Therefore, we obviously have:
  $\components = \{a, b, z\}$ and $\cs = \{ab\}$,
  which respects \pref{cr:compcs}.
\end{example}
}


We note that any sort $a$ that is hit by no action (\ie $\forall h \in \PHh, \PHsort(\target{h}) \neq a$)
does comply with both definitions of well-formed components and well-formed cooperative sorts.
In this case, such a sort can be arbitrarily affected to either set.
Such a case may occur for “input” sorts (whose state is initially fixed and does not evolve)
or some special cases of cooperative sorts produced by the translations of \pref{sec:encodings}.
In the following we simply write “component” (\resp “cooperative sort”) instead of “well-formed component” (\resp “well-formed cooperative sort”).

Because of \pref{def:cs}(\ref{csai}), we denote in the following: $\pfp_s(a) = a_i$.
Furthermore, because of \pref{def:cs}(\ref{css}), we denote by $\csState(a_i)$ the set of sub-states modelled by the process $a_i$ of any cooperative sort $a$ (\pref{def:csState}),
that is, the sub-states predecessors for which the focal process of $a$ is $a_i$.
\begin{definition}[$\csState : \PHproc \rightarrow \powerset(\PHproc)$]
\label{def:csState}
  If $a \in \cs$ and $a_i \in \PHl_a$, 
  $$\csState(a_i) \DEF \{ \toset{ps} \mid ps\in\PHsubl[\PHl]_{\compin(a)} \wedge
  							\exists s\in\PHl, (ps\subseteq s\wedge \pfp_s(a)=a_i)
							\}$$
\end{definition}

\modMF{%
\begin{example}
  We have, for the cooperative sort $ab$ of \pref{fig:ph-livelock}
  $\csState(ab_{10}) = \{\{a_1, b_0\}\}$,
  which is a direct consequence of the fact that $\pfp_{s}(ab) = \{ab_{10}\}$
  with $s = \PHstate{a_1, b_0, ab_{00}, z_0}$.
\end{example}
}%



\subsection{Consequences of the Conditions}

In this subsection, we give several general lemmas that can be derived from the restrictions of \pref{ssec:hypothesis},
in the special case of a PH model with $2$ classes of priorities $\PH = (\PHs; \PHl; \PHa^{\langle 2 \rangle})$.
These results will help building the static analysis of \pref{sec:sa}.

We first denote by $\update(s)$ (\pref{def:update}) the state equivalent to $s$
\modMF{%
but in which the active process of all cooperative sorts is changed to the process
that represent the current state of the components.
}%
This state is unique due to the properties of $\pfp$ given in the previous subsection.
Then, \pref{lem:update} states that from any state, there exists a scenario updating the cooperative sorts of this state,
\modMF{%
thus allowing to end in $\update(s)$.
}%

\begin{definition}[$\update : \PHl \rightarrow \PHl$]
\label{def:update}
  For all $s \in \PHl$, we define:
  \begin{align*}
    \update(s) = s \Cap \{ \pfp_{s}(a) \mid a \in \cs \} \enspace.
  \end{align*}
\end{definition}
%
\begin{lemma}
\label{lem:update}
  $\forall s \in \PHl, \exists \delta \in \restriction{\Sce}{1}(s), s \PHplay \delta = \update(s)$
\end{lemma}
%
\begin{proof}
  Let $a$ be a cooperative sort so that $\PHget{s}{a} \neq \pfp_s(a)$.
  Given the definition of $\pfp_s(a)$, there exists a scenario $\delta$ updating $a$ in $s$ so that
  $\forall \delta' \in \restriction{\Sce}{1}(s \PHplay \delta)$, $\PHget{(s \PHplay \delta \PHplay \delta')}{a} = \pfp_s(a)$.
\end{proof}

\modMF{%
\begin{example}
  Regarding the model of \pref{fig:ph-livelock}, we have:
  $\update(s) = \PHstate{a_1, b_0, ab_{10}, z_0}$,
  with $s = \PHstate{a_1, b_0, ab_{00}, z_0}$.
  We note that in $\update(s)$, only the active process of the cooperative sort
  $ab$ is changed
  to the process representing the presence of both $a_1$ and $b_0$.
  Furthermore, \pref{lem:update} states that there exists a scenario $\delta$
  of actions in $\PH^{(1)}$
  so that $s \play \delta = \update(s)$.
  Indeed, $\delta = (\PHfrappe{a_1}{ab_{00}}{ab_{10}})$ is such a scenario,
  here containing only one action.
\end{example}
}%

\pref{lem:hcompcomp} states that for a given state $s$, and for any action $h = \PHhit{a_i}{b_j}{b_k} \in \PHh^{(2)}$ where $a$ and $b$ are components,
if $\PHget{s}{a} = a_i$ and $\PHget{s}{b} = b_j$, then
$h$ can always be played after a series of prioritised hits (and these hits do not prevent it to be fired).
\pref{lem:hcscomp} states the same if $a$ is a cooperative sort, under the condition that $a$ is updated in $s$.
In other words, prioritized actions cannot prevent an action of priority $2$
to be eventually played; they only ``delay'' it,
provided only that the hitter of this action
matches the active process of its sort in the updated state.
\modMF{
We note that in some cases, namely, when all cooperative sorts are already updated,
this scenario of prioritized actions is empty
and the action can be immediately played.
We note also that there may exist several scenarios that fit into these lemmas;
however, the choice of one scenario over another has no consequence due to
the unicity of the focal process of the cooperative sorts
(see (\ref{csai}) in \pref{def:cs}).
}

\begin{lemma}
\label{lem:hcompcomp}
  $\forall s \in \PHl, \forall a,b \in \components, \forall h = \PHhit{a_i}{b_j}{b_k} \in \PHh,$\\
  $(\PHget{s}{a} = a_i \wedge \PHget{s}{b} = b_j) \Rightarrow (\exists \delta \in \restriction{\Sce}{1}(s),
  (s \PHplay \delta) \PHPtrans (s \PHplay \delta \PHplay h))$
\end{lemma}
%
\begin{proof}
  From \pref{lem:update}, there exists a scenario $\delta$ with: $(s \PHplay \delta) = \update(s)$.
  As $a,b \in \components$, $a_i \in (s \PHplay \delta)$ and $b_j \in (s \PHplay \delta)$.
  Finally, by definition of $\update(s)$, no prioritised action can be played in $(s \PHplay \delta)$, thus $h$ can be played in $(s \PHplay \delta)$.
\end{proof}
%
\begin{lemma}
\label{lem:hcscomp}
  $\forall s \in \PHl, \forall h = \PHhit{a_i}{b_j}{b_k} \in \PHh, a \in \cs, b \in \components,$\\
  $(\PHget{s}{a} = a_i \wedge \PHget{s}{b} = b_j \wedge \pfp_s(a) = a_i) \Rightarrow (\exists \delta \in \restriction{\Sce}{1}(s),
  (s \PHplay \delta) \PHPtrans (s \PHplay \delta \PHplay h))$
\end{lemma}
\begin{proof}
  Similar to the proof of \pref{lem:hcompcomp};
  as $a_i \in \pfp_s(a)$, $a_i \in (s \PHplay \delta)$.
\end{proof}

\modMF{%
\begin{example}
  Considering the model of \pref{fig:ph-livelock} and starting from the state
  $s = \PHstate{a_1, b_0, ab_{00}, z_0}$,
  \pref{lem:hcompcomp} states that the action $h = \PHfrappe{a_1}{a_1}{a_0}$
  is playable after a scenario $\delta$ of prioritized actions.
  In this case, $h$ is indeed playable in the state
  $s \play \delta = \PHstate{a_1, b_0, ab_{10}, z_0}$,
  where $\delta = (\PHfrappe{a_1}{ab_{00}}{ab_{10}}) \in \restriction{\Sce}{1}(s)$.
\end{example}
}%

\medskip

\modMF{%
This section aimed at defining the PH framework in a general fashion,
and focused on a more specific class of PH models
that will be used later in this paper.
We gave a definition of the notion of cooperative sort,
a special sort with no explicit biological existence or meaning,
which is used to model the joint action of several components
on another one.
Using this definition, we defined a specific class of PH models
that assign a specific role to the set of actions
of highest priority:
these actions can only be used to update cooperative sorts to ensure that
they are equivalent to Boolean gates.
Technically, the static analysis developed in the next section has to be applied
to PH models belonging to this specific class with only two sets of priorities.
However, it has to be noted that this class is in fact as expressive as
the class of all possible PH models with priorities (see \pref{ssec:flattening}),
and that the static analysis developed is thus not limited to this sub-class of models.
}%
