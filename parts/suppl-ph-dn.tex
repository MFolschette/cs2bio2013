\section{Proof of the Weak Bisimulation of Asynchronous Discrete Networks (\pref{sec:dn})}
\label{suppl:proofbisimADN}

\newproof{proofbisimADN}{Proof of \pref{th:bisimDN}}
\begin{proofbisimADN}
(\ref{adn2ph}) From \pref{def:DN}, $x\DNtrans x'\Rightarrow \exists i\in\segm{1}{n},
f^i(x)=\get{x'}{i} \wedge \forall j\in\segm{1}{n},i\neq j, \get{x}{j}=\get{x'}{j}$.
Let us assume (without loss of generality) that $f^i(x)=k'$, $\get{x}{i}=k$ and
$\varsigma\in\underset{j\in\DNdep(f^i)}{\times} \PHl_{a^j}$ such that
$\forall j\in\DNdep(f^i), \get{\varsigma}{j}=a^j_{\get{x}{j}}$.
From \pref{def:DN2PH}, $h=\hit{f^i_\varsigma}{a^i_k}{a^i_{k'}}\in\Hits^{(2)}$.
From the definition of $\encode x$,
$a^i_k\in \encode x$ and $f^i_\varsigma\in \encode x$;
moreover, as there is no action in $\Hits^{(1)}$ applicable in $\encode x$,
$h$ is applicable in $\encode x$:
$\encode x\PHPtrans \encode x\play h$.
In $\encode x\play h$, the only applicable actions of priority $1$ are those having
$a^i_{k'}$ as hitter and hitting cooperative sorts, giving a finite number of transitions towards
$\encode{x'}$.

(\ref{ph2adn}) $s\PHPtrans s'$ only if there exists an action $h$ applicable in $s$ such that
$s\play h=s'$.
If $\prio(h)=1$, then, by definition of $\Hits^{(1)}$, 
$\decode s=\decode {s'}$.
If $\prio(h)=2$, then $\forall i\in\segm{1}{n}$,
if $\get{s}{f^i} = f^i_\varsigma$, then, $\forall j\in\DNdep(f^i),
\get{\varsigma}{a^j}=\get{s}{a^j}$.
Let $i\in\segm{1}{n}$ such that $\get{s}{a^i}\neq\get{s'}{a^i}$ ($i$ is unique for this
transition).
By \pref{def:DN2PH}, if $\get{s'}{a^i}=a^i_{k'}$, necessarily $f^i(\decode s)=k'$, hence
$\decode s\DNtrans \decode{s'}$.
\end{proofbisimADN}
