% vi:spell spelllang=en:
\section{Introduction}
\label{sec:intro}

Discrete modelling frameworks for biological networks is an active research field where formal methods have proven that they were very powerful.
Such a work started in the seventies.
It was later enriched in many directions and widely used to elucidate many biological questions.
Among these questions, a major one is to understand precisely how biological systems evolve and behave; why and how they change their usual behaviours, etc.
These questions are strongly linked to the (possible or inevitable) reachability of some states.
The ultimate goal is to discover how it could be possible to prevent biological systems from reaching some pathological conditions.
\todo{Ajouter des références}

Of course, such formal models on which analyses are performed are abstract representations of the actual studied systems.
They are associated with parameters that have to be synthesised %so as to be as much as possible fitting with the real systems having some observed behaviors.
to give the most faithful representation of the real systems with their observed behaviours.
As a matter of fact, the abstractions we get are more or less rough or accurate.
Prevalent formal frameworks for such modelling activities are state-transition systems or process algebras. % Petri nets
We developed a quite similar framework named the Process Hitting~\cite{PMR10-TCSB},
consisting in a restriction of these frameworks where the evolution of a component is determined by the state of at most one other component that does not evolve.
In a sense, these kind of actions are of the form $X + Y \rightarrow X + Y'$ where $X$ behaves like a catalyst molecule that “hits” another molecule $Y$ and changes it into $Y'$, without being changed itself.
Assuming catalysts are always available, this can represent any biochemical system made of monomolecular reactions, and can also represent catalytic networks such as metabolic networks.
%, with the aim of avoiding to build the whole state space in order to be able to tackle very large systems
Our motivation behind this framework was to design a model and analysis techniques adapted to biological modelling.
These analyses avoid to build the whole state space, which allows to tackle very large systems (that would have led to a huge number of states, hopelessly too huge to be analysed).
They are based on the fact that most biological models have few levels of expression per component:
in Boolean networks~\cite{kauffman69,Thomas73} there are only two levels per component, and in their multivalued equivalent~\cite{deJong02}, components rarely have more than four levels.

Besides, one further objective of our work is now %to be more accurate in the description of the studied systems.
to improve the accuracy of the description of the studied systems dynamics.
The idea for this is to introduce timing features into models:
we are interested in taking into account some knowledge about the relative length of some phenomena as it is a way to refute some models (or parameters) that are inconsistent with the observed dynamic behaviours.
In this paper, we are dealing with these timing properties through priorities,
that are based on the simple founding idea that actions with higher priority have to be processed before the ones with lower priority.
Furthermore, due to the Process Hitting framework restrictions, multimolecular reactions were previously not immediately available, but one could simulate them with an encoding called “cooperative sort”.
That encoding however introduces extra reactions,
that produce a temporal shift between the presence of the reactants
and the playability of the reaction.
This is where the priorities become useful, if not necessary:
the extra reactions can for example be given “infinite speed” (highest priority) so that they do not affect the behaviour of “normal” (lower priority) reactions, including the multimolecular ones.
Another approach consists in considering another class of models,
that we call Asynchronous Automata Network,
and that allows to naturally model these cooperations
by defining several requisites for a reaction.
Furthermore, such automata networks are still compatible with the notion of priority,
that can also be used to model different reaction rates in the model.

Until now, such a priority scheduling of the actions was not studied extensively in the different formal modelling frameworks dedicated to systems biology.
Nevertheless, such an attempt has been carried out for Petri nets by F.~Bause~\cite{Bause97},
and the concept of priority relations among the transitions of a network has also more recently been introduced by A.~K.~Wagler \textit{et~al.}~\cite{waw,WaglerW12} in order to allow modelling deterministic systems for biological applications.
The concept of priority is much straightforward in the approach of process algebras as it was shown by R.~Cleaveland and M.~Hennessy in~\cite{Cleaveland199058,Cleaveland99prioritiesin} and their abstractions and equivalences were studied in~\cite{Cleaveland:2007:PAP:1282576.1282847}.
It was later extended for applications in the field of systems biology by M.~John \textit{et~al.}~\cite{jlnu2010}.

\subsection*{Contributions}
Since our formalisms (the Asynchronous Automata Network and the Process Hitting)
can be considered as a subset of Calculus of Communicating Systems,
or Synchronous Automata Networks,
%is inspired from the $\pi$-calculus,
our work is related to such semantic ramifications of extending traditional process algebras with the concept of priority that allows for some transitions to be given precedence over others.
The concept is derived in two directions: dynamic versus static, the difference being naturally that the former one refers to a semantics where priority values may change during execution according to some evolution rules.
In our work, actions exhibit a static priority structure.
The actions modelling biological processes are divided in several classes of priorities which stand for the different reaction rates or relative preemptions.
Furthermore, the multiple causality of the actions in Asynchronous Automata Networks
allow to abstract a highest class of priority that would be used
to model particular nonbiological actions
that encode Boolean functions inside the model.

This paper is an extension of~\cite{FPMR13-CS2Bio} where we
introduced a new extension to the semantics of Process Hitting
by partitioning actions into classes of priorities.
One of our objectives was to reach an accurate representation of cooperating components in the model, that was not fulfilled with the initial semantics.
Although in~\cite{FPMR13-CS2Bio} this was achieved with a particular class of high priority,
the work presented in this paper, however, focuses on Asynchronous Automata Networks.
The simple version of this formalism (without classes of priorities)
is defined in \pref{sec:ph}
and consists of an extension of Process Hitting with classes of priorities
that allows to simplify the formalisation with the same benefits.
In \pref{sec:sa},
we develop an efficient under-approximation of the reachability
of several local states of components,
thus allowing to compute efficient static analysis of the dynamics.
This local reachability under-approximation can be also easily extended to study the reachability of a global or partial state.
Furthermore, we provide a new extended under-approximation method to refine this method
in the case of a successive reachability property.

Furthermore,
we define in \pref{sec:encodings} the notion of Asynchronous Automata Networks
with classes of priorities,
that consists of a convenient way to introduce time parameters
into the models, under the form of preemptions.
We also provide in \pref{sec:encodings} a “flattening”
of any Asynchronous Automata Networks with classes of priorities
into a model without classes of priorities,
thus extending the range of applicability of the static analysis.
We also give in the same section a formal encoding of Asynchronous Discrete Networks,
that is, the multivalued version of Asynchronous Boolean Networks.
These models,
which encompass variables and evolution functions with a limited number of discrete values,
are also called “Logical Networks” in~\cite{Thomas95,deJong02}.
This encoding thus allows to efficiently study this kind of widespread models.

In \pref{sec:example}, we give two examples of the use of our method.
The first consists of a detailed application of the analysis of a biological network expressed as
an ANN with 2 classes of priorities,
which requires to be encoded into a simple ANN (without priorities).
The second consists of applying our method
on a large-scale biological model containing 94 components;
the under-approximation turned out to be conclusive in all cases and results were computed in hundredths of seconds,
thus overtaking the efficiency of usual model-checkers.

\modMF{
The main addition in this paper compared to~\cite{FPMR13-CS2Bio} is
the representation under the form of ANNs, that allow to simplify the formalisation.
Furthermore, we give new tools to improve the applicability of our method:
\pref{th:bisimPHP} states that any ANN with any number of classes of priorities
can be represented into a simple ANN (or, equivalently, with only one class of priority),
thus extending the scope of our method to any possible Process Hitting model;
\pref{thm:ordered-ua} gives a method to refine successive reachability,
and obtain more conclusive results in the case of several successive objectives;
\pref{def:concret} allows to extract a concrete scenario when our static analysis method
is conclusive.
}



\subsection*{Notations}
\label{notations}
%We denote: $\segm{a}{b} = \{ a, a+1, \dots, b-1, b \}$.

\paragraph*{Sets}
If $A$ is a finite set,
$|A|$ is the cardinality of $A$
and $\powerset(A)$ is the power set of $A$.
$\sN$ is the set of natural numbers,
$\sN^* = \sN \setminus \{ 0 \}$ is the set of positive natural numbers,
$\sNN = \sN \setminus \{ 0, 1 \}$ is the set of natural numbers strictly greater than 1,
and $\segm{x}{y} = \{ x, x+1, \dots, y-1, y \}$ is the set of natural numbers from $x$ to $y$ included.
The Cartesian product is denoted $\times$;
if $z$ is a tuple of $n$ components, $\toset{z}$ denotes the corresponding set:
$\toset{z} = \{z_1, \cdots, z_n\}$.

\paragraph*{Sequences}
We denote by $\emptyseq$ the empty sequence.
If $n \in \sN$ and
$x = (x_i)_{i \in \segm{1}{n}}$ is a sequence of elements indexed by $i \in \segm{1}{n}$,
%we denote %$|x| = (b-a)+1$ the size of this sequence and
then $\indexes{x} = \segm{1}{n}$ is the set of indexes of this sequence.
Furthermore, if $a, b \in \indexes{x}$ with $a \leq b$,
then $x_{a..b} = (x_i)_{i \in \segm{a}{b}}$ is a subsequence of $x$.
%from element $n$ to element $m$ inclusive.
Finally, if $x$ is a sequence, $\toset{x}$ also denotes the corresponding set:
$\toset{x} = \{x_1, \cdots, x_{\card{x}}\}$.

\paragraph*{Functions and least fixed point}
If $A$ and $B$ are sets,
$f : A \rightarrow B$ denotes an application $f$ that maps the elements of $A$ to elements of $B$.
Furthermore, if $f$ is a monotonically increasing and bounded function, then
$\lfp{x_0}{x}{x'}$ is the least fixed point of the function $x \mapsto x'$ which is greater than $x_0$.
