% vim:set spell spelllang=en:

\section{Biological Examples}\label{sec:example}

\subsection{\ldots (metazoan)}
\todo{Maxime}

\subsection{Large-scale Application}

In order to support the scalability and applicability of our under-approximation of reachability, we
apply our new approach for the analysis of large-scale model of the T-cell receptor (TCR)
signalling pathway \cite{tcrsig94}.
This model gathers 94 interacting components and is specified as a Boolean network.
The under-approximation presented in this paper has been implemented in the existing Pint
software\footnote{Pint is freely available at \url{http://process.hitting.free.fr}.}.

The Boolean model has been automatically encoded into a Process Hitting with 2 classes of priority%
\footnote{Model and scripts are available at
\url{http://www.irccyn.ec-nantes.fr/~folschet/underapprox-tcrsig94.zip}.}.
Then, we verified the reachability for the independent activation of 4 outputs of the signalling
cascade (SRE, AP1, NFkB, NFAT) from all possible input combinations (CD45, CD28, TCRlib) using our
new reachability under-approximation (answering either \emph{yes} or \emph{inconclusive}) and a 
previously defined reachability over-approximation \cite{PMR12-MSCS} (answering either \emph{no} or
\emph{inconclusive}).
All result in conclusive decisions, and the under-approximation has been satisfied in 12 cases (over
32) proving the satisfiability of the concerned reachability property in the encoded Boolean network
(and non-satisfiability in the other cases).

Computations times are in the order of a few hundredths of a second on a 2.4GHz processor with 2GB
of RAM.
To give a comparison, we did the same experiments with a standard symbolic model-checker, libDDD
\cite{libddd}, known for its good performances, the input model being the Boolean network expressed
as a Petri net.
However, due to the large scale of the model, the program runs out of memory for all the experiments.

While ensuring a low complexity for the analysis of reachability in Boolean and discrete networks, our
under-approximation method reveals to be conclusive in numerous cases when applied to real
large-scale biological models, which were not tractable with exact model-checking.

