% vim:set spell spelllang=en:

\section{Biological Examples}\label{sec:example}

\modMF{
This section aims at giving application examples of the static analysis method
that we developed in \pref{sec:sa}.
In \pref{ssec:ex-metazoan}, we apply our method to a small model of the
metazoan segmentation process,
and demonstrate how the classes of priorities help in the modelling process,
and how the flattening of \pref{ssec:flattening}
can be used in such a case.
In \pref{ssec:ex-tcrsig}, we apply our method to a large-scale model
of the T-cell receptor signalling pathway
in order to show the scalability of our method.
}



\subsection{Flattening of a model: Metazoan segmentation}
\label{ssec:ex-metazoan}

\modMF{
We give in this section a detailed example of the use of classes of priorities
in order to model a system with temporal constraints.
This model also allows us to give a detailed example of the application of the sequential
under-approximation proposed in \pref{ssec:ordered-ua},
which consists of several applications of the methods developed in
\pref{ssec:ua}.
For this, we first have to use the flattening method presented in \pref{ssec:flattening}
because the considered model contains 2 classes of priorities.
}

We consider in the following a model of metazoan segmentation
inspired from a first translation to Process Hitting given in~\cite{PMR10-TCSB}.
This model was originally established in silico in~\cite{MSB:MSB4100192}
in a differential equations framework.
It is composed of a wavefront gene $f$ that activates the gap-gene $a$ whose products are responsible for stripes formation.
Gene $f$ also activates a gene $c$ whose products represses the gene $a$.
The auto-inhibition of $c$ generalises a chain of repressors on $a$.
The auto-inhibition of $f$, which normally terminates
the stripes formation in the original model,
has been removed in order to focus on the stationary dynamics of the model.

The actions of the original model make $2$ classes of priorities, as represented in \pref{fig:metazoan-php}:
\modMF{%
\begin{align*}
  \ov{\PHh}^{(1)} = \{ \quad
    & \PHfrappes{c_1}{a_1}{a_0} \quad , \quad
    \PHfrappes{f_1, c_0}{a_0}{a_1}
  \quad \} \\
  \ov{\PHh}^{(2)} = \{ \quad
    & \PHfrappem{c_1}{c_0} \quad , \quad
    \PHfrappes{f_1}{c_0}{c_1}
  \quad \} \\
\end{align*}
Indeed, without this use of priorities,
some unwanted behaviours emerge, allowing the formation of irregular stripes.
In order to fix this, a high priority is affected to the actions hitting $a$,
a low priority to the actions hitting $c$,
in order to model the fact that the switching of $c$ has to be immediately followed by
a switching of $a$.
This forces the the switches of genes $a$ and $c$
to alternate in order not to miss a stripe;
$a$ and $c$ thus have intertwined oscillations.
We can thus consider that these two classes of priorities
are derived from known reaction relative rates,
the evolution of the clock $c$ being slower and regular,
while the evolution of $a$ has to follow the changes of $c$.
}%

\begin{figure}[p]
  \centering
  \scalebox{1}{
  \begin{tikzpicture}[aan]
    \TSort{(0,4)}{c}{2}{l}
    \TSort{(1,0)}{f}{2}{l}
    \TSort{(5,4)}{a}{2}{r}
    
    \TAction{f_1}{c_0.west}{c_1.south west}{bend left=30, in=90}{left}
    \TActionPlur{}{c_1.west}{c_0.north west}{}{-1,5.5}{right}
    
    \TAction{c_1}{a_1.west}{a_0.north west}{prio}{right}
    \TActionPlur{f_1, c_0}{a_0.west}{a_1.south west}{prio}{2.5,2.5}{left}
    
    \TState{f_1, a_0, c_0}
  \end{tikzpicture}
  }
  \caption{
  \label{fig:metazoan-php}
    An example of AAN$2$
    modelling the process of metazoan segmentation.
    Component $a$ models the pigment production, and is influenced by
    component $c$ that has the role of a clock,
    while $f$ represents the wavefront propagation.
    If component $a$ oscillates (that is, its active local state changes regularly)
    then regular stripes are created on the metazoan.
    Actions of $\ov{\PHh}^{(2)}$ (low priority) are represented in thin lines
    and actions of $\ov{\PHh}^{(1)}$ (high priority) are in thick lines.
    The greyed local states represent a possible initial state:
    $\PHstate{f_1, a_0, c_0}$.
  }
\end{figure}

\pref{fig:metazoan-ph} gives the flattening of this model,
\modMF{
that is, an AAN with the equivalent dynamics
(but only one class of priority).
Its actions are:
\begin{align*}
  \PHh = \{ \quad
    & \PHfrappes{c_1}{a_1}{a_0} \quad , \quad
    \PHfrappes{f_1, c_0}{a_0}{a_1} \quad , \\
    & \PHfrappes{a_0}{c_1}{c_0} \quad , \quad
    \PHfrappes{f_1, a_1}{c_0}{c_1}
  \quad \}
\end{align*}
We note that the two actions in $\ov{\PHh}^{(2)}$ have been replaced by
the equivalent actions
$\PHfrappes{a_0}{c_1}{c_0}$ and $\PHfrappes{f_1, a_1}{c_0}{c_1}$,
in order to model the preemption of the actions in $\ov{\PHh}^{(1)}$.
}%

\begin{figure}[p]
  \centering
  \scalebox{1}{
  \begin{tikzpicture}[aan]
    \TSort{(-3,4)}{c}{2}{l}
    \TSort{(0,1)}{f}{2}{l}
    \TSort{(3,4)}{a}{2}{r}
    
%    \TAction{f_1}{c_0.west}{c_1.south west}{bend left=30, in=90}{left}
    \TActionPlur{f_1, a_1}{c_0.south east}{c_1.south east}{}{-1.5,3.3}{right}
    \TAction{a_0.west}{c_1.east}{c_0.north east}{}{left}
    
    \TAction{c_1.north east}{a_1.north west}{a_0.north west}{}{right}
    \TActionPlur{f_1, c_0}{a_0.south west}{a_1.south west}{}{1.5,3.5}{left}
    
    \TState{f_1, a_0, c_0}
  \end{tikzpicture}
  }
  \caption{
  \label{fig:metazoan-ph}
    An example of AAN, which is the flattening of the AAN$2$ in \pref{fig:metazoan-php};
    in other words, this model has exactly the same dynamics as
    \pref{fig:metazoan-php}, but its actions make only one class of priority.
    The greyed local states represent the same initial state:
    $\PHstate{f_1, a_0, c_0}$.
  }
\end{figure}

\modMF{
Finally, the static analysis results presented in
\pref{ssec:ua} and \pref{ssec:ordered-ua} can be used to check if the model is functional,
\ie if gene $a$ can oscillate, thus leading to the formation of stripes.
Starting from context $\ctx = \PHstate{f_1, a_0, c_0}$,
we thus want to check the reachability of
$\w = \PHobj{a_0}{a_1} \PHconcat \PHobj{a_1}{a_0} \PHconcat \PHobj{a_0}{a_1}$.
Rather than checking this whole objective sequence with \pref{th:approxinf},
we break it using \pref{thm:ordered-ua}.
We first build the first saturated graph of local causality
based on the first objective, $\w_1 = \PHobj{a_0}{a_1}$, in the initial context $\ctx$.
This graph is depicted in \pref{fig:metazoan-sa}(left).
%and given \pref{tmh:approxinf}, we have:
We have especially:
\[\allprocs(\thisB{\w_1}{\ctx}) = \{ a_0, c_0, f_1 \} \enspace \text{and}\]
\[\lastprocs(\PHobj{a_0}{a_1}) = \{\{ a_1, c_0, f_1 \}\} \enspace.\]
Therefore, the second step consists of computing the saturated graph of local causality
of the objective $\w_2 = \PHobj{a_1}{a_0}$
in the context
\[\ctx' = \ctx \Cap \allprocs(\thisB{\w_1}{\ctx}) \Cap \lastprocs(\PHobj{a_0}{a_1}) = \PHstate{a_1, c_0, f_1} \enspace.\]
This second graph is given in \pref{fig:metazoan-sa}(middle).
Finally, because
\[\allprocs(\thisB{\w_2}{\ctx'}) = \{ a_1, c_0, c_1, f_1 \} \enspace \text{and}\]
\[\lastprocs(\PHobj{a_1}{a_0}) = \{\{ a_0, c_1 \}\} \enspace,\]
The last step is then checking
the objective $\w_3 = \PHobj{a_0}{a_1}$
in the context
\[\ctx'' = \ctx' \Cap \allprocs(\thisB{\w_2}{\ctx'}) \Cap \lastprocs(\PHobj{a_1}{a_0}) = \PHstate{a_0, c_1, f_1} \enspace.\]
This last graph is depicted in \pref{fig:metazoan-sa}(right).
In conclusion, by using \pref{thm:ordered-ua} on these saturated graphs of local causality,
we can conclude that the initial objective $\w$ is reachable,
and thus that the AAN of \pref{fig:metazoan-ph} is functional.
This result can be extended to the model of \pref{fig:metazoan-php}
because they have the same dynamics (given \pref{th:bisimPHP}).
}

% $\w = \PHobj{a_0}{a_1} \PHconcat \PHobj{c_0}{c_1} \PHconcat \PHobj{a_1}{a_0} \PHconcat \PHobj{c_1}{c_0}$.
% Rather than checking this whole objective sequence,
% we decompose it into the following ones:
% \begin{align*}
%   \ctx^1 &= \PHstate{f_1, a_0, c_0} \text{ and } P_1 = \PHobj{a_0}{a_1} &
%   \ctx^3 &= \PHstate{f_1, a_1, c_1} \text{ and } P_2 = \PHobj{a_1}{a_0} \\
%   \ctx^2 &= \PHstate{f_1, a_1, c_0} \text{ and } P_3 = \PHobj{c_0}{c_1} &
%   \ctx^4 &= \PHstate{f_1, a_0, c_1} \text{ and } P_4 = \PHobj{c_1}{c_0}
% \end{align*}
% Given \pref{th:approxinf}, we have: $\forall i \in \segm{1}{4}, \muconcr_{\ctx^i}(P_i) \neq \emptyset$.
% Finally, after checking that it can be applied for $i = 3$ down to $i = 1$,
% \pref{thm:ordered-ua} gives: $\uconcr(\w) \neq \emptyset$.
% Thus the model is functional.

\begin{figure}[tp]
  \centering
  \begin{tikzpicture}[aS]
    % STEP 1
    \node[Aobj] (a01) {$\PHobj{a_0}{a_1}$};
    \node[Asol,below of=a01] (a01s) {};
    \node[Assol,below of=a01s] (a01ss) {$\{ c_0, f_1 \}$};

    \node[Aproc,below right of=a01ss] (f1) {$f_1$};
    \node[Aobj,below of=f1] (f11) {$\PHobj{f_1}{f_1}$};
    \node[Asol,below of=f11] (f11s) {};
    \node[Assol,below of=f11s] (nf11s) {$\emptyset$};

    \node[Aproc,below left of=a01ss] (c0) {$c_0$};
    \node[Aobj,below of=c0] (c00) {$\PHobj{c_0}{c_0}$};
    \node[Asol,below of=c00] (c00s) {};
    \node[Assol,below of=c00s] (nc00s) {$\emptyset$};

    \path
    (a01) edge (a01s)
    (a01s) edge (a01ss)
    (a01ss) edge (f1) edge (c0)

    (f1) edge (f11)
    (f11) edge (f11s)
    (f11s) edge (nf11s)

    (c0) edge (c00)
    (c00) edge (c00s)
    (c00s) edge (nc00s)
    ;
    
    % STEP 2
    \node[Aobj,right of=a01,node distance=3.5cm] (a10) {$\PHobj{a_1}{a_0}$};
    \node[Asol,below of=a10] (a10s) {};
    \node[Assol,below of=a10s] (a10ss) {$\{ c_1 \}$};

    \node[Aproc,below of=a10ss] (c1) {$c_1$};
    \node[Aobj,below right of=c1] (c01) {$\PHobj{c_0}{c_1}$};
    \node[Asol,below of=c01] (c01s) {};
    \node[Assol,below of=c01s] (c01ss) {$\{ a_1, f_1 \}$};
    \node[Aobj,below left of=c1] (c11) {$\PHobj{c_1}{c_1}$};
    \node[Asol,below of=c11] (c11s) {};
    \node[Assol,below of=c11s] (nc11s) {$\emptyset$};

    \node[Aproc,below left of=c01ss] (a1) {$a_1$};
    \node[Aobj,below of=a1] (a11) {$\PHobj{a_1}{a_1}$};
    \node[Asol,below of=a11] (a11s) {};
    \node[Assol,below of=a11s] (na11s) {$\emptyset$};

    \node[Aproc,below right of=c01ss] (f1) {$f_1$};
    \node[Aobj,below of=f1] (f11) {$\PHobj{f_1}{f_1}$};
    \node[Asol,below of=f11] (f11s) {};
    \node[Assol,below of=f11s] (nf11s) {$\emptyset$};

    \path
    (a10) edge (a10s)
    (a10s) edge (a10ss)
    (a10ss) edge (c1)

    (c1) edge (c01) edge (c11)
    (c01) edge (c01s)
    (c01s) edge (c01ss)
    (c01ss) edge (f1) edge (a1)
    (c11) edge (c11s)

    (f1) edge (f11)
    (f11) edge (f11s)
    (f11s) edge (nf11s)

    (a1) edge (a11)
    (a11) edge (a11s)
    (a11s) edge (na11s)
    ;
    
    % STEP 3
    \node[Aobj,right of=a10,node distance=4cm] (ta01) {$\PHobj{a_0}{a_1}$};
    \node[Asol,below of=ta01] (ta01s) {};
    \node[Assol,below of=ta01s] (ta01ss) {$\{ c_0, f_1 \}$};

    \node[Aproc,below left of=ta01ss] (tf1) {$f_1$};
    \node[Aobj,below of=tf1] (tf11) {$\PHobj{f_1}{f_1}$};
    \node[Asol,below of=tf11] (tf11s) {};
    \node[Assol,below of=tf11s] (tnf11s) {$\emptyset$};

    \node[Aproc,below right of=ta01ss] (tc0) {$c_0$};
    \node[Aobj,below of=tc0] (tc10) {$\PHobj{c_1}{c_0}$};
    \node[Asol,below of=tc10] (tc10s) {};
    \node[Aobj,right of=tc10] (tc00) {$\PHobj{c_0}{c_0}$};
    \node[Asol,below of=tc00] (tc00s) {};
    \node[Assol,below of=tc00s] (tnc00s) {$\emptyset$};

    \node[Aproc,below of=tc10s] (ta0) {$a_0$};
    \node[Aobj,below of=ta0] (ta00) {$\PHobj{a_0}{a_0}$};
    \node[Asol,below of=ta00] (ta00s) {};
    \node[Assol,below of=ta00s] (tna00s) {$\emptyset$};

    \path
    (ta01) edge (ta01s)
    (ta01s) edge (ta01ss)
    (ta01ss) edge (tf1) edge (tc0)

    (tf1) edge (tf11)
    (tf11) edge (tf11s)
    (tf11s) edge (tnf11s)

    (tc0) edge (tc00)
    (tc00) edge (tc00s)
    (tc00s) edge (tnc00s)
    (tc0) edge (tc10)
    (tc10) edge (tc10s)

    (tc10s) edge (ta0)
    (ta0) edge (ta00)
    (ta00) edge (ta00s)
    (ta00s) edge (tna00s)
    ;
  \end{tikzpicture}
  \caption{%
  \label{fig:metazoan-sa}%
    The three successive saturated graphs of local causality
    of the AAN in \pref{fig:metazoan-ph}
    for the objective sequence
    $\w = \PHobj{a_0}{a_1} \concat \PHobj{a_1}{a_0} \concat \PHobj{a_0}{a_1}$
    and the initial context
    $\ctx = \PHstate{a_0, c_0, f_1}$.
    The (left) graph allows to check objective $\PHobj{a_0}{a_1}$
    from the initial context $\ctx$.
    The (middle) graph is for the objective $\PHobj{a_1}{a_0}$
    and the context $\PHstate{a_1, c_0, f_1}$.
    The (right) graph is for the last objective, $\PHobj{a_0}{a_1}$,
    and the context $\PHstate{a_0, c_1, f_1}$.
%     Rectangular nodes containing a single local state are elements in $\Proc$,
%     borderless nodes containing a couple of local states are elements in $\Obj$
%     and circle nodes are elements in $\Sol$.
%     \pref{th:approxinf} is inconclusive on this example as edge $(ab_{11}, \{ a_1, b_1 \}) \in \Proc \times \Sol$ (here represented with a double line) is not coherent (\pref{def:coherent}).
%     Indeed, $a_0 \in \Proc$ is a child of $\{ a_1, b_1 \}$, but $a_0 \neq a_1$ (and the same also stands for $b_0$).
  }
\end{figure}



\subsection{Large-scale Application: T-cell receptor}
\label{ssec:ex-tcrsig}

In order to support the scalability and applicability of our under-approximation of reachability, we
apply our new approach to the analysis of a large-scale model of the T-cell receptor (TCR)
signalling pathway \cite{tcrsig94}.
This model gathers 94 interacting components and is specified as a Boolean network.
The under-approximation presented in this paper has been implemented in the existing Pint
software\footnote{Pint is freely available at \url{http://loicpauleve.name/pint}.}.

The Boolean model has been automatically encoded into
a Process Hitting model with two classes of priorities,
and a dynamics equivalent to an AAN%
\footnote{Files are available at
\url{http://maxime.folschette.name/underapprox-tcrsig94.zip}.}
\modMF{
by using a translation equivalent to the one given in \pref{sec:dn}.
}
Then, we verified the reachability for the independent activation of 4 outputs of the signalling
cascade (SRE, AP1, NFkB, NFAT) for all possible input combinations (CD4, CD28, TCRlig) using our
new reachability under-approximation (answering either \emph{yes} or \emph{inconclusive}) and a 
previously defined reachability over-approximation \cite{PMR12-MSCS} (answering either \emph{no} or
\emph{inconclusive}).
All result in conclusive decisions, and the under-approximation has been satisfied in 12 cases (over
32) proving the satisfiability of the concerned reachability property in the encoded Boolean network
(and non-satisfiability in the other cases).

\modMF{
By analysing the detail of the results, one can find out that one of the outputs, NFkB, can never be activated.
In other words, FkB$_1$ is never reachable, whatever the initial state of the inputs,
the other outputs (SRE, AP1, NFAT) can be independently activated
for some configurations of inputs.
We thus want to check if these three outputs can be activated together,
that is, if the activation of one of the outputs does not prevent the activation of one of the others.
For this, as proposed in \pref{ssec:simult-ua},
we add a new automaton $\reach$ into the model
with two local states ($\PHl_\reach = \{ \reach_0, \reach_1 \}$)
and the action:
\[ \PHhits{\text{SRE}_1, \text{AP1}_1, \text{NFAT}_1}{\reach_0}{\reach_1} \enspace. \]
Finally, checking the reachability of $\reach_1$ in all configurations of the inputs
is always conclusive,
and the existence of 4 positive answers
(amongst all 8 possible initial configurations of the inputs) allows to conclude that
it is possible to reach a state where SRE, AP1 and NFAT are simultaneously active.
}

We held all these experiments on a laptop computer with a 2GHz processor
and 1GB of RAM.
\modMF{
Computations times were in the order of a few hundredths to tenths of a second
in each case (reachability of $\reach_1$ included).
}
To give a comparison, we did the same experiments with a standard symbolic model-checker, libDDD
\cite{libddd}, known for its good performances, the input model being the Boolean network expressed
as a Petri net.
However, due to the large scale of the model, the program runs out of memory for all the experiments.
\modMF{
This limitation shows that studying the dynamics of models is strongly limited to the size of the model for standards model checkers,
while our method is able to conclude with limited memory and computation time usage.
Furthermore, given that the complexity of the static analysis presented in this paper
is polynomial in the size of the models,
it is expected that our work will be scalable to models that are larger,
even by orders of magnitude.
%---~that is, containing thousands of components.
}

\modMF{
Finally, we note that similar experiments were conducted in~\cite{PMR12-MSCS}.
However, if these experiments allowed to obtain some information
on this model to some extent,
it did not provide a formal proof of the reachabilities that have been
checked in the example above.
The semantics of Process Hitting does not allow to model accurate Boolean gates,
as explained in \pref{ssec:cooperation},
thus leading to some unwanted behaviours that take the form of temporal shifts.
Only the adding of classes of priorities allows to remove these temporal shifts,
as explained in~\cite{FPMR13-CS2Bio},
with a construction that is equivalent to the AANs presented in \pref{se:ph}.
Therefore, only the method presented in this paper provides a formal proof that
the observed behaviours are the result of the true dynamics of the system,
which was originally described in a format equivalent to synchronous Discrete Networks.
}

In conclusion,
while ensuring a low complexity for the analysis of reachability in Boolean and discrete networks,
our under-approximation method turns out to be conclusive in numerous cases when applied to real
large-scale biological models, which were not tractable with exact model-checking.
