% vim:set spell spelllang=en:

\section{Biological Examples}\label{sec:example}

This section aims at giving application examples of the static analysis method
that we developed in \pref{sec:sa}.
In \pref{ssec:ex-metazoan}, we apply our method to a small model of the
metazoan segmentation process,
and demonstrate how classes of priorities help in the modelling process,
and how the flattening of \pref{sec:flattening}
can be used in such a case.
In \pref{ssec:ex-tcrsig}, we apply our method to two large-scale models
in order to show the scalability of our method.



\subsection{Under-approximation of a Model with Priorities: Metazoan Segmentation}
\label{ssec:ex-metazoan}

We give in this section a detailed example of the use of classes of priorities
in order to model a system with temporal constraints.
This model also allows us to give a detailed example of the application of the sequential
under-approximation proposed in \pref{ssec:ordered-ua},
which consists of several applications of the method developed in
\pref{ssec:ua}.
For this, we first have to use the flattening method presented in \pref{sec:flattening}
because the considered model contains 2 classes of priorities.

We consider in the following a model of metazoan segmentation
inspired from a first translation to Process Hitting model given in~\cite{PMR10-TCSB}.
This model was originally established in silico in~\cite{MSB:MSB4100192}
in a differential equations framework.
It is composed of a wavefront gene $f$ that activates the gap-gene $a$ whose products are responsible for stripes formation.
Gene $f$ also activates a gene $c$ whose products represses the gene $a$.
The auto-inhibition of $c$ generalises a chain of repressors on $a$.
The auto-inhibition of $f$, which normally terminates
the stripes formation in the original model,
has been removed in order to focus on the stationary dynamics of the model.

The actions of the original model are split into $2$ classes of priorities, as represented in \pref{fig:metazoan-php}:
\begin{align*}
  \ov{\PHh}^{(1)} = \{ \quad
    & \PHfrappes{c_1}{a_1}{a_0} \quad , \quad
    \PHfrappes{f_1, c_0}{a_0}{a_1}
  \quad \} \\
  \ov{\PHh}^{(2)} = \{ \quad
    & \PHfrappem{c_1}{c_0} \quad , \quad
    \PHfrappes{f_1}{c_0}{c_1}
  \quad \} \\
\end{align*}
Indeed, without this use of priorities,
some unwanted behaviours emerge, allowing the formation of irregular stripes.
In order to fix this, a high priority is affected to the actions hitting $a$
and a low priority to the actions hitting $c$,
in order to model the fact that the switch of $c$ has to be immediately followed by
a switch of $a$.
This forces the the evolution of genes $a$ and $c$
to alternate in order not to miss a stripe;
$a$ and $c$ thus have intertwined oscillations.
We can thus consider that these two classes of priorities
are derived from known relative reaction rates,
the evolution of the clock $c$ being slower and regular,
while the evolution of $a$ has to follow the changes of $c$.

\begin{figure}[p]
  \centering
  \scalebox{1}{
  \begin{tikzpicture}[aan]
    \TSort{(0,4)}{c}{2}{l}
    \TSort{(1,0)}{f}{2}{l}
    \TSort{(5,4)}{a}{2}{r}
    
    \TAction{f_1}{c_0.west}{c_1.south west}{bend left=30, in=90}{left}
    \TActionPlur{}{c_1.west}{c_0.north west}{}{-1,5.5}{right}
    
    \TAction{c_1}{a_1.west}{a_0.north west}{prio}{right}
    \TActionPlur{f_1, c_0}{a_0.west}{a_1.south west}{prio}{2.5,2.5}{left}
    
    \TState{f_1, a_0, c_0}
  \end{tikzpicture}
  }
  \caption{
  \label{fig:metazoan-php}
    An example of AAN$2$
    modelling the process of metazoan segmentation.
    Component $a$ models the pigment production, and is influenced by
    component $c$ that has the role of a clock,
    while $f$ represents the wavefront propagation.
    If component $a$ oscillates (that is, its active local state changes regularly)
    then regular stripes are created on the metazoan.
    Actions of $\ov{\PHh}^{(2)}$ (low priority) are represented in thin lines
    and actions of $\ov{\PHh}^{(1)}$ (high priority) are in thick lines.
    The greyed local states represent a possible initial state:
    $\PHstate{f_1, a_0, c_0}$.
  }
\end{figure}

\pref{fig:metazoan-ph} gives the flattening of this model,
that is, an AAN with the equivalent dynamics
(but only one class of priority).
Its actions are:
\begin{align*}
  \PHh = \{ \quad
    & \PHfrappes{c_1}{a_1}{a_0} \quad , \quad
    \PHfrappes{f_1, c_0}{a_0}{a_1} \quad , \\
    & \PHfrappes{a_0}{c_1}{c_0} \quad , \quad
    \PHfrappes{f_1, a_1}{c_0}{c_1}
  \quad \}
\end{align*}
We note that the two actions in $\ov{\PHh}^{(2)}$ have been replaced by
the equivalent actions
$\PHfrappes{a_0}{c_1}{c_0}$ and $\PHfrappes{f_1, a_1}{c_0}{c_1}$,
in order to model the preemption of the actions in $\ov{\PHh}^{(1)}$.

\begin{figure}[p]
  \centering
  \scalebox{1}{
  \begin{tikzpicture}[aan]
    \TSort{(-3,4)}{c}{2}{l}
    \TSort{(0,1)}{f}{2}{l}
    \TSort{(3,4)}{a}{2}{r}
    
%    \TAction{f_1}{c_0.west}{c_1.south west}{bend left=30, in=90}{left}
    \TActionPlur{f_1, a_1}{c_0.south east}{c_1.south east}{}{-1.5,3.3}{right}
    \TAction{a_0.west}{c_1.east}{c_0.north east}{}{left}
    
    \TAction{c_1.north east}{a_1.north west}{a_0.north west}{}{right}
    \TActionPlur{f_1, c_0}{a_0.south west}{a_1.south west}{}{1.5,3.5}{left}
    
    \TState{f_1, a_0, c_0}
  \end{tikzpicture}
  }
  \caption{
  \label{fig:metazoan-ph}
    An example of AAN, which is the flattening of the AAN$2$ in \pref{fig:metazoan-php};
    in other words, this model has exactly the same dynamics as
    \pref{fig:metazoan-php}, but its actions make only one class of priority.
    The greyed local states represent the same initial state:
    $\PHstate{f_1, a_0, c_0}$.
  }
\end{figure}

\medskip

At this point, the static analysis results presented in
\pref{ssec:ua} can be used to check if the model is functional,
\ie if gene $a$ can oscillate, thus leading to the formation of stripes.
Starting from context $\ctx^1 = \PHstate{f_1, a_0, c_0}$,
we thus want to check the reachability of $\myp^1 = a_1$;
then, starting from any new state obtained,
we want to check the reachability of $\myp^2 = a_0$,
and finally $\myp^3 = a_1$ once again to ensure that we entered a cycle.
This consists in using \pref{th:approxinf} several times
to check these reachabilities successively,
by splitting it following the method given in \pref{ssec:ordered-ua}.

We first build the local causality graph $\thisB{\myp^1}{\ctx^1}$
related to the reachability of $\myp^1 = a_1$ from the initial state $\ctx^1$.
This graph is depicted in \pref{fig:metazoan-sa}(left).
\pref{th:approxinf} allows to conclude that this reachability is true.
One can thus extract a concretizing scenario from this local causality graph
with a traversal of the graph, as explained in \pref{ssec:concret}.
Such a traversal is detailed in \pref{tab:concret-metazoan};
the scenario extracted from this example consists of only one action:
$\PHfrappes{c_0, f_1}{a_0}{a_1}$.
Another traversal of the same graph
would consist in visiting node $f_1$ before node $c_0$
when first descending from node $\{ c_0, f_1 \}$;
however, the same scenario would be extracted.
As there exists no other traversal, we thus have in this case:
$\Delta(\thisB{\myp^1}{\ctx^1}) = \{ \PHfrappes{c_0, f_1}{a_0}{a_1} \}$.

We denote in the following:
$\ctx^2 = \PHstate{f_1, a_0, c_0} \play \PHfrappes{c_0, f_1}{a_0}{a_1} =
\PHstate{f_1, a_1, c_0}$
the state resulting from the play of the scenario concretizing
$\myp^1$ in $\ctx^1$.
As explained in \pref{ssec:ordered-ua},
one can use this resulting state in order to carry on with another
successive reachability,
such as the reachability of $\myp^2 = a_0$ in our case.
The local causality graph used to check the reachability of $\myp^2$
from $\ctx^2$ is given in \pref{fig:metazoan-sa}(middle).
The same reasoning allows to conclude that this reachability is true, and
to extract the following set containing only one concretizing scenario:
$\Delta(\thisB{\myp^2}{\ctx^2}) =
\{ \PHfrappes{a_1, f_1}{c_0}{c_1} \concat \PHfrappes{c_1}{a_1}{a_0} \}$.
We note that once again, two traversals are possible
(by visiting node $a_1$ before or after note $f_1$)
but both output the same scenario, and therefore end up in the same sate
$\ctx^3 = \PHstate{f_1, a_0, c_1}$.
Finally, the last local causality graph $\thisB{\myp^3}{\ctx^3}$,
depicted in \pref{fig:metazoan-sa}(right),
allows to conclude that this final reachability is true.

In conclusion, by following \pref{ssec:ordered-ua},
we showed that it is possible to reach successively
$a_1$, $a_0$ and $a_1$
and thus that the AAN of \pref{fig:metazoan-ph} is functional.
This result can be extended to the model of \pref{fig:metazoan-php}
because they have the same dynamics (given \pref{th:bisimPHP}).

\begin{figure}[tp]
  \centering
  \begin{tikzpicture}[aS]
    % STEP 1
    \node[Aproc] (a1) {$a_1$};
    \node[Aobj,below of=a1] (a01) {$\PHobj{a_0}{a_1}$};
    \node[Asol,below of=a01] (a01s) {};
    \node[Assol,below of=a01s] (a01ss) {$\{ c_0, f_1 \}$};

    \node[Aproc,below right of=a01ss] (f1) {$f_1$};
    \node[Aobj,below of=f1] (f11) {$\PHobj{f_1}{f_1}$};
    \node[Asol,below of=f11] (f11s) {};
    \node[Assol,below of=f11s] (nf11s) {$\emptyset$};

    \node[Aproc,below left of=a01ss] (c0) {$c_0$};
    \node[Aobj,below of=c0] (c00) {$\PHobj{c_0}{c_0}$};
    \node[Asol,below of=c00] (c00s) {};
    \node[Assol,below of=c00s] (nc00s) {$\emptyset$};

    \path
    (a1) edge (a01)
    (a01) edge (a01s)
    (a01s) edge (a01ss)
    (a01ss) edge (f1) edge (c0)

    (f1) edge (f11)
    (f11) edge (f11s)
    (f11s) edge (nf11s)

    (c0) edge (c00)
    (c00) edge (c00s)
    (c00s) edge (nc00s)
    ;
    
    % STEP 2
    \node[Aproc,right of=a1,node distance=3.5cm] (a0) {$a_0$};
    \node[Aobj,below of=a0] (a10) {$\PHobj{a_1}{a_0}$};
    \node[Asol,below of=a10] (a10s) {};
    \node[Assol,below of=a10s] (a10ss) {$\{ c_1 \}$};

    \node[Aproc,below of=a10ss] (c1) {$c_1$};
    \node[Aobj,below right of=c1] (c01) {$\PHobj{c_0}{c_1}$};
    \node[Asol,below of=c01] (c01s) {};
    \node[Assol,below of=c01s] (c01ss) {$\{ a_1, f_1 \}$};
    \node[Aobj,below left of=c1] (c11) {$\PHobj{c_1}{c_1}$};
    \node[Asol,below of=c11] (c11s) {};
    \node[Assol,below of=c11s] (nc11s) {$\emptyset$};

    \node[Aproc,below left of=c01ss] (a1) {$a_1$};
    \node[Aobj,below of=a1] (a11) {$\PHobj{a_1}{a_1}$};
    \node[Asol,below of=a11] (a11s) {};
    \node[Assol,below of=a11s] (na11s) {$\emptyset$};

    \node[Aproc,below right of=c01ss] (f1) {$f_1$};
    \node[Aobj,below of=f1] (f11) {$\PHobj{f_1}{f_1}$};
    \node[Asol,below of=f11] (f11s) {};
    \node[Assol,below of=f11s] (nf11s) {$\emptyset$};

    \path
    (a0) edge (a10)
    (a10) edge (a10s)
    (a10s) edge (a10ss)
    (a10ss) edge (c1)

    (c1) edge (c01) edge (c11)
    (c01) edge (c01s)
    (c01s) edge (c01ss)
    (c01ss) edge (f1) edge (a1)
    (c11) edge (c11s)
    (c11s) edge (nc11s)

    (f1) edge (f11)
    (f11) edge (f11s)
    (f11s) edge (nf11s)

    (a1) edge (a11)
    (a11) edge (a11s)
    (a11s) edge (na11s)
    ;
    
    % STEP 3
    \node[Aproc,right of=a0,node distance=4cm] (ta1) {$a_1$};
    \node[Aobj,below of=ta1] (ta01) {$\PHobj{a_0}{a_1}$};
    \node[Asol,below of=ta01] (ta01s) {};
    \node[Assol,below of=ta01s] (ta01ss) {$\{ c_0, f_1 \}$};

    \node[Aproc,below left of=ta01ss] (tf1) {$f_1$};
    \node[Aobj,below of=tf1] (tf11) {$\PHobj{f_1}{f_1}$};
    \node[Asol,below of=tf11] (tf11s) {};
    \node[Assol,below of=tf11s] (tnf11s) {$\emptyset$};

    \node[Aproc,below right of=ta01ss] (tc0) {$c_0$};
    \node[Aobj,below of=tc0] (tc10) {$\PHobj{c_1}{c_0}$};
    \node[Asol,below of=tc10] (tc10s) {};
    \node[Assol,below of=tc10s] (tnc10s) {$\{ a_0 \}$};
    \node[Aobj,right of=tc10] (tc00) {$\PHobj{c_0}{c_0}$};
    \node[Asol,below of=tc00] (tc00s) {};
    \node[Assol,below of=tc00s] (tnc00s) {$\emptyset$};

    \node[Aproc,below of=tnc10s] (ta0) {$a_0$};
    \node[Aobj,below of=ta0] (ta00) {$\PHobj{a_0}{a_0}$};
    \node[Asol,below of=ta00] (ta00s) {};
    \node[Assol,below of=ta00s] (tna00s) {$\emptyset$};

    \path
    (ta1) edge (ta01)
    (ta01) edge (ta01s)
    (ta01s) edge (ta01ss)
    (ta01ss) edge (tf1) edge (tc0)

    (tf1) edge (tf11)
    (tf11) edge (tf11s)
    (tf11s) edge (tnf11s)

    (tc0) edge (tc00)
    (tc00) edge (tc00s)
    (tc00s) edge (tnc00s)
    (tc0) edge (tc10)
    (tc10) edge (tc10s)

    (tc10s) edge (tnc10s)
    (tnc10s) edge (ta0)
    (ta0) edge (ta00)
    (ta00) edge (ta00s)
    (ta00s) edge (tna00s)
    ;
  \end{tikzpicture}
  \caption{%
  \label{fig:metazoan-sa}%
    The three successive saturated graphs of local causality
    of the AAN in \pref{fig:metazoan-ph}
    for the successive reachability of $a_0$, $a_1$ and $a_0$
    from the initial context
    $\ctx = \PHstate{f_1, a_0, c_0}$.
    The (left) graph allows to check the reachability of $a_1$
    from the initial context $\ctx$.
    The (middle) graph is for the reachability of $a_0$
    and the context $\PHstate{f_1, a_1, c_0}$.
    The (right) graph is for the last reachability, $a_1$,
    and the context $\PHstate{f_1, a_0, c_1}$.
  }
\end{figure}

\newcommand{\xproc}[1]{#1}
%\newcommand{\xsol}[1]{\circ^{{#1}}}
\newcommand{\xsol}{\circ}

\begin{table}[p]
  \centering
  \begin{tabular}{|c|c|c|l|l|l|}
    \hline
    \# & Dir. & Node & Marking & Output & State \\\hline\hline
    1  & $\searrow$ & $\xproc{a_1}$ &  &  & $\PHstate{f_1, a_0, c_0}$ \\\hline
    2  & $\searrow$ & $\PHobj{a_0}{a_1}$ &  &  & $\PHstate{f_1, a_0, c_0}$ \\\hline
    3  & $\searrow$ & $\xsol$ & $\PHfrappes{c_0, f_1}{a_0}{a_1}$ &  & $\PHstate{f_1, a_0, c_0}$ \\\hline
    4  & $\searrow$ & $\{ c_0, f_1 \}$ & $\{ c_0, f_1 \}$ &  & $\PHstate{f_1, a_0, c_0}$ \\\hline
    \phantom{*}%
    5* & $\searrow$ & $\xproc{c_0}$ &  &  & $\PHstate{f_1, a_0, c_0}$ \\\hline
    6  & $\searrow$ & $\PHobj{c_0}{c_0}$ &  &  & $\PHstate{f_1, a_0, c_0}$ \\\hline
    7  & $\searrow$ & $\xsol$ & $\emptyseq$ &  & $\PHstate{f_1, a_0, c_0}$ \\\hline
    8  & $\nearrow$ & $\PHobj{c_0}{c_0}$ &  &  & $\PHstate{f_1, a_0, c_0}$ \\\hline
    9  & $\nearrow$ & $\xproc{c_0}$ &  &  & $\PHstate{f_1, a_0, c_0}$ \\\hline
    10 & $\nearrow$ & $\{ c_0, f_1 \}$ & $\{ f_1 \}$ &  & $\PHstate{f_1, a_0, c_0}$ \\\hline
%    \phantom{*}%
    11 & $\searrow$ & $\xproc{f_1}$ &  &  & $\PHstate{f_1, a_0, c_0}$ \\\hline
    12 & $\searrow$ & $\PHobj{f_1}{f_1}$ &  &  & $\PHstate{f_1, a_0, c_0}$ \\\hline
    13 & $\searrow$ & $\xsol$ & $\emptyseq$ &  & $\PHstate{f_1, a_0, c_0}$ \\\hline
    14 & $\nearrow$ & $\PHobj{f_1}{f_1}$ &  &  & $\PHstate{f_1, a_0, c_0}$ \\\hline
    15 & $\nearrow$ & $\xproc{f_1}$ &  &  & $\PHstate{f_1, a_0, c_0}$ \\\hline
    16 & $\nearrow$ & $\{ c_0, f_1 \}$ & $\emptyset$ &  & $\PHstate{f_1, a_0, c_0}$ \\\hline
    17 & $\nearrow$ & $\xsol$ & $\emptyseq$ & $\PHfrappes{c_0, f_1}{a_0}{a_1}$ & $\PHstate{f_1, a_1, c_0}$ \\\hline
    18 & $\nearrow$ & $\PHobj{a_0}{a_1}$ &  &  & $\PHstate{f_1, a_1, c_0}$ \\\hline
    19 & $\nearrow$ & $\xproc{a_1}$ &  &  & $\PHstate{f_1, a_1, c_0}$ \\\hline
  \end{tabular}
  \caption{\label{tab:concret-metazoan}%
    Example of the extraction of a concretizing scenario
    from the local causality graph of \pref{fig:metazoan-sa}(left),
    by using the algorithm of \pref{ssec:concret}.
    The first column denotes the step number in the traversal,
    the second depicts the direction of traversal,
    either “$\searrow$” for descending or “$\nearrow$” for ascending,
    the third one is the name of the current node,
    the fourth one gives the marking of this node when it is left,
    the fifth one gives the actions output by the algorithm
    and the last column gives the current state of the model when leaving each node.
    In step \#5, marked with an asterisk,
    the traversal visits node $c_0$, but this was arbitrarily chosen
    amongst the set $\{ c_0, f_1 \}$.
    Another traversal thus consists in visiting node $f_1$ first,
    although in this example it does not change the result.
  }
\end{table}



\subsection{Large-scale Applications}
\label{ssec:ex-tcrsig}

In order to support the scalability and applicability of our under-approximation of reachability, we
apply our new approach to the analysis of two large-scale models:
a T-cell receptor (TCR) signalling pathway~\cite{tcrsig94}
and an epidermal growth factor receptor (EGFR) signalling pathway~\cite{Samaga2009}.
These models both gather about a hundred components and are detailed below.
They are originally specified as Boolean networks,
and have been automatically encoded into
a Process Hitting model with two classes of priorities,
whose dynamics is equivalent to an AAN%
\footnote{Files are available at
\url{http://maxime.folschette.name/underapprox-aan.zip}.}.
Boolean networks being a subclass of AANs, this translation is straightforward.
In the following, we first present the Pint implementation,
then the two models that were used for our tests,
and the finally summarise the quantitative results of these tests.

\subsubsection*{The Pint Implementation}

The under-approximation presented in \pref{sec:sa} has been implemented
in the existing Pint software%
\footnote{Pint gathers tools related to the Process Hitting
and is freely available at \url{http://loicpauleve.name/pint}.
The release “2015-02-11” was used for these experiments.}.
It comes in two versions:
\begin{itemize}
  \item One version consists in building the local causality graph once
    and checking \pref{th:approxinf} on this graph only.
    This version is polynomial in the size of the model,
    and therefore scalable by nature by its low complexity.
    However, it may be non-conclusive due to the presence of cycles
    or non-independent synchronizations that could be avoided.
  \item The other version consists in checking \pref{th:approxinf}
    on every sub-graph obtained by considering sub-sets of the nodes in $\Sol$
    of the local causality graph.
    This enumeration of “sub-solutions” may allow to find conclusive
    responses by removing cycles or unnecessary local states in the graph.
    However, due to its exhaustive nature,
    it is exponential in the number of solutions of each objective,
    and thus not as scalable.
\end{itemize}
In other words, when this implementation is unable to conclude with the
local causality graph alone, it tries to enumerate sub-solutions
in order to reach a conclusion.
Although this search is guided in order to remove in priority
the solutions that could create cycles or depended synchronizations,
in the worst case all combinations have to be checked.
Therefore, for the following tests, we capped the execution time
of each call to Pint to 3 seconds,
considering that the implementation is inconclusive above this threshold.

The under-approximation presented in this paper can therefore answer
either \emph{True} or \emph{Inconclusive} regarding the reachability
of a given local state.
We used it together with a previously defined
reachability over-approximation~\cite{PMR12-MSCS}
that allows to answer either \emph{False} or \emph{Inconclusive},
but which was not tailored specifically, but still valid, for AANs.

\subsubsection*{The T-cell Receptor Model}

The TCR signalling pathway consists of 94 components.
We checked the reachability for the independent activation of
the 4 outputs of the signalling cascade (SRE, AP1, NFkB, NFAT)
for all possible combinations of the 3 inputs (CD4, CD28, TCRlig).
All result in conclusive decisions,
and our under-approximation has been satisfied in 12 cases (over
32) proving the satisfiability of the concerned reachability properties in the encoded Boolean network
(and non-satisfiability in the other cases, using the previously existing over-approximation).

By analysing the detail of the results, one can find out that one of the outputs, NFkB, can never be activated.
In other words, FkB$_1$ is never reachable, whatever the initial state of the inputs,
while the other outputs (SRE, AP1, NFAT) can be independently activated
for some configurations of inputs.
We thus wanted to check if these three outputs could be activated together,
that is, if the activation of one of the outputs did not prevent the activation another one.
For this, as proposed in \pref{ssec:simult-ua},
we have added a new automaton $\reach$ into the model
with two local states ($\PHl_\reach = \{ \reach_0, \reach_1 \}$)
and the action:
\[ \PHhits{\text{SRE}_1, \text{AP1}_1, \text{NFAT}_1}{\reach_0}{\reach_1} \enspace. \]
Finally, checking the reachability of $\reach_1$ in all configurations of the inputs
has always been conclusive,
and the existence of 4 positive answers
(amongst all 8 possible initial configurations of the inputs) allows to conclude that
it is possible to reach a state where SRE, AP1 and NFAT are simultaneously active.

\subsubsection*{The Epidermal Growth Factor Receptor Model}

The EGFR signalling pathway gathers 104 components, amongst which one can distinguish
21 inputs and 12 outputs.
We selected 13 of all inputs
(erbb1, erbb2, erbb3, erbb4, bir, btc, egf, epr, nrg1a, nrg1b, nrg2b, nrg4, tgfa)
and observed the impact of their variations on all outputs
(elk1, creb, ap1, hsp27, actin\_reorg, cmyc, pro\_apoptotic, p70s6\_2, pkc, stat1, stat3, stat5).
We note that some of the experiments trigger the exponential search
of sub-solutions described above, and are cut after 3 seconds of computation
(thus leading to \emph{Inconclusive} cases).
These “interrupted” experiments represent about 10\% of all tests.
%amongst the remaining 90\%, while all the others are conclusive.
We note that amongst all tests that terminated “normally” (without being cut after 3 seconds)
%none of the tests terminated normally (without being cut after 3 seconds)
none of them responded with an inconclusive answer, which is however theoretically possible.

\subsubsection*{Synthesis of the Results}

The results of all tests performed with the implementation of our under-approximation
are summarised in \pref{tab:results} (columns “AAN”).
We held all these experiments on a personal computer,
and regarding the experiments that were not cut at the 3 seconds limit,
computations times were in the order of a few tenths of a second to about one second.
To give a comparison, we did the same experiments with a standard symbolic model-checker, LibDDD
\cite{libddd}, known for its good performances, the input model being the Boolean network expressed
as a Petri net.
However, due to the large scale of the model,
this program takes at least several minutes to terminate,
and runs out of memory for the majority of all experiments.
On the other hand,
our method is able to conclude with limited memory and computation time usage
in the majority of the cases,
and is expected to be scalable to models that are even larger,
even by orders of magnitude.

Finally, we note that similar experiments were conducted in~\cite{PMR12-MSCS}.
However, if these experiments allowed to obtain some information
on the TCR and EGFR models to some extent,
they did not provide a formal “True” response regarding the reachabilities that have been
checked in the examples above.
Indeed, the semantics of Process Hitting (without classes of priorities)
does not allow to model accurate Boolean gates,
thus leading to some unwanted spurious behaviours
that especially take the form of temporal shifts.
The adding of classes of priorities allows to remove these temporal shifts,
as explained in~\cite{FPMR13-CS2Bio},
with a construction that is equivalent to the AANs presented in \pref{sec:ph}.
Therefore, only the method presented in this paper provides a formal proof that
the observed behaviours are the result of the true dynamics of the systems.
Thus, additionally to the experiments previously mentioned,
we conducted similar experiments with the previous version
of the under-approximation, that are reported alongside in \pref{tab:results}
(in columns “PH”).
We note that indeed, about 15\% of the positive results regarding the EGFR model
could not be proven with our new under-approximation,
and are thus still formally uncertain.

\definecolor{lightgraycell}{gray}{0.85}
\newcommand{\grcl}{\cellcolor{lightgraycell}}

\begin{table}[htp]
  \centering
  \begin{tabular}{|c||c|c||c|c|}
  \hline
    & \multicolumn{2}{c||}{TCR} & \multicolumn{2}{c|}{EGFR} \\
  \hline
  \hline
    Inputs & \multicolumn{2}{c||}{3} & \multicolumn{2}{c|}{13} \\
  \hline
    Outputs & \multicolumn{2}{c||}{4 + $\reach$} & \multicolumn{2}{c|}{12} \\
  \hline
    Total tests & \multicolumn{2}{c||}{40} & \multicolumn{2}{c|}{98'304} \\
  \hline
  \hline
               & PH        & AAN       & PH               & AAN              \\
  \hline
  \hline
    True       & 16 (40\%) & \grcl 16 (40\%) & 74'268 (75,55\%) & \grcl 64'282 (65,39\%) \\
  \hline
    Inconclusive & 0 (0\%) & \grcl 0 (0\%)   & 0 (0\%)          & \grcl 9'986 (10,16\%)  \\
  \hline
    False      & \multicolumn{2}{c||}{24 (60\%)} & \multicolumn{2}{c|}{24'036 (24,45\%)} \\
  \hline
  \hline
    Max time   & 0.043s    & \grcl 0.20s     & 0.37s            & \grcl 0.87s            \\
  \hline
    Total time & <1s       & \grcl <1s       & 45min            & \grcl 9h50min          \\
  \hline
  \end{tabular}
  \caption{\label{tab:results}%
    Results of the tests on large-scale examples.
    The “AAN” column gives the related results on AAN models,
    using the under-approximation presented in this paper,
    while the “PH” column gives the results for PH models
    using cooperative sorts to model actions with multiple hitters,
    and using the under-approximation of~\cite{PMR12-MSCS}.
    The lines labelled “True”, “Inconclusive” and “False” give respectively
    the number of positive answers,
    experiments cut after 3 seconds and negative answers;
    while “Max time” and “Total time” depict respectively
    the maximum time of the individual computations (except those cut at 3 seconds)
    and the overall execution time of all tests (including those cut at 3 seconds).
    The greyed cells highlight the results that are proper to the
    under-approximation presented in this paper.
  }
\end{table}

In conclusion,
while ensuring a low complexity for the analysis of reachability in Boolean and discrete networks,
our under-approximation method turns out to be conclusive in numerous cases when applied to real
large-scale biological models, which were not tractable in most cases with exact model-checking.
