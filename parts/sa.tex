
\towrite{Idée : prise en compte des actions perturbatrices, ou des réactions en chaîne pouvant rendre de telles actions jouables}

\newcommand{\abstr}[1]{#1^\wedge}%\text{\textasciicircum}}
\def\BS{\mathbf{BS}}
%\def\priomax{\mathsf{prio}_{max}}
\def\procs{\mathsf{procs}}
\def\allprocs{\mathsf{allProcs}}
\def\pfp{\mathsf{pfp}}

\def\ctx{\varsigma}
\def\w{\omega}
\def\aBS{\abstr{\BS}}

\def\Req{\mathrm{Req}}
\def\Sol{\mathrm{Sol}}
\def\Cont{\mathrm{Cont}}
\def\A{\mathcal{A}}
\def\cwA{\A_\ctx^\w}
\def\cwReq{\Req_\ctx^\w}
\def\cwSol{\Sol_\ctx^\w}
\def\cwCont{\Cont_\ctx^\w}

\def\aB{\mathcal{B}}
\def\sat#1{\lceil #1\rceil}
\def\cwB{\sat{\aB_\ctx^\w}}
\def\mycwB#1#2{\sat{\aB_{#1}^{#2}}}
\def\Bsol{\sat{\Sol^\w_\ctx}}
\def\Breq{\sat{\Req^\w_\ctx}}
\def\Bcont{\sat{\Cont^\w_\ctx}}

\def\myB{\aB^\w_\ctx}
\def\mysol{\overline{\Sol^\w_\ctx}}
\def\myreq{\overline{\Req^\w_\ctx}}
\def\mycont{\overline{\Cont^\w_\ctx}}

%\def\mlfp#1{\f{pppf}\{#1\}}

\def\PHobjp#1#2#3{\PHobj{{#1}_{#2}}{{#1}_{#3}}}
\def\Obj{\mathbf{Obj}}
\def\powerset{\wp}
\def\gCont{\f{maxCont}}
\def\w{\omega}

Approximation : pour qu'une séquence de bonds soit jouable, il faut pouvoir atteindre tous les processus nécessaires depuis tous les points fixes possibles locaux, uniquement à l'aide d'actions de priorité maximale pour la sorte considérée.



%Plus haute priorité :
%$$\forall a \in \PHs, \priomax(a) = \max_{h \in \PHh, \PHtarget(h) = a}(\prio(h))$$

Processus possibles :
\begin{align*}
\procs((\cwSol,\cwReq,\cwCont)) = \{ p \in \PHproc \mid & \exists (P,ps) \in \cwSol, p \in ps
\\ & \vee p = \PHtarget(P)
\\ & \vee (P \neq \omega \Rightarrow p = \PHbounce(P)) \}
%\\ & \vee \exists h \in \BS(P), (p = \PHhitter(h) \vee p = \PHbounce(h)) \}
\end{align*}

Processus rencontrés :
\begin{align*}
\allprocs((\cwSol,\cwReq,\cwCont)) = \{ p \in \PHproc \mid & \exists (P,ps) \in \cwSol, p \in ps
\\ & \vee p = \PHtarget(P)
%\\ & \vee (P \neq \omega \Rightarrow p = \PHbounce(P)) \}
\\ & \vee \exists h \in \BS(P), (p = \PHhitter(h) \vee p = \PHbounce(h)) \}
\end{align*}

\subsection{Local fixed points}
Actions chaînées :
\begin{align*}
\PHh(a, b) = \{ (h_i)_{i \in \segm{0}{s}} \in \PHh^{s+1} &\mid s \in \sN, \PHsort(\PHhitter(h_0)) = a \wedge \PHsort(\PHtarget(h_n)) = b \\
  & \wedge \prio(h_{s-1}) \leq n \wedge \forall i \in \segm{0}{s-1}, \\
  & (\prio(h_i) \leq n \wedge \PHsort(\PHhitter(h_i)) = \PHsort(\PHtarget(h_{i+1}))) \}
\end{align*}

Voisinage :
$$
V(a) = \{ b \in \PHs \mid \PHh^n(a, b) \neq \emptyset \}
$$

Points fixes possibles : pour toute sorte $a$ et tout contexte partiel $\ctx$ sur $V^n(a)$:
\begin{align*}
  \pfp_\ctx: \PHs &\rightarrow \wp(\PHproc) \\
  a &\mapsto \{ s \in \times_{b \in V^n(a)} \PHl_b \mid \text{$s$ est accessible depuis $\ctx$ et aucune action n'y est jouable} \}
\end{align*}

Points fixes possibles sur un ensemble de sortes :
\begin{align*}
  \pfp: \mathbb{A} &\rightarrow \wp(\PHproc) \\
  \myB &\mapsto \bigcup_{a \in A} \pfp_{\allprocs(\myB)}(a)
\end{align*}

\begin{comment}
Séquences de bonds abstraites :
$$\BS^\wedge(P) = \{ \zeta^\wedge \mid \zeta \in \BS(P), \nexists \zeta' \in \BS(P), \zeta'^\wedge \subsetneq \zeta^\wedge \}$$
where $\zeta^\wedge = (\zeta^\wedge_A, \zeta^\wedge_B, \zeta^\wedge_{max})$ with:
\begin{itemize}
  \item $\zeta^\wedge_A = \{ \PHhitter(\zeta_n) \mid n \in \indexes{\zeta} \wedge \PHsort(\PHhitter(\zeta_n)) \neq \PHsort(P) \}$ : ens. des requis d'autres sortes (frappeurs)
  \item $\zeta^\wedge_B = \{ \PHhitter(\zeta_n) \mid n \in \indexes{\zeta} \} \cup \{ \PHtarget(\zeta_n) \mid n \in \indexes{\zeta} \}$ : ens. des processus nécessaires (à ne pas perturber)
  \item $\zeta^\wedge_{max} = \max_{n \in \indexes{\zeta}}(\prio(\zeta_n))$ : plus faible priorité
\end{itemize}
\end{comment}

\subsection{Abstract structure}
Séquences de bonds abstraites :
$$\BS^\wedge(P) = \{ \zeta^\wedge \mid \zeta \in \BS(P), \nexists \zeta' \in \BS(P), \zeta'^\wedge \subsetneq \zeta^\wedge \}$$
where:
$$\zeta^\wedge = \{ \PHhitter(\zeta_n) \mid n \in \indexes{\zeta} \wedge \PHsort(\PHhitter(\zeta_n)) \neq \PHsort(P) \}$$

The abstract structure $\cwB=(\Breq,\Bsol,\Bcont)$ is defined as
$
 \cwB = \sfp{\myB}{\myB}{\aB^\w_{\ctx \Cap (\procs(\myB) \cup \pfp(\myB))}}
$,
with $\myB=(\myreq,\mysol,\mycont)$:
\begin{align*}
\myreq = &\{ (a_i,\PHobjp{a}{j}{i}) \in \PHproc \times \Obj \mid a_j \in \PHget{\ctx}{a}
  \wedge (\exists (P,ps) \in \mysol, a_i \in ps \\
& \qquad\qquad
          \vee \exists n \in \indexes{\w}, \PHbounce(\w)=a_i\}
\\
\mysol \subseteq & \{ (P,ps) \in \Obj \times \powerset(\PHproc) \mid
        \exists (a_i,P)\in \myreq \wedge ps \in \aBS(P)  \}
\\
\mycont = & \{ (P,\PHobj{q}{\PHbounce(P)}) \in \Obj \times \Obj \mid 
        \exists (P,ps)\in \mysol \\
&\qquad\qquad
        \wedge q \in \gCont_\ctx(\PHs(P),P)\}
\end{align*}

\begin{comment}
Voisinage :
\begin{equation*}
\begin{split}
    V: \wp(\PHproc) \times \segm{1}{k} &\rightarrow \wp(\PHh) \\
    (ps; m) &\mapsto \sfp{\PHh^{(m)+}_{cibles}(ps)}{hs}{\PHh^{(m)+)}_{bonds}(\widehat{B}(hs)) \cup hs)}
  \end{split}
\end{equation*}
where:
\begin{equation*}
\begin{split}
    \widehat{B}: \wp(\PHh) &\rightarrow \wp(\PHproc) \\
    hs &\mapsto \{ \PHhitter(h) \mid h \in hs \} \cup \{ \PHtarget(h) \mid h \in hs \}
  \end{split}
\end{equation*}
\begin{equation*}
\begin{split}
    \PHh^{(m)+}_{\mathsf{ref}}: \wp(\PHproc) &\rightarrow \wp(\PHh) \quad,\quad m \in \segm{0}{k} \text{ and } \mathsf{ref} \in \{ \PHhitter, \PHtarget, \PHbounce \} \\
    ps &\mapsto \{ h \in \PHh \mid \mathsf{ref}(h) \in ps \wedge \prio(h) \leq m \}
  \end{split}
\end{equation*}
\end{comment}



