

\towrite{Idée : prise en compte des actions perturbatrices, ou des réactions en chaîne pouvant rendre de telles actions jouables}



%%% N'est plus utile (?)
%Plus haute priorité :
%$$\forall a \in \PHs, \priomax(a) = \max_{h \in \PHh, \PHtarget(h) = a}(\prio(h))$$

%%% N'est plus utile (?)
%Processus possibles :
%\begin{align*}
%\procs((\cwSol,\cwReq,\cwCont)) = \{ p \in \PHproc &\mid \exists (P,ps) \in \cwSol, p \in ps
%\\ & \vee p = \PHtarget(P)
%\\ & \vee (P \neq \omega \Rightarrow p = \PHbounce(P)) \}
%%\\ & \vee \exists h \in \BS(P), (p = \PHhitter(h) \vee p = \PHbounce(h)) \}
%\end{align*}

Processus rencontrés :
\begin{align*}
\allprocs((\cwSol,\cwReq,\cwCont)) = \{ p \in \PHproc &\mid \exists (P,ps) \in \cwSol, p \in ps
\\ & \vee p = \PHtarget(P)
%\\ & \vee (P \neq \omega \Rightarrow p = \PHbounce(P)) \}
\\ & \vee \exists h \in \BS(P), (p = \PHhitter(h) \vee p = \PHbounce(h)) \}
\end{align*}

\subsection{Local fixed points}

%%% Pas utile
%Points fixes possibles : pour toute sorte $a$ et tout contexte partiel $\ctx$ sur $V^n(a)$:
%\begin{align*}
%  \pfp_\ctx: \PHs &\rightarrow \wp(\PHproc) \\
%  a &\mapsto \{ s \in \underset{b \in V^n(a)}{\times} \PHl_b \mid \text{$s$ est accessible depuis $\ctx$ et aucune action n'y est jouable} \}
%\end{align*}

\towrite{Formaliser}

Points fixes possibles : pour toute sorte $a$ et tout contexte $\ctx$:
\begin{align*}
  \pfp_\ctx: \PHs &\rightarrow \wp(\PHproc) \\
  a &\mapsto \{ s \PHplay \delta \in \restriction{\PHl}{\Vs(a)} \mid s \in \restriction{\ctx}{\Vs(a)} \wedge \delta \in \restriction{\Sce}{\Vs(a)} \\
  &\qquad\qquad\qquad\qquad \wedge \forall h \in \Vh(a), \text{$h$ n'est pas jouable dans $s \PHplay \delta$} \}
\end{align*}

Processus rencontrés comme résultats d'un point fixe :
\begin{align*}
\pfpprocs_\ctx(a) = \{ \PHget{s}{a} \mid s \in \pfp_\ctx(a) \}
\end{align*}



Le nouvel ensemble de processus à prendre en compte à chaque itération du point fixe :
\begin{align*}
\newprocs_\ctx(\myB) = %\allprocs(\myB) \cup \bigcup_{a \in \PHs} \pfpprocs_{\allprocs(\myB)}(a)
  \{ a_k \in \PHproc &\mid a_k \in \allprocs(\myB) \vee \\
  & (a_i, \PHobjp{a}{j}{i}) \in \cwReq \wedge a_k \in \pfpprocs_{\allprocs(\myB)}(a) \wedge \\
  & \quad (\exists (P, ps) \in \cwSol, a_i \in ps \wedge \prio(\PHsort(P)) > \prio(a) \\
  & \quad \vee \exists (\PHobjp{a}{j}{i}, ps) \in \cwSol, \exists p \in ps, \exists (p, P) \in \cwReq,\\
  & \qquad \text{$P$ is not trivial} \wedge \prio(\PHsort(P)) > \prio(a) \}
\end{align*}

%%% N'est plus adapté
%Points fixes possibles sur un ensemble de sortes :
%\begin{align*}
%  \pfp: \mathbb{A} &\rightarrow \wp(\PHproc) \\
%  \myB &\mapsto \bigcup_{a \in A} \pfp_{\allprocs(\myB)}(a)
%\end{align*}

\begin{comment}
Séquences de bonds abstraites :
$$\BS^\wedge(P) = \{ \zeta^\wedge \mid \zeta \in \BS(P), \nexists \zeta' \in \BS(P), \zeta'^\wedge \subsetneq \zeta^\wedge \}$$
where $\zeta^\wedge = (\zeta^\wedge_A, \zeta^\wedge_B, \zeta^\wedge_{max})$ with:
\begin{itemize}
  \item $\zeta^\wedge_A = \{ \PHhitter(\zeta_n) \mid n \in \indexes{\zeta} \wedge \PHsort(\PHhitter(\zeta_n)) \neq \PHsort(P) \}$ : ens. des requis d'autres sortes (frappeurs)
  \item $\zeta^\wedge_B = \{ \PHhitter(\zeta_n) \mid n \in \indexes{\zeta} \} \cup \{ \PHtarget(\zeta_n) \mid n \in \indexes{\zeta} \}$ : ens. des processus nécessaires (à ne pas perturber)
  \item $\zeta^\wedge_{max} = \max_{n \in \indexes{\zeta}}(\prio(\zeta_n))$ : plus faible priorité
\end{itemize}
\end{comment}

\subsection{Abstract structure}
%Séquences de bonds abstraites :
%$$\BS^\wedge(P) = \{ \zeta^\wedge \mid \zeta \in \BS(P), \nexists \zeta' \in \BS(P), \zeta'^\wedge \subsetneq \zeta^\wedge \}$$
%where:
%$$\zeta^\wedge = \{ \PHhitter(\zeta_n) \mid n \in \indexes{\zeta} \wedge \PHsort(\PHhitter(\zeta_n)) \neq \PHsort(P) \}$$

\begin{definition}[$\gCont_\ctx : \Sigma \times \Obj \mapsto \powerset(\Proc)$]
  \label{def:maxCont}
  \begin{align*}
    \gCont_\ctx(a,P) = 
    \{ p \in \PHproc &\mid \exists ps \in \aBS(P), \exists b_i \in ps, b = a \wedge p = b_i \\
      & \vee b \neq a \wedge p \in \gCont_\ctx(a, \PHobj{b_j}{b_i}) \wedge b_j \in \PHget{\ctx}{b} \}
    \enspace.
  \end{align*}
\end{definition}

\begin{definition}
  \label{def:aS}
  The abstract structure $\cwB=(\Breq,\Bsol,\Bcont)$ is defined as
  $
  \cwB = \sfp{\myB}{\myB}{\aB^\w_{\ctx \Cap \newprocs_\ctx(\myB)}}
  $,
  with $\myB=(\myreq,\mysol,\mycont)$:
  \begin{align*}
    \myreq &= \{ (a_i,\PHobjp{a}{j}{i}) \in \PHproc \times \Obj \mid
      a_j \in \PHget{\ctx}{a} \\ % \vee a_j \in \pfpprocs_\ctx(a) \\
      & \qquad \wedge (\exists (P,ps) \in \mysol, a_i \in ps \vee \exists n \in \indexes{\w}, \PHbounce(\w_n)=a_i) \}
    \\
    \mysol &\subseteq \{ (P,ps) \in \Obj \times \powerset(\PHproc) \mid
            \exists (a_i, P) \in \myreq \wedge ps \in \aBS(P) \\
      & \qquad\qquad \vee \exists (Q, P) \in \mycont \wedge ps \in \aBS(P) \}
    \\
    \mycont & = \{ (P, \PHobj{q}{\PHbounce(P)}) \in \Obj \times \Obj \mid
      \exists (P, ps) \in \mysol \\
      & \qquad\qquad \wedge q \in \gCont_\ctx(\PHsort(P),P) \}
  \end{align*}
\end{definition}

\def\muconcr{\ell}
\def\uconcr{\muconcr_\ctx}

\begin{theorem}[Approximation inf.]\label{th:approxinf}
If the graph $\cwB$ contains no cycle and all objectives have at least one solution, then $\uconcr(\w) \neq \emptyset$.
\end{theorem}

\begin{proof}
We note $max\ctx = \ctx \Cap \allprocs(\cwB)$ the context supported by $\cwB$.
As there is no cycle in $\cwB$, we show by induction that $\forall s\in L, s\subseteq max\ctx$, 
for all objective $P$ in $\cwB$ so that $\PHtarget(P) \in s$ and
$\exists \delta \in \muconcr_s(P)$.% and $\ceil(\delta) \subseteq max\ctx$.

\begin{itemize}
  \item If $(P, \emptyset) \in \Bsol$, either $\PHtarget(P) = \PHbounce(P)$ and $\delta = \emptyseq$,
    or $\forall \zeta \in \BS(P), \zeta \in \Sce \wedge \PHsort(\zeta) = \{ \PHsort(P) \}$ and $\delta$ is given by \pref{th:autohits}.

  \item Suppose all children objectives of $P$ are concretizable. \todo{Problème : ce n'est plus suffisant avec cette saturation (il faut que les frères de $P$ obtenus par saturation soient concrétisables aussi).}
  \begin{itemize}
    \item If $\exists Q \in \Bcont$, then by hypothesis,
      $\muconcr_{s}(\obj{\PHtarget(P)}{\PHtarget(Q)} \concat Q) \neq \emptyset$, thus
      $\muconcr_{s}(P) \neq \emptyset$.
    \item Else, by \pref{def:maxCont}, the concretizations of the children of $P$ require no process of sort $\PHsort(P)$.
      Furthermore, there exists $\zeta \in \BS(P)$ so that $(P, \aZ) \in \Bsol$.
      We build recursively a scenario $\delta$. Let $m = |\indexes{\zeta}|$.
      \todo{Idée avec les nouvelles defs : au rang 0, on se replace sur un processus de $\pfp$. Au rang n, il existe par saturation un objectif adéquat qui “reprend” la progression.}
      \begin{itemize}
%        \item[*] Let $\zeta_1 = \PHhit{b_i}{a_j}{a_k}$. By hypothesis, $\exists \delta_1 \in \muconcr_s(\PHobj{\any}{b_i}), \PHget{s \PHplay \delta_1}{a} \in \pfp_s(a)$.
%          If $\PHget{s \PHplay \delta_1}{a} \neq a_j$, then by construction of $\cwB$, $(\PHobj{a_j}{a_k}, \PHobj{\PHget{s \PHplay \delta_1}{a}}{a_k}) \in \Bsat$.
%          By hypothesis, $\muconcr_s(\PHobj{\any}{a_k}) \neq \emptyset$.
%          If $\PHget{s \PHplay \delta_1}{a} = a_j$, then from \pref{th:vplay}, $\exists \delta'_1 \in \Sce$, $\zeta_1$ can be played in $s \PHplay \delta_1 \PHplay \delta'_1$.
        \item[*] For $n \in \indexes{\zeta}$, let $s_n = s \PHplay \delta_1 \PHplay \delta'_1 \PHplay \zeta_1 \PHplay \dots \PHplay \delta_{n-1} \PHplay \delta'_{n-1} \PHplay \zeta_{n-1}$ (or $s_1 = s$),
          and $\zeta_n = \PHhit{b_i}{a_j}{a_k}$. By hypothesis, $\exists \delta_n \in \muconcr_{s_n}(\PHobj{\any}{b_i})$, and we have: $\PHget{s_n \PHplay \delta_n}{a} \in \pfp_{s_n}(a)$.
          If $\PHget{s_n \PHplay \delta_n}{a} \neq a_j$, then by construction of $\cwB$, $(\PHobj{a_j}{a_k}, \PHobj{\PHget{s_n \PHplay \delta_n}{a}}{a_k}) \in \Bsat$.
          By hypothesis, $\muconcr_{s_n \PHplay \delta_n}(\PHobj{\any}{a_k}) \neq \emptyset$.
          If $\PHget{s_n \PHplay \delta_n}{a} = a_j$, then from \pref{th:vplay}, $\exists \delta'_n \in \Sce$, $\zeta_n$ can be played in $s \PHplay \delta_n \PHplay \delta'_n$
          \todo{Il reste à montrer que $\delta'_n$ ne perturbe pas $b$. Ça se montre par contradiction grâce à l'hyopthèse de cycle-freeness.}
          .
%        \item[*] For $n \in \segm{1}{m-1}$, $m = |\indexes{\zeta}|$, let $s_n = s \PHplay \delta_1 \PHplay \delta'_1 \PHplay \zeta_1 \PHplay \dots \PHplay \delta_n \PHplay \delta'_n \PHplay \zeta_n$ and $\zeta_{n+1} = \PHhit{c_i}{a_j}{a_k}$.
%          By hypothesis, $\exists \delta_1 \in \muconcr_{s_n}(\PHobj{\any}{c_i}), \PHget{s_n \PHplay \delta_{n+1}}{a} \in \pfp_s(a)$.
      \end{itemize}
      Thus, $\delta = \delta_m \in \muconcr_s(P)$. % and $\ceil(\delta) \subseteq max\ctx$.
  \end{itemize}
\end{itemize}

Finally, as $\muconcr_{max\ctx}(\w) \neq \emptyset$, $\uconcr(\w) \neq \emptyset$ (property \towrite{xxx}).
\end{proof}



\begin{comment}
Voisinage :
\begin{equation*}
\begin{split}
    V: \wp(\PHproc) \times \segm{1}{k} &\rightarrow \wp(\PHh) \\
    (ps; m) &\mapsto \sfp{\PHh^{(m)+}_{cibles}(ps)}{hs}{\PHh^{(m)+)}_{bonds}(\widehat{B}(hs)) \cup hs)}
  \end{split}
\end{equation*}
where:
\begin{equation*}
\begin{split}
    \widehat{B}: \wp(\PHh) &\rightarrow \wp(\PHproc) \\
    hs &\mapsto \{ \PHhitter(h) \mid h \in hs \} \cup \{ \PHtarget(h) \mid h \in hs \}
  \end{split}
\end{equation*}
\begin{equation*}
\begin{split}
    \PHh^{(m)+}_{\mathsf{ref}}: \wp(\PHproc) &\rightarrow \wp(\PHh) \quad,\quad m \in \segm{0}{k} \text{ and } \mathsf{ref} \in \{ \PHhitter, \PHtarget, \PHbounce \} \\
    ps &\mapsto \{ h \in \PHh \mid \mathsf{ref}(h) \in ps \wedge \prio(h) \leq m \}
  \end{split}
\end{equation*}
\end{comment}



\begin{figure}
  \centering
  \begin{tikzpicture}[aS,node distance=1.5cm]
    \node[Aproc] (c1) {$c_1$};
    \node[Aobj,right of=c1] (c01) {$\PHobj{c_0}{c_1}$};
    \node[Asol,right of=c01] (c01s) {};
    \node[Aproc,right of=c01s] (b1) {$b_1$};
    \node[Aobj,right of=b1] (b11) {$\PHobj{b_1}{b_1}$};
    \node[Asol,right of=b11] (b11s) {};
    \node[Aobj,below right of=b1] (b01) {$\PHobj{b_0}{b_1}$};
    \node[Anos,right of=b01] (b01nos) {$\bottom$};

    \path
    (c1) edge (c01)
    (c01) edge (c01s)
    (c01s) edge (b1)
    (b1) edge (b11) edge (b01)
    (b11) edge (b11s)
    ;
  \end{tikzpicture}
  \label{fig:sa-conca}
  \caption{The local causality graph of the PH in \pref{fig:ph-conca}.}
\end{figure}



\begin{figure}
  \centering
  \begin{tikzpicture}[aS,node distance=1.5cm]
    \node[Aproc] (c2) {$c_2$};
    \node[Aobj,right of=c2] (c12) {$\PHobj{c_1}{c_2}$};
    \node[Asol,right of=c12] (c12s) {};
    \node[Aproc,right of=c12s] (b1) {$b_1$};
    \node[Aobj,right of=b1] (b01) {$\PHobj{b_0}{b_1}$};
    \node[Asol,right of=b01] (b01s) {};
    \node[Aproc,right of=b01s] (a1) {$a_1$};
    \node[Aobj,right of=a1] (a11) {$\PHobj{a_1}{a_1}$};
    \node[Asol,right of=a11] (a11s) {};

    \node[Aobj,above right of=c2] (c22) {$\PHobj{c_2}{c_2}$};
    \node[Asol,right of=c22] (c22s) {};
    \node[Aobj,below right of=c2] (c02) {$\PHobj{c_0}{c_2}$};
    \node[Anos,right of=c02] (c02nos) {$\bottom$};

    \path
    (c2) edge (c12) edge (c22) edge (c02)
    (c22) edge (c22s)
    (c12) edge (c12s)
    (c12s) edge (b1)
    (b1) edge (b01)
    (b01) edge (b01s)
    (b01s) edge (a1)
    (a1) edge (a11)
    (a11) edge (a11s)
    ;
  \end{tikzpicture}
  \label{fig:sa-concb}
  \caption{The local causality graph of the PH in \pref{fig:ph-concb}.}
\end{figure}





