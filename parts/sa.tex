\section{Static analysis}\label{sec:sa}

The aim of this section is to define the problem of reachability in a PH,
and propose an under-approximation allowing to efficiently solve it.
The static analysis presented here is inspired from~\cite{PMR12-MSCS}.

\subsection{Preliminary definitions}
\label{ssec:sa-def}

The reachability of a process $a_j$ of a given sort $a$ from another process $a_i$ is called an objective and is denoted $\PHobjp{a}{j}{i}$ (\pref{def:obj}).
\begin{definition}[Objective ($\Obj$)]
\label{def:obj}
  If $a \in \components$, the reachability of a process $a_j$ from a process $a_i$ is called an \emph{objective}, noted $\PHobj{a_i}{a_j}$.
  The set of all objectives is called $\Obj \DEF \{ \PHobj{a_i}{a_j} \mid a \in \components \wedge (a_i, a_j) \in \PHl_a^2 \}$.
  For an objective $P = \PHobj{a_i}{a_j} \in \Obj$, we define: $\PHsort(P) = a$, $\PHtarget(P)=a_i$, $\PHbounce(P)=a_j$,
  and $P$ is \emph{trivial} if $a_i = a_j$.

  We define an \emph{objective sequence} as a sequence of objectives in which each objective target must be equal to the previous objective bounce of the same sort, if it exists.
  The set of all objective sequences is denoted by $\OS$.
\end{definition}

A context (\pref{def:context}) extends the notion of state to a set of possible initial states.
%the extension of the override operator follows (\pref{def:ctxcap}).
We also extend the override operator to contexts (\pref{def:ctxcap}).
\begin{definition}[Context ($\Ctx$)]
\label{def:context}
  A \emph{context} $\ctx$ associates to each sort in $\PHs$ a non-empty subset of its processes:
  $\forall a \in \PHs, \PHget{\ctx}{a} \subseteq \PHl_a \wedge \PHget{\ctx}{a} \neq \emptyset$.
  $\Ctx$ is the set of all contexts.
\end{definition}
%
\begin{definition}[$\Cap: \Ctx \times \powerset(\PHproc) \rightarrow \Ctx$]
\label{def:ctxcap}
  For any $\ctx\in\Ctx$ and set of processes $ps \in \powerset(\PHproc)$,
  the override of $\ctx$ by $ps$ is noted $\ctx \Cap ps$ and is defined by:
  \[ \forall a \in \PHs, \PHget{(\ctx \Cap ps)}{a} =
  \begin{cases}
    \{ p \in ps \mid \PHsort(p)=a \} & \text{if } \exists p \in ps, \PHsort(p)=a,\\
    \PHget{\ctx}{a} & \text{else.}
  \end{cases}
  \]
\end{definition}
\noindent
For a given context $\ctx$, we note $a_i \in \ctx$ if and only if $a_i \in \PHget{\ctx}{a}$,
and for all $ps \in \powerset(\PHproc)$ or $ps \in \PHl$, $ps \subseteq \ctx \EQDEF \forall a_i \in ps, a_i \in \ctx$.
A sequence of actions $\delta$ is \emph{playable} in a context $\ctx$ if and only if 
$\exists s \subseteq \ctx, \delta \in \Sce(s)$.
We denote then: $\delta \in \Sce(\ctx)$,
%$\supp(\delta) \subseteq \ctx$. 
and the play of $\delta$ in $\ctx$ is $\ctx \PHplay \delta = \ctx \Cap \ceil(\delta)$,
where $\ceil(\delta)$ is the set containing the last process (hitter or bounce) of each sort present in $\delta$.

\begin{comment}
  \begin{definition}[Objective sequence ($\OS$)]
  \label{def:OS}
  \towrite{Simplifier / réécrire en dehors d'une def}
    An \emph{objective sequence} is a sequence $\w = P_1 \concat \dots \concat P_{|\w|}$,
    where $\forall n \in \indexes{\w}, \w_n \in \Obj$ and $a_i = \PHtarget(\w_n) \Rightarrow \last_a(\w_{1..n-1}) \in \{ \varnothing, a_i \}$.
    The set of all objective sequences is denoted by $\OS$.
    The definitions of $\last_a$ \todo{À définir}, $\first_a$ \todo{À définir}, $\supp$ \todo{À définir} and $\ceil$ \todo{À définir}
    are simply derived by omitting the case of hitters.
  \end{definition}
\end{comment}

Finally, a bounce sequence on a sort $a$ (\pref{def:bs}) is a sequence of actions hitting $a$
in which the bounce process of each action is the hitter process of the following action.
Bounce sequences are used to find local solutions to a given objective.
%Bounce sequences are used to find sequences of actions that locally solve a given objective.
%solve an objective locally, by finding the sequences of actions that allow to reach a process.
A bounce sequence can be abstracted into sets of all its hitters that are not in sort $a$.
%of every action that are different from the considered sort (\pref{def:aBS}).
This abstraction allows to propagate an objective into objectives on other sots.
In the following, we denote: $\Sol = \powerset(\PHproc)$.
\begin{definition}[Bounce sequence ($\BS$)]
\label{def:bs}
  A \emph{bounce sequence} $\zeta$ is a sequence of actions so that $\forall n \in \indexes{\zeta}, n < |\zeta|, \PHbounce(\zeta_{n}) = \PHtarget(\zeta_{n+1})$.
  $\BS$ denotes the set of all bounce sequences, and
  $\BS(P)$ denotes the set of bounce sequences \emph{solving} an objective $P$:
  \[
    \BS(\PHobj{a_i}{a_j}) = \{ \zeta \in \BS \mid \PHtarget(\zeta_1)= a_i \wedge \PHbounce(\zeta_{|\zeta|}) = a_j \} \enspace.
  \]
%  Obviously, $\emptyseq \in \BS(\obj{a_i}{a_i})$; and $\BS(\obj{a_i}{a_j}) = \emptyset$ if there is no way to reach $a_j$ from $a_i$.
  $\BS(\obj{a_i}{a_j}) = \emptyset$ if there is no way to reach $a_j$ from $a_i$ and $\emptyseq \in \BS(\obj{a_i}{a_i})$.
\end{definition}
%
\begin{definition}[$\aBS:\Obj \rightarrow \powerset(\Sol)$]
\label{def:aBS}
  \[
    \aBS(P) = \{ \abstr{\zeta} \in \Sol \mid \zeta \in \BS(P), \nexists \zeta' \in \BS(P), \abstr{\zeta'} \subsetneq \abstr{\zeta} \} \enspace,
  \]
  where $\abstr{\zeta} = \{ \PHhitter(\zeta_n) \mid  n \in \indexes{\zeta} \wedge \PHsort(\PHhitter(\zeta_n)) \neq \PHsort(P) \}$.
\end{definition}



\subsection{Under-approximation}

We define $\concr(\w)$ as the set of scenarios concretising an objective sequence $\w$ in the context $\ctx$.
In \pref{def:uconcr}, we define $\uconcr(\w)$ as equal to $\concr(\w)$ if and only if $\concr(\w)$ contains scenarios starting from all states $s \subseteq \ctx$.
\pref{th:uconcr-ctx} is used to over-approximate the initial context $\ctx$.
%if such a scenario exists in all states $s \subseteq \ctx$.
%
\begin{comment}
\begin{definition}[$\concr: \OS \mapsto \powerset(\Sce)$]
\label{def:concr}
\towrite{Simplifier ? Supprimer la def formelle ?}
  For a given $\w \in \OS$, $\concr(\w)$ is the set of scenarios concretising $\w$ in the context $\ctx$:
  \begin{align*}
    \concr(\w) = \{ \delta \in \Sce \mid & (\w^\vartriangle = \emptyseq \wedge \delta = \emptyseq) 
      \vee (\w^\vartriangle \neq \emptyseq \wedge \supp(\delta) \subseteq \ctx
    \\ &
      \wedge \exists \phi:\indexes{\w} \mapsto \indexes{\delta}, (\forall n, m \in \indexes{\w}, n < m \Leftrightarrow \phi(n) \leq \phi(m)) 
    \\ &
      \wedge \forall n \in \indexes{\w}, \PHbounce(\w_n) \in \ctx \PHplay \delta_{1..\phi(n)})
    \}
    \enspace,
  \end{align*}
  where $\omega^\vartriangle$ refers to the sequence of objectives $\omega$ where the trivial objectives have been removed.
\end{definition}
%
\begin{definition}[$\concr: \powerset(\OS) \mapsto \powerset(\Sce)$]
\label{def:concr-set}
  $\concr(\W) = \{ \delta \in \concr(\w) \mid \w \in \W \} \enspace.$
\end{definition}
\end{comment}
%
\begin{definition}[$\uconcr: \OS \rightarrow \powerset(\Sce)$]
\label{def:uconcr}
  \[
  \uconcr(\w) = 
  \begin{cases}
    \concr(\w) & \text{if } \forall s \in \PHl, s \subseteq \ctx, \exists \delta \in \concr(\w), \delta \in \Sce(s) \\
    \emptyset & \text{else.}
  \end{cases}
  \]
\end{definition}
% 
\begin{theorem}
\label{th:uconcr-ctx}
  $\ctx \subseteq \ctx' \wedge \muconcr_{\ctx'}(\w) \neq \emptyset \Longrightarrow \muconcr_{\ctx}(\w) \neq \emptyset$.
\end{theorem}

\begin{comment}
\begin{definition}[$\uconcr: \powerset(\OS) \mapsto \powerset(\Sce)$]
\label{def:uconcr-set}
  $\uconcr(\Omega) = \{ \delta \in \uconcr(\w) \mid \w \in \Omega\}$
\end{definition}
\end{comment}

For any objective $P$ and context $\ctx$, \pref{def:maxCont} allows to obtain the set of processes of sort $\PHsort(P)$ are required to solve $P$ in $\ctx$, given by $\gCont_\ctx(\PHsort(P), P)$.
\begin{comment}
\begin{definition}
\label{def:allprocs}
  \begin{align*}
  \allprocs(\V, \E) = \{ p \in \PHproc &\mid p \in \V \vee \exists P \in \V \cap \Obj,\\
  &\qquad p = \PHtarget(P) \vee p = \PHbounce(P) \}
%    &\qquad\qquad p \in ps \vee p = \PHtarget(P) \vee p = \PHbounce(P) \\
%    &\qquad\qquad \vee \exists (a_i,ps) \in \cwRSP, p \in ps \}
  \end{align*}
  \begin{align*}
  \allprocs(\V, \E) = (V \cap \Proc) \cup \{ \PHtarget(P), \PHbounce(P) \mid P \in \V \cap \Obj \}
  \end{align*}
\end{definition}
\end{comment}
%
\begin{comment}
\begin{definition} \todo{À revoir avec la nouvelle sémantique}
\label{def:allprocs}
  \begin{align*}
  &\allprocs((\cwSol,\cwReq,\cwRSP,\cwCont)) = \{ p \in \PHproc \mid \exists (P,ps) \in \cwSol, \\
    &\qquad\qquad p \in ps \vee p = \PHtarget(P) \vee p = \PHbounce(P) \\
    &\qquad\qquad \vee \exists (a_i,ps) \in \cwRSP, p \in ps \}
  %  &\qquad\qquad \vee p = \PHtarget(P) \vee \exists h \in \BS(P), (p = \PHhitter(h) \vee p = \PHbounce(h)) \\
  %  &\qquad\qquad \vee \exists(Q, \PHobj{p}{\PHbounce(Q)} \in \cwPrioCont \}
  \end{align*}
\end{definition}
\end{comment}
%
%\subsection{Abstract structure}
%
%
\begin{definition}[$\gCont_\ctx : \Sigma \times \Obj \rightarrow \powerset(\PHproc)$]
  \label{def:maxCont}
  \begin{align*}
    \gCont_\ctx(a,P) = 
    \{ p \in \PHproc &\mid \exists ps \in \aBS(P), \exists b_i \in ps, b = a \wedge p = b_i \\
      & \vee b \neq a \wedge p \in \gCont_\ctx(a, \PHobj{b_j}{b_i}) \wedge b_j \in \PHget{\ctx}{b} \}
    \enspace.
  \end{align*}
\end{definition}

\begin{comment}
\begin{definition}
  $\cwB = (\V, \E)$, $V \subset \PHproc \cup \Obj \cup \powerset(\PHproc)$ and $E \subset V \times V$
  so that:\\
  $\w \subset \V$\\
  $\{ \PHbounce(P) \mid P \in \V \} \subset \V$\\
  $\{ (P, ps) \in \Obj \times \powerset(\PHproc) \mid P \in \V \wedge ps \in \BS(P) \} \subset \E$\\ % Sol
  $\{ ps \in \powerset(\PHproc) \mid (P, ps) \in \E \} \subset \V$\\
  $\{ (ps, a_i) \in \powerset(\PHproc) \times \PHproc \mid a_i \in ps \} \subset \E$\\ % Sol
  $\{ a_i \in \PHproc \mid (ps, a_i) \in \E \} \subset \V$\\
  $\{ (a_i, \PHobjp{a}{j}{i}) \in \PHproc \times \Obj \mid a \in \components \wedge a_i \in \V \wedge a_j \in \ctx \} \subset \E$\\ % Req
  $\{ \PHobjp{a}{j}{i} \in \Obj \mid (a_i, \PHobjp{a}{j}{i}) \in \E \} \subset \V$\\
  $\{ (P, \PHobj{q}{\PHbounce(P)}) \in \Obj \times \Obj \mid P \in \V \wedge q \in \gCont_\ctx(\PHsort(P), P) \} \subset \E$\\ % Cont
  $\{ (a_i, ps) \in \Proc \times \powerset(\PHproc) \mid a \in \cs \wedge a_i \in \V \wedge ps \in \csState(a_i) \} \subset \E$\\ % ReqSolPrio
\end{definition}

\begin{definition}
  $\cwB = (\V, \E)$, $V \subset \PHproc \cup \Obj \cup \powerset(\PHproc)$ and $E \subset V \times V$
  so that:
  \begin{align*}
    \w &\subset \V\\
    \{ \PHbounce(P) \mid P \in \V \} &\subset \V\\
    \{ (P, ps) \in \Obj \times \powerset(\PHproc) \mid P \in \V \wedge ps \in \BS(P) \} &\subset \E\\ % Sol
    \{ ps \in \powerset(\PHproc) \mid (P, ps) \in \E \} &\subset \V\\
    \{ (ps, a_i) \in \powerset(\PHproc) \times \PHproc \mid a_i \in ps \} &\subset \E\\ % Sol
    \{ a_i \in \PHproc \mid (ps, a_i) \in \E \} &\subset \V\\
    \{ (a_i, \PHobjp{a}{j}{i}) \in \PHproc \times \Obj \mid a \in \components \wedge a_i \in \V \wedge a_j \in \ctx \} &\subset \E\\ % Req
    \{ \PHobjp{a}{j}{i} \in \Obj \mid (a_i, \PHobjp{a}{j}{i}) \in \E \} &\subset \V\\
    \{ (P, \PHobj{q}{\PHbounce(P)}) \in \Obj \times \Obj \mid P \in \V \wedge q \in \gCont_\ctx(\PHsort(P), P) \} &\subset \E\\ % Cont
    \{ (a_i, ps) \in \Proc \times \powerset(\PHproc) \mid a \in \cs \wedge a_i \in \V \wedge ps \in \csState(a_i) \} &\subset \E % ReqSolPrio
  \end{align*}
\end{definition}
\end{comment}

The reachability of a process $a_i$ is approximated by the ability to solve all objectives $\PHobjp{a}{j}{i}$ so that $a_j$ is in the initial context.
To this end, the abstractions of bounce sequences (\pref{def:aBS}) solving these objectives are computed, leading to new required processes, and thus new objectives on other sorts.
Thus, the graph of local causality $\cwB = (\V, \E)$ defined in \pref{def:glc} is a graph where $\V \subset \Proc \cup \Obj \cup \Sol$ and $\E \subset \V \times \V$
in which edges link any required process $a_i$ in $\Proc$ to all corresponding objectives $P$ in $\Obj$,
and any objective $P$ to abstractions of bounce sequences $ps \in \Sol$ (\pref{def:aBS}) solving this objective;
such sets $ps$ then stand for new required processes.
In the case where a process $a_i$ belongs to a cooperative sort, it is directly linked to the set of the processes $\csState(a_i)$ (\pref{def:csState}) that it represents.
If the solving of an objective $P$ requires a process of $\PHsort(P)$, a continuity edge is added to re-target this objective (\pref{def:maxCont}).
Finally, as the active process of every sort may evolve, $\cwB$ is obtained iteratively by saturation with every process it contains,
\ie by overriding its initial context $\ctx$ by $\allprocs(\V, \E)$, defined by:
%in which the initial context $\ctx$ is updated by iteratively adding every process that appears in the graph ($\allprocs$).
%The aim of the this subsection is to build a graph of local causality that describes the propagation of objectives due to the bounce sequence abstraction of \pref{def:aBS}.
%This oriented graph has the form $(\V, \E)$, where $\V \subset \Proc \cup \Obj \cup \Sol$ and $\E \subset \V \times \V$.
%In order to build an accurate under-approximation, this graph is defined as a fixed point, in which each iteration saturates 
%We denote $\allprocs(\V, \E)$ the set of all processes involved in such a graph $(\V, \E)$, defined by: %in \pref{def:allprocs}.
$$\allprocs(\V, \E) = (V \cap \Proc) \cup \{ \PHtarget(P), \PHbounce(P) \mid P \in \V \cap \Obj \}$$

%The final graph of local causality is computed by a fixed point 
%is defined as a fixed point

\begin{definition}
\label{def:glc}
  The graph of local causality $\cwB = (\Bv, \Be)$ is defined as: $\cwB = \lfp{\aB^\w_\ctx}{\myB}{\aB^\w_{\ctx \Cap \allprocs(\myB)}}$,
  where $\myB = (\cwV, \cwE)$ is the smallest graph where:
  %$\cwV = \VProc \cup \VObj \cup \VSol$ with $\VProc \subset \PHproc$, $\VObj \subset \Obj$, $\VSol \subset \Sol$ and $\cwE \subset \cwV \times \cwV$
  $\cwV \subset \Proc \cup \Obj \cup \Sol$ and $\cwE \subset \cwV \times \cwV$
  so that:
  \begin{align}
    \w &\subset \cwV \label{eq:Vw} \\
%      \{ \PHbounce(P) \mid P \in \VProc \} &\subset \V\\
    P \in \VProc &\Rightarrow \PHbounce(P) \in \cwV \label{eq:Vproc} \\
    (x, y) \in \E &\Rightarrow y \in \cwV \label{eq:VE} \\
%    P \in \VObj &\Rightarrow \PHbounce(P) \in \V\\
%      \{ (P, ps) \in \Obj \times \powerset(\PHproc) \mid P \in \V \wedge ps \in \BS(P) \} &\subset \E\\ % Sol
    P \in \VObj \wedge ps \in \BS(P) &\Rightarrow (P, ps) \in \cwE \label{eq:ESol1} \\
%    \{ ps \in \powerset(\PHproc) \mid (P, ps) \in \E \} &\subset \V\\
%      \{ (ps, a_i) \in \powerset(\PHproc) \times \PHproc \mid a_i \in ps \} &\subset \E\\ % Sol
    ps \in \VSol \wedge a_i \in ps &\Rightarrow (ps, a_i) \in \cwE \label{eq:ESol2} \\
%    \{ a_i \in \PHproc \mid (ps, a_i) \in \E \} &\subset \V\\
%      \{ (a_i, \PHobjp{a}{j}{i}) \in \PHproc \times \Obj \mid a \in \components \wedge a_i \in \V \wedge a_j \in \ctx \} &\subset \E\\ % Req
    a \in \components \wedge a_i \in \VProc \wedge a_j \in \ctx &\Rightarrow (a_i, \PHobjp{a}{j}{i}) \in \cwE \label{eq:EReq} \\
%      \{ (a_i, ps) \in \Proc \times \powerset(\PHproc) \mid a \in \cs \wedge a_i \in \V \wedge ps \in \csState(a_i) \} &\subset \E\\ % ReqSolPrio
    a \in \cs \wedge a_i \in \VProc \wedge ps \in \csState(a_i) &\Rightarrow (a_i, ps) \in \cwE \label{eq:EPrio} \\
%    \{ \PHobjp{a}{j}{i} \in \Obj \mid (a_i, \PHobjp{a}{j}{i}) \in \E \} &\subset \V\\
%      \{ (P, \PHobj{q}{\PHbounce(P)}) \in \Obj \times \Obj \mid P \in \V \wedge q \in \gCont_\ctx(\PHsort(P), P) \} &\subset \E\\ % Cont
    P \in \VObj \wedge q \in \gCont_\ctx(\PHsort(P), P) &\Rightarrow (P, \PHobj{q}{\PHbounce(P)}) \in \cwE \label{eq:ECont}
  \end{align}
\end{definition}

\pref{eq:Vw}, \eqref{eq:Vproc} and \eqref{eq:VE} ensure that all required nodes are in $\cwV$.
Elements in $\PHproc$ are required processes,
elements in $\Obj$ are the objectives to reach these processes
and elements in $\Sol$ are the sets of processes that allow to solve these objectives.
\pref{eq:ESol1} links an objective $P$ to its solutions in $\BS(P)$
and \pref{eq:ESol2} links a solution $ps$ to the processes $a_i \in ps$ it contains.
Then, \pref{eq:EReq} links a required component process $a_i$ to all objectives $\PHobjp{a}{j}{i}$ with $a_j \in \ctx$
while \pref{eq:EPrio} links a required cooperative sort process $a_i$ to its solutions in $\csState(a_i)$.
Finally, if the resolution of an objective $P$ requires a process of $\PHsort(P)$, then \pref{eq:ECont} re-targets this objective ($\gCont$).

%In the following, we denote:
%$\BvProc = \Bv \cap \PHproc$, $\BvObj = \Bv \cap \Obj$ and $\BvSol = \Bv \cap \Sol$.
%$\Bee{X}{Y} = \Be \cap (X \times Y)$, with $X, Y$ amongst $\PHproc$, $\Obj$ and $\Sol$.
%
\begin{comment}
\begin{definition}
  \label{def:aS}
  The abstract structure $\cwB=(\Breq,\Bsol,\Brsp,\Bcont)$ is defined as
  $\cwB = \sfp{\aB^\w_\ctx}{\myB}{\aB^\w_{\ctx \Cap \allprocs_\ctx(\myB)}}$,\\
  with $\myB=(\myreq,\mysol,\myrsp,\mycont)$:
  \begin{align*}
    \myreq &= \{ (a_i,\PHobjp{a}{j}{i}) \in \PHproc \times \Obj \mid
      a \in \components \wedge a_j \in \PHget{\ctx}{a} \\ % \vee a_j \in \pfpprocs_\ctx(a) \\
      & \qquad \wedge ((\exists (P,ps) \in \mysol \vee \exists (b_j, ps) \in \myrsp), a_i \in ps \\
      & \qquad\qquad \vee \exists n \in \indexes{\w}, \PHbounce(\w_n)=a_i) \}
    \\
    \mysol &\subseteq \{ (P,ps) \in \Obj \times \powerset(\PHproc) \mid
            \exists (a_i, P) \in \myreq \wedge ps \in \aBS(P) \\
      & \qquad\qquad \vee \exists (Q, P) \in \mycont \wedge ps \in \aBS(P) \}
    \\
    \myrsp &= \{ (a_i,ps) \in \PHproc \times \powerset(\PHproc) \mid a \in \cs \\
      & \qquad \wedge (\exists (P,ps') \in \mysol \vee \exists (b_j,ps') \in \myrsp), \\
      & \qquad\qquad a_i \in ps' \wedge \csState(a_i) \in ps \}
    \\
    \mycont & = \{ (P, \PHobj{q}{\PHbounce(P)}) \in \Obj \times \Obj \mid
      \exists (P, ps) \in \mysol \\
      & \qquad\qquad \wedge q \in \gCont_\ctx(\PHsort(P),P) \}
  \end{align*}
\end{definition}

\begin{definition}[Coherent solution]
\label{def:coherent}
  A cooperative sort requisite $(a_i, ps)$ in $\Brsp$ is said coherent iff
  it has no successor $(b_k,\PHobjp{b}{j}{k})$ in $\Breq$ so that there exists $b_n \in ps$, $b_k \neq b_n$.
\end{definition}
\end{comment}
%
%\towrite{Expliquer}
%We define in \pref{def:coherent} the notion of coherent edge in the graph of local causality.
In the graph of local causality, an edge linking a process of a cooperative sort to a solution set $ps$ is said coherent (\pref{def:coherent}) if none of the processes in $ps$ conflict with the children processes of $ps$.
Then, \pref{th:approxinf} gives a sufficient condition for the concretization of a sequence of objectives in a given context,
which is derived immediately from the graph of local causality.
A proof of this theorem is given in Annex~\ref{suppl:demoapproxinf}.
\begin{definition}[Coherent edge]
\label{def:coherent}
  An edge $(x, y) \in \cwE$ is said coherent if and only if:
%  $(x, y) \in \Bee{\Proc}{\Sol} \Rightarrow y$ has no process successor $a_j \in \Bv \cap \Proc$ so that $\exists a_i \in y$, $a_i \neq a_k$.
  $(x, y) \in \Be \cap (\Proc \times \Sol) \Rightarrow y$ has no children process $a_j \in \Bv \cap \Proc$ so that $\exists a_i \in y$, $a_i \neq a_k$.
\end{definition}

\begin{theorem}[Under-Approximation]
\label{th:approxinf}
  If the graph $\cwB$ contains no cycle,
  all objectives have at least one solution
  and all edges are coherent,
  then $\uconcr(\w) \neq \emptyset$.
\end{theorem}



\begin{example}
  %As an example showing the usefulness of our method, we can consider \pref{fig:ph-livelock}.
  Let $\PH' = (\PHs, \PHl, \PHh'^{\langle 1 \rangle})$ be the “flattened” version of the PH in \pref{fig:ph-livelock}, that is: $\PHh'^{\langle 1 \rangle} = \PHh^{(1)} \cup \PHh^{(2)}$,
  %As this PH model contains only one class of priority, 
  which is equivalent to a PH in the semantics without priorities.
  The under-approximation on such models developed in~\cite{PMR12-MSCS} concludes that $c_1$ is reachable from $\ctx = \PHstate{a_1, b_0, c_0, ab_{10}}$;
  this is due to an over-approximation of the behaviours inherent to cooperative sorts in the semantics without priorities.
  
  This unwanted behaviour is palliated by the semantics of PH with priorities proposed in this paper.
  Indeed, the under-approximation given in \pref{th:approxinf} does not conclude regarding the reachability of $c_1$,
  as one of the edges of the resulting graph of local causality is not coherent (\pref{def:coherent}),
  as shown in in \pref{fig:sa-livelock}.
  
  We note that if the actions $\PHhit{a_0}{b_0}{b_1}$ and $\PHhit{b_0}{a_0}{a_1}$ are replaced by the self-hits
  $\PHhit{a_0}{a_0}{a_1}$ and $\PHhit{b_0}{b_0}{b_1}$,
  then \pref{th:approxinf} concludes that $c_1$ is reachable from $\ctx$.
\end{example}

\begin{figure}[p]
  \centering
  \begin{tikzpicture}[aS]
    \node[Aproc] (c1) {$c_1$};
    \node[Aobj,below of=c1] (c01) {$\PHobj{c_0}{c_1}$};
    \node[Asol,below of=c01] (c01s) {};

    \node[Aproc,below of=c01s] (ab11) {$ab_{11}$};
    \node[AsolPrio,below of=ab11] (ab11s) {};

    \node[Aproc,below left of=ab11s] (a1) {$a_1$};
    \node[Aobj,below of=a1] (a11) {$\PHobj{a_1}{a_1}$};
    \node[Asol,below of=a11] (a11s) {};
    \node[Aobj,below left of=a1] (a01) {$\PHobj{a_0}{a_1}$};
    \node[Asol,below of=a01] (a01s) {};
    \node[Aproc,below of=a01s] (b0) {$b_0$};
    \node[Aobj,below of=b0] (b00) {$\PHobj{b_0}{b_0}$};
    \node[Asol,below of=b00] (b00s) {};
    \node[Aobj,below left of=b0] (b10) {$\PHobj{b_1}{b_0}$};
    \node[Asol,below of=b10] (b10s) {};

    \node[Aproc,below right of=ab11s] (b1) {$b_1$};
    \node[Aobj,below of=b1] (b11) {$\PHobj{b_1}{b_1}$};
    \node[Asol,below of=b11] (b11s) {};
    \node[Aobj,below right of=b1] (b01) {$\PHobj{b_0}{b_1}$};
    \node[Asol,below of=b01] (b01s) {};
    \node[Aproc,below of=b01s] (a0) {$a_0$};
    \node[Aobj,below of=a0] (a00) {$\PHobj{a_0}{a_0}$};
    \node[Asol,below of=a00] (a00s) {};
    \node[Aobj,below right of=a0] (a10) {$\PHobj{a_1}{a_0}$};
    \node[Asol,below of=a10] (a10s) {};

    \path
    (c1) edge (c01)
    (c01) edge (c01s)
    (c01s) edge (ab11)
    (ab11) edge (ab11s)
    (ab11s) edge (a1) edge (b1)

    (a1) edge (a01) edge (a11)
    (a01) edge (a01s)
    (a01s) edge (b0)
    (a11) edge (a11s)
    (a0) edge (a10) edge (a00)
    (a10) edge (a10s)
    (a00) edge (a00s)

    (b0) edge (b10) edge (b00)
    (b10) edge (b10s)
    (b00) edge (b00s)
    (b1) edge (b01) edge (b11)
    (b01) edge (b01s)
    (b01s) edge (a0)
    (b11) edge (b11s)
    ;
    \end{tikzpicture}
  \caption{
  \label{fig:sa-livelock}
    The graph of local causality of the PH model in \pref{fig:ph-livelock}.
    \pref{th:approxinf} is inconclusive on this example as the only solution of process $ab_{11}$ is not coherent (\pref{def:coherent}).
  }
\end{figure}



\subsection{Reachability of a state}

\newcommand{\total}{\tau}
\newcommand{\reach}{\sigma}

The semantics of PH with $2$ classes of priorities studied in this paper allows to model cooperative sorts accurately representing a coherent configuration of a set of sorts.
%As the method presented in this section allows to efficiently conclude on the reachability of a process,
Therefore, we can derive a new method to conclude about the reachability of a state (considering only components).
Indeed, let $\PH = (\PHs, \PHl, (\PHh^{(1)}, \PHh^{(2)}))$ be a PH and suppose that we want to study the reachability of a state $s \in \PHl$.
Let $\PH' = (\PHs', \PHl', (\PHh'^{(1)}, \PHh'^{(2)}))$
with: $\PHs' = \PHs \cup \{ \total, \reach \}$ and $\PHl' = \PHl \times \PHl_\total \times \PHl_\reach$,
where $\total$ is a cooperative sort on all components $\components$ of $\PH$ (thus $\PHl_\total = \underset{a \in \components}{\times} \PHl_a$)
and $\reach$ is a component with $\PHl_\reach = \{ \reach_0, \reach_1 \}$;
furthermore, $\PHh'^{(1)}$ is the set $\PHh^{(1)}$ completed with all actions updating the cooperative sort $\total$,
and $\PHh'^{(2)} = \PHh^{(2)} \cup \{ \PHhit{\pfp_s(\total)}{\reach_0}{\reach_1} \}$.

Given an initial context $\ctx$, the reachability of $s$ in $\PH$ is equivalent to the concretization of $\PHobjp{\reach}{0}{1}$ in $\PH'$ from the initial context $\ctx \cup \{ \reach_0 \}$ (the initial state of $\total$ does not matter), which can be efficiently under-approximated using \pref{th:approxinf}.
Indeed, the additional action $\PHhit{\pfp_s(\total)}{\reach_0}{\reach_1}$ in $\PHh'^{(2)}$ allows to conclude on the reachability of process $\pfp_s(\total)$, that is, on the reachability of the state $s$ (considering only the components).

%It is also possible to compute the reachability of a partial state
%by modelling in $\total$ the cooperation of a subset of $\components$,
It is also possible to compute the reachability of a set of states $S \subset \PHl$ by creating several actions $\PHhit{\total_s}{\reach_0}{\reach_1}$ in $\PHh^{(2)}$ for each state $s \in S$.
