\section{Proof of Under-approximation (\pref{ssec:ua})}
\label{suppl:demoapproxinf}

We introduce for this proof the notion of context (\pref{def:context}),
which extends the notion of state to a set of possible initial states:
to each automaton in the model, a context maps a set of local states in this automaton.
\begin{definition}[Context]
\label{def:context}
  A \emph{context} $\varsigma$ associates to each automaton in $\PHs$ a non-empty subset of its local states:
  $\forall a \in \PHs, \varsigma(a) \subseteq \PHl_a \wedge \varsigma(a) \neq \emptyset$.
%  $\Ctx$ is the set of all contexts.
\end{definition}

For a given context $\varsigma$,
%we note $a_i \in \varsigma$ if and only if $a_i \in \varsigma(a)$,
and for all state $s \in \PHl$,
we note $s \subseteq \varsigma$ if and only if $\forall a_i \in s, a_i \in \varsigma(a)$.

In the following, we also denote:
$\Bee{X}{Y} = \Be \cap (X \times Y)$, with $X, Y$
amongst $\PHproc$, $\Obj$, $\sSol$ and $\Sol$.

\newproof{proofapproxinf}{Proof of \pref{th:approxinf}}
\begin{proofapproxinf}
Given the LCG $\cwB=(\cwV,\cwE)$,
we note $\varsigma = \{a\mapsto \cwV\cap\PHl_a \mid a\in\PHs\}$ the context supported by $\cwB$.

Let $ps \in \Bv \cap \sSol$ be a set of hitters
that is a node in the LCG,
and suppose all of its successors are concretizable.
We first want to demonstrate that
there exists a scenario that activates all the local states it contains.
We label all local states of $ps$ by an integer: $ps = \{ p_m \}_{m \in \indexes{ps}}$.
Let us prove by induction that for all $n \in \{ 0 \} \cup \indexes{ps}$,
there exists a scenario $\delta_n$ so that:
$\forall i \in \segm{1}{n}, \PHget{(s \PHplay \delta_n)}{\PHsort(p_i)} = p_i$.
\begin{itemize}
  \item It is straightforward that $\varepsilon$ is a valid value for $\delta_0$.
  \item Suppose such $\delta_n$ exists and let $q = \PHget{(s \PHplay \delta_n)}{\PHsort(p_{n+1})}$.
    By construction, $p_{n+1} \in \Bv \cap \Proc$ is a child of $ps$.
    Furthermore, by hypothesis, $ps$ is independent (see \pref{def:coherent}).
    This means that amongst all the successors in $\Proc$ of $p_{n+1}$,
    there does not exist a local state $b_j$ to that
    $\exists b_k \in ps, \PHsort(b_j) = \PHsort(b_k) \wedge b_j \neq b_k$;
    in other words, the resolution of $p_{n+1}$ does not require a local state
    that may change the other local states of the set $ps$.
    Therefore, there exists $\delta' \in \mconcr_{s \PHplay \delta_n}(\PHobj{q}{p_{n+1}})$,
    so that $\forall i \in \segm{1}{n+1}$, $\PHget{(s \PHplay \delta_n \PHplay \delta')}{\PHsort(p_{i})} = p_{i}$.
    We denote then: $\delta_{n+1} = \delta_n \play \delta'$.
\end{itemize}
Therefore, $\delta = \delta_{\card{ps}}$ exists, and given its properties, we have:
$\forall i \in \segm{1}{\card{ps}}, \PHget{(s \PHplay \delta)}{\PHsort(p_{i})} = p_{i}$.

As there is no cycle in $\cwB$, we show by induction in the following that
$\forall s\in \PHl$ so that $s\subseteq \varsigma$ and $\forall P \in \Bv \cap \Obj,
\PHtarget(P) \in s \Longrightarrow \exists \delta \in \mconcr_s(P)$.
\begin{itemize}
  \item If $(P, \langle P, \{ \emptyset \} \rangle) \in \Bee{\Obj}{\Sol}$,
    either $\PHtarget(P) = \PHbounce(P)$ and $\delta = \emptyseq$;
    or $\exists \zeta \in \BS(P), \zeta \in \Sce(s) \wedge
      \forall i \in \indexes{\zeta}, \hitter{\zeta_i} = \emptyset$
    and in this case, $\delta = \zeta$ is a valid scenario in $s$.

  \item Suppose all successors objectives of $P$ are concretizable.
    If $\exists (P, Q) \in \Bee{\Obj}{\Obj}$, then by hypothesis,
      $\mconcr_{s}(\obj{\PHtarget(P)}{\PHtarget(Q)} \concat Q) \neq \emptyset$, thus
      $\mconcr_{s}(P) \neq \emptyset$.
    Else, by \pref{def:lcg}-\ref{lcg-conn}, the concretizations of the successors of $P$ require no local state of automaton $\PHsort(P)$.
      Furthermore, there exists $\zeta \in \BS(P)$ so that $(P, \aZ) \in \Bee{\Obj}{\Sol}$.
      We show by induction that for all $n \in \indexes{\zeta}$, there is a scenario $\delta_n$ so that $\PHget{(s \PHplay \delta_n)}{\PHsort(P)} = \PHbounce(\zeta_n)$.
      \begin{itemize}
%        \item[$\circ$] If $\zeta = \emptyseq$, then trivially, $\delta = \emptyseq$.
        \item[$\circ$] Suppose that $\delta_n$ exists and let $\zeta_n = \PHhit{A}{a_j}{a_k}$.
        By construction, there exists $A \in \Bv \cap \sSol$
        amongst the children of $\aZ$.
        By the first result of this demonstration,
        there exists a scenario $\delta'$ in $s \play \delta_n$ so that
        $\forall a_i \in A, \PHget{(s \play \delta_n \PHplay \delta')}{a} = a_i$.
        Therefore, $\zeta_n$ is playable in $s \play \delta_n \PHplay \delta'$,
        and $\delta_{n+1} = \delta_n \concat \delta' \concat \zeta_n$.
      \end{itemize}
      Thus, $\delta_{|\zeta|} \in \mconcr_s(P)$. % and $\ceil(\delta) \subseteq maxs$.
\end{itemize}
\end{proofapproxinf}
