\documentclass[11pt]{article}

%\usepackage{babel}
\usepackage[utf8]{inputenc}
\usepackage[T1]{fontenc}

\usepackage{color}
\usepackage{fullpage}

\def\BibTex{{\rm B\kern-.05em{\sc i\kern-.025em b}\kern-.08em
         T\kern-.1667em\lower.7ex\hbox{E}\kern-.125emX}}
\usepackage{amsmath}
\usepackage{amssymb}

% \newcounter{pointno}
% \setcounter{pointno}{0}
% \def\thepoint{\arabic{pointno}}
% \newcommand{\ptno}[1]{\textcolor{blue}{\refstepcounter{pointno} \thepoint. \#1}}
% \newcommand{\question}[1]{\bigskip{\it\ptno{\#1}}}
% \def\answer{}

\newcommand{\todo}[1]{\textcolor{red}{\textbf{[TODO: #1]}}}
%+\newcommand{\answertodo}[1]{\textcolor{red}{\textbf{À appliquer :}} \textcolor{magenta}{#1}}

\newcommand{\answer}[1]{\textcolor{blue}{#1}\vspace*{1em}}

\title{Response to the reviewers --- First review of “Sufficient Conditions for Reachability in Automata Networks with Priorities”}
\author{Maxime Folschette, Loïc Paulevé, Morgan Magnin, Olivier Roux}
\date{}



\begin{document}

\maketitle



\section*{Summary of the main changes}

We wish to thank the reviewers for their comments.
Please find below a summary of the main changes brought to the paper, and a detailed response to all comments.

\begin{itemize}
  \item \todo{Main changes}
\end{itemize}



\section*{Comments and answers to the reviewers}

\subsection*{Reviewer \#1}


Reviewer \#1: In this article the  Asynchronous Automata Networks (AANs) are studied. An AAN is an Automata Network with synchronised transitions between automata, where each transition changes the local state of exactly one automaton, but the number of synchronazing state sis not restricted.  The AAN with priorities are also studied. 

Authors build a method of efficient under-approximation for the problem of state reachability. The computational cost is lower than the cost of usual model checker methods. Also, a reduction from AANs with priorities to the AANs without priorities is given.

Clearly, the paper is based on real applications of the system studied. It seems that the method and the theory are build for and from some specific applications with a interesting idea. Therefore, from the purely automata theoretical point of view,  the methods and results are not very strong nor interesting as themself. On the other hand, the method is used in two example applications in the end of the paper to support the fact that the model and the method work well in practise. This is probably the strongest part of the paper.  

General: 1) somehow the system of AANs seems such that it has conection or roots or generalization by cellural automata. Obviously, CA are a more general model and AANs are a restriction, but it could be interesting to see whether can you generalize or simulate in simple manners your system with CA?

2) Just a note: As I see it, the AANs is an interactive model of computation, and that is interesting. based on this, can you consider the reachability of state as a game in (multiplayer) game of AAN? Does  your work give a fast method for checking for winning strategy?   
 
Minor coments to the authors: 

page 1, authors: names are not in alphabetical order probably for a good reason, but it strikes my mathematician eye badly.

page 3, line 13: is much straightforward. More?

page 4, line 20: size, I would say length.

page 8, line -5 -- -4. Index a and b for delta are a bad choise a and b are for the automata and states, and delta's subscripts are numbers.



\subsection*{Reviewer \#2}



Reviewer \#2: SUMMARY OF THE RESULTS

In this paper the authors introduce so-called asynchronous automata networks, which are automata networks with transitions synchronized between the automata in such a way that each transition changes the local state of exactly one automaton. There is also a generalization of asynchronous automata networks, called asynchronous automata networks with classes of priorities. This model is shown to be equivalent to the named networks without priorities.

The authors have developed a method of performing reachability analysis and a way of getting a valid execution path if the reachability question has a positive answer. The method of performing reachability analysis is based upon abstract interpretation, and the complexity of such an analysis is polynomial in the total number of local states and exponential in the number of local states within a single automaton.

The applicability of the method has been demonstrated by the authors by considering two large-scale biological models.

My main concern about the paper is that it is written in a rather unclear manner. Some of the definitions are given in a very formal way; this does not help understanding the intuition behind them. On the other hand, the authors have managed to give a very good intuition in description of the running examples of the paper.

In my opinion, the presentation of the material of the paper should be improved.

COMMENTS TO AUTHORS

page 1, line 49 ---
I would reformulate the phrase "with [4,5]" somehow in different way.

page 1, line 52 ---
Replace "behaviours" in the phrase "their usual behaviours" with "behaviour".

page 2, line 21 ---
Remove the second verb "are" in the phrase "This is modelled by actions are of the form".

page 2, line 27, "avoid to build" ---
Reformulate as "avoid building".
The same comment applies to many other places within the paper (for instance, page 5, line 51, page 3, line 43).

page 7, line 29 ---
I would replace "Subsect." with "Subsection" throughout the paper.

page 7, Definition 4, line 55, in the sentence "The set of all objectives is called ..." ---
Using the logical AND operator "$\land$" is too formal and this unnecessarily complicates reading of the definition.
I would replace "$\land$" with a word "and".

page 8, Definition 6 ---
The comment "($\gamma_s: OSeq \to$ ...)" in the parentheses after "Definition 6" complicates the perception of the material. This also applies to other definitions.

page 8, line 52 ---
To which definition does the phrase "in the previous definition" refer to?

page 8, Lemma 1 ---
I recommend changing the formulation of the lemma so that it contains at least some words.
The comment right before the formulation of the lemma states essentially the same thing as the lemma, without any intuition.

page 9, line 53 ---
Replace "would consists" either with "would consist" or with "consists".

page 10, comment before Definition 8 ---
First of all, I would give a shorter intuition before Definition 8 and only then this delaited explanation of the definition. The definition itself is very difficult to follow and the explanation is written in a really succinct way, that does not make it easier to get. Perhaps the authors could state how this definition is different from a similar definition in the cited work [16].

page 10, line 53 ---
In the phrase "we prove that if ..., if ..., and if ..., then" I would only leave the first "if" and remove the two others.

page 11, line 33 ---
Remove "we know" in the sentence "Then, we know (by induction)".

page 12, lines 10-12 ---
In the sentence "in order to reduce the number ..., and potentially removing cycles" replace "removing" with "remove".

page 12, line 14 ---
Something seems to be wrong with the phrase "would imply to enumerate".

page 12, lind 18 ---
In the sentence "from the initial state ... depicted" I would replace "depicted" to "illustrated" and actually put this word to another place within the sentence.

page 12, line 20 ---
I think the phrase "but does not conclude regarding the reachability of $c_1$" could be rewritten in a simpler way.

page 15, line 12 ---
Put a comma before "where" in the phrase "having n+1 local states where n".
This also concerns some other places of the paper.

page 15, line 16 ---
Remove the second period in "etc.).".

page 15, line 26 ---
I would consider using the term "transformation" instead of "translation".

page 15, paragraph before Definition 10 ---
This is a comment both to Definition 10 and Definition 11. I would try to "separate" them.

page 16, line 56 ---
Consider replacing "in the following" with "in what follows".
This comment applies to several other places in the paper.

page 19, Section 5 ---
The phrases "this section aims at giving application examples", "we give in this section a detailed example", "we consider in the following a model of" repeat the same idea within a half of the page. I would try to restructure the introductory text to Section 5 and its Subsection 5.1.

page 20, line 25 ---
Remove the second article in "the the".

page 20, line 46 ---
The sentence starting with "This consists in using Theorem 1 ..." is written in a very sophisticated way.

page 29, reference [18] ---
The year and the pages of the publication are missing.

\end{document}
